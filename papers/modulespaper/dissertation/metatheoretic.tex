%!TEX root = formalsys.tex

\section{Metatheoretic Properties} 
$\Upsilon$ is a sequence of bindings of entity variables to entities. Let ($\sqsubseteq_{ee}, \Upsilon$) be the total ordering of the bindings in $\Upsilon$ according to the ordering in the sequence. For example, for $\Upsilon = [\rho_0 \mapsto \tau^0][\rho_1 \mapsto \tau^0][\rho_2 \mapsto \tau^0]$, $[\rho_0 \mapsto \tau^0] \sqsubseteq_{ee} [\rho_2 \mapsto \tau^0]$. 

$\Upsilon, \Sigma$, and $\Gamma$ are sequences of bindings. The first two are injective but $\Gamma$ is not. In either case, the notation $b \in \Upsilon$ denotes that the binding $b$ is in the sequence of bindings $\Upsilon$. 

\begin{definition}
$\Gamma(x)$ denotes the rightmost binding such that $[y\mapsto \gamma] \in \Gamma$ and $x=y$. 
\end{definition}


\begin{lemma}
For instantiation, let the set of all entity variables $V$ be such that $\rho\in V$ iff $[x\mapsto(\rho,s^v)]$ is in the signature. It must be the case that $V\cap dom(\Upsilon) = \emptyset$.
\end{lemma}
\begin{proof}
\begin{description}
\item[Rule (38) (opaque signature match)] $\Rightarrow_{sig}$ produced $\Sigma_{spec}$ such that all $\rho$'s fresh in $\Gamma,\Upsilon$, dom($\Upsilon_{spec}$)$\cap EV(\Gamma) = \emptyset$ and $dom(\Upsilon_{spec})\cap EV(\Upsilon) = \emptyset$. 
\end{description}
\end{proof}

Properties of $\Upsilon$: 
\begin{enumerate}
\item $\Upsilon$ is injective
\end{enumerate}


\begin{definition}[$\Upsilon(\vec{\rho})$]
If $\vec{\rho}$ is singleton $\rho_0$, $\Upsilon(\rho_0) = \upsilon$ such that $[\rho_0\mapsto\upsilon] \in \Upsilon$ of which there is one because $\Upsilon$ is injective. 
\end{definition}

\begin{definition}[$\Upsilon^{-1}(\tau^n)$]
Let $\rho$ such that $[\rho\mapsto\tau^n]$ is in $\Upsilon$ and it is the least binding in $\Upsilon$ that maps to $\tau^n$. 
\end{definition}

\begin{definition}
Let $\Sigma(x)$ denote $spec$ where $[x\mapsto spec] \in \Sigma$.
\end{definition}

\begin{lemma}

\end{lemma}

\section{Key Properties From Other Work}
\begin{theorem}[Strong Normalization of Tycons]
If $\emptyset,\emptyset\vdash C^s :: K$, then there exists $\tau^n$ such that $C^s \mapsto^*  \tau^n$
\end{theorem}

\begin{proof}[by induction on the derivation of well-kinding]
\begin{enumerate}
\item Type Variable: Type variable is not closed. 
\item Application: By inversion, 
\item Projection
\item Universal quantification
\end{enumerate}
\end{proof}


Shao proves the NRC to TGC (System F$_\omega$) elaboration terminates, that the elaboration process produces unique (principal) typings, that the elaboration relation preserves typing.  

\begin{theorem}[Termination]
Suppose B is a well-formed basis and d is an NRC declaration, if d is
normalized with respect to B, then elaboration $B\vdash^d
d:N;D;R\Rightarrow \bullet$ always terminates. Similar proposition
holds for module expressions. 
\end{theorem}
\begin{proof}[by contradiction]
Suppose that elaboration does not terminate. Then there is an infinite
elaboration derivation tree. One can construct a corresponding
derivation tree for the simply-typed lambda calculus (extended with
records and integers) operational
semantics. Functor realizations correspond to function closures that
calculuate the total number of functor applications. But that would mean that there exists a non-terminating
simply-typed lambda calculus term. Contradiction. 
\end{proof}

\begin{definition}
A well-formed basis $B=(\Delta,N,D,R)$ preserves TGC typing if
\begin{enumerate}
\item for each type component $p_t \in defPaths(D)$, suppose $D(p_t)$ has kind $\kappa_t$, and $\mu_t = m2t(R(p_t))$, then $\Delta \triangleright \mu_t ::\kappa_t$;
\item for each structure component $p_s \in defPaths(D)$, suppose $M_s = D(p_s)$ and $r_s = R(p_s)$ and $\vdash^x M_s \Rightarrow \kappa_t$ and $\vdash^x (M_s, r_s) \Rightarrow \mu_t;\sigma_t$, then $\Delta\triangleright \mu_t::\kappa_t$ and $\Delta\triangleright\sigma_t$;
\item for each functor component $p_f \in defPaths(D)$, suppose $M_f = D(p_f)$ and $r_f = R(p_f)$ and $\vdash^x M_f \Rightarrow \kappa_t$ and $\vdash^x (M_f,r_f) \Rightarrow \mu_t;\sigma_t$, then there exists TGC entities $\kappa'_t, \kappa''_t, \mu'_t, \sigma'_t$, and $\sigma''_t$ such that $\kappa_t = \kappa'_t \to \kappa''_t$ and $\Delta \triangleright \mu_t \equiv_t (\lambda s_i::\kappa'_t.\mu'_t)::\kappa_t$ and $\Delta\triangleright \sigma_t \equiv_t (\forall s_i::\kappa'_t.\sigma'_t\to\sigma''_t)$; furthermore, if $r_f = (m'_b, B',\bullet)$, then B' preserves TGC typing as well. 
\end{enumerate}
\end{definition}

\begin{theorem}[Shao Type Preservation]
Given a well-formed basis B, suppose B preserves TGC typing and $\Delta$ and $\Gamma$ are its derived TGC kind and type environments, then
\begin{enumerate}
\item for each NRC module expression m\_, if m\_ is normalized with respect to B and $B \vdash^m m\_: N; M\_; r\_ \Rightarrow e_t$ and $\vdash^x M\_ \Rightarrow \kappa_t$ and $\vdash^x (M\_,r\_) \Rightarrow \mu_t;\sigma_t$, then $\Delta\triangleright\mu_t::\kappa_t$ and $\Delta;\Gamma\vdash^t e_t : \sigma_t$;
\item for each NRC module declaration d, if d is normalized with respect to B and $B\vdash^d d: N;D;R \Rightarrow d_t$, and $\Gamma'$ is a the derived TGC type environment constructed from R and D, then $B\oplus(\emptyset_\Delta,N,D,R)$ preserves TGC typing and $\Delta;\Gamma\vdash^t d_t:\Gamma'$. 
\end{enumerate}
\end{theorem}
\begin{proof}[by induction on the derivations.]
\end{proof}

Dreyer et al show Soundness of Elaboration and Decidability of Elaboration (this is straightforward except for the nondeterministic signature matching instantiation). 
\begin{theorem}[Dreyer Soundness]
Provided $\Gamma\vdash \Box$ we have:
\begin{enumerate}
\item If $\Gamma\vdash T: \kappa \leadsto \tau$, then $\Gamma\vdash \tau:\kappa$
\item If $\Gamma\vdash E:\tau\leadsto e$, then $\Gamma\vdash e:\tau$
\item If $\Gamma\vdash \tau \le \tau' \leadsto f$ and $\Gamma\vdash \tau:\Omega$ and $\Gamma\vdash \tau':\Omega$, then $\Gamma\vdash f:\tau \to \tau'$. 
\item If $\Gamma\vdash S\leadsto \Xi$, then $\Gamma\vdash \Xi:\Omega$. 
\item If $\Gamma\vdash D\leadsto \Xi$, then $\Gamma\vdash \Xi:\Omega$.
\item If $\Gamma\vdash M:\Xi\leadsto e$, then $\Gamma\vdash e:\Xi$. 
\item If $\Gamma\vdash B:\Xi\leadsto e$, then $\Gamma\vdash e:\Xi$.
\item If $\Gamma\vdash P:\Sigma\leadsto e$, then $\Gamma\vdash e:\Sigma$.
\item If $\Gamma\vdash \Xi \le \Xi'\leadsto f$ and $\Gamma\vdash \Xi:\Omega$ and $\Gamma\vdash \Xi':\Omega$, then $\Gamma\vdash f : \Xi\to \Xi'$. 
\item If $\Gamma\vdash \Sigma \le \exists \overline{\alpha}.\Sigma' \uparrow \overline{\tau} \leadsto f$ and $\Gamma\vdash \Sigma:\Omega$ and $\Gamma,\overline{\alpha}\vdash\Sigma':\Omega$, then $\overline{\Gamma\vdash \tau:\kappa_\alpha}$ and $\Gamma\vdash f:\Sigma\to\Sigma'[\overline{\tau}/\overline{\alpha}]$. 
\end{enumerate}
\end{theorem}
\begin{proof}[by induction on the derivations. Proved in Coq.]
\end{proof}



