
\chapter{Translation}\label{ch:translation}

One way to show the soundness of the module system is to provide a
translation to an applied F$_\omega$ language. What this translation
amounts to is a decoupling of the static content from the dynamic
content of modules. Module systems that permit decoupling are
said to admit \emph{phase separation}. 

The elaboration semantics given in the previous chapter is sufficient
and self-contained. However, to support translation into F$_\omega$,
the elaboration must be instrumented to annotate module language terms
with their full signatures, full functor signatures, signature
matching coercions, and free instantiation of functor parameters. 

\section{Annotated Module Language}

%!TEX root = ../main.tex

\begin{figure}
\hrule
\[
\begin{array}{lrcl}
\textrm{Decls} & d^c & ::= & \mathsf{val}~x=e~
|~\mathsf{type}~t=C^\lambda\\
 & & ~~| & \mathsf{datatype}~\vec{\alpha}~t\\
        & spec & ::= & \mathsf{structure}~x :
        sigexp~|~\mathsf{type}~\vec{\alpha}~t\\
        & & ~~| & \mathsf{type}~t = C^\lambda\\ 
	& & ~~| & \mathsf{functor}~f(x:sigexp_1) : sigexp_2\\
        & & ~~| & \mathsf{val}~x:T\\
        & specs & ::= & \emptyset_{specs}~|~spec, specs\\
	& sigexp & ::= &
        x~|~\mathsf{sig}~specs~\mathsf{end}\\
        & & ~~| & sigexp~\mathsf{where~
          type}~p=C^\lambda\\
	& strexp & ::= & p\an{\vec{\rho}}~|~\mathsf{struct} ~d^m~\mathsf{end}\\
        & & ~~| & p\an{\vec{\rho},\rho_{par}}(strexp\an{M,\rho_u})\\
        & & ~~| & strexp:sigexp\\
        & & ~~| & strexp:>sigexp\\
\textrm{Mod decls}	& d^m & ::= & \circ~|~\mathsf{structure}~x\an{\rho} = strexp,
d^m\\
       & & ~~| & \mathsf{functor}\an{\rho,F,\Upsilon}
       ~f(x:sigexp)\\
       & &  & =strexp, d^m\\
       & & ~~| & d^c, d^m\\
\textrm{Top decls} & d^t & ::= &
\circ~|~\mathsf{signature}~x=sigexp,d^t\\ 
& & ~~| & d^m,d^t

\end{array}
\]
\hrule
\caption{Annotated module language}
\label{fig:annotatedlang}
\end{figure}



Fig.~\ref{fig:annotatedlang} gives the grammar for the module language
annotated with semantic objects from elaboration. The semantic objects
are enclosed by angle brackets $\an{}$. Structure paths must be
annotated with the entity path corresponding to that symbolic
path. The functor path in a functor application should be annotated
with its entity path and the entity variable for the formal
parameter. The entity variable for the formal parameter is necessary
for producing the necessary coercions to get the type and value
content of an argument structure into the proper form. The argument
structure should be annotated with its coercion full signature and coercion
structure entity expression. The associated entity variable should
annotate both structure and functor declarations. The functor
declaration further requires the annotation of a full functor
signature and free instantiation entity environment. All this
information is readily available during the elaboration process. Thus,
annotating is straightforward. Below, I show a select modified rule:  

\begin{equation} 
\infer{
\begin{array}{c}
\Gamma,\Upsilon\vdash p(strexp)\Rightarrow_{str}
(p\an{\vec{\rho},\rho_{par}}(strexp\an{M_{c},\rho_u}),(\Sigma_{body},R_{app}),\varphi_{app})
\end{array}}
	{
\begin{array}{c}
\Gamma(p) = (\vec{\rho}, (\Pi \rho_{par}:\Sigma_{par}.\Sigma_{body},
\langle\theta; \Upsilon'\rangle)) \qquad
\Gamma,\Upsilon\vdash strexp\Rightarrow_{str}
(M,\varphi)\\ 
\Upsilon\vdash (M,\varphi) : \Sigma_{par} \Rightarrow_{match}
(M_{c},\letin{\rho_u=\varphi}{\llparenthesis \eta \rrparenthesis})\\
\varphi_{app} = \vec{\rho}(\varphi_{c})\qquad \Upsilon\vdash \varphi_{app} \Downarrow_{str} R_{app}
\end{array}}
\label{eq:annotating-strapp}
\end{equation}

As shown in the above rule, the entity path for the functor and the
functor parameter entity variable are in the
static environment at elaboration time. The coercion full signature
$M_c$ and entity variable $\rho_u$ for the uncoerced argument
structure. 

%coercion entity expression $\varphi_c$ are produced by
%signature matching. 

%The signature matching semantics
% (rule~\ref{eq:sigmatch}) guarantee that the coercion entity expression
% is always of the form $\letin{\rho_u=\varphi_u}{\llparenthesis \eta
%   \rrparenthesis}$, a fact that will be used later in this chapter.  
 
\section{Target Language: F$_\omega$}

\begin{figure}
	\centering
	\fixedCodeFrame{
	\small
        ~\\[2mm]
        \[
        \begin{array}{rcll}
           k & ::= & \Omega~|~k\to k~|~\{l_0::k_0,\ldots,l_n::k_n\} & \textrm{kinds}\\
           c & ::= &
           \alpha~|~\inj(\tau^n)~|~\newx(n)~|~\lambda\alpha:k.c~|~c(c)~& \textrm{tycons}\\
           & ~~| & \{l_0=c_0,\ldots,l_n=c_n\}~|~c.l\\
 
           t & ::= & \mathsf{ty}(c)~|~\forall\alpha::k.t~|~\{l_0:t_0,\ldots,l_n:t_n\} & \textrm{types}\\
            e_{\omega} & ::= & x~|~\lambda
            x:t.e_{\omega}~|~e_{\omega}(e_{\omega})~|~\Lambda\alpha::k.e_{\omega}~|~e_{\omega}[c]\\
            &~~| & ~\{l_0=e_{\omega},\ldots,l_n=e_{\omega}\}~|~e_{\omega}.l\\
            & ~~| & \mathrm{let}~d~\mathrm{in}~e_{\omega}~|~\lettyc{\alpha=c}{e_\omega} & \textrm{exprs}\\
            v & ::= & \lambda x:t.e_{\omega}~|~\Lambda\alpha::k.e_{\omega} &
            \textrm{values} \\
            d & ::= & \emptyset_{\omega dec}~|~x=e_\omega;d & \textrm{decl}\\
            E & ::= & \emptyset_{\omega env}~|~E,\alpha::k~|~E,x:t &
            \textrm{environs} 
        \end{array}
        \]\\
        Let $t \to t'$ be shorthand for $\to^2(t,t')$. 
      }
      \caption{System F$_{\omega}$} 
      \label{fig:fomega}
\end{figure}

Fig.~\ref{fig:fomega} gives the grammar for the applied
F$_\omega$. The language is nearly identical to the one Shao
used~\cite{shao98}. It is an F$_\omega$ core enriched with record
types and declarations. 

An F$_\omega$ kind $k$ is either a monokind or an arrow kind. A tycon
$c$ can be a type variable, an atomic tycon $\mathsf{inj}(\tau^n)$, an
explicitly kinded tycon-level $\lambda$ abstraction, and a tycon
application. The tycon for a record is represented as
$\{l_0 : c_0,\ldots,l_n : c_n\}$ where $l_i$'s are record field labels and
$c_i$'s are the tycons for the fields corresponding to those
labels. The tycon form $c.l$ can project out the field type of field
$l$ assuming that $c$ is a record type containing a field of that
label. Type expressions $t$ are either a tycon or an explicitly quantified
and kinded polymorphic type $\forall\alpha::k.t$. Unlike the version of
the F$_\omega$ type system given by Shao, this type
system includes atomic tycons. Also, this type system simplifies
Shao's by subsuming base types (int) and arrow types $\tau\to\tau'$ using an atomic tycon
and type application $\to^2(\tau)(\tau')$. Types $t$ omit arrow and
record types from Shao's version. 

F$_\omega$ expressions include variables, an explicitly-typed
$\lambda$-abstractions, applications of expressions, an explicitly-kinded type
abstraction $\Lambda\alpha::k.e_\omega$, and type application
$e_\omega[c]$. An expression can also be a record expression
$\{l_0=e_{\omega_0},\ldots,l_n=e_{\omega_n}\}$, a record field
projection $e_\omega.l$, or a let expression. Within let expressions
are declarations, which are sequences of variables bound to
expressions terminated by $\cdot$, the empty declaration. The same
environment is used for both kinding $\alpha::k$ and typing $x:t$. Note that the
abstraction ($\lambda x:t.e_\omega$) and type application
$e_{\omega}[t]$ both permit application to polymorphic types
$\forall\alpha::k.t$. This is necessary because functors and
functor applications in the module language support abstraction over
polymorphic types ($\mathsf{functor}~F(X:\mathsf{sig~val}~id :
\forall\alpha.\alpha\to\alpha~\mathsf{end})=\mathsf{struct~end}$) and
application to such types. When these are translated to F$_\omega$,
the polymorphism must be maintained. 

\subsection{Kind System}

\begin{figure}
	\centering
	\fixedCodeFrame{
	\small
        ~\\[2mm]

        \fbox{$E\vdash c :: k$}
          \begin{equation}
              \infer{E\vdash \alpha :: E(\alpha)}
              {\strut}
              \label{eq:f-k-var}
          \end{equation}

          \begin{equation}
              \infer{E \vdash \inj(\tau^n) :: \Omega_1 \to \ldots \Omega_n \to \Omega}
              {\strut}
              \label{eq:f-k-at}
          \end{equation}

          \begin{equation}
              \infer{E \vdash \newx(n) :: \Omega_1 \to \ldots \Omega_n
                \to \Omega}
              {\strut}
              \label{eq:f-k-new}
          \end{equation}

          \begin{equation}
              \infer{E \vdash (\lambda\alpha::k.c) :: k \to k'}
              {E[\alpha::k]\vdash c :: k'}
              \label{eq:f-k-lam}
          \end{equation}

          \begin{equation}
              \infer{E \vdash c_1(c_2) :: k_2}
              {E\vdash c_1 :: k_1 \to k_2\qquad E\vdash c_2 :: k_1}
              \label{eq:f-k-app}
           \end{equation}

           \begin{equation}
             \infer{E \vdash \{l_1=c_1,\ldots,l_n=c_n\} 
               :: \{l_1::k_1,\ldots,l_n::k_n\}}
             {E\vdash c_1::k_1\ldots E\vdash c_n::k_n}
             \label{eq:f-k-rec}
           \end{equation}

           \begin{equation}
             \infer{E\vdash c.l :: k}
             {E\vdash c :: \{lks\}\qquad l::k\in\{lks\}}
             \label{eq:f-k-proj}
           \end{equation}
            
        }
        \caption{F$_\omega$ well-kinding}
        \label{fig:fomega-kinding}
\end{figure}


Fig.~\ref{fig:fomega-kinding} describes well-kinding of F$_\omega$
tycons. The kind system is standard. The judgment relies only on the
kind part of the environment. Rule~\ref{eq:f-k-at} gives the curried
kind for the $n$-ary atomic tycon. The resultant kind is curried $n$
times.  Rules~\ref{eq:f-k-rec}
and~\ref{eq:f-k-proj} kind record tycons and record projection tycons
respectively in the usual way. 

\subsection{Type System}


\begin{figure}
	\centering
	\fixedCodeFrame{
	\small
        ~\\[2mm]
        \fbox{$E\vdash e_\omega : t$}

        \begin{equation}
          \infer{E\vdash x : E(x)}{\vdash E(x)~\mathsf{type}}
        \label{eq:fvar}
        \end{equation}
        
        \begin{equation}
          \infer{E\vdash (\lambda x:t.e_\omega) : t\to t'}
          {E[x:t]\vdash e_\omega : t'}
          \label{eq:flam}
          \end{equation}

          \begin{equation}
            \infer{E\vdash e_{\omega_1}(e_{\omega_2}) : t_2}
            {E\vdash e_{\omega_1} : t_1 \to t_2\qquad E\vdash e_{\omega_2} : t_1 }
            \label{eq:fapp}
          \end{equation}

          \begin{equation}
            \infer{E\vdash (\Lambda\alpha::k.e_\omega) : (\forall\alpha::k.t)}
            {E[\alpha::k]\vdash e_\omega : t}
            \label{eq:ftfun}
            \end{equation}
            
            \begin{equation}
              \infer{E\vdash e_\omega[c] : t'\{c/\alpha\}}
              {E\vdash e_\omega : \forall\alpha::k.t'\qquad 
                E\vdash c :: k }
              \label{eq:ftapp}
              \end{equation}
            
           \begin{equation}
              \infer{E\vdash
                \{l_0=e_{\omega_0},\ldots,l_n=e_{\omega_n}\} : 
                \{l_0:t_0,\ldots,l_n:t_n\} }
              {E\vdash e_{\omega_0} : t_0 \ldots E\vdash e_{\omega_n} : t_n }
              \label{eq:frec}
           \end{equation}

           \begin{equation}
             \infer{E\vdash e_\omega.l_i : t_i}
             {E\vdash e_\omega : \{\ldots,l_i:t_i,\ldots\}}
             \label{eq:fproj}
             \end{equation}

           \begin{equation}
             \infer{E\vdash \letin{d}{e_\omega} : t}
             {E\vdash d :_{dec} E'\qquad E'\vdash e_\omega : t}
             \label{eq:flet}
           \end{equation}

           \begin{equation}
             \infer{E \vdash \lettyc{\alpha = c}{e_\omega} :: t}
             {E\vdash c :: k\qquad E[\alpha::k]\vdash e_\omega :: t}
             \label{eq:flettyc}
           \end{equation}

           %%%%%
           \fbox{$E\vdash d :_{dec} E'$} 
           %%%%%

           \begin{equation}
             \infer{E\vdash \emptyset_{\omega dec} :_{dec} E}{\strut}
             \label{eq:fdecempty}
           \end{equation}

           \begin{equation}
             \infer{E\vdash x=e;d :_{dec} E'}
             {E\vdash e : t\qquad E[x:t]\vdash d :_{dec} E'}
             \label{eq:fdecrest}
             \end{equation}

       }
        \caption{F$_\omega$ Type System}
        \label{fig:fomegatype}
\end{figure}

Fig.~\ref{fig:fomegatype} gives the type system for
F$_\omega$. $\vdash t~\mathsf{type}$ is an axiom. In particular
$\vdash k~\mathsf{type}$ is not true. Type application
(rule~\ref{eq:ftapp}) assumes that the left-hand side has a type
$\lambda\alpha.k.t'$. The type of the type application is
$t'\{t/\alpha\}$. Rule~\ref{eq:frec} gives the record tycon
$\{l_0:c_0,\ldots,l_n:c_n\}$ for a
record expression $\{l_0=e_0, \ldots, l_n=e_n\}$ as long as all its
field expressions have the correct type,\ie, $e_i : t_i$. 

The notation $\{c/\alpha\}_{tyc}$ denotes tycon substitution. This
kind of substitution does not act on values that may coincidentally
share the same name $\alpha$. This distinct form of substitution is
necessary because tycons and values will have distinct namespaces. 
    
\section{Translation Semantics}
The translation rules all assume an annotated module language that was
produced by an error-free elaboration. Since the program was
successfully elaborated, the semantics can make some assumptions about
the program, especially what variables will or will not be accessed,
and thereby simplify the rules. 


\begin{figure}
	\centering
	\fixedCodeFrame{
	\small
        ~\\[1mm]

           \[\begin{array}{rcl} decl & ::= &
            \mathbf{val}~x=e~|~\mathbf{type}~t=C^\lambda~|~\dt~\vec{\alpha}~x~\\
             & ~~| & \mathbf{structure}~x\an{\rho}=strexp\\
            & ~~| & \mathbf{functor}~f\an{\ldots}(X:sigexp)=strexp 
            \end{array}\]

        ~\\[1mm]

        \fbox{$ d^m \leadsto_{dec} d$}

        \begin{equation}
          \infer{ \circ \leadsto_{dec}
            \emptyset_{\omega dec}}{\strut}
          \label{eq:transdec-empty}
       \end{equation}

        \begin{equation}
          \infer{ \mathbf{val}~x=e \leadsto_{dec} x=e_\omega}
          { e \leadsto_{exp} e_\omega}
          \label{eq:transdec-val}
        \end{equation}

       \begin{equation}
          \infer{ \mathbf{type}~t=C^\lambda\leadsto_{dec}
            \emptyset_{\omega dec}}
          {\strut}
          \label{eq:transdec-typedef}
        \end{equation}

       \begin{equation}
           \infer{
             \dt~\vec{\alpha}~x\leadsto_{dec}
             \emptyset_{\omega dec}}
           {\strut}
           \label{eq:transdec-dt}
       \end{equation}

         \begin{equation}
            \infer{ \structure~x\an{\rho}=strexp\leadsto_{dec}
              \rho=e_\omega}
            { strexp \leadsto_{exp} e_\omega}
            \label{eq:transdec-str}
          \end{equation}

        \begin{equation}
          \infer{
            \begin{array}{c}
              \mathbf{functor}~f\an{\rho,F,\Upsilon_{ins}}(X:sigexp)=strexp\\
            \leadsto_{dec} \rho=\Lambda \rho_x::k.\lambda
            \rho_x:t.e_\omega
          \end{array}}
          {\begin{array}{c}
              F=(\Pi\rho_x:\Sigma_{par}.\Sigma', \psi)\qquad
            \vdash \Sigma_{par} \leadsto_{knd} k\\
            \Upsilon_{ins}\vdash \Sigma_{par}
            \leadsto_{type} t\qquad
             strexp \leadsto_{exp}  e_\omega
          \end{array}}
          \label{eq:transdec-fct}
        \end{equation}

        \begin{equation}
          \infer{ decl,d^m \leadsto_{dec} d_1; d_2 }
          { decl \leadsto_{dec} d_1\qquad
           d^m \leadsto_{dec} d_2}
        \label{eq:transdec-comp}
        \end{equation}
            }
\caption{Translation for declarations}
\label{fig:transdec}
\end{figure}

\begin{figure}
  \centering
  \fixedCodeFrame{
    \small
    ~\\[2mm]
\fbox{$strexp \leadsto_{exp} e_\omega$}
           
    \begin{equation}
    \infer{  p\an{\vec{\rho}} \leadsto_{exp} \vec{\rho}}
    {\strut}
    \label{eq:transexp-path}
     \end{equation}

    \begin{equation}
       \infer{  \mathsf{struct}~d^m~\mathsf{end}
         \leadsto_{exp} \mathbf{let}~d~\mathbf{in}~e_\omega}
       {  d^m \leadsto_{dec} d\qquad
         e_\omega=\{\rho_i=\rho_i,\ldots\} \forall \rho_i \in dom(d) \textrm{
         and }\rho_i\textrm{ is not a type}}
     \label{eq:transexp-basestruct}
     \end{equation}

     \begin{equation}
       \infer{ \begin{array}{c}
         p\an{\vec{\rho},\rho_{par}}
         (strexp\an{M_c,\rho_u}) 
         \leadsto_{exp}
         \letin{\rho_u=e_{\omega_u}}{\vec{\rho}[c](e_{\omega_c})}
       \end{array}}
       {\begin{array}{c}  
           strexp \leadsto_{exp} e_\omega\qquad
           M_c \leadsto_{tyc}  c\qquad M_c=(\Sigma, R) \\
           R\vdash \Sigma\leadsto_{type} lts_c\qquad
           e_{\omega_c} = \coerce(\rho_u,c,\{lts_c\})
         \end{array}}
       \label{eq:transexp-app}
     \end{equation}

     \begin{equation}
       \infer{  strexp : sigexp \leadsto_{exp} e_\omega}
       {  strexp \leadsto_{exp} e_\omega}
     \label{eq:transexp-transparent}
     \end{equation}

     \begin{equation}
       \infer{ strexp :> sigexp
         \leadsto_{exp} e_\omega}
       {strexp \leadsto_{exp} e_\omega}
       \label{eq:transexp-opaque}
     \end{equation}     
    }
        \caption{Translation of structure expressions to System F$_\omega$}
        \label{fig:transexp}
\end{figure}

The notation $dom(d)$ here denotes the set of
all the bound names,\ie, the $\rho$ in $\rho=e$. 

Fig.~\ref{fig:transdec} gives the rules for translating module
declarations to F$_\omega$ declarations. Rule~\ref{eq:transdec-empty}
is the empty declaration case. Rule~\ref{eq:transdec-val} translates
the core language expression $e$ to a corresponding $e_\omega$. Unlike
the static entity case, shadowing a previous variable $x$ is harmless
so $x$ will be used as the variable to
bind. Rules~\ref{eq:transdec-typedef} and~\ref{eq:transdec-dt} do not
produce F$_\omega$ expressions because type definitions and datatype
declarations play no role in the
dynamic semantics. Rules~\ref{eq:transdec-str}
and~\ref{eq:transdec-fct} must produce declarations binding the entity
variable of the structure or functor instead of the syntactic
variable. % Is this necessary? Yes, structures and functors are static
          % entities but these rules only care about the dynamic
          % components for which there is no avoidance problem. 
The rule~\ref{eq:transdec-fct} will use the judgment $\vdash \Sigma
\leadsto_{sig} k$ defined in Fig.~\ref{fig:semsigtokinds} to extract
the kind information from the semantic signature for the functor
parameter. The type for the F$_\omega$ $\lambda$ abstraction is
calculated separately from the same semantic signature. 

Fig.~\ref{fig:transexp} gives the translation of the module
language to F$_\omega$. Rule~\ref{eq:transexp-app} translates
structure applications. The functor being applied can be translated by
translating its entity path to the corresponding F$_\omega$ record
projection form. The actual argument structure expression is
translated to $e_\omega$, but the functor is expecting a parameter of
the form defined in full signature $M$. The associated coercion
structure expression $\letin{\rho_u=\varphi_u}{\llparenthesis \eta
  \rrparenthesis}$ can help produce coerced type and value
arguments. For the type argument, the rule translates the coercion entity
declaration to F$_\omega$ via the $\tycon$ function defined in
Fig.~\ref{fig:xlate-entdec}. For the value argument, reusing the
uncoerced expression by aliasing $\rho_{par} = \rho_u$ is sufficient
because the prior elaboration already guarantees that none of
components in $\rho_u$ that is not specified in the formal parameter
signature will be accessed. Furthermore, none of the variables are
being renamed. 
   

\begin{figure}
  \centering
  \fixedCodeFrame{
    \small
    ~\\[2mm]
    \fbox{$\vdash \Sigma \leadsto_{knd} k$}
    \begin{equation}
      \infer{\vdash \emptyset_{sig} \leadsto_{knd} \{ \}}
      {\strut}
      \label{eq:f-semsig-empty}
    \end{equation}

    \begin{equation}
      \infer{\vdash [x\mapsto\mathbb{T}]\Sigma \leadsto_{knd} k}
      {\vdash \Sigma \leadsto_{knd} k}
      \label{eq:f-semsig-val}
    \end{equation}

    \begin{equation}
      \infer{\vdash [x\mapsto\mathbb{C}^\lambda]\Sigma \leadsto_{knd} k}
      {\vdash \Sigma \leadsto_{knd} k}
      \label{eq:f-semsig-typedef}
    \end{equation}

    \begin{equation}
      \infer{\vdash [x\mapsto(\rho,n)]\Sigma \leadsto_{knd} \{\rho::\Omega^n\to\Omega\}\uplus k}
      {\vdash \Sigma \leadsto_{knd} k}
      \label{eq:f-semsig-opentype}
     \end{equation}

    \begin{equation}
      \infer{\vdash [x\mapsto(\rho,\Sigma)]\Sigma' \leadsto_{knd} \{\rho::k\}\uplus k'}
      {\vdash \Sigma \leadsto_{knd} k\qquad\vdash \Sigma' \leadsto_{knd} k'}
      \label{eq:f-semsig-str}
    \end{equation}

    \begin{equation}
      \infer{\vdash [x\mapsto(\rho,\Pi\rho_x:\Sigma_x.\Sigma)]\Sigma' 
        \leadsto_{knd} \{\rho::k_x\to k\}\uplus k'}
      {\vdash \Sigma_x \leadsto_{knd} k_x\qquad \vdash \Sigma\leadsto_{knd} k
        \qquad \vdash \Sigma' \leadsto_{knd} k'}
      \label{eq:f-semsig-fct}
    \end{equation}
  }
\caption{F$_\omega$ kind synthesis}
\label{fig:semsigtokinds}
\end{figure}

\begin{figure}
	\centering
	\fixedCodeFrame{
	\small
        ~\\[2mm]
     \fbox{$\vdash M \leadsto_{tyc} c$}
        
        \begin{equation}
          \infer{\vdash (\Sigma, \langle \Upsilon^{lcl},\Upsilon^{clo} \rangle) 
            \leadsto_{tyc} \{ lcs \}}
          {\Upsilon^{clo}\Upsilon^{lcl}\vdash \Sigma \leadsto_{spec} lcs}
          \label{eq:transfullsig}
        \end{equation}

        \fbox{$\Upsilon\vdash \Sigma \leadsto_{spec} lcs$}
        
        \begin{equation}
          \infer{\Upsilon\vdash \emptyset_{sig} \leadsto_{spec} \{ \} }
          {\strut}
          \label{eq:transspec-empty}
        \end{equation}

        \begin{equation}
          \infer{\Upsilon\vdash [x\mapsto\mathfrak{T}]\Sigma \leadsto_{spec} lcs}
          {\Upsilon\vdash \Sigma \leadsto_{spec} lcs}
          \label{eq:transspec-val}
        \end{equation}

        \begin{equation}
          \infer{\Upsilon\vdash [x\mapsto\mathfrak{C}^\lambda]\Sigma \leadsto_{spec} lcs}
          {\Upsilon\vdash \Sigma \leadsto_{spec} lcs}
          \label{eq:transspec-typedef}
        \end{equation}

        \begin{equation}
          \infer{\Upsilon\vdash [x\mapsto(\rho, n)]\Sigma 
            \leadsto_{spec} \rho=\inj(\Upsilon(\rho)),lcs}
          {\Upsilon\vdash \Sigma\leadsto_{spec} lcs}
          \label{eq:transspec-opentype}
        \end{equation}

        \begin{equation}
          \infer{\Upsilon\vdash [x\mapsto(\rho, \Sigma)]\Sigma'
            \leadsto_{spec} \rho=\{lcs\},lcs'}
          {\Upsilon(\rho)\vdash \Sigma \leadsto_{spec} lcs\qquad 
            \Upsilon\vdash \Sigma' \leadsto_{spec} lcs'}
          \label{eq:transspec-str}
        \end{equation}

        \begin{equation}
          \infer{\Upsilon\vdash [x\mapsto (\rho, \Sigma^f)]\Sigma
               \leadsto_{spec} \rho=c, lcs'}
             {\vdash (\Sigma^f, \Upsilon(\rho)) \leadsto_{fctspec} c\qquad 
               \Upsilon\vdash \Sigma \leadsto_{spec} lcs'}
           \label{eq:transspec-fct}
         \end{equation}

\fbox{$\vdash F \leadsto_{fctspec} c$}
         \begin{equation}
           \infer{\vdash (\Pi\rho:\Sigma.\Sigma',
             \langle\lambda\rho.\varphi;\Upsilon\rangle)
             \leadsto_{fctspec} \lambda\rho::k.c}
           {\vdash \Sigma \leadsto_{knd} k\qquad \vdash \varphi
             \leadsto^{strexp}_{tyc} c }
           \label{eq:trans-fctspec}
           \end{equation}
        }
\caption{F$_\omega$ tycon synthesis}
\label{fig:f-tycon-synthesis}
\end{figure}

\begin{figure}
  \centering
  \fixedCodeFrame{
    \small
    ~\\[2mm]
    \fbox{$\vdash \mathfrak{C}^{nf} \leadsto^{nf}_{type} t$}
    \begin{equation}
      \infer{\vdash \alpha \leadsto^{nf}_{type} \alpha}
      {\strut}
    \end{equation}

    \begin{equation}
      \infer{\vdash \tau^n\left(\vv{\mathfrak{C}^{nf}}\right) 
        \leadsto^{nf}_{type} \mathsf{inj}(\tau^n)(t_1)\ldots(t_n)}
      {\vdash\mathfrak{C}^{nf}_1 \leadsto^{nf}_{type} t_1 \ldots 
        \vdash \mathfrak{C}^{nf}_2 \leadsto^{nf}_{type} t_n}
     \end{equation}

    \fbox{$\Delta,\Upsilon\vdash \mathbb{C}^s \leadsto^{tyc}_{type} t$}
    
    \begin{equation}
      \infer{\Delta,\Upsilon\vdash \alpha \leadsto^{tyc}_{type} \alpha}
      {\alpha\in\Delta}
      \label{eq:f-tyc-type-var}
    \end{equation}

    \begin{equation}
      \infer{\Delta,\Upsilon\vdash \vec{\rho}\left(\vv{\mathbb{C}^s}\right) 
          \leadsto^{tyc}_{type} t}
      {\begin{array}{c}
          \vv{\mathfrak{C}^s}~s.t.~\Delta,\Upsilon\vdash\mathbb{C}^s_i\Downarrow_{mt}\mathfrak{C}^s_i~\forall
        \mathbb{C}^s_i \in \vv{\mathbb{C}^s}\\
        \Upsilon(\vec{\rho})(\vv{\mathfrak{C}^s})
        \Downarrow_{mt} 
        \mathfrak{C}^{nf}
      \qquad \vdash\mathfrak{C}^{nf}\leadsto^{nf}_{type} t
    \end{array}}
    \label{eq:f-tyc-type-app}
    \end{equation}

    \fbox{$\Upsilon\vdash \mathbb{T} \leadsto^t_{type} t$}
    
    \begin{equation}
      \infer{\Upsilon\vdash \mathsf{typ}(\mathbb{C}^s) \leadsto^t_{type} t}
      {\emptyset_{knd},\Upsilon\vdash \mathbb{C}^s \leadsto^{tyc}_{type} t }
    \end{equation}

    \begin{equation}
      \infer{\Upsilon\vdash \forall\vec{\alpha}.\mathbb{C}^s 
        \leadsto^t_{type} \forall\vec{\alpha}.t}
      {[\vec{\alpha}],\Upsilon\vdash \mathbb{C}^s \leadsto^t_{type} t}
    \end{equation}
}
\caption{F$_\omega$ synthesis of types from normal form semantic tycons, relativized monotypes, and type expressions}
\label{fig:f-semtyc-mt-te}
\end{figure}

\begin{figure}
  \centering
  \fixedCodeFrame{
    \small
    ~\\[2mm]

    \fbox{$\Upsilon\vdash \Sigma \leadsto_{type} lts$}

    \begin{equation}
      \infer{\Upsilon\vdash \emptyset_{sig} \leadsto_{type} \{\}}
      {\strut}
      \label{eq:fty-empty}
     \end{equation}

     \begin{equation}
       \infer{\Upsilon\vdash [x\mapsto\mathbb{T}]\Sigma' \leadsto_{type} x:t, lts}
       {\Upsilon\vdash\mathbb{T}\leadsto^{t}_{type} t\qquad
         \Upsilon\vdash\Sigma'\leadsto_{type} lts}
       \label{eq:fty-val}
     \end{equation}

     \begin{equation}
       \infer{\Upsilon\vdash [x\mapsto\mathbb{C}^\lambda]\Sigma' \leadsto_{type} lts}
       {\Upsilon\vdash\Sigma' \leadsto_{type} lts}
       \label{eq:fty-typedef}
     \end{equation}

     \begin{equation}
       \infer{\Upsilon\vdash [x\mapsto(\rho,n)]\Sigma' \leadsto_{type} lts}
       {\Upsilon\vdash \Sigma' \leadsto_{type} lts}
       \label{eq:fty-opentyc}
     \end{equation}

    \begin{equation}
      \infer{\Upsilon\vdash [x\mapsto(\rho,\Sigma)]\Sigma' 
        \leadsto_{type} \rho:\{lts\}, lts'}
      {\Upsilon(\rho)\vdash \Sigma \leadsto_{type} lts\qquad 
        \Upsilon\vdash \Sigma' \leadsto_{type} lts'}
      \label{eq:fty-str}
    \end{equation}

    \begin{equation}
      \infer{\begin{array}{c}
          \Upsilon\vdash [x\mapsto(\rho,\Pi\rho_x:\Sigma_x.\Sigma)]\Sigma'\\ 
        \leadsto_{type} \rho:(\forall\rho_x::k.\{lts_x\}\to \{lts\}),
        lts'
      \end{array}}
      {\begin{array}{c}\vdash \Sigma_x \leadsto_{knd} k\qquad
 \Upsilon\vdash \Sigma_x \leadsto_{type} lts_x \\
      \Upsilon\vdash \Sigma \leadsto_{type} lts\qquad \Upsilon\vdash \Sigma'
      \leadsto_{type} lts'
    \end{array}}
    \label{eq:fty-fct}
    \end{equation}

    }
\caption{F$_\omega$ type synthesis}
\label{fig:fty-sig}
\end{figure}



The F$_\omega$ kinds, tycons, and types for tycon content of structures, type
content of structures, and structures themselves are record kinds,
record tycons, and record types respectively. Each label-field pair is
a pair of an entity variable and the kind, tycon, and type of the
entity referred to by that entity variable. Let $lks$, $lcs$, and $lts$ be a sequence of record label-field for
kinds, tycons, and types respectively. The notation
$lks\uplus lks'$ appends two record kinds. The notation is extended in
the usual way to append tycons and types. No shadowing is expected
because all records field names will be entity variables. 

Fig.~\ref{fig:semsigtokinds} gives the rules for producing $F_\omega$
kinds from semantic signatures. Value specs and type definition specs
do not contribute to the kind (rules~\ref{eq:f-semsig-val} and~
\ref{eq:f-semsig-typedef}). Open tycon specs do
contribute. Rule~\ref{eq:f-semsig-opentype} produces the kind field
$\rho::\Omega^n \Rightarrow \Omega$ such that $n$ is the arity of the
open tycon spec. Rules~\ref{eq:f-semsig-str} and~\ref{eq:f-semsig-fct}
construct kinds for structure and functor specs from the semantic
signature and functor signature respectively. 

Fig.~\ref{fig:f-tycon-synthesis} defines the judgment
that calculates the tycons from the full signature of a
structure. Again, value specs, and type definition specs do not
contribute to the tycon (rules~\ref{eq:transspec-val} and~\ref{eq:transspec-typedef}). Rule~\ref{eq:transspec-opentype} translates
an open spec by
looking up the entity environment for the tycon entity associated with
the open tycon spec. The rule~\ref{eq:transspec-str} translates the
structure spec signature in the context of the entity environment
calculated from the structure entity
($\Upsilon(\rho)=\langle\Upsilon^{lcl},\Upsilon^{clo}\rangle=\Upsilon^{clo}\Upsilon^{lcl}$).
To calculate the tycon for a functor component,
rule~\ref{eq:transspec-fct} must supply rule~\ref{eq:trans-fctspec}
the functor signature and entity from looking up the entity
environment. Rule~\ref{eq:trans-fctspec} calculates the kind for the
functor parameter using the parameter signature and also the tycon for
the functor body from the functor body signature and closure entity
environment. The judgment relies on $\vdash \varphi
\leadsto^{strexp}_{tyc} c$ which translates the structure entity
expression to the corresponding F$_\omega$ tycon expression, which is
the same because there is an embedding of the structure entity
expression language to the F$_\omega$ tycon expression language. The
above statement fully defines $\vdash \varphi
\leadsto^{strexp}_{tyc} c$. 

Fig.~\ref{fig:fty-sig} gives the rules
for producing $F_\omega$ types from semantic signatures. These two
judgments decompose semantic signatures into kinds (for classifying
the static components) and types (for classifying the dynamic
components). Kind synthesis is used in rule~\ref{eq:fty-fct} to
produce the kind annotation for the polymorphic quantifier. 

Let the $\mathsf{env}(lts) = [\rho:t]$ where $\rho_i:t_i \in lts$ for the purpose of fig.~\ref{fig:xlate-entdec}. 
  
\begin{figure}
	\centering
	\fixedCodeFrame{
	\small
        ~\\[2mm]
        \begin{equation}
          \coerce(\rho_u, lts) = \left\{ \begin{array}{ll}
              \{\} & lts = \{\}\\
              \{l_0=\rho_u.l_0\} & lts = \{l_0:t_0\}\uplus lts'\\
              \uplus\coerce(\rho_u, lts')
         \end{array} \right.
        \end{equation}

       }
        \caption{Coercion of dynamic components}
        \label{fig:xlate-entdec}
\end{figure}
    
Fig.~\ref{fig:xlate-entdec} defines the function $\coerce$ that
coerces an F$_\omega$ expression $e_\omega$ to a record type $lts$. 
It thins out the value part by dropping components not
in the specified type. The resultant expression should have the target
type $\{lts\}\{c/\rho\}_{tyc}$ where $lts$ is parameterized by
$\rho$. The type-level coercion was already taken care of by
F$_\omega$ tycon synthesis. 

\begin{lemma}[Correctness of Value-Level Coercion]
If $E\vdash \{lts_u\}::k$, $E\vdash
\rho_u : \{lts_u\}$, for all $l\in dom(lts_u)\cap dom(lts_c)$
$t_c\{c/\alpha\}=t_u$ such that $l:t_c \in lts_c$ and $l:t_u \in lts_u$, and
 $\coerce(\rho_u,c,lts_c) = e_\omega$, then $E\vdash e_\omega : \{lts_c\{c/\alpha\}\}$. 
\end{lemma}

\section{Soundness}

\begin{theorem}
If a closed structure expression $strexp$ elaborates to $(M, \_)$ and $strexp$
translates to $e_\omega$, then $\vdash e_\omega : t$ and $\vdash M \leadsto_{type} t$. 
\end{theorem}

The proof of the above lemma is sketched out in more detail in the
proof appendix. It relies on much of the same machinery as Shao's
proof~\cite{shao98}. The proof diverges in Shao's memoization of
pre-translated F$_\omega$ tycon and types in the functor
representation during elaboration of the functor. Furthermore, this
proof must deal with the coercive signature matching semantics which
is not present in Shao's main elaboration-translation semantics. 

\section{Type Generativity}

Shao's calculus and many of the related accounts of the module system
do not fully support the type generativity present in ML. Type
generativity stems from two constructs, opaque ascription and datatype
declarations. 

Seemingly benign examples that feature datatype declarations are not supported. 

\begin{lstlisting}
structure M = 
struct 
  datatype t
  val id : t -> t = fn x:t => x
end
\end{lstlisting}

\begin{lstlisting}
functor F() = 
struct
  datatype t
  val id : t -> t = fn x:t => x
end
\end{lstlisting}

Datatype declarations are an interesting case because unlike the rest
of the static content of a module, datatype declarations translate to
F$_\omega$ tycons that are not referentially transparent. Whereas all
other forms of F$_\omega$ tycons can be duplicated and be supplied as
arguments to type applications, the tycons from datatype declarations
cannot. This is because the identity of the tycon from a datatype must
be maintained. The tycon cannot be copied. Consequently, such a tycon
must be bound (by the special let tyc form) to a type variable that is
in the scope of all subsequent occurrences. It turns out that in
general, the expressions constructing these tycons are exactly the
entity expressions from the module system. The referentially
transparent static content such as type definitions can continue to be packaged together in a tycon record to which $\Lambda$ type abstractions can
be directly applied. 

Another difference between datatype declarations and type definitions
is what is translated into F$_\omega$. For datatype declarations, the
entity expression is translated. For type definitions, the full
signature is directly translated to an F$_\omega$ tycon. 

Translation decomposes functors into a tycon part and a value
part. The hat notation $\widehat{x}$ denotes the variable is a type
variable that refers to the type part of a structure or functor x. Let
$k$ represent $\Omega^0 \to \Omega$. 

\begin{lstlisting}
functor f(x:sig type s type t end) = struct datatype w type v = x.t end
\end{lstlisting}
 
\[\lettycline{\widehat{f}=\lambda \widehat{x}::\{s::k,t::k\}.\{w =
  \newx(0), v=\widehat{x}.t \}}
{\letin{f = \Lambda\widehat{x}::\{s::k, 
t:: k
    \}.\lambda x:\{\}.\{\}}
 {\ldots}}
\]

For the above example functor $f$, the primary tycon part is a tycon
function $\lambda \widehat{x}
. \{w = \newx(0) \}$, which is exactly a straightforward
translation of the structure entity expression for the functor
body. The value part is $\Lambda\widehat{x}::\{s::k, 
t :: k \}.\lambda x.\{\}.\{\}$. The
secondary tycon part $v = \widehat{x}.t$ corresponds to the type definitions
\lstinline{type v = x.t}.

The above translation is sufficient in absence of value components,
whose type annotations may refer to tycons (the type part of the
functor). The value components in the functor parameter may refer to
tycons in the functor parameter. The value components in the functor
body may refer to tycons in either the functor parameter or
body. However, since the tycons in the functor body are not accessible
until the type part of the functor is applied, the value part of the
functor must take an extra type parameter, the result of applying the
type part of the functor. 

\begin{lstlisting}
functor f(x:sig type s end) = struct datatype w val n : w -> w = fn z : w => z end
\end{lstlisting}

\[\lettycline{\widehat{f}=\lambda\widehat{x}::\{s::k\}.\{ w = \newx(0) \}}
  {\letin{f = \Lambda \widehat{x}::\{s::k\}.\Lambda
      \widehat{f_{res}}::\{ w :: k \} . \lambda x::\{\}.\{n = \lambda z:\widehat{f_{res}}.w\}
    }{\ldots}}
\]

The first curried type abstraction $\Lambda\widehat{x}::\{s::k\}$ is
the type part of the functor parameter. The second type
abstraction $\Lambda \widehat{f_{res}}::\{ w :: k \}$ is for the
result of applying the type part $\widehat{f}$ to the same type part
of the functor argument $\widehat{x}$. The type annotation in the
body of the value part $n$ can refer to the tycon $w$ through
$\widehat{f_{res}}.w$. 
 
Functor application must coerce both the tycon and value parts
of the argument structure. 

\begin{lstlisting}
structure m = f(struct datatype s type t = s end)
\end{lstlisting}

\[\begin{array}{l}
  \letx~\tyc~\widehat{u} = \{s'=\newx(0) \}\\
  \inlet~\tyc~\widehat{m} = \widehat{f}(\{s = \widehat{u}.s', t =
  \widehat{u}.s' \})\\
  \inlet~m = f[\{s = \widehat{u}.s', t =
  \widehat{u}.s'\}](\{ \})\\
  \inx~\ldots
 \end{array}\]

In the above example, the translation first decomposes the argument
structure into a
primary tycon part $\{s'=\newx(0)\}$, a value part $\{ \}$, and a
secondary tycon part $t = \widehat{u}.s'$. Note that the entity
variable for the argument structure $s'$ that is distinct from
the entity variable in the formal parameter $s$. The entity
variable $\widehat{u}$ binds the raw, uncoerced primary tycon part of the
argument structure. The raw, uncoerced secondary tycon part is
directly inlined because the syntactic functor application inlines the
argument structure. The tycon part of $\widehat{f}$ is applied to the coerced
tycon part of the argument structure, $\widehat{f}(\{s=\widehat{u}.s',
t=\widehat{u}.s'\})$ where $\widehat{f}$ here refers to the tycon
function bound in the previous example. The value part of $f$ is
applied to the coerced tycon and value parts of the argument
structure, $f[\{s=\widehat{u}.s',t=\widehat{u}.s'\}]
(\{\})$.

Since the tycon part of functor is a translation of the functor entity
expression, the tycon part also represents the nested functors in the
functor body. 

\begin{lstlisting}
functor f() = struct functor g() = struct datatype t end end
\end{lstlisting}

\[ \begin{array}{l}
\letx~\tyc~\widehat{f} = \lambda().\{\widehat{g} = \lambda().\{ t =
\newx(0) \} \}\\
   \inlet~f = \Lambda()::\{\}.\lambda():\{\}.\{\}
\end{array}
\]

The tycon part of the nested functor $g$ is $\lambda().\{t =
\newx(0)\}$. Thus, the tycon part $\widehat{f}$ must be the first line in the
example above. In this example, the value part is degenerate since
there are no value components. 

Formal functors are handled in the same way as above. 

\begin{lstlisting}
functor f(x:sig functor g(y: sig type t end) : sig type v end end) =
struct end

structure m = f(struct functor g(y: sig type t end) = struct type v = y.t end end)
\end{lstlisting}

The above example translates to the following F$_\omega$ program.
\[\begin{array}{l}
\letx~\tyc~\widehat{f} = \lambda\widehat{x}::\{\widehat{g}::\{\widehat{y}::\{t :: k\}\}\to\{ v :: k \}\}.\{\}\\
\inlet~f = \Lambda\widehat{x}::\{ \widehat{g}::\{\widehat{y}::\{t :: k\}\}\to\{ v :: k \}
\}.\lambda x:\{ g : \forall\widehat{y}::\{ t ::
k \}\to \{ v :: k \}\}. \{\} \}\\
\inlet~\tyc~\widehat{u} = \{ \rho_g' = \lambda\rho_y'.\{ \} \}\\
\inlet~u = \{ g' = \Lambda \widehat{y'}::\{ \widehat{g'} :: \{ t'
:: k\} \to
\{ v' :: k\} \}.\lambda y'.\{ \} \}\\
\inlet~\tyc~\widehat{m} = \widehat{f}(\{\widehat{g} = u.\widehat{g'} \})\\
\inlet~m = f[\{ \widehat{g} = \lambda \widehat{y'} :: \{ \}  . \{ v' =
\widehat{y'}.t  \} \}](\{ g = u.g' \})
\end{array}\]

The arrow kinds are given to the tycon abstractions representing the formal functor, reflective of how functors can be viewed as type operators. 
The tycon and value parts
of the uncoerced functor argument are $\{\widehat{g'}=\lambda \widehat{y'}. \{
\}\}$ and
$\{g'=\lambda y'.\{\}\}$ respectively.  

Support for type generativity requires some modification of the
translation rules for functor declaration and functor application.

        \begin{equation}
          \infer{
            \begin{array}{c}
              \mathbf{functor}~f\an{\rho,F,\Upsilon_{ins}}(X:sigexp)=strexp\\
            \leadsto_{dec} \rho=_{tyc} \theta, \rho=\Lambda \rho_x::k.\lambda
            \rho_x:t.e_\omega
          \end{array}}
          {\begin{array}{c}
              F=(\Pi\rho_x:\Sigma_{par}.\Sigma', \langle\theta; \Upsilon^{clo}\rangle)\qquad
            \vdash \Sigma_{par} \leadsto_{knd} k\qquad \Upsilon_{ins}\vdash \Sigma_{par}
            \leadsto_{type} t\\
             strexp \leadsto_{exp}  e_\omega
          \end{array}}
       \end{equation}

     \begin{equation}
       \infer{ \begin{array}{c}
         \structure~x\an{\rho_x}=p\an{\vec{\rho},\rho_{par}}
         (strexp\an{M_c,\rho_u}) \\
         \leadsto_{exp}
         \letx~\tyc~\rho_x= \vec{\rho}(\coerce_{tyc}(\rho_u,c)) \inx~\letx~\rho_x=
         \vec{\rho}(\coerce_{tyc}(\rho_u,c)),
         \letin{\rho_u=e_{\omega_u}}{\vec{\rho}[c](e_{\omega_c})}
       \end{array}}
       {\begin{array}{c}  
           strexp \leadsto_{exp} e_\omega\qquad
           M_c \leadsto_{tyc}  c\qquad M_c=(\Sigma, R) \\
           R\vdash \Sigma\leadsto_{type} lts_c\qquad
           e_{\omega_c} = \coerce(\rho_u,c,\{lts_c\})
         \end{array}}
     \end{equation}


%%% Local Variables: 
%%% mode: latex
%%% TeX-master: "main"
%%% End: 
