\documentclass[9pt]{sigplanconf}
\usepackage{cite} % sort citations 
\usepackage{amsmath,xspace,aux/math-envs,aux/math-cmds,aux/code,aux/proof,stmaryrd}
% code.sty: -*- latex -*-
% Latex macros for a "weak" verbatim mode.
% -- like verbatim, except \, {, and } have their usual meanings.

% Environments: code, tightcode,  codeaux, codebox, centercode
% Commands: \dcd, \cddollar, \cdmath, \cd, \codeallowbreaks, \codeskip, \^
% Already defined in LaTeX, but of some relevance: \#, \$, \%, \&, \_, \{, \}

% Changelog at the end of the file.

% These commands give you an environment, code, that is like verbatim
% except that you can still insert commands in the middle of the environment:
%     \begin{code}
%     for(x=1; x<loop_bound; x++)
%         y += x^3; /* {\em Add in {\tt x} cubed} */
%     \end{code}
%
% All characters are ordinary except \{}. To get \{} in your text, 
% you use the commands \\, \{, and \}.

% These macros mess with the definition of the special chars (e.g., ^_~%).
% The characters \{} are left alone, so you can still have embedded commands:
%	\begin{code} f(a,b,\ldots,y,z) \end{code}
% However, if your embedded commands use the formerly-special chars, as in
%    	\begin{code} x := x+1 /* \mbox{\em This is $y^3$} */ \end{code}
% then you lose. The $ and ^ chars are scanned in as non-specials,
% so they don't work. If the chars are scanned *outside* the code env,
% then you have no problem:
% 	\def\ycube{$y^3$}
% 	\begin{code} x := x+1 /* {\em This is \ycube} */ \end{code}
% If you must put special chars inside the code env, you do it by
% prefixing them with the special \dcd ("decode") command, that
% reverts the chars to back to special status:
%    	\begin{code} x := x+1 /* {\dcd\em This is $y^3$} */ \end{code}
% \dcd's scope is bounded by its enclosing braces. It is only defined within
% the code env. You can also turn on just $ with the \cddollar command;
% you can turn on just $^_ with the \cdmath command. See below.
%
% Alternatively, just use \(...\) for $...$, \sp for ^, and \sb for _.

% WARNING:
% Like \verb, you cannot put a \cd{...} inside an argument to a macro
% or a command. If you try, for example,
%     \mbox{\cd{$x^y$}}
% you will lose. That is because the text "\cd{$x^y$}" gets read in
% as \mbox's argument before the \cd executes. But the \cd has to
% have a chance to run before LaTeX ever reads the $x^y$ so it can
% turn off the specialness of $ and ^. So, \cd has to appear at
% top level, not inside an argument. Similarly, you can't have
% a \cd or a \code inside a macro (Although you could use \gdef to
% define a macro *inside* a \cd, which you could then use outside.
% Don't worry about this if you don't understand it.)

% BUG: In the codebox env, the effect of a \dcd, \cddollar, or \cdmath
%   command is reset at the end of each line. This can be hacked by
%   messing with the \halign's preamble, if you feel up to it.

% Useage note: the initial newline after the \begin{code} or 
%   \begin{codebox} is eaten, but the last newline is not.
%   So,
%     \begin{code}
%     foo
%     bar
%     \end{code}
%  leaves one more blank line after bar than does
%     \begin{code}
%     foo
%     bar\end{code}
%  Moral: get in the habit of terminating code envs without a newline
%  (as in the second example).
%

%
% get the paragraph indentation (use \edef and \the to force evaluation)
%
%\edef\codeindent{\the\parindent}
\edef\codeindent{0pt}


% All this stuff tweaks the meaning of space, tab, and newline.
%===============================================================================
% \cd@obeyspaces
% Turns all spaces into non-breakable spaces.
% Note: this is like \@vobeyspaces except without spurious space in defn.
% @xobeysp is basically a space; it's defined in latex.tex.
%
{\catcode`\ =\active\gdef\cd@obeyspaces{\catcode`\ =\active\let =\@xobeysp}}



% \cd@obeytabs
% Turns all tabs into 8 non-breakable spaces (which is bogus).
%
{\catcode`\^^I=\active %
  \gdef\cd@obeytabs{\catcode`\^^I=\active\let^^I=\cd@tab}}

\def\cd@tab{\@xobeysp\@xobeysp\@xobeysp\@xobeysp\@xobeysp\@xobeysp\@xobeysp\@xobeysp}



% \cd@obeylines
% Turns all cr's into linebreaks. Pagebreaks are not permitted between lines.
% This is copied from lplain.tex's \obeylines, with the cr def'n changed.
%
{\catcode`\^^M=\active % these lines must end with %
  \gdef\cd@obeylines{\catcode`\^^M=\active\let^^M=\cd@cr}}

% What ^M turns into.
\def\cd@cr{\par\penalty10000 } 	% TeX magicness
%
% If the "\leavevmode" is included, the blank lines are not compressed out
% but you will end up with extra space at the bottom of your code if you
% put the "\end{code}" on a new line.
%\def\cd@cr{\par\penalty10000\leavevmode} 	% TeX magicness
%\def\cd@cr{\par\penalty10000\mbox{}}		% LaTeX


% \codeallowbreaks
% Same as \cd@obeylines, except pagebreaks are allowed.
% Put this command inside a code env to allow pagebreaks.

{\catcode`\^^M=\active % these lines must end with %
  \gdef\codeallowbreaks{\catcode`\^^M\active\let^^M\cd@crbr}}

\def\cd@crbr{\leavevmode\endgraf} % What ^M turns into.


% \cd@obeycrsp 
% Turns cr's into non-breakable spaces. Used by \cd.

{\catcode`\^^M=\active % these lines must end with %
  \gdef\cd@obeycrsp{\catcode`\^^M=\active\let^^M=\@xobeysp}}

% =============================================================================

% Set up code environment, in which most of the common special characters
% appearing in code are treated verbatim, namely: $&#^_~%
% \ { } are still enabled so that macros can be called in this
% environment.  Use \\, \{, and \} to use these characters verbatim
% in this environment.
% 
% Inside a group, you can make
% all the hacked chars	special with the	\dcd		command
% $			special with the 	\cddollar	command
% $^_			special with the	\cdmath		command.
% If you have a bunch of math $..$'s in your code env, then a global \cddollar
% or \cdmath at the beginning of the env can save a lot of trouble.
% When chars are special (e.g., after a \dcd), you can still get #$%&_{} with
% \#, \$, \%, \&, \_, \{, and \} -- this is standard LaTeX.
% Additionally, \\ gives \ inside the code env, and when \cdmath
% makes ^ special, it also defines \^ to give ^.

%The hacked characters can be made special again
% within a group by using the \dcd command.

% Note: this environment allows no breaking of lines whatsoever; not
% at spaces or hypens.  To arrange for a break use the standard \- command,
% or a \discretionary{}{}{} which breaks, but inserts nothing.  This is useful,
% for example for allowing hypenated identifiers to be broken, e.g.
% \def\={\discretionary{}{}{}} %optional break
% FOO-\=BAR.

% setupsmallcode added by JHR
%
\def\setupsmallcode{\parsep=0pt\parindent=0pt%
  \renewcommand{\baselinestretch}{1.0}%
  \small\tt\frenchspacing\catcode``=13\@noligs%
  \def\\{\char`\\}\def\_{\char`\_}%
  \def\{{\char`\{}\def\}{\char`\}}%
  \let\dcd=\cd@dcd\let\cddollar=\cd@dollarspecial\let\cdmath=\cd@mathspecial%
  \@makeother\$\@makeother\&\@makeother\#%
  \@makeother\^\@makeother\_\@makeother\~%
  \@makeother\%\cd@obeytabs\cd@obeyspaces}

\def\setupcode{\parsep=0pt\parindent=0pt%
  \tt\frenchspacing\catcode``=13\@noligs%
  \def\\{\char`\\}\def\_{\char`\_}%
  \def\{{\char`\{}\def\}{\char`\}}%
  \let\dcd=\cd@dcd\let\cddollar=\cd@dollarspecial\let\cdmath=\cd@mathspecial%
  \@makeother\$\@makeother\&\@makeother\#%
  \@makeother\^\@makeother\_\@makeother\~%
  \@makeother\%\cd@obeytabs\cd@obeyspaces}
% other: $&#^_~%
% left special: \{}
% unnecessary: @`'"


%% codebox, centerbox
%%=============================================================================
%% The codebox env makes a box exactly as wide as it needs to be
%% (i.e., as wide as the longest line of code is). This is useful
%% if you want to center a chunk of code, or flush it right, or
%% something like that. The optional argument to the environment,
%% [t], [c], or [b], specifies how to vertically align the codebox,
%% just as with arrays or other boxes. Default is [c].

%% Must be a newline immediately after "\begin{codebox}[t]"!

{\catcode`\^^M=\active % these lines must end with %
  \gdef\cd@obeycr{\catcode`\^^M=\active\let^^M=\cr}}

% If there is a [<letter>] option, then the following newline will
% be read *after* ^M is bound to \cr, so we're cool. If there isn't
% an option given (i.e., default to [c]), then the @\ifnextchar will
% gobble up the newline as it gobbles whitespace. So we insert the
% \cr explicitly. Isn't TeX fun?
\def\codebox{\leavevmode\@ifnextchar[{\@codebox}{\@codebox[c]\cr}} %]

\def\@codebox[#1]%
  {\hbox\bgroup$\if #1t\vtop \else \if#1b\vbox \else \vcenter \fi\fi\bgroup%
   \tabskip\z@\setupsmallcode\cd@obeycr% just before cd@obey
   \halign\bgroup##\hfil\span}

\def\endcodebox{\crcr\egroup\egroup\m@th$\egroup}

% Center the box on the page:
\newenvironment{centercode}%
  {\begin{center}\begin{codebox}[c]}%
  {\end{codebox}\end{center}}


%% code, codeaux, tightcode
%%=============================================================================
%% Code environment as described above. Lines are kept on one page.
%% This actually works by setting a huge penalty for breaking
%% between lines of code. Code is indented same as other displayed paras.
%% Note: to increase left margin, use \begin{codeaux}{\leftmargin=1in}.

% To allow pagebreaks, say \codeallowbreaks immediately inside the env.
% You can allow breaks at specific lines with a \pagebreak form.

%% N.B.: The \global\@ignoretrue command must be performed just inside
%% the *last* \end{...} before the following text. If not, you will
%% get an extra space on the following line. Blech.

%% This environment takes two arguments. 
%% The second, required argument is the \list parameters to override the
%%     \@listi... defaults.
%%     - Usefully set by clients: \topsep \leftmargin
%%     - Possible, but less useful: \partopsep
%% The first, optional argument is the extra \parskip glue that you get around
%%     \list environments. It defaults to the value of \parskip.
\def\codeaux{\@ifnextchar[{\@codeaux}{\@codeaux[\parskip]}} %]
\def\@codeaux[#1]#2{%
  \bgroup\parskip#1%
    \begin{list}{}%
      {\parsep\z@\leftmargin=\codeindent\rightskip\z@\listparindent\z@\itemindent\z@#2}%
        \item[]\setupsmallcode\cd@obeylines}%
\def\endcodeaux{\end{list}\leavevmode\egroup\ignorespaces\global\@ignoretrue}

%% Code env is codeaux with the default margin and spacing \list params:
\def\code{\codeaux{}} \let\endcode=\endcodeaux

%% Like code, but with no extra vertical space above and below.
\def\tightcode{\codeaux[=0pt]{\topsep\z@}}%
\let\endtightcode\endcodeaux
%  {\vspace{-1\parskip}\begin{codeaux}{\partopsep\z@\topsep\z@}}%
%  {\end{codeaux}\vspace{-1\parskip}}



% Reasonable separation between lines of code
\newcommand{\codeskip}{\penalty0\vspace{2ex}}


% \cd is used to build a code environment in the middle of text.
% Note: only difference from display code is that cr's are taken
% as unbreakable spaces instead of linebreaks.

\def\cd{\leavevmode\begingroup\ifmmode\let\startcode=\startmcode\else%
	\let\startcode\starttcode\fi%
	\setupcode\cd@obeycrsp\startcode}

\def\cdm{\leavevmode\begingroup\ifmmode\let\startcode=\startmcode\else%
	\let\startcode\starttcode\fi%
	\setupcode\cd@obeycrsp\cd@mathspecial\startcode}

\def\starttcode#1{#1\endgroup}
\def\startmcode#1{\hbox{#1}\endgroup}


% Restore $&#^_~% to their normal catcodes
% Define \^ to give the ^ char.
% \dcd points to this guy inside a code env.
\def\cd@dcd{\catcode`\$=3\catcode`\&=4\catcode`\#=6\catcode`\^=7%
	   \catcode`\_=8\catcode`\~=13\catcode`\%=14\def\^{\char`\^}}

% Selectively enable $, and $^_ as special.
% \cd@mathspecial also defines \^ give the ^ char.
% \cddollar and \cdmath point to these guys inside a code env.
\def\cd@dollarspecial{\catcode`\$=3}
\def\cd@mathspecial{\catcode`\$=3\catcode`\^=7\catcode`\_=8%
		    \def\^{\char`\^}}


% Change log:
% Started off as some macros found in C. Rich's library.
% Olin 1/90:
% Removed \makeatletter, \makeatother's -- they shouldn't be there,
%   because style option files are read with makeatletter. The terminal
%   makeatother screwed things up for the following style options.
% Olin 3/91:
% Rewritten. 
% - Changed things so blank lines don't get compressed out (the \leavevmove
%   in \cd@cr and \cd@crwb). 
% - Changed names to somewhat less horrible choices. 
% - Added lots of doc, so casual hackers can more easily mess with all this.
% - Removed `'"@ from the set of hacked chars, since they are already
%   non-special. 
% - Removed the bigcode env, which effect can be had with the \codeallowbreaks
%   command.
% - Removed the \@noligs command, since it's already defined in latex.tex.
% - Win big with the new \dcd, \cddollar, and \cdmath commands.
% - Now, *only* the chars \{} are special inside the code env. If you need
%   more, use the \dcd command inside a group.
% - \cd now works inside math mode. (But if you use it in a superscript,
%   it still comes out full size. You must explicitly put a \scriptsize\tt
%   inside the \cd: $x^{\cd{\scriptsize\tt...}}$. A \leavevmode was added
%   so that if you begin a paragraph with a \cd{...}, TeX realises you
%   are starting a paragraph.
% - Added the codebox env. Tricky bit involving the first line hacked
%   with help from David Long.
%
% JHR 8/19/91:
% - Added \setupsmallcode to use in multi-line code displays (code, codeaux and
%   codebox environments).
%
% JHR 8/31/91:
% - changed size of small code to \small (from \footnotesize).  Also added
%   code to set the baselinestretch to 1 in smallcode.
%
% JHR 9/12/91:
% - added \codeindent (set to \parindent)
%
% JHR 11/19/91
% - added \cdm{} command for supporting math mode in \cd{}

%%%%%%%%%%%%%%%%%%%%%%%%%%%%%%%%%%%%%%%%%%%%%%%%%%%%%%%%%%%%%%%%%%%%%%
%% mac.tex
%%
%% Umut A. Acar
%% Macros for adaptive computation paper.
%%%%%%%%%%%%%%%%%%%%%%%%%%%%%%%%%%%%%%%%%%%%%%%%%%%%%%%%%%%%%%%%%%%%%%
\newcommand{\AFL}{\textsf{AFL}\xspace}
\newcommand{\IFL}{\textsf{IFL}\xspace}
\newcommand{\MFL}{\textsf{MFL}\xspace}
\newcommand{\AMFL}{\textsf{AMFL}\xspace}
%\newcommand{\SAL}{\textsf{SAL}\xspace}
\newcommand{\SLF}{\textsf{SLf}\xspace}
\newcommand{\SLI}{\textsf{SLi}\xspace}
\newcommand{\afl}{\AFL}
\newcommand{\ifl}{\IFL}
\newcommand{\mfl}{\MFL}
\newcommand{\amfl}{\AMFL}
%\newcommand{\sal}{\SAL}
\newcommand{\slf}{\SLF}
\newcommand{\sli}{\SLI}

\newcommand{\readarrow}{\ensuremath{\Longrightarrow}}


\newcommand{\cutspace}{\vspace{-4mm}}

\newcommand{\bomb}[1]{\fbox{\mbox{\emph{\bf {#1}}}}}

%\renewcommand{\paragraph}[1]{{\bf {#1}}}
% formatting stuff
\newcommand{\codecolsep}{1ex}

%\newcommand{\todo}[1]{{\bf{[NOTE:{#1}]}}}
\newcommand{\todo}[1]{}

\newcommand{\rlabel}[1]{\mbox{\small{\bf ({#1})}}}

\newcommand{\tablerow}{\\[5ex]}
\newcommand{\tableroww}{\\[7ex]}
\newcommand{\tableline}{
\vspace*{2ex}\\
\hline\\ 
\vspace*{2ex}}

% Don't care
\newcommand{\dontcare}{\_}


%% filter and quicksort stuff
\newcommand{\ncf}[2]{C^{fil}_{\ensuremath{{#1},{#2}}}}
\newcommand{\ncq}[1]{C^{qsort}_{\ensuremath{#1}}}
\newcommand{\nuf}[3]{P^{fil}_{#1,(\ensuremath{{#2},{#3}})}}
\newcommand{\nuq}[2]{P^{qsort}_{(\ensuremath{{#1},{#2}})}}

%% shorthands
\newcommand{\ddg}{{\sc ddg}}
\newcommand{\ncpa}{change-propagation algorithm}
\newcommand{\adg}{{\sc adg}}
\newcommand{\nwrite}{\texttt{write}}
\newcommand{\nread}{\texttt{read}}
\newcommand{\nmodr}{\texttt{mod}}
\newcommand{\ttt}[1]{\texttt{#1}}
\newcommand{\nmodl}{\texttt{modl}}
\newcommand{\nnil}{\ttt{NIL}}
\newcommand{\ncons}[2]{\ttt{CONS({\ensuremath{#1},\ensuremath{#2}})}}
\newcommand{\nfilter}{\texttt{filter}}
\newcommand{\nfilterp}{\texttt{filter'}}
\newcommand{\naqsort}{\texttt{qsort'}}
\newcommand{\nqsort}{\texttt{qsort}}
\newcommand{\nqsortp}{\texttt{qsort'}}
\newcommand{\nnewMod}{\texttt{newMod}}
\newcommand{\nchange}{\texttt{change}}
\newcommand{\npropagate}{\texttt{propagate}}
\newcommand{\ndest}{\texttt{d}}
\newcommand{\ninit}{\texttt{init}}



%% Comment sth out. 
\newcommand{\out}[1] {}
\newcommand{\sthat}{\ensuremath{~|~}}

%% definitions
\newcommand{\defi}[1]{{\bfseries\itshape #1}}


% Code listings.
\newcounter{codeLineCntr}
\newcommand{\codeLine}
 {\refstepcounter{codeLineCntr}{\thecodeLineCntr}}
\newcommand{\codeLineL}[1]
 {\refstepcounter{codeLineCntr}\label{#1}{\thecodeLineCntr}}

\newenvironment{codeListing}
 {\setcounter{codeLineCntr}{0}
%  \fontsize{10}{12}
 % the first one is width the second is height
 \fontsize{9}{11}
  \vspace{-.1in}
  \ttfamily\begin{tabbing}}
  {\end{tabbing}
   \vspace{-.1in}}

\newenvironment{codeListing8}
 {\setcounter{codeLineCntr}{0}
  \fontsize{8}{10}
  \vspace{-.1in}
  \ttfamily
  \begin{tabbing}}
 {\end{tabbing}
 \vspace{-.1in}
}

\newenvironment{codeListing8h}
 {\setcounter{codeLineCntr}{0}
  \fontsize{8.5}{10.5}
  \vspace{-.1in}
  \ttfamily
  \begin{tabbing}}
 {\end{tabbing}
 \vspace{-.1in}
}


\newenvironment{codeListing9}
 {\setcounter{codeLineCntr}{0}
  \fontsize{9}{11}
  \vspace{-.1in}
  \ttfamily
  \begin{tabbing}}
 {\end{tabbing}
 \vspace{-.1in}
}

\newenvironment{codeListing10}
 {\setcounter{codeLineCntr}{0}
  \fontsize{10}{12}
  \vspace{-.1in}
  \ttfamily
  \begin{tabbing}}
 {\end{tabbing}
 \vspace{-.1in}
}


\newenvironment{codeListingNormal}
 {\setcounter{codeLineCntr}{0}
  \vspace{-.1in}
  \ttfamily
  \begin{tabbing}}
 {\end{tabbing}
 \vspace{-.1in}
}

\newcommand{\codeFrame}[1]
{\begin{center}\fbox{\parbox[t]{5in}{#1}}\end{center}
% \vspace*{-.15in}
}

\newcommand{\fixedCodeFrame}[1]
{
\vspace*{-4mm}
\begin{center}
\fbox{
\vspace*{-2mm}
\parbox[t]{0.999\columnwidth}{
\vspace*{-2mm}
#1
}
}\end{center}
\vspace*{-4mm}
}

% Footnote commands.
\newcommand{\footnotenonumber}[1]{{\def\thempfn{}\footnotetext{#1}}}

% Margin notes - use \notesfalse to turn off notes.
\setlength{\marginparwidth}{0.6in}
\reversemarginpar
\newif\ifnotes
\notestrue
\newcommand{\longnote}[1]{
  \ifnotes
    {\medskip\noindent Note:\marginpar[\hfill$\Longrightarrow$]
      {$\Longleftarrow$}{#1}\medskip}
  \fi}
\newcommand{\note}[1]{
  \ifnotes
    {\marginpar{\raggedright{\tiny #1}}}
  \fi}

% Stuff not wanted.
\newcommand{\punt}[1]{}

% Sectioning commands.
\newcommand{\subsec}[1]{\subsection{\boldmath #1 \unboldmath}}
\newcommand{\subheading}[1]{\subsubsection*{#1}}
\newcommand{\subsubheading}[1]{\paragraph*{#1}}

% Reference shorthands.
\newcommand{\spref}[1]{Modified-Store Property~\ref{sp:#1}}
\newcommand{\prefs}[2]{Properties~\ref{p:#1} and~\ref{p:#2}}
\newcommand{\pref}[1]{Property~\ref{p:#1}}


\newcommand{\partref}[1]{Part~\ref{part:#1}}
\newcommand{\chref}[1]{Chapter~\ref{ch:#1}}
\newcommand{\chreftwo}[2]{Chapters \ref{ch:#1} and~\ref{ch:#2}}
\newcommand{\chrefthree}[3]{Chapters \ref{ch:#1}, and~\ref{ch:#2}, and~\ref{ch:#3}}
\newcommand{\secref}[1]{Section~\ref{sec:#1}}
\newcommand{\subsecref}[1]{Subsection~\ref{subsec:#1}}
\newcommand{\secreftwo}[2]{Sections \ref{sec:#1} and~\ref{sec:#2}}
\newcommand{\secrefthree}[3]{Sections \ref{sec:#1},~\ref{sec:#2},~and~\ref{sec:#3}}
\newcommand{\appref}[1]{Appendix~\ref{app:#1}}
\newcommand{\figref}[1]{Figure~\ref{fig:#1}}
\newcommand{\figreftwo}[2]{Figures \ref{fig:#1} and~\ref{fig:#2}}
\newcommand{\figpageref}[1]{page~\pageref{fig:#1}}
\newcommand{\tabref}[1]{Table~\ref{tab:#1}}
\newcommand{\stref}[1]{step~\ref{step:#1}}
\newcommand{\caseref}[1]{case~\ref{case:#1}}
\newcommand{\lineref}[1]{line~\ref{line:#1}}
\newcommand{\linereftwo}[2]{lines \ref{line:#1} and~\ref{line:#2}}
\newcommand{\linerefthree}[3]{lines \ref{line:#1},~\ref{line:#2},~and~\ref{line:#3}}
\newcommand{\linerefrange}[2]{lines \ref{line:#1} through~\ref{line:#2}}
\newcommand{\thmref}[1]{Theorem~\ref{thm:#1}}
\newcommand{\thmreftwo}[2]{Theorems \ref{thm:#1} and~\ref{thm:#2}}
\newcommand{\thmpageref}[1]{page~\pageref{thm:#1}}
\newcommand{\lemref}[1]{Lemma~\ref{lem:#1}}
\newcommand{\lemreftwo}[2]{Lemmas \ref{lem:#1} and~\ref{lem:#2}}
\newcommand{\lemrefthree}[3]{Lemmas \ref{lem:#1},~\ref{lem:#2},~and~\ref{lem:#3}}
\newcommand{\lempageref}[1]{page~\pageref{lem:#1}}
\newcommand{\corref}[1]{Corollary~\ref{cor:#1}}
\newcommand{\defref}[1]{Definition~\ref{def:#1}}
\newcommand{\defreftwo}[2]{Definitions \ref{def:#1} and~\ref{def:#2}}
\newcommand{\defpageref}[1]{page~\pageref{def:#1}}
%\newcommand{\eqref}[1]{Equation~(\ref{eq:#1})}
\newcommand{\eqreftwo}[2]{Equations (\ref{eq:#1}) and~(\ref{eq:#2})}
\newcommand{\eqpageref}[1]{page~\pageref{eq:#1}}
\newcommand{\ineqref}[1]{Inequality~(\ref{ineq:#1})}
\newcommand{\ineqreftwo}[2]{Inequalities (\ref{ineq:#1}) and~(\ref{ineq:#2})}
\newcommand{\ineqpageref}[1]{page~\pageref{ineq:#1}}
\newcommand{\itemref}[1]{Item~\ref{item:#1}}
\newcommand{\itemreftwo}[2]{Item~\ref{item:#1} and~\ref{item:#2}}

% Useful shorthands.
\newcommand{\abs}[1]{\left| #1\right|}
\newcommand{\card}[1]{\left| #1\right|}
\newcommand{\norm}[1]{\left\| #1\right\|}
\newcommand{\floor}[1]{\left\lfloor #1 \right\rfloor}
\newcommand{\ceil}[1]{\left\lceil #1 \right\rceil}
  \renewcommand{\choose}[2]{{{#1}\atopwithdelims(){#2}}}
\newcommand{\ang}[1]{\langle#1\rangle}
\newcommand{\paren}[1]{\left(#1\right)}
\newcommand{\prob}[1]{\Pr\left\{ #1 \right\}}
\newcommand{\expect}[1]{\mathrm{E}\left[ #1 \right]}
\newcommand{\expectsq}[1]{\mathrm{E}^2\left[ #1 \right]}
\newcommand{\variance}[1]{\mathrm{Var}\left[ #1 \right]}
\newcommand{\twodots}{\mathinner{\ldotp\ldotp}}

% Standard number sets.
\newcommand{\reals}{{\mathrm{I}\!\mathrm{R}}}
\newcommand{\integers}{\mathbf{Z}}
\newcommand{\naturals}{{\mathrm{I}\!\mathrm{N}}}
\newcommand{\rationals}{\mathbf{Q}}
\newcommand{\complex}{\mathbf{C}}

% Special styles.
\newcommand{\proc}[1]{\ifmmode\mbox{\textsc{#1}}\else\textsc{#1}\fi}
\newcommand{\procdecl}[1]{
  \proc{#1}\vrule width0pt height0pt depth 7pt \relax}
  \newcommand{\func}[1]{\ifmmode\mathrm{#1}\else\textrm{#1}fi} %
%  Multiple cases.  
\renewcommand{\cases}[1]{\left\{
  \begin{array}{ll}#1\end{array}\right.}
  \newcommand{\cif}[1]{\mbox{if $#1$}} 

%% spacing hacks
\newcommand{\longpage}{\enlargethispage{\baselineskip}}
\newcommand{\shortpage}{\enlargethispage{-\baselineskip}}

%!TEX root = manual.tex
%

\usepackage{times}
%-------------------------
% the following magic makes the tt font in math mode be the same as the
% normal tt font (i.e., Courier)
%
\SetMathAlphabet{\mathtt}{normal}{OT1}{pcr}{n}{n}
\SetMathAlphabet{\mathtt}{bold}{OT1}{pcr}{bx}{n}
%-------------------------

\usepackage{color}
\definecolor{Red}{rgb}{0.9,0.0,0.0}
\definecolor{DarkBlue}{rgb}{0.0,0.0,0.75}
\definecolor{Purple}{rgb}{0.5,0.0,0.4}
\definecolor{DarkGreen}{rgb}{0.0,0.5,0.0}
\newcommand{\cdColor}{DarkBlue}
\newcommand{\kwColor}{Purple}
\newcommand{\strColor}{DarkGreen}
\newcommand{\comColor}{Red}

\usepackage{listings}

\lstdefinelanguage{SML}{%
  morekeywords={%
    abstype, and, andalso, as, case, datatype, do, else, end, eqtype, exception,%
    fn, fun, functor, handle, if, in, include, infix, infixr, let, local, nonfix,%
    of, op, open, orelse, raise, rec, sharing, sig, signature, struct, structure,%
    then, type, val, where, while, with, withtype%
  },%
  otherkeywords={[,],\{,\},\,,:,...,_,|,=,=>,->,\#,:>},
  sensitive,%
  alsoletter={_},
  morecomment=[n]{(*}{*)},%
  morestring=[d]",%
}[keywords,comments,strings]%

\lstdefinelanguage{CM}{%
  morekeywords={%
    functor, signature, structure,%
    Library, Group, is,%
  },%
  sensitive,%
  alsoletter={_},
  morecomment=[n]{(*}{*)},%
  morestring=[d]",%
}[keywords,comments,strings]%

\lstset{
  basicstyle=\small\ttfamily\color{\cdColor},
  keywordstyle=\color{\kwColor}\bfseries,
  commentstyle=\color{\comColor}\itshape,
  stringstyle=\color{\strColor}\itshape,
  showstringspaces=false,
  language=SML
}

\newcommand{\nt}[1]{{\it #1}}
\newcommand{\tl}[1]{{\underline{\bf #1}}}
\newcommand{\ttl}[1]{{\underline{\tt #1}}}
\newcommand{\ar}{$\rightarrow$\ }
\newcommand{\vb}{~$|$~}

\usepackage{hyperref}

%!TEX root = manual.tex
%
\chapter{ASDL Syntax}
\label{chap:syntax}

This section describes the syntax of the input language to \asdlgen{}.
The syntax is described using EBNF notation.
Literal terminals are typeset in \lit{bold} and enclosed in single quotes.
Optional terms are enclosed in square brackets and terms that are
repeated zero or more times are enclosed in braces.
Each section describes a fragment of the syntax and its meaning.

\section{Lexical Tokens}

The lexical conventions for \asdl{} are given in \figref{fig:lexical-syntax}.
ASDL is a case-sensitive language and, furthermore, classifies identifiers
into initial-lower-case (\synt{lc-id}) and initial-upper-case (\synt{uc-id}) identifiers.
Type identifiers are initial-lower-case, while constructor identifiers are initial-upper-case.
Module and field identifiers can be either upper or lower-case.

\begin{figure}[t]
  \begin{quote}
    \begin{grammar}
      <upper>     ::= `A' | ... | `Z'

      <lower>     ::= `a' | ... | `z'

      <alpha>     ::= `_' | <upper> | <lower>

      <alpha-num> ::= <alpha> | `0' | ... | `9'

      <lc-id>     ::= <lower> \{ <alpha-num> \}
      
      <uc-id>     ::= <upper> \{ <alpha-num> \}

      <id>        ::= <lc-id> | <uc-id>
                  
      <typ-id>    ::= <lc-id>

      <con-id>    ::= <uc-id>

      <comment>   ::= `--' \{ <text-character> \} <end-of-line>

      <text>      ::= `:' \{ <text-character> \} <end-of-line>
               \alt{} `\%\%' \{ <text-character> | <end-of-line> \} <end-of-line> `\%\%'
    \end{grammar}
  \end{quote}
  \caption{Lexical rules for \asdl{} terminals}
  \label{fig:lexical-syntax}
\end{figure}%

Comments begin with \lit{--}
and continue to the end of the line.

Verbatim text is denoted by \synt{text} and can be specified in one of two ways.
Either by an initial \lit{:} followed by a sequence of \synt{text-character}s that
continues to the end of the line or by a \lit{\%\%} terminated by a \lit{\%\%}
at the beginning of a line by itself. 
Text included using the `\lit{:} notation will have trailing and leading 
whitespace removed.

ASDL has the following keywords:
\begin{quote}
  \lit{alias}  \lit{attributes} \lit{import}
  \lit{include} \lit{module} \lit{primitive} \lit{view}
\end{quote}%
Note that it is allowed to use a keyword as an identifier wherever a \synt{lc-id} is permitted.

\section{File Syntax}

An \asdl{} file consists of one or more \synt{definition}s possibly preceded by \lit{include}
directives.
A definition specifies either a module (see \secref{sec:module-syntax}),
primitive module (see \secref{sec:primitive-syntax}),
or view (see \secref{sec:view-syntax}).

Include directives allow one to split a large \asdl{} specification into multiple files,
while allowing \asdlgen{} to check references from one module to another.
\asdlgen{} will parse included files, but will not generate code for the definitions in included
files.
Also, included files will only be parsed once.

\begin{figure}[t]
  \begin{quote}
    \begin{grammar}
      <file>  ::=  \{ `include' <text> \} <definition> \{ <definition> \}
      
      <definition> ::= <module>
        \alt{} <primitive-module>
        \alt{} <view>
    \end{grammar}
  \end{quote}
  \caption{\asdl{} file syntax}
  \label{fig:file-syntax}
\end{figure}%

\section{Module Syntax}
\label{sec:module-syntax}

\figref{fig:module-syntax} gives the syntax for modules.
An \asdl{} module declaration consists of the keyword \lit{module}
followed by an identifier, an optional set of imported modules, and a
sequence of type definitions enclosed in braces.

\begin{figure}[t]
  \begin{quote}
    \begin{grammar}
      <module>  ::=  `module' <id> [ <imports> ] `{' \{ <type-definition> \} `}'

      <imports> ::=  `(' \{ `import' <id> [ `alias' <id> ] \} `)'
    \end{grammar}
  \end{quote}
  \caption{\asdl{} module syntax}
  \label{fig:module-syntax}
\end{figure}%

For example the
following example declares modules \lstinline[language=ASDL]!A!,
\lstinline[language=ASDL]!B!, and \lstinline[language=ASDL]!C!.
\lstinline[language=ASDL]!B! imports types from \lstinline[language=ASDL]!A!.
\lstinline[language=ASDL]!C! imports types from both \lstinline[language=ASDL]!A! and
\lstinline[language=ASDL]!B!.
Imports cannot be recursive; for example, it is an error for \lstinline[language=ASDL]!B! to
import \lstinline[language=ASDL]!C!, since \lstinline[language=ASDL]!C!
imports \lstinline[language=ASDL]!B!.
\begin{code}\begin{lstlisting}[language=ASDL]
 module A { ... } 
 module B (import A) { ... }
 module C (import A 
           import B) { ... }
\end{lstlisting}\end{code}% 

To refer to a type imported from another module the type must
\emph{always} be qualified by the module name from which it is
imported.
The following declares two different types called ``\texttt{t}.''
One in module \texttt{A} and one in module \texttt{B}.
The type ``\texttt{t}'' in module \texttt{B} defines a type ``\texttt{t}'' that
recursively mentions itself and also references the type ``\texttt{t}'' imported
from module \texttt{A}.
\begin{code}\begin{lstlisting}[language=ASDL]
module A { t = ... } 
module B (import A) { t = T(A.t, t) | N  ... }
\end{lstlisting}\end{code}% 

\section{Type Definitions}

The syntax of type definitions is given in \figref{fig:type-syntax}.
A type defintion begins with a type identifier, which is the name of the type.
The name must be unique within the module, but the order of definitions is unimportant.
When translating type definitions from a module they are placed in what would be considered a
module, package, or name-space of the same name.
If the output language does not support such features and only has one global
name space the module name
is used to prefix all the globally exported identifiers.

\begin{figure}[t]
  \begin{quote}
    \begin{grammar}
      <type-definition>  ::=  <typ-id> `=' <type>

      <type>         ::= <alias-type> | <sum-type> | <product-type>

      <alias-type>   ::= <typ-exp>
      
      <product-type> ::= <fields>

      <sum-type>     ::= <constructor> \{ `|' <constructor> \} [ `attributes' <fields> ]

      <constructor>  ::= <con-id> [ <fields> ]

      <fields>       ::= `(' \{ <field>  `,' \} <field> `)'

      <field>        ::=  <typ-exp> [ <id> ]
      
      <typ-exp>      ::= [ <id> `.' ] <typ-id> [ `?' | `*' | `!' ]
    \end{grammar}
  \end{quote}
  \caption{\asdl{} type definition syntax}
  \label{fig:type-syntax}
\end{figure}%

Type definitions are either alias types, which bind a name to a type expression;
product types, which are simple record definitions;
or sum type, which represent a discriminated union of possible values.
Unlike sum types, product types cannot form recursive type definitions, but they can
contain recursively declared sum types.

\subsection{Alias Types}

Alias types are the simplest form of type definition.
They provide a way to give a name to a type or type expression, similar
to \sml{}'s \lstinline[language=SML]@type@ and \Cplusplus{}'s
\lstinline[language=C++]@typedef@ constructs.

\subsection{Type expressions}

A type expression (\synt{typ-exp}) consists of a possibly qualified type name followed
by an optional type operator.
If the specified type is an ASDL primitive type or is defined in the current module,
then its name is not qualified; all other types defined outside the current module must
be qualified by their module name (or module alias).

The type operators are:
\begin{itemize}
  \item
    option (\lit{?}), which specifies either zero or one value of the specified type.
  \item
    sequence (\lit{*}), which specifies a sequence of zero or more values of the
    specified type, or
  \item
    shared (\lit{!}), which specifies that a value is shared across multiple points
    in the data structure (\ie{}, the structure has a DAG shape instead of just a tree).
\end{itemize}%

Note that while at most one type operator is allowed in a type expression, one can use alias
types to combine two or more operators.
For example, a sequence of optional integers could be defined by:
\begin{code}\begin{lstlisting}[language=ASDL]
int_opt = integer?
int_opt_seq = int_opt*
\end{lstlisting}\end{code}%

\subsection{ASDL Primitive Types}
There are seven pre-defined primitive types in ASDL, which are available without qualification:
\begin{description}
  \item[\normalfont\texttt{\color{\cdColor}bool}] describes Boolean values.
  \item[\normalfont\texttt{\color{\cdColor}int}] describes signed-integer values that are representable in 30 bits
    (\ie{}, in the range ${-}2^{29}$ to~\mbox{$2^{29}-1$}).
  \item[\normalfont\texttt{\color{\cdColor}uint}] describes unsigned-integer values that representable in 30 bits
    (\ie{}, in the range $0$ to~$2^{30}-1$).
  \item[\normalfont\texttt{\color{\cdColor}integer}] describes arbitrary-precision signed-integer values.
  \item[\normalfont\texttt{\color{\cdColor}natural}] describes arbitrary-precision unsigned-integer values.
  \item[\normalfont\texttt{\color{\cdColor}string}] describes length encoded strings of 8-bit characters.
  \item[\normalfont\texttt{\color{\cdColor}identifier}] describes strings with fast equality testing
    analogous to Lisp symbols.
\end{description}%

% TODO: Not implemented yet
%In addition, the \lstinline!StdTypes! module defines additional fixed-precision numeric types
%that can be used when necessary (\eg{}, \lstinline!StdTypes.int32!).

\subsection{Product Types}
Product types are defined by a non-empty  sequence of fields separated by
commas enclosed in parenthesis.
A field consists of a type expression followed by an optional label.
The fields of a product or sum type must either all be labeled or unlabeled.
We use \emph{record} to refer to products of labeled fields and \emph{tuple}
to products of unlabeled fields.
Labels aid in the readability of
descriptions and are used by \asdlgen{} to name the fields of records
and classes for languages.

For example, the declaration
\begin{code}\begin{lstlisting}[language=ASDL] 
pair_of_ints = (int, int) 
\end{lstlisting}\end{code}% 
defines the tuple type \lstinline[language=ASDL]!pair_of_ints! that consists of two integers,
whereas the declaration
\begin{code}\begin{lstlisting}[language=ASDL] 
size = (int width, int height) 
\end{lstlisting}\end{code}%
defines the record type \lstinline[language=ASDL]!size! that consists of two
labeled fields: \lstinline[language=ASDL]@width@ and \lstinline[language=ASDL]@height@.
Note that ASDL requires that if any field in a product type has a label, then all
of them must have labels.

For the \sml{} target, product types without labels are translated to tuples, while
those with labels are translated to records.

\subsection{Sum Types}

Sum types are the most useful types in ASDL. They provide concise notation
used to describe a type that is the tagged union of a finite set of other
types.  Sum types consists of a series of constructors separated by a
vertical bar. Each constructor consist of a constructor identifier followed
by an optional product type. 

Constructor names must be unique within the module in which they are
declared. Constructors can be viewed as functions who take some number of
arguments of arbitrary type and create a value belonging to the sum type in
which they are declared.
For example
\begin{code}\begin{lstlisting}[language=ASDL]
module M {
  sexpr = Int(int)
	| String(string)
	| Symbol(identifier)
	| Cons(sexpr, sexpr)
	| Nil
}
\end{lstlisting}\end{code}%
declares that values of type \lstinline[language=ASDL]!sexpr! can either be
constructed from an \lstinline[language=ASDL]!int! using the
\lstinline[language=ASDL]!Int! constructor or a \lstinline[language=ASDL]!string!
from a \lstinline[language=ASDL]!String! constructor, an
\lstinline[language=ASDL]!identifier! using the \lstinline[language=ASDL]!Symbol!
constructor, from two other \lstinline[language=ASDL]!sexpr! using the
\lstinline[language=ASDL]!Cons! constructor, or from no arguments
using the \lstinline[language=ASDL]!Nil! constructor.
Notice that the \lstinline[language=ASDL]!Cons! constructor
recursively refers to the \lstinline[language=ASDL]!sexpr! type.
\asdl{} allows sum types to be mutually recursive.
Recursion, however, is limited to sum types defined within the same module.

\subsubsection{Sum Types as Enumerations}
\label{sec:enumerations}

Sum types that consist completely of nullary constructors
are often treated specially and translated into static constants of a
enumerated value in languages that support them.
For example, the following \asdl{} specification:
%
\begin{code}\begin{lstlisting}[language=ASDL]
module Op {
  op = PLUS | MINUS | TIMES | DIVIDE 
}
\end{lstlisting}\end{code}%
%
Is translated into the following \Cplusplus{} code:
%
\begin{code}\begin{lstlisting}[language=c++]
namespace M {
    enum class op {
        PLUS = 1, MINUS, TIMES, DIVIDE
    };
}
\end{lstlisting}\end{code}%

\subsubsection{Attribute Fields}
A sum-type definition may optionally be followed by a list of attribute fields, which
provide a concise way to specify fields that are common to all of the constructors
of a sum type.
For example, the definition
%
\begin{code}\begin{lstlisting}[language=ASDL]
module M {
  pos = (string file, int linenum, int charpos)
  sexpr = Int(int)
	| String(string)
	| Symbol(identifier)
	| Cons(sexpr, sexpr)
	| Nil
        attribute(pos)
}
\end{lstlisting}\end{code}%
adds a field of type \lstinline[language=ASDL]!pos! to all the constructors
in \lstinline[language=ASDL]!sexpr!.
One can think of an \lstinline!attribute! annotation as syntactic sugar for just including
the extra fields at the \emph{beginning} of each constructor's fields.
For example, the above definition can be viewed as syntactic sugar for
%
\begin{code}\begin{lstlisting}[language=ASDL]
module M {
  pos = (string file, int linenum, int charpos)
  sexpr = Int(pos, int)
	| String(pos, string)
	| Symbol(pos, identifier)
	| Cons(pos, sexpr, sexpr)
	| Nil(pos)
}
\end{lstlisting}\end{code}%
Note that this interpretation implies that attribute fields are labeled if, and only if, all
of the constructor fields are labeled.

Attribute fields are treated specially when translating to some targets.
For example in \Cplusplus{} code, the attribute field is defined in the base class for the sum type.

\section{Primitive Modules}
\label{sec:primitive-syntax}

\begin{figure}[t]
  \begin{quote}
    \begin{grammar}
      <primitive-module> ::= `primitive' `module' <id> `{' \{ <id> \} `}'
    \end{grammar}%
  \end{quote}%
  \caption{\asdl{} primitive module syntax}
  \label{fig:prim-module-syntax}
\end{figure}%

Primitive modules (see \figref{fig:prim-module-syntax}) provide a way to introduce abstract
types that are defined outside
of \asdl{} and which have their own pickling and unpickling code.
For example, we might want to include GUIDs (16-byte globally-unique IDs) in our pickles.
We can do so by first defining a primitive module \lstinline!Prim!:
%
\begin{quote}\begin{lstlisting}[language=ASDL]
primitive module Prim { guid }
\end{lstlisting}\end{quote}%
%
Then, depending on the target language, we define
supporting code to read and write guids from the byte stream.
In \sml{}, we would define two modules:
\begin{enumerate}
  \item \lstinline[language=SML]!structure Prim! that defines the representation of the
    \lstinline[language=SML]!guid! type.
  \item
    \lstinline[language=SML]!structure PrimPickle! that defines functions
    for pickling/unpickling a GUID using an imperative stream API.
\end{enumerate}%
The \sml{} implementation of these modules could be written as follows:
\begin{code}\begin{lstlisting}[language=SML]
structure Prim : sig
    type guid
  end = struct
    type guid = GUID.guid
  end

structure PrimPickle : sig

    val read_guid : (unit -> Word8.word) -> unit -> Prim.guid
    val write_guid : (Word8.word -> unit) -> Prim.guid -> unit
    
  end = struct
  
    val guidSize = 16
    fun read_guid getByte () =
          GUID.fromBytes(Word8Vector.tabulate(guidSize, fn _ => getByte()))
    fun write_guid putByte guid = let
          Word8Vector.app putByte (GUID.toBytes guid)

   end
\end{lstlisting}\end{code}%
(assuming that the \lstinline!GUID! module implements the application's representation
of GUIDs).

For \Cplusplus{}, a primitive module requires a corresponding header file that declares
the primitive types and instances of the overloaded \lstinline[language=c++]!<<! and
\lstinline[language=c++]!>>! operators on the primitive types.  These declarations should
all be in a \lstinline[language=c++]!namespace! with the type name of the primitive module.
For example, the module from above would require the provision of a \texttt{Prim.hxx} header
file that contained something like the following code:
%
\begin{code}\begin{lstlisting}[language=c++]
#include <iostream>
#include "guid.hxx"

namespace Prim {

    typedef GUID::guid guid;

    std::istream &operator>> (std::istream &is, guid &g);
    std::ostream &operator<< (std::ostream &os, guid const &g);

}
\end{lstlisting}\end{code}%
(assuming that the \texttt{guid.hxx} header defines the application's representation
of GUIDs).
%For the \Cplusplus{} target, one should use string streams (in binary mode)
%to support pickling/unpickling of to/from memory.


\section{View Syntax}
\label{sec:view-syntax}

A view defines how an \asdl{} specification is translated to a target.
Each of the supported targets (\eg{}, \sml{} or \Cplusplus{}) has a default
view, but it is possible to customize the translation using \synt{view}
definitions.
The syntax of view declarations is given in \figref{fig:view-syntax}.
This section covers the syntax of views, but leaves the semantics to
\chapref{chap:views}.

\begin{figure}[t]
  \begin{quote}
    \begin{grammar}
      <view>        ::= `view' <id> `{' \{ <view-entry> \} `}'

      <view-entry>  ::=  <view-entities> `<=' <view-properties>
         \alt{} `<=' <id> `{' \{ <view-entity> <text> \} `}'

      <view-entities> ::= <view-entity>
         \alt{} `{' \{ <view-entity> \} `}'

      <view-entity> ::= `<file>'
         \alt{} `module' <id>
         \alt{} <id> `.' <typ-id> [ `.' `*' ]
         \alt{} <id> `.' <typ-id> `.' <con-id>

      <view-properties> ::= <id> <text>
          \alt{} `{' \{ <id> <text> \} `}'
    \end{grammar}%
  \end{quote}%
  \caption{\asdl{} view syntax}
  \label{fig:view-syntax}
\end{figure}%

\subsection{Basic View Syntax}
Views are named and consist of series of entries.
In its basic form, a view entry consists of a \synt{view-entity}, which specifies a file, module,
type, or constructor entity, and a view property, which is a name-value pair that is associated with
the entity.
%
\begin{quote}\begin{grammar}
  <view-entry>  ::= <view-entity> `<=' <id> <text>
\end{grammar}\end{quote}%
%
The meaning of an entry is to associate the specified view property with the specified view entity.
%%
%\begin{code}\begin{lstlisting}[language=ASDL]
%view Doc {
%  module  M <= doc_string
%%%
%  Types for representing LISP s-expressions.
%%%
%  M.sexpr  <= doc_string : s-expressions 
%  M.Int    <= doc_string : s-expression constructor
%  M.Symbol <= doc_string : s-expression constructor
%  M.Cons   <= doc_string : s-expression constructor
%  M.Nil    <= doc_string : s-expression constructor
%}
%
%view Java {
% M.sexpr <= source_name : Sexpr
% M.sexpr <= base_class  : MyClass
%}
%\end{lstlisting}\end{code}%
%%
%associates with the module \lstinline[language=ASDL]!M! the type
%\lstinline[language=ASDL]!M.sexpr! and the
%constructor \lstinline[language=ASDL]!M.Int!
%strings that will be added to the automatically generated documentation
%produced by the \texttt{--doc} command of \asdlgen{}.
%(\emph{In future we will probably dump them in comments in the output code too.})
%The view named \lstinline[language=ASDL]!Java! causes the type
%\lstinline[language=ASDL]!M.sexpr! to be renamed \lstinline[language=ASDL]!Sexpr! when
%generating Java output, and causes the abstract class normally generated to
%inherit from \lstinline[language=ASDL]!MyClass!. 

There can be multiple views with the same name.
The entries of two views with the same name are merged and consist of the
union of the entries in both.
It is an error, for two views of the same name to assign different values
to the same property of an entity.

\subsection{View Entry Derived Forms}
To make it easier to specify view entries, \asdl{} generalizes the basic syntax
to remove some of the redundancy of the basic syntax.
First, it is possible to specify multiple view entities on the left-hand-side of
the \lit{<=} symbol.
Likewise, it is possible to specify multiple view properties on the right-hand-side
of the \lit{<=} symbol.

It is also possible to assign different values to different entities for a fixed property
using the syntax.
%
\begin{quote}\begin{grammar}
  <view-entry> ::= `<=' <id> `{' \{ <view-entity> <text> \} `}'
\end{grammar}\end{quote}%
%
Here the property name is given first, followed by a sequence of view-entity-value pairs.

Lastly, \asdl{} allows a \lit{.*} suffix to be added to sum-type entities.
This suffix means that the entity specifies the set of all of the constructors of the type.

%Examples of the sugared notation are shown
%below in their respective order.
%\begin{code}\begin{lstlisting}[language=ASDL]
%view Doc {
% { M.Int  M.Symbol  
%   M.Cons M.Nil } <= doc_string : s-expression constructor
%
% <= doc_string {
%  module  M 
%%%
%  Types for representing LISP s-expressions.
%%%
%  M.sexpr : s-expressions 
%  }
%}
%
%view Cxx {
%  M.sexpr <= {
%    source_name : Sexpr
%    base_class  : MyClass
%  }
%}
%\end{lstlisting}\end{code}%


\bibliographystyle{plain}

\begin{document}
	\permission{\copyright ACM, 2008. http://doi.acm.org/10.1145/nnnnnn.nnnnnn}
	%\conferenceinfo{ML'07,} {October 5, 2007, Freiburg, Germany.}
	\CopyrightYear{2008}
	\copyrightdata{978-1-59593-676-9/07/0010}
	
\title{Principles for an Extensible Module System}
\authorinfo{}{University of Chicago}{}
\maketitle

\begin{abstract}
The ML module system has inspired a series of formalism describing the workings and mechanisms of variations. These formalisms are quite varied and run the gamut from Leroy's presentation based on syntactic mechanisms alone, to Harper-Lillibridge and Dreyer-Crary-Harper's type-theoretic approach, to the elaboration semantics approach as represented by the Definition. 	
\end{abstract}

\category{D.3.1}{Programming Languages}{Formal Definitions and Theory}
\category{D.3.3}{Programming Languages}{Language Constructs and Features}[Abstract data types, Modules] 
\category{F.3.3}{Logics and Meanings of Programs}{Studies of Program Constructs}[Type structure]

\terms 
Languages, Theory

\keywords
modularity, translucent sum, singleton types, type theory, elaboration semantics, abstract data types, functors, generativity

{
\section{Outline}
\begin{enumerate}
	\item ML module system is powerful because:
	\begin{enumerate}
		\item functors can be typechecked independent of functor applications
		\item enforces type abstraction by opaque ascription
		\item hierarchical modularity
		\item higher-order functors
		\item type definitions
	\end{enumerate}
	\item Background
	\begin{enumerate}
		\item Modular module\cite{leroy00} provides a syntactic model 
		\begin{enumerate}
			\item Functorized over core language syntax and typechecker
			\item Core language must be aware of paths
			\item functor can only be applied to rooted path (can be extended to anonymous argument module if parameter dependency in the result signature can be reduced away)
			\item Because notion of core language (especially types) is abstract, the system is not easily extensible to richly typed languages. 
			\item Does not support shadowing in core language declarations. Shadowing would fundamentally break syntactic notion of opaque ascription. 
		\end{enumerate}
		\item Definition \cite{mthm97} provides a semantic approach
		\item Harper-Lillibridge\cite{lillibridge94}, Dreyer-Crary-Harper\cite{dhc03} provide a type theoretic model
		\begin{enumerate}
			\item purity
			\item totality/partiality
			\item static vs. dynamic effects; strong and weak sealing
			\item comparability and projectibility
		\end{enumerate}
		\item Harper-Pierce \cite{ATTAPL} provides high-level design principles and issues
		\begin{enumerate}
			\item sharing type constraints cannot always be expanded out as claimed by Pierce-Harper. Symmetric constraints are necessary in the absence of an ability for type definitions to reference their enclosing signatures and thus specifications that come after the type definition in question. 
			\item determinacy versus static/dynamic
			\item first-class modules
			\item In Stone, full signatures are called ``very precise'' versus abstract; he argues that the avoidance problem is one reason why translucent sums do not have full signatures (most precise); also that restricting all programs to Leroy's named form guarantees the existence of ``most-specific'' interfaces. This terminology leads to the term ``natural interface'' -- the most precise interface that can be computed without appealing to strengthening (M-SELF). 
		\end{enumerate}
		\item Treatments of first-class modules
		\begin{enumerate}
			\item Harper-Lillibridge
			\item Russo \cite{russo00}
			\item Dreyer-Crary-Harper
		\end{enumerate}
		\item Leroy \cite{Leroy-generativity} gives a module system where type generativity and SML90-style definitional sharing = path equivalence + A-normalization (for functor applications) + S-normalization (a consolidation of sharing constraints). Leroy shows that a module system with generative datatypes (but no constructors), sharing between type paths, and abstract type specifications can be expressed in terms of a module system with generative datatypes and manifest types. Leroy's simplified module system does not include value specifications and datatype constructors both of which can constrain the order in which specifications must be written in and therefore result in situations where sharing constraints cannot be in general reduced to manifest types. 
	\end{enumerate}
	\item Design space: Propagation of types 
	\begin{enumerate}
		\item Primarily through functor application
		\item Shao fully transparent signature calculus
		\item SML90 - only explicit sharing equations and structure (identity) sharing 
		\item SML93 - plus definitional sharing
		\item SML97 - plus where type and definitional specs; structure identity sharing eliminated
		\item existential types, dependent sums, translucent sums, singleton kinds/signatures, Shao flexroot
	\end{enumerate}
	\item Key problem in modularity: modularizing types and their interpretations
	\begin{enumerate}
		\item Example: Symbol table versus Ord
		\item type class an imperfect solution because limits interpretation to a single instance and the type to only generative nonstructured types (i.e., datatypes)
		\item applicative functors imperfect because they permit too much sharing
		\item constructors of higher kind
		\item relationship to expression problem?
	\end{enumerate}
	\item True higher-order functors (MacQueen-Tofte 94) -- {\bf full transparency}
	\begin{enumerate}
		\item functor parameter type information should be propagated through functor application; in other words, should transparent signature matching semantics carry over to higher-order functor setting?
		\item Shao \cite{shao98} cites optimal propagation of types (ensuring that inlined and separately compiled modules receive the same typing) as a benefit of full transparency
		\item \begin{verbatim}
			signature FPS = sig type t end
			signature FRS = sig type t end
			structure M = struct type t = int end
			functor F(X: FPS): FRS = 
			  struct type t = X.t end
			functor G(functor F(X: FPS):FRS) = 
			  struct structure R = F(M) end
		\end{verbatim}
		\item \verb|t = int| should propagate through the HO functor application to \verb|G(F).R|
		\item type definitions in signatures insufficient because not all parameter functor F's will propagate this type information
		\item MacQueen-Tofte 94 appears to be the only module system that accounts for this class of type sharing
		\item Unfortunately, this feature apparently conflicts with separate compilation as noted by Dreyer-Crary-Harper \cite{dhc03}
		\item Primary criticism of stamp-based operational semantics is the difficulty of extending such a semantics
		\item The MT94 semantics also stratifies the stamp computation in the peculiar way 
		\item Shao \cite{shao99} offers an alternative example for fully transparent higher-order functors\\
		\begin{verbatim}
			signature S = sig type t val x : t end
			funsig FS = fsig (X:S): S
			structure S = struct type t=int val x=1 end
			functor F1(X:S) = struct type t=X.t val x=X.x end
			functor F2(X:S) = struct type t=int val x=1 end
			functor APPS(F:FS) = F(S)
			structure R = 
			  struct structure R1 = APPS(F1)
				 structure R2 = APPS(F2)
				 val res = (R1.x = R2.x)
			  end
		\end{verbatim}
		\item Shao offers a signature language based on gathering all flexible components in a higher-order type constructor that can be applied to obtain the fully transparent signature at a later point. The resultant signature language superficially resembles applicative functors. However, applications in the signature language must be on paths. Consequently, it does not address fully transparency in the general case. 
		\item Shao \cite{shao98} extends MacQueen-Tofte fully transparency modules with support for type definitions, type sharing (normalized into type definitions), and hidden module components. 
	\end{enumerate}
	\item Extending modules to support signature bindings as components
	\begin{enumerate}
		\item In ML modules, structures can be arranged in a hierarchy. This feature enables flexible namespace management. In contrast, signatures cannot be arranged in such a hierarchy. In the ML module system, signatures must be defined at the top-level and can never be enclosed in any other signature or module. For complex hierarchies such the SML/NJ's Control module that contains layers of submodules, the corresponding signature CONTROL and the signatures of the submodules PRINT and ELAB are related only incidentally by occurrence in structure specifications in CONTROL. This shortcoming in the signature language unnecessarily pollutes the signature namespace and complicates browsing through and working with highly nested hierarchies. It would be desirable to permit (transparent) signature specifications within signatures. For added flexibility and perhaps increased expressiveness, it may be useful permit signature definitions within structures and functors. Furthermore, in order for modules to match these signatures enriched with signature specifications, modules must permit corresponding signature definitions. 
		\item Leroy \cite{leroy94} offers an example that introducing signature bindings into structures would add polymorphic modules and F$_\omega$-like type operators. In particular, he offers 
		\begin{verbatim}
			functor(x: sig signature X end) (m{x.X/X})
		\end{verbatim}
		as an encoding of $\Lambda X.m$
		\item Swasey \cite{swasey06} and Leroy \cite{leroy94} both cite Harper Lillibridge's proof of the undecidability of $\lambda^{\rightarrow,\exists,\exists=}$ as reason for their skepticism that such a feature can be added to a module system without breaking type-checking
		\item Harper and Lillibridge's proof establishes that in a type calculus with opaque and binary sums, subtyping is undecidable in the presence of a Forget rule that forgets transparency. The example they use is a subtyping relation on transparent and opaque sums containing a type constructor with a contravariant subtyping rule such as $T\rightarrow \alpha'$
		\item Adding signature components does not necessarily provide parametric polymorphism in the style of System F because functor application uses coercive subtyping 
	\end{enumerate}
	\item Polymorphism and modules
	\begin{enumerate}
		\item Interaction with Hindley-Milner polymorphism in core language
		\item Moscow ML's first-class modules provides first-class polymorphism
		\item Example: polymorphic data structures, continuation monad \cite{kahrs94}
	\end{enumerate}
	\item Claim: Instantiation is an analysis process that can detect cyclic sharing and other behaviors that may result in an unrealizable signature (though certainly not all behaviors)
\end{enumerate}	

The key observation in Leroy's syntactic presentation of a module system is that we can check a sense of type equality by comparing rooted paths that uniquely determine type identity. Unfortunately, this technique suffers from the inability to support core and module language level shadowing of bindings. 
 
True separate compilation poses an interesting challenge in the ML module system. In order to have true separate compilation, the surface signature language must be able to express the full signature of all structures and functors. Even in a module language that only supports first-order functors, this requirement proves to be a problem because the signature language would be unable to express generative types in the body of a functor. Generative types in the body of a functor do not have externally expressible names prior to functor application. 

\begin{figure}
	\tiny
\begin{tabular}{|l|l|l|l|l|l|l|l|}
	\hline
System & higher-order & first-class & sep comp & rec mod & app fct & gen & phase sep\\
	\hline
	HL \cite{lillibridge94} & & \checkmark & & & & \\
	\hline
	Leroy 94 & & & & & & \\
	\hline
	Russo \cite{russo01} & & \checkmark & & \checkmark & & \\
	\hline
	DCH (\cite{dhc03}) & & & & & & \\
	\hline
	RTG (\cite{dreyer05}) & & & & & & \\
	\hline
	$\lambda^{\llparenthesis~\rrparenthesis}$ (\cite{ATTAPL}) & & & & & & \\
	\hline
	$\lambda^S$ (\cite{ATTAPL}) & & N & & N & N & N \\
	\hline
	MT94 (\cite{mt94}) & \checkmark & & & & & \checkmark & \checkmark \\
	\hline
\end{tabular}
\end{figure}
%Would a completely nondependent module calculus work? More to the point, signature specs may depend on module terms, but is there a way to avoid this kind of dependence? 

%!TEX root = ../principles.tex
\begin{figure}
\centering
\fixedCodeFrame{
\small
\[
\setlength{\tabcolsep}{0ex}
\renewcommand{\arraystretch}{1.1}
\begin{array}{rcll}
	d & ::= &\signature~s=\sig{\overline{spec}}\\
	  & ~~\bnfalt& \local~\overline{d}~\inx~\overline{d}~\en\\
	  & ~~\bnfalt&~ld\\
	ld & ::= & \structure~X~=~m\\
	   & ~~\bnfalt&\functor~F\overline{(X:sigexp)}~cnstr = m\\
	   & ~~\bnfalt&\val~x=e~\bnfalt~\type~\overline{\alpha}~t=\tau\\
	   & ~~\bnfalt&\local~\overline{ld}~\inx~\overline{ld}~\en\\
	   & ~~\bnfalt&\open~\overline{q}\\
	m & ::= & q~\bnfalt~\struct{\overline{ld}}~\bnfalt~q~arg~\bnfalt~\letx~\overline{d}~\inx~m~\en\\
	  & ~~\bnfalt~&~m:sigexp~\bnfalt~m:>sigexp\\
	arg & ::= & ( \overline{d} )~\overline{arg}~\bnfalt~(m)~arg~\bnfalt~(m)~\bnfalt~(\overline{d})\\
	cnstr & ::= & :~sigexp~\bnfalt~:>~sigexp\\
	sigexp & ::= & s~\bnfalt~\sig{\overline{spec}}~\bnfalt~sigexp~\where~\type~\overline{\alpha}~q=\tau\\
	spec & ::= & \structure~q:sigexp[=q]\\
	     & ~~\bnfalt &\functor~q~\overline{(X:sigexp)} : sigexp\\
	     & ~~\bnfalt &\type~\overline{\alpha}~t[=\tau]~\bnfalt~\val~x:\tau~\bnfalt~\sharing~\type~p=p\\
	     & ~~\bnfalt &\sharing~p=p~\bnfalt~\exception~exn\\
	     & ~~\bnfalt & \eqtype~\overline{\alpha}~t[=\tau]\\
	\tau & ::= & \tau\rightarrow\tau~\bnfalt~\Int~\bnfalt~t\\
\end{array}
\]
}
The overline notation indicates a vector of the components. 
\caption{Module surface language: Design caveat -- In SML/NJ the AST permits signature declarations within structures. The parser, however, does not support this. The surface language follows the parser. The implementation AST has an Abstract Structure form, but the parser does not seem to ever produce it. }
\label{fig:pureml}
\end{figure}
%!TEX root = ../principles.tex
\begin{figure}
\centering
\fixedCodeFrame{
\small
\[
\setlength{\tabcolsep}{0ex}
\renewcommand{\arraystretch}{1.1}
\begin{array}{ll}
	\infer[\rlabel{local}]{\begin{array}{l}\Gamma\vdash\local~\overline{d_1}~\inx~\overline{d_2}~\en\\
	\Rightarrow(\underline{\local}~\underline{\overline{d_1}}~\inx~\underline{\overline{d_2}}~\en,entdec,\Gamma_2,\Delta)\end{array}}
		{\Gamma\vdash\overline{d_1}\Rightarrow(\underline{\overline{d_1}},entdec_1,\Gamma_1,\Delta_1)\\ 
 \Gamma\oplus\Gamma_1\vdash\overline{d_2}\Rightarrow(\underline{\overline{d_2}},entdec_2,\Gamma_2,\Delta_2)}
\\
\qquad
\\
\infer[\rlabel{type}]{\Gamma\vdash\type~\overline{\alpha}~t=\tau\Rightarrow(,entdec,\Gamma,\Delta)}
{\strut}
\\
\qquad
\\
\infer[\rlabel{sig}]{\Gamma\vdash\sig{\overline{spec}}\Rightarrow(\sig{\overline{spec}},\bullet,\Gamma',\emptyset)}{\strut}
\\
\qquad
\\
\infer[\rlabel{structure}]{\Gamma\vdash\struct{\overline{ld}}\Rightarrow\struct{\overline{ld}}}{}
\end{array}
\]
}
The underline notation denotes the elaborated version of the nonterminal, i.e., the explicitly typed and elaborated variant.
\nocaptionrule
\caption{Static Semantics}
\label{fig:statsem}
\end{figure}
\input{figs/fig-rlzn}
\input{figs/fig-evalent}
%!TEX root = ../principles.tex
\begin{figure}
\centering
\fixedCodeFrame{
\small
~\\[2mm]
\fbox{$\Upsilon\vdash\theta\Downarrow_{fct}\psi$}
\begin{equation}
\infer{\Upsilon\vdash\vec{\rho}\Downarrow_{fct}\Upsilon(\vec{\rho})}
{\strut} 
\label{eq:fctentpath}
\end{equation}

\begin{equation}
\infer{\Upsilon\vdash\lambda\rho.\varphi
\Downarrow_{fct}\langle
  \lambda\rho.\varphi;\Upsilon\rangle}
{\strut}	
\label{eq:fctentlam}
\end{equation}

\begin{equation}
\infer{\Upsilon\vdash\lambda\rho.\Sigma \Downarrow_{fct} \langle
  \lambda\rho.\Sigma; \Upsilon
  \rangle}
{\strut}
\label{eq:fctentformal}
\end{equation}

}
\caption{Functor entity evaluation}
\label{fig:fctenteval}
\end{figure}
 %!TEX root = ../principles.tex
\begin{figure}
\centering

\fixedCodeFrame{
\small
~\\[0.5mm]
\fbox{$\Upsilon\vdash \eta \Downarrow_{decl} \Upsilon'$}
\begin{equation}
	\infer{\Upsilon\vdash\bullet\Downarrow_{decl} \emptyset_{ee}}
        {\strut}
\label{eq:entdecempty}
\end{equation}

\begin{equation}
	 \infer{\Upsilon\vdash\rho=_{str}\varphi,\eta\Downarrow_{decl}[\rho\mapsto R]\Upsilon'}
	{\Upsilon\vdash\varphi\Downarrow_{str}R\qquad 
          \Upsilon[\rho\mapsto R]\vdash\eta\Downarrow_{decl}
          \Upsilon'} 
\label{eq:entdecstr}
\end{equation}
	
\begin{equation}
	\infer{\Upsilon\vdash\rho=_{fct}\theta,\eta\Downarrow_{decl}[\rho\mapsto\psi]\Upsilon'}
	  {\Upsilon\vdash\theta\Downarrow_{fct}\psi\qquad 
            \Upsilon[\rho\mapsto\psi]\vdash\eta\Downarrow_{decl}
            \Upsilon'}
\label{eq:entdecfct}
\end{equation}

\begin{equation}
	\infer{\Upsilon\vdash\rho=_{tyc}\newx(n),\eta\Downarrow_{decl}
          [\rho\mapsto\tau^n]\Upsilon'}
	    {\begin{array}{c}
                \Upsilon[\rho\mapsto\tau^n]\vdash\eta\Downarrow_{decl}\Upsilon'\qquad
            (\tau\textrm{ is fresh in }\Upsilon)
          \end{array}} 
\label{eq:entdecnew}
\end{equation}

\begin{equation}
         \infer{\Upsilon\vdash[\rho=_{def}\mathbb{C}^\lambda],\eta\Downarrow_{decl}
           [\rho\mapsto\mathbb{C}^\lambda]\Upsilon'}
         {\Upsilon[\rho\mapsto\mathbb{C}^\lambda]\vdash\eta\Downarrow_{decl}
           \Upsilon'}
         \label{eq:entdectypedef}
\end{equation}

\begin{equation}
         \infer{\Upsilon\vdash[\rho=_{def}\vec{\rho}],\eta\Downarrow_{decl}
           [\rho\mapsto\Upsilon(\vec{\rho})]\Upsilon'}
         {\Upsilon[\rho\mapsto\Upsilon(\vec{\rho})]\vdash \eta
           \Downarrow_{decl} \Upsilon'}
         \label{eq:entdecalias}
\end{equation}
}
\caption{Entity declaration semantics}
\label{fig:entdecsems}
\end{figure}
%!TEX root = ../principles.tex
\begin{figure}
\centering
\fixedCodeFrame{
\small
\setlength{\tabcolsep}{0ex}
\renewcommand{\arraystretch}{1.1}
~\\[2mm]
\fbox{$\Gamma,\Upsilon,\Sigma\vdash sigexp \Rightarrow_{sig} \Sigma'$}
	
\begin{equation}
\infer{\Gamma,\Upsilon,\Sigma\vdash x \Rightarrow_{sig} \Gamma(x)}
{\strut} 
\label{eq:emptysig}
\end{equation}

\begin{equation}
\infer{\Gamma,\Upsilon,\Sigma\vdash sigexp~\textbf{where type}~p = C^\lambda\Rightarrow_{sig}\mathsf{rebind}(p,\mathbb{C}^\lambda,\Sigma')}
	{\begin{array}{c}
	  \Gamma,\Upsilon,\Sigma\vdash sigexp\Rightarrow_{sig}\Sigma'\qquad
	  \Sigma'(p) = (\rho,n)\\ \Gamma,\Upsilon\vdash C^\lambda \Rightarrow_{tyc} \mathfrak{C}^\lambda\qquad 
          \Gamma,\Upsilon,\Sigma\Sigma'\vdash C^\lambda \searrow^{tyc} \mathfrak{C}^\lambda \\
          \Upsilon\vdash \mathfrak{C}^\lambda \searrow^{tyc} 
          \mathbb{C}^\lambda\qquad |\mathbb{C}^\lambda|=n
	 \end{array}} 
\label{eq:wheretype}
\end{equation}

\begin{equation}
\infer{\Gamma,\Upsilon,\Sigma\vdash \textbf{sig } specs \textbf{ end} \Rightarrow_{sig} \Sigma'}
	{\begin{array}{c}
	\Gamma,\Upsilon,\Sigma\vdash specs \Rightarrow_{specs} \Sigma'
	\end{array}} 
\label{eq:sigspecs}
\end{equation}

$\mathsf{rebind}(p,\mathbb{C}^\lambda,\Sigma)$ replaces the
binding $[x\mapsto (\rho,n)]$ in $\Sigma$ with $[x \mapsto 
\mathbb{C}^\lambda]$ where $p$ ends in $x$. 

\fbox{$\Gamma,\Upsilon,\Sigma\vdash specs \Rightarrow_{specs} \Sigma$}
\begin{equation}
\infer{\Gamma,\Upsilon,\Sigma\vdash \emptyset_{specs} \Rightarrow_{specs} \emptyset_{sig}}
{\strut}
\end{equation}

\begin{equation}
\infer{\Gamma,\Upsilon,\Sigma\vdash spec,specs \Rightarrow_{specs} \Sigma'\Sigma''}
{\Gamma,\Upsilon,\Sigma\vdash spec \Rightarrow_{spec} \Sigma' \qquad \Gamma,\Upsilon,\Sigma\Sigma'\vdash specs \Rightarrow_{specs} \Sigma''}
\label{eq:specs}
\end{equation}

}
\caption{Signature elaboration}
\label{fig:elabsig}
\end{figure}

\begin{figure}
	\centering
	\fixedCodeFrame{
	\small
	~\\[2mm]
	\fbox{$\Gamma, \Upsilon, \Sigma \vdash spec \Rightarrow_{spec} \Sigma'$}
\begin{equation}
\infer{\Gamma,\Upsilon,\Sigma\vdash\textbf{type
                  }\vec{\alpha}~t\Rightarrow_{spec} [t\mapsto (\rho,|\vec{\alpha}|)]}
{(\rho\textrm{ fresh in }\Gamma\textrm{ and
                  }\Upsilon)} 
\end{equation}
% Do we need to extend the static environment during elaboration of
% the subsequent specs? The only reason we might need to is to support
% relativization. I think we have to. 
\begin{equation}
\infer{\Gamma,\Upsilon,\Sigma\vdash\textbf{type }t=C^\lambda\Rightarrow_{spec} [t\mapsto \mathbb{C}^\lambda]}
{\Gamma,\Upsilon\vdash C^\lambda \Rightarrow_{tyc} \mathfrak{C}^\lambda\qquad \Upsilon,\Sigma\vdash \mathfrak{C}^\lambda \searrow^{tyc}
 \mathbb{C}^\lambda}
\label{eq:typedefspec}
\end{equation}

\begin{equation}
              \infer{\Gamma,\Upsilon,\Sigma\vdash\textbf{val }x:T
                 \Rightarrow_{spec} [x\mapsto\mathbb{T}]}
{\Gamma,\Upsilon\vdash T \Rightarrow_{te} \mathfrak{T}\qquad \Upsilon,\Sigma\vdash \mathfrak{T} \searrow^{tyc} \mathbb{T}}
\label{eq:valspec}
\end{equation}

\begin{equation}
\infer{\begin{array}{c}
    \Gamma,\Upsilon,\Sigma\vdash\textbf{structure }x :
                  sigexp
                  \Rightarrow_{spec} [x\mapsto
                  (\rho,\Sigma')]
\end{array}}
		{\begin{array}{c}
                    \Gamma,\Upsilon,\Sigma\vdash sigexp
                    \Rightarrow_{sig}
                    \Sigma'\qquad
                    (\rho~\textrm{fresh in }\Gamma\textrm{ and
                    }\Upsilon) 
              \end{array}}
\end{equation}

\begin{equation}
\infer{\begin{array}{c}
\Gamma,\Upsilon,\Sigma\vdash\textbf{functor }f(X: sigexp_1):sigexp_2\\
   \Rightarrow_{spec} [f\mapsto(\rho,\Pi\rho_x:\Sigma_1.\Sigma_2)]
\end{array}}
		{\begin{array}{c}
\Gamma,\Upsilon,\Sigma\vdash sigexp_1 \Rightarrow_{sig}
\Sigma_1 \qquad
(\rho_x\textrm{ and }\rho~\textrm{fresh in }\Gamma\textrm{ and }\Upsilon )\\
\Gamma,\Upsilon,\Sigma[X\mapsto(\rho_x,\Sigma_1)]\vdash
sigexp_2 \Rightarrow_{sig} \Sigma_2\\
% \qquad \psi=\langle\lambda\rho_x.\Sigma_2 ;
%  \Upsilon\rangle % [4/8/10] Is this the correct closure environment? Can \Sigma_2 mention local entities? Yes, but those are local, therefore, they should be interpreted locally and not by the closure, which only interpret nonlocal entities.  
\end{array}}
\label{eq:fctspec}
\end{equation}

	}
	\caption{Signature spec elaboration}
	\label{fig:elabspec}
\end{figure}
%!TEX root = ../principles.tex
\begin{figure}
\centering
\fixedCodeFrame{
\small
\setlength{\tabcolsep}{0ex}
\renewcommand{\arraystretch}{1.1}
~\\[2mm]
\fbox{$\Gamma,\Upsilon\vdash d^m \Rightarrow_{decl} (\eta, \Gamma', \Upsilon')$}

	\begin{equation} 
          \infer{\Gamma,\Upsilon\vdash \circ
            \Rightarrow_{decl} (\bullet, \emptyset_{se},
            \emptyset_{ee})}{\strut}  
          \label{eq:emptydecl}
        \end{equation}

        \begin{equation}
          \infer{\Gamma,\Upsilon\vdash \mathbf{val}~x=e,d^m
            \Rightarrow_{decl} (\eta, [x\mapsto\mathfrak{T}]\Gamma', \Upsilon')}
          {\Gamma \vdash e \Rightarrow_{core} \mathfrak{T} \qquad
            \Gamma[x\mapsto\mathfrak{T}], \Upsilon \vdash d^m \Rightarrow_{decl}
            (\eta, \Gamma', \Upsilon')}
          \label{eq:valdecl}
        \end{equation}

	\begin{equation} 
          \infer{\begin{array}{l} 
              \Gamma,\Upsilon\vdash \mathbf{type}~t=C^\lambda,d^m
              \Rightarrow_{decl}(\eta,[t\mapsto \mathfrak{C}^\lambda]\Gamma',\Upsilon')
	\end{array}}
	{\begin{array}{c}
            \Gamma,\Upsilon \vdash C^\lambda \Rightarrow_{tyc} \mathfrak{C}^\lambda\qquad
            \Gamma[t\mapsto \mathfrak{C}^\lambda],\Upsilon\vdash
            d^m\Rightarrow_{decl}(\eta,\Gamma',
            \Upsilon')
          \end{array}} 
        \label{eq:typedefdecl}
      \end{equation}

        \begin{equation} 
       \infer{\begin{array}{c}
           \Gamma,\Upsilon\vdash
         \mathbf{datatype}~\vec{\alpha}~t,d^m\\
         \Rightarrow_{decl}([\rho_t =_{tyc} \newx(n)]\eta, [t\mapsto
         \tau^n]\Gamma', [\rho_t\mapsto \tau^n]\Upsilon')
       \end{array}}
{\begin{array}{c}
    n=|\vec{\alpha}|\qquad
    \Gamma[t\mapsto\tau^n],\Upsilon[\rho_t\mapsto \tau^n]\vdash d^m \Rightarrow_{decl} (\eta, \Gamma',
    \Upsilon')\\ (\rho_t\textrm{ and }\tau\textrm{ are fresh})
\end{array}}
      \label{eq:dtdecl}
        \end{equation}

\begin{equation} 
          \infer{\begin{array}{c}
              \Gamma,\Upsilon\vdash \mathbf{structure}~X=strexp,d^m\\
  \Rightarrow_{decl} ([\rho=_{str}\varphi]\eta, [X\mapsto (\rho, M)]\Gamma',
  [\rho\mapsto R]\Upsilon')
\end{array}}
	{\begin{array}{c}
\Gamma,\Upsilon\vdash strexp\Rightarrow_{str} (M, \varphi)\qquad 
M = (\Sigma,R)\qquad (\rho~\textrm{fresh})\\
\Gamma[X\mapsto (\rho, M)],\Upsilon[\rho\mapsto R]\vdash d^m\Rightarrow_{decl}(\eta, \Gamma', \Upsilon')
	\end{array}} 
      \label{eq:strdecl}
\end{equation}


	\begin{equation} 
          \infer{\begin{array}{c}
              \Gamma,\Upsilon\vdash
              \mathbf{functor}~f(X:sigexp)=strexp,d^m \\
              \Rightarrow_{decl} ([\rho=_{fct}\theta]\eta, [f\mapsto(\rho,(\Pi\rho_x:\Sigma_x.\Sigma_{res},\psi))]\Gamma',
              [\rho\mapsto\psi]\Upsilon')
            \end{array}}
	        {\begin{array}{c} 
                    \Gamma,\Upsilon, \emptyset_{sig} \vdash
                    sigexp\Rightarrow_{sig} \Sigma_x\qquad
                    \Upsilon,\emptyset_{ee}\vdash \Sigma_x \uparrow
                    \Upsilon_x \\
                    R_x = \langle \Upsilon_x,\Upsilon \rangle\\
                    \Gamma[X\mapsto(\rho_x, (\Sigma_x,
                   R_x))],\Upsilon[\rho_x\mapsto R_x]\vdash
                    strexp\Rightarrow_{str}((\Sigma_{res},\_),\varphi)\\
                    % \Upsilon_\Delta is out of scope at the
                    % declaration level, so it is dropped. 
        \theta =
        \lambda\rho_x.\varphi\qquad \psi =
        \langle\theta;\Upsilon\rangle\\
	\Gamma [f\mapsto(\rho,(\Pi
        \rho_x:\Sigma_x.\Sigma_{res},\psi))],
        \Upsilon[\rho\mapsto\psi]\vdash
        d^m \Rightarrow_{decl}(\eta,\Gamma',\Upsilon')\\
        % No need for extending entity environment because \rho_F
        % won't be looked up
	(\rho_x,\rho~\textrm{fresh})\\
        %\Gamma, \Upsilon \vdash \emptyset_{se}\gamma \hookrightarrow (M_{ext}, \eta_{ext})
	         \end{array}} 
               \label{eq:fctdecl}
        \end{equation}
}
\vspace{1em}
The resultant $\Upsilon$ must be the local entity environment in order
for the structure expression judgment for struct $d^m$ end to properly
construct a structure realization. 
\caption{Module declaration elaboration}
\label{fig:elabmod}
\end{figure}

\begin{figure}
\centering
\fixedCodeFrame{
\small
~\\[2mm]
\fbox{$\Gamma,\Upsilon\vdash strexp \Rightarrow_{str} (M,\varphi)$}
	\begin{equation} 
\infer{\Gamma,\Upsilon\vdash p \Rightarrow_{str} (M, \vec{\rho})}
	          {\Gamma(p)=(\vec{\rho}, M)} 
\label{eq:strpath}
\end{equation}

	\begin{equation} 
\infer{\Gamma,\Upsilon\vdash \mathbf{struct}~d^m~\mathbf{end}
  \Rightarrow_{str} ( (\Sigma,\langle \Uloc,\Upsilon\rangle),\llparenthesis\eta\rrparenthesis)}
	{\begin{array}{c}
\Gamma,\Upsilon\vdash d^m\Rightarrow_{decl}(\eta,\Gamma', \Uloc)\qquad
\Upsilon \Uloc \vdash\Gamma'\hookrightarrow \Sigma
\end{array}} 
\label{eq:basestr}
\end{equation}

% [4/8/2010] How do local environments work with functor entities? 
\begin{equation} 
\infer{\begin{array}{c}
\Gamma,\Upsilon\vdash p(strexp)\Rightarrow_{str}
((\Sigma_{body},R_{app}),\varphi_{app})
\end{array}}
	{\begin{array}{c}
\Gamma(p) = (\vec{\rho}, (\Pi X:\Sigma_{par}.\Sigma_{body}, \langle\theta; \Upsilon'\rangle))\\
\Gamma,\Upsilon\vdash strexp\Rightarrow_{str}
(M,\varphi)\\ 
\Upsilon\vdash (M,\varphi) : \Sigma_{par} \Rightarrow_{match} (R_{c},\varphi_{c})\\
\varphi_{app} = \vec{\rho}(\varphi_{c})\qquad \Upsilon\vdash \varphi_{app} \Downarrow_{str} R_{app}
\end{array}}
\label{eq:strapp}
\end{equation}

	\begin{equation} 
\infer
	{\Gamma,\Upsilon\vdash \mathbf{let}~d^m~\mathbf{in}~strexp\Rightarrow_{str}(M,\mathbf{let}~\eta_{def}~\mathbf{in}~\varphi)}
	{\begin{array}{c}\Gamma,\Upsilon\vdash d^m\Rightarrow_{decl}(\eta_{def},\Gamma_{def},\Upsilon_{def})\\ \Gamma_{def},\Upsilon_{def}\vdash strexp\Rightarrow_{str}(M, \varphi)
\end{array}} 
\label{eq:letexp}
\end{equation}

\begin{equation} 
\infer{\begin{array}{c}
\Gamma,\Upsilon\vdash strexp : sigexp
 \Rightarrow_{str} ((\Sigma_{spec},R_c),\varphi_c)
\end{array}}
{\begin{array}{c}
	   \Gamma,\Upsilon,\emptyset_{sig}\vdash sigexp \Rightarrow_{sig} \Sigma_{spec} \qquad
	   \Gamma,\Upsilon\vdash strexp \Rightarrow_{str} (M_{u},\varphi_{u})\\
	   \Upsilon\vdash (M_{u},\varphi_u) : \Sigma_{spec} \Rightarrow_{match} (R_c,\varphi_{c})
\end{array}} 
\label{eq:transascription}
\end{equation}
     
 \begin{equation} 
\infer{\begin{array}{c}
\Gamma,\Upsilon \vdash strexp :> sigexp 
\Rightarrow_{str}
   ((\Sigma_{spec},\langle\Upsilon_{spec},\Upsilon\rangle),
   \varphi_{c})
\end{array}}
{\begin{array}{cc}
\Gamma,\Upsilon,\emptyset_{sig}\vdash sigexp
     \Rightarrow_{sig} \Sigma_{spec}\qquad
   \Gamma,\Upsilon\vdash strexp \Rightarrow_{str} (M_u, \varphi_u)\\
  \Upsilon\vdash (M_u,\varphi_u) : \Sigma_{spec}
  \Rightarrow_{match} (R_{c},\varphi_{c})\\
  \Upsilon, \emptyset_{ee} \vdash \Sigma_{spec} \uparrow \Upsilon_{spec}
% Does it matter whether we return the uncoerced or coerced varphi?
% The compiler returns the coerced version.  
\end{array}
}
\label{eq:opaqueascription}
\end{equation}

}
\caption{Structure expression elaboration}
\label{fig:strexpelab}
\end{figure}


%!TEX root = ../principles.tex
\begin{figure}
	\centering
	\fixedCodeFrame{
	\small
	\begin{align}
		& E;\Upsilon\vdash tycexp \Rightarrow_{tyc} t & \notag \\[3mm]
		& E;\Upsilon\vdash & 
	\end{align}	
	}
	\caption{Type constructor elaboration}
	\label{fig:elabtyc}
\end{figure}
%!TEX root = ../principles.tex

% Observation: Don't need the entire static environment as context
% because all symbolic names should have already been reduced
% away. The place where the entire static environment does play a role
% is in relativization of type expressions for values and (derived) tycon
% expressions. 

% There are some fundamental flaws in the rules as given. First, there
% is no base case, for the empty static environment. That would
% establish what the realization part of the resultant full signature
% really is. Second, none of the rules extend the realization part of
% the full signature at this point. Thus, I expect the resultant
% realization to end up empty anyway. 

\begin{figure}
	\centering
	\fixedCodeFrame{
	\small
        ~\\[2mm]
	\fbox{$\Upsilon\vdash \Gamma \hookrightarrow \Sigma $}
               \begin{equation}
                 \infer{\Upsilon\vdash\emptyset_{se}\hookrightarrow
                   \emptyset_{sig}}
                 {\strut}
               \end{equation}

		\begin{equation} 
                  \infer{\Upsilon\vdash
                    [x\mapsto \mathfrak{T}]\Gamma \hookrightarrow
                  [x\mapsto \mathbb{T}]\Sigma_r
                  }
		{\begin{array}{c}
                  \Upsilon\vdash \mathfrak{T} \searrow^{te}
                  \mathbb{T}\qquad
                  \Upsilon\vdash \Gamma \hookrightarrow \Sigma_r
                \end{array}} 
\label{eq:extraval}
            \end{equation}
% The value binding rule doesn't add any entity decl. It merely adds a value spec with a relativized version of value binding's type.

% There are no open tycon bindings in the static environment. All tycons in the static environment are defined or instantiated. 
% Perhaps combine \upharpoonright and C_{ep}() notation together because one is just an extension to type expressions
              %\begin{equation} \infer{\begin{array}{c} C_{ep},C_{elab}\vdash
               % [q\mapsto[=t]]\Gamma\\ \hookrightarrow
                %(([q\mapsto (\rho, \mathsf{arity}(t))]\Sigma, \Upsilon),
                %\eta)\end{array}}
              %{C_{ep},C_{elab}\vdash\Gamma\hookrightarrow((\Sigma,\Upsilon),
               % \eta)\qquad C_{ep}(t)=[\rho]} & \\[5mm]
              \begin{equation} 
\infer{\Upsilon\vdash [t\mapsto \mathfrak{C}^\lambda]\Gamma \hookrightarrow [t\mapsto \mathbb{C}^\lambda]\Sigma_r}
{\Upsilon\vdash \mathfrak{C}^\lambda
  \searrow^{tyc} \mathbb{C}^\lambda\qquad \Upsilon\vdash \Gamma \hookrightarrow
  \Sigma_r}
\label{eq:extratypedef}
\end{equation}

% SML/NJ's representation of tycon entities included type definitions
% (represented by an entity path). In this new semantics, there is no
% need for any form other than new(arity). 
% The question remains, what is the relationship between the scoping
% in the entity environment and in the static environment. That is to
% say, do we expect \Upsilon^{-1}(\tau^n) to be anything other than
% singleton? If so, what is an example?
\begin{equation} 
\infer{\Upsilon\vdash[t\mapsto\tau^n]\Gamma \hookrightarrow [t\mapsto (\rho, n)]\Sigma_r}
{\inv(\Upsilon,\tau^n)=\rho\qquad\Upsilon\vdash
  \Gamma \hookrightarrow\Sigma_r} 
\label{eq:extraatomictyc}
\end{equation}

% The interesting part about producing a tycon spec is how we compute the entity variable. The tycon definition may be local, in which case we can reuse that tycon's (RHS) entity variable. Otherwise, we need to produce a new entity variable and a corresponding entity declaration mapping this new entity variable to a CONSTtyc or a VARtyc for new and nonlocal tycons respectively. 
% What we need is a judgment that produces an entity variable, entity environment, and entity declarations from a static entity.
\begin{equation}
  \infer{\begin{array}{c}
      \Upsilon\vdash[x\mapsto (\rho, (\Sigma_1,R_1))]\Gamma_r\hookrightarrow [x\mapsto
                (\rho,\Sigma_1)]\Sigma_r
              \end{array}}
              {\begin{array}{c}
                  \Upsilon\vdash\Gamma_r\hookrightarrow \Sigma_r
                \end{array}}
\label{eq:extrastr}
            \end{equation}

% The structure binding rule first looks up the entity path for the full signature 
% If the structure has an entity path which is a singleton, then it uses that as entity variable in the structure spec we are producing. 
% If the entity path is not singleton, then the entdec is a structure e_new = ep such that ep is the non-singleton entity path. It is this e_new that is used in the structure spec. 
% Otherwise, no entity path exists yet so we produce a CONSTstr with the realization from the full signature, mapping a e_new to it. 
% In the latter two cases, the entity is not local so we need to add a
% local version of the entity to the entity environment, mapping e_new
% to rlzn.  
              \begin{equation} 
                \infer{\Upsilon\vdash[f\mapsto (\rho, (\Sigma^f_1,\psi))]\Gamma_r \hookrightarrow [f\mapsto (\rho,\Sigma^f_1)]\Sigma_r}
              {\begin{array}{c}
                 \Upsilon\vdash\Gamma_r\hookrightarrow\Sigma_r
              \end{array}} 
\label{eq:extrafct}
              \end{equation}

              % Ignoring signature and functor signature
              % bindings. Since this is the cumulative static
              % environment, it may contain top-level declared
              % signatures and functor signatures. 
             %\begin{equation}
             %\infer{\Upsilon\vdash[x\mapsto \Sigma]\Gamma\hookrightarrow \Sigma'}
             %{\Upsilon\vdash\Gamma\hookrightarrow \Sigma'}
             %\end{equation}

             %\begin{equation}
             %\infer{\Gamma_0,\Upsilon\vdash[x\mapsto\Sigma^f]\Gamma\hookrightarrow (M,\eta)}
             %{\Gamma_0,\Upsilon\vdash\Gamma\hookrightarrow (M,\eta)} 
             %\end{equation}

	}
\caption{Signature extraction}
\label{fig:extractsig}
\end{figure}

\section{Comparison to MacQueen-Tofte}
In MacQueen-Tofte, signatures support higher-order functors by including a functor signature environment, denoted $\Phi$, that maps functor maps to functor signatures. Because functor signature components may depend on specifications that came earlier in the enclosing structure signature, the system introduces a binding $\lambda\rho$ that binds $\rho$, the entity variable representing the entire enclosing structure signature. 

The MT structure signatures require this $\lambda\rho$ binding because they incorporate functor signature environments indexed by functor paths, $\Phi$. Without the $\lambda\rho$-binding, $\Phi$ cannot depend on entities in the enclosing signatures. Structure matching first looks up all $fp$s in $\Phi$ and then attempts to match the static functor with $\Phi[\varphi/\rho]$.

In contrast, in the current language, SML/NJ no longer permits nonlocal forms of sharing of the flavor illustrated in the example in the MT94 paper. Instead, structure definitions can express the same sharing. Signature specifications contain the signatures and realizations for structure definitions ({\it i.e.}, $\rho,\Sigma=\Sigma$ and $\rho,\Sigma=_{\overline{\rho}} \Sigma$) at the point of structure signature matching. These definition structure signatures and realizations are filled in during signature elaboration by looking up the static environment and entity path context.  

\section{Primary and secondary components}
The form of a functor argument is constrained by the functor parameter signature possibly modified by a where type definition. In the parameter signature, there are structure specifications, formal functor specifications, structure/type sharing constraints, and two classes of type specifications. Type specifications may be abstract or definitional. Abstract type specifications that remain abstract after the elaborator resolves all sharing and where type constraints are called flexible or primary components. These primary type components are those essential components that must be kept to maintain the semantics of functor application (i.e., the type application associated with the functor application). The specific function of primary type components is to capture a canonical representative of an equivalence class of abstract types induced by type sharing constraints. Each equivalence class has exactly one primary type component that serves as a representative element. References to all other members of the equivalence class should be redirected to the associated primary type component. The remaining type components are secondary and therefore should be fully derivable from the primary components and externally defined types. Secondary types do not have to be explicitly represented in the parameter signature because all occurrences of these secondary types can be expanded out according to their definitions. 

\begin{lstlisting}
functor F(type s type t type u = s * t
          sharing type t = s) = ...
\end{lstlisting}

In the above example, \lstinline|s| can be primary, representative for the equivalence class containing both \lstinline|s| and \lstinline|t|, and \lstinline|u| is secondary.

\section{Relationship to Harper-Stone Semantics}

\section{A Fully Expressive Signature Language}
The ML signature language permits type definitions that may refer to general type expressions. Type expressions may involve both primitive type constructors such as $\rightarrow$ and type operators. It is the inclusion of type operators that gives the signature language much of its expressiveness. The semantics of type sharing constraints differs significant between SML90 and SML97. Type sharing constraints could be imposed on two type constructors without restriction in SML90. In SML97, the designers partitioned the semantics of type sharing into type definitions which expressed sharing between an abstract type and an arbitrary type expression, and regular type sharing constraints which can only be imposed between two flexible (or primary) types whose names must be in scope. 

A module system that permits both type definitions and type sharing constraints in signatures introduces significant new complexity. For example, whereas in Leroy's \cite{Leroy-generativity} TypModl language, which only permits SML90-style definitional type sharing constraints and no type definitions, type sharing constraints can be ``normalized'' by pushing them up the signature and eliminated by turning them into type definitions, type sharing constraints cannot be eliminated in a language that permits both type definitions and type sharing constraints.

\section{FLINT Compilation}
The goal of FLINT compilation was to enable all type-based optimizations to work across module boundaries by compiling the module calculus into System F. 

%!TEX root = ../principles.tex
\begin{figure}
	\centering
	\fixedCodeFrame{
	\small
	\[
	\setlength{\tabcolsep}{0ex}
	\renewcommand{\arraystretch}{1.1}
	\begin{array}{rcll}
		p_s & ::= & s_i~|~p_s.s_i\\
		p_f & ::= & f_i~|~p_s.f_i\\
		p_t & ::= & t_i~|~p_s.t_i\\
		D & ::= & \epsilon~|~D~D'~|~type~t_i::\kappa_c\\
		  & ~~| & type~t_i::\kappa_c = \mu_c~|~structure~s_i:M_s\\
		  & ~~| & functor~f_i:M_f\\
		M_s & ::= & sig~D~end\\
		M_f & ::= & fsig(s_i:M_s)M'_s\\
		\kappa_c & ::= & \Omega~|~\Omega\rightarrow\kappa_c\\
		\mu_c & ::= & p_t~|~int~|~\mu_c \rightarrow \mu'_c~|~\lambda t_i::\Omega.\mu_c~|~\mu_c[\mu'_c]\\
		d & ::= & \epsilon~|~d d'~|~local~d~in~d'~end~|~type~t_i::\kappa_c = \mu_c\\
		  & ~~| & structure~s_i=m_s~|~functor~f_i=m_f\\
		m_s & ::= & p_s~|~f_i(s_i)~|~(s_i:M_s)~|~m_b\\
		m_b & ::= & struct~d~end\\
		m_f & ::= & p_f~|~funct (s_i:M_s)m_b
	\end{array}
	\]
	}
\caption{Normalized module calculus NRC}
\end{figure}
%!TEX root = ../principles.tex
\begin{figure}
	\centering
	\fixedCodeFrame{
	\small
	\[
	\setlength{\tabcolsep}{0ex}
	\renewcommand{\arraystretch}{1.1}
	\begin{array}{rcll}
		\kappa_t & ::= & \Omega~|~\kappa_t\rightarrow\kappa'_t~|~\{l::\kappa_t,\ldots,l'::\kappa'_t\}\\
		\mu_t & ::= & \alpha~|~Int~|~\mu_t\rightarrow\mu'_t~|~\lambda\alpha::\kappa_t.\mu_t~|~\mu_t[\mu'_t]\\
		      & ~~| & \{l=\mu_t,\ldots,l'=\mu'_t\}~|~\mu_t.l\\
		\sigma_t & ::= & T(\mu_t)~|~\sigma_t\rightarrow\mu'_t~|~\{l:\sigma_t,\ldots,l':\sigma'_t\}~|~\forall\alpha::\kappa_t.\sigma_t\\
		e_t & ::= & x~|~i~|~\lambda x:\sigma_t.e_t~|~@e_t e'_t~|~\Lambda\alpha::\kappa_t.e_t~|~e_t[\mu_t]\\
		    & ~~| & \{l=e_t,\ldots,l'=e'_t\}~|~e_t.l~|~let~d_t~in~e_t]\\
		d_t & ::= & \epsilon~|~(x=e_t);d_t
	\end{array}
	\]
	}
	\caption{F$_\omega$-based target calculus TGC}
\end{figure}
}

\bibliography{modules}
\end{document}
