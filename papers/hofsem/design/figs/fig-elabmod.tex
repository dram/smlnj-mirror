%!TEX root = ../principles.tex
\begin{figure}
\centering
\fixedCodeFrame{
\small
\setlength{\tabcolsep}{0ex}
\renewcommand{\arraystretch}{1.1}
~\\[2mm]
\fbox{$\Gamma,\Upsilon\vdash d^m \Rightarrow_{decl} (\eta, \Gamma', \Upsilon')$}

	\begin{equation} 
          \infer{\Gamma,\Upsilon\vdash \circ
            \Rightarrow_{decl} (\bullet, \emptyset_{se},
            \emptyset_{ee})}{\strut}  
          \label{eq:emptydecl}
        \end{equation}

        \begin{equation}
          \infer{\Gamma,\Upsilon\vdash \mathbf{val}~x=e,d^m
            \Rightarrow_{decl} (\eta, [x\mapsto\mathfrak{T}]\Gamma', \Upsilon')}
          {\Gamma \vdash e \Rightarrow_{core} \mathfrak{T} \qquad
            \Gamma[x\mapsto\mathfrak{T}], \Upsilon \vdash d^m \Rightarrow_{decl}
            (\eta, \Gamma', \Upsilon')}
          \label{eq:valdecl}
        \end{equation}

	\begin{equation} 
          \infer{\begin{array}{l} 
              \Gamma,\Upsilon\vdash \mathbf{type}~t=C^\lambda,d^m
              \Rightarrow_{decl}(\eta,[t\mapsto \mathfrak{C}^\lambda]\Gamma',\Upsilon')
	\end{array}}
	{\begin{array}{c}
            \Gamma,\Upsilon \vdash C^\lambda \Rightarrow_{tyc} \mathfrak{C}^\lambda\qquad
            \Gamma[t\mapsto \mathfrak{C}^\lambda],\Upsilon\vdash
            d^m\Rightarrow_{decl}(\eta,\Gamma',
            \Upsilon')
          \end{array}} 
        \label{eq:typedefdecl}
      \end{equation}

        \begin{equation} 
       \infer{\begin{array}{c}
           \Gamma,\Upsilon\vdash
         \mathbf{datatype}~\vec{\alpha}~t,d^m\\
         \Rightarrow_{decl}([\rho_t =_{tyc} \newx(n)]\eta, [t\mapsto
         \tau^n]\Gamma', [\rho_t\mapsto \tau^n]\Upsilon')
       \end{array}}
{\begin{array}{c}
    n=|\vec{\alpha}|\qquad
    \Gamma[t\mapsto\tau^n],\Upsilon[\rho_t\mapsto \tau^n]\vdash d^m \Rightarrow_{decl} (\eta, \Gamma',
    \Upsilon')\\ (\rho_t\textrm{ and }\tau\textrm{ are fresh})
\end{array}}
      \label{eq:dtdecl}
        \end{equation}

\begin{equation} 
          \infer{\begin{array}{c}
              \Gamma,\Upsilon\vdash \mathbf{structure}~X=strexp,d^m\\
  \Rightarrow_{decl} ([\rho=_{str}\varphi]\eta, [X\mapsto (\rho, M)]\Gamma',
  [\rho\mapsto R]\Upsilon')
\end{array}}
	{\begin{array}{c}
\Gamma,\Upsilon\vdash strexp\Rightarrow_{str} (M, \varphi)\qquad 
M = (\Sigma,R)\qquad (\rho~\textrm{fresh})\\
\Gamma[X\mapsto (\rho, M)],\Upsilon[\rho\mapsto R]\vdash d^m\Rightarrow_{decl}(\eta, \Gamma', \Upsilon')
	\end{array}} 
      \label{eq:strdecl}
\end{equation}


	\begin{equation} 
          \infer{\begin{array}{c}
              \Gamma,\Upsilon\vdash
              \mathbf{functor}~f(X:sigexp)=strexp,d^m \\
              \Rightarrow_{decl} ([\rho=_{fct}\theta]\eta, [f\mapsto(\rho,(\Pi\rho_x:\Sigma_x.\Sigma_{res},\psi))]\Gamma',
              [\rho\mapsto\psi]\Upsilon')
            \end{array}}
	        {\begin{array}{c} 
                    \Gamma,\Upsilon, \emptyset_{sig} \vdash
                    sigexp\Rightarrow_{sig} \Sigma_x\qquad
                    \Upsilon,\emptyset_{ee}\vdash \Sigma_x \uparrow
                    \Upsilon_x \\
                    R_x = \langle \Upsilon_x,\Upsilon \rangle\\
                    \Gamma[X\mapsto(\rho_x, (\Sigma_x,
                   R_x))],\Upsilon[\rho_x\mapsto R_x]\vdash
                    strexp\Rightarrow_{str}((\Sigma_{res},\_),\varphi)\\
                    % \Upsilon_\Delta is out of scope at the
                    % declaration level, so it is dropped. 
        \theta =
        \lambda\rho_x.\varphi\qquad \psi =
        \langle\theta;\Upsilon\rangle\\
	\Gamma [f\mapsto(\rho,(\Pi
        \rho_x:\Sigma_x.\Sigma_{res},\psi))],
        \Upsilon[\rho\mapsto\psi]\vdash
        d^m \Rightarrow_{decl}(\eta,\Gamma',\Upsilon')\\
        % No need for extending entity environment because \rho_F
        % won't be looked up
	(\rho_x,\rho~\textrm{fresh})\\
        %\Gamma, \Upsilon \vdash \emptyset_{se}\gamma \hookrightarrow (M_{ext}, \eta_{ext})
	         \end{array}} 
               \label{eq:fctdecl}
        \end{equation}
}
\vspace{1em}
The resultant $\Upsilon$ must be the local entity environment in order
for the structure expression judgment for struct $d^m$ end to properly
construct a structure realization. 
\caption{Module declaration elaboration}
\label{fig:elabmod}
\end{figure}

\begin{figure}
\centering
\fixedCodeFrame{
\small
~\\[2mm]
\fbox{$\Gamma,\Upsilon\vdash strexp \Rightarrow_{str} (M,\varphi)$}
	\begin{equation} 
\infer{\Gamma,\Upsilon\vdash p \Rightarrow_{str} (M, \vec{\rho})}
	          {\Gamma(p)=(\vec{\rho}, M)} 
\label{eq:strpath}
\end{equation}

	\begin{equation} 
\infer{\Gamma,\Upsilon\vdash \mathbf{struct}~d^m~\mathbf{end}
  \Rightarrow_{str} ( (\Sigma,\langle \Uloc,\Upsilon\rangle),\llparenthesis\eta\rrparenthesis)}
	{\begin{array}{c}
\Gamma,\Upsilon\vdash d^m\Rightarrow_{decl}(\eta,\Gamma', \Uloc)\qquad
\Upsilon \Uloc \vdash\Gamma'\hookrightarrow \Sigma
\end{array}} 
\label{eq:basestr}
\end{equation}

% [4/8/2010] How do local environments work with functor entities? 
\begin{equation} 
\infer{\begin{array}{c}
\Gamma,\Upsilon\vdash p(strexp)\Rightarrow_{str}
((\Sigma_{body},R_{app}),\varphi_{app})
\end{array}}
	{\begin{array}{c}
\Gamma(p) = (\vec{\rho}, (\Pi X:\Sigma_{par}.\Sigma_{body}, \langle\theta; \Upsilon'\rangle))\\
\Gamma,\Upsilon\vdash strexp\Rightarrow_{str}
(M,\varphi)\\ 
\Upsilon\vdash (M,\varphi) : \Sigma_{par} \Rightarrow_{match} (R_{c},\varphi_{c})\\
\varphi_{app} = \vec{\rho}(\varphi_{c})\qquad \Upsilon\vdash \varphi_{app} \Downarrow_{str} R_{app}
\end{array}}
\label{eq:strapp}
\end{equation}

	\begin{equation} 
\infer
	{\Gamma,\Upsilon\vdash \mathbf{let}~d^m~\mathbf{in}~strexp\Rightarrow_{str}(M,\mathbf{let}~\eta_{def}~\mathbf{in}~\varphi)}
	{\begin{array}{c}\Gamma,\Upsilon\vdash d^m\Rightarrow_{decl}(\eta_{def},\Gamma_{def},\Upsilon_{def})\\ \Gamma_{def},\Upsilon_{def}\vdash strexp\Rightarrow_{str}(M, \varphi)
\end{array}} 
\label{eq:letexp}
\end{equation}

\begin{equation} 
\infer{\begin{array}{c}
\Gamma,\Upsilon\vdash strexp : sigexp
 \Rightarrow_{str} ((\Sigma_{spec},R_c),\varphi_c)
\end{array}}
{\begin{array}{c}
	   \Gamma,\Upsilon,\emptyset_{sig}\vdash sigexp \Rightarrow_{sig} \Sigma_{spec} \qquad
	   \Gamma,\Upsilon\vdash strexp \Rightarrow_{str} (M_{u},\varphi_{u})\\
	   \Upsilon\vdash (M_{u},\varphi_u) : \Sigma_{spec} \Rightarrow_{match} (R_c,\varphi_{c})
\end{array}} 
\label{eq:transascription}
\end{equation}
     
 \begin{equation} 
\infer{\begin{array}{c}
\Gamma,\Upsilon \vdash strexp :> sigexp 
\Rightarrow_{str}
   ((\Sigma_{spec},\langle\Upsilon_{spec},\Upsilon\rangle),
   \varphi_{c})
\end{array}}
{\begin{array}{cc}
\Gamma,\Upsilon,\emptyset_{sig}\vdash sigexp
     \Rightarrow_{sig} \Sigma_{spec}\qquad
   \Gamma,\Upsilon\vdash strexp \Rightarrow_{str} (M_u, \varphi_u)\\
  \Upsilon\vdash (M_u,\varphi_u) : \Sigma_{spec}
  \Rightarrow_{match} (R_{c},\varphi_{c})\\
  \Upsilon, \emptyset_{ee} \vdash \Sigma_{spec} \uparrow \Upsilon_{spec}
% Does it matter whether we return the uncoerced or coerced varphi?
% The compiler returns the coerced version.  
\end{array}
}
\label{eq:opaqueascription}
\end{equation}

}
\caption{Structure expression elaboration}
\label{fig:strexpelab}
\end{figure}

