\section{A Fully Expressive Signature Language}
The ML signature language permits type definitions that may refer to general type expressions. Type expressions may involve both primitive type constructors such as $\rightarrow$ and type operators. It is the inclusion of type operators that gives the signature language much of its expressiveness. The semantics of type sharing constraints differs significant between SML90 and SML97. Type sharing constraints could be imposed on two type constructors without restriction in SML90. In SML97, the designers partitioned the semantics of type sharing into type definitions which expressed sharing between an abstract type and an arbitrary type expression, and regular type sharing constraints which can only be imposed between two flexible (or primary) types whose names must be in scope. 

A module system that permits both type definitions and type sharing constraints in signatures introduces significant new complexity. For example, whereas in Leroy's \cite{Leroy-generativity} TypModl language, which only permits SML90-style definitional type sharing constraints and no type definitions, type sharing constraints can be ``normalized'' by pushing them up the signature and eliminated by turning them into type definitions, type sharing constraints cannot be eliminated in a language that permits both type definitions and type sharing constraints.