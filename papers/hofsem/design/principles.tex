\documentclass[9pt]{sigplanconf}
\usepackage{cite} % sort citations 
\usepackage{amsmath,xspace,aux/math-envs,aux/math-cmds,aux/code,aux/proof,stmaryrd}
\newcommand{\keywordaccent}[1]{{#1}}
\newcommand{\kfun}{\keywordaccent{fun}\xspace}
\newcommand{\kmfun}{\keywordaccent{mfun}\xspace}
\newcommand{\klet}{\keywordaccent{let}\xspace}
\newcommand{\kletx}{\keywordaccent{letX}\xspace}
\newcommand{\kin}{\keywordaccent{in}\xspace}
\newcommand{\kend}{\keywordaccent{end}\xspace}
\newcommand{\kendp}{\keywordaccent{end}}
\newcommand{\kif}{\keywordaccent{if}\xspace}
\newcommand{\kmif}{\keywordaccent{mif}\xspace}
\newcommand{\kthen}{\keywordaccent{then}\xspace}
\newcommand{\kelse}{\keywordaccent{else}\xspace}
\newcommand{\kreturn}{\keywordaccent{return}\xspace}
\newcommand{\kval}{\keywordaccent{val}\xspace}
\newcommand{\kcase}{\keywordaccent{case}\xspace}
\newcommand{\kmcase}{\keywordaccent{mcase}\xspace}
\newcommand{\kof}{\keywordaccent{of}\xspace}
\newcommand{\ktrue}{\keywordaccent{true}\xspace}
\newcommand{\kfalse}{\keywordaccent{false}\xspace}
\newcommand{\kand}{\keywordaccent{and}\xspace}
\newcommand{\knil}{\keywordaccent{nil}\xspace}
\newcommand{\kNIL}{\keywordaccent{NIL}\xspace}
\newcommand{\kcons}{\keywordaccent{cons}}
\newcommand{\kCONS}{\keywordaccent{CONS}}

%%selective memoization
\newcommand{\bang}[1]{\mathcd{!}\,{#1}}
\newcommand{\nletx}{\ensuremath{\mathcd{let*}}\xspace}
\newcommand{\nletbang}{\ensuremath{\mathcd{let!}}\xspace}
\newcommand{\nmcase}{\ensuremath{\mathcd{mcase}}\xspace}
\newcommand{\nreturn}{\ensuremath{\mathcd{return}}\xspace}
\newcommand{\nmfun}{\ensuremath{\mathcd{mfun}}\xspace}
\newcommand{\bxed}[1]{{#1}\,\mathcd{box}}
\newcommand{\nletq}{\ensuremath{\mathcd{let?}}\xspace}
%%%%%%%%%%%%%%%%%%%%%%%%%%%%%%%%%%%%%%%%%%%%%%%%%%%%%%%%%%%%%%%%%%%%%%
%% mac.tex
%%
%% Umut A. Acar
%% Macros for adaptive computation paper.
%%%%%%%%%%%%%%%%%%%%%%%%%%%%%%%%%%%%%%%%%%%%%%%%%%%%%%%%%%%%%%%%%%%%%%
\newcommand{\AFL}{\textsf{AFL}\xspace}
\newcommand{\IFL}{\textsf{IFL}\xspace}
\newcommand{\MFL}{\textsf{MFL}\xspace}
\newcommand{\AMFL}{\textsf{AMFL}\xspace}
%\newcommand{\SAL}{\textsf{SAL}\xspace}
\newcommand{\SLF}{\textsf{SLf}\xspace}
\newcommand{\SLI}{\textsf{SLi}\xspace}
\newcommand{\afl}{\AFL}
\newcommand{\ifl}{\IFL}
\newcommand{\mfl}{\MFL}
\newcommand{\amfl}{\AMFL}
%\newcommand{\sal}{\SAL}
\newcommand{\slf}{\SLF}
\newcommand{\sli}{\SLI}

\newcommand{\readarrow}{\ensuremath{\Longrightarrow}}


\newcommand{\cutspace}{\vspace{-4mm}}

\newcommand{\bomb}[1]{\fbox{\mbox{\emph{\bf {#1}}}}}

%\renewcommand{\paragraph}[1]{{\bf {#1}}}
% formatting stuff
\newcommand{\codecolsep}{1ex}

%\newcommand{\todo}[1]{{\bf{[NOTE:{#1}]}}}
\newcommand{\todo}[1]{}

\newcommand{\rlabel}[1]{\mbox{\small{\bf ({#1})}}}

\newcommand{\tablerow}{\\[5ex]}
\newcommand{\tableroww}{\\[7ex]}
\newcommand{\tableline}{
\vspace*{2ex}\\
\hline\\ 
\vspace*{2ex}}

% Don't care
\newcommand{\dontcare}{\_}


%% filter and quicksort stuff
\newcommand{\ncf}[2]{C^{fil}_{\ensuremath{{#1},{#2}}}}
\newcommand{\ncq}[1]{C^{qsort}_{\ensuremath{#1}}}
\newcommand{\nuf}[3]{P^{fil}_{#1,(\ensuremath{{#2},{#3}})}}
\newcommand{\nuq}[2]{P^{qsort}_{(\ensuremath{{#1},{#2}})}}

%% shorthands
\newcommand{\ddg}{{\sc ddg}}
\newcommand{\ncpa}{change-propagation algorithm}
\newcommand{\adg}{{\sc adg}}
\newcommand{\nwrite}{\texttt{write}}
\newcommand{\nread}{\texttt{read}}
\newcommand{\nmodr}{\texttt{mod}}
\newcommand{\ttt}[1]{\texttt{#1}}
\newcommand{\nmodl}{\texttt{modl}}
\newcommand{\nnil}{\ttt{NIL}}
\newcommand{\ncons}[2]{\ttt{CONS({\ensuremath{#1},\ensuremath{#2}})}}
\newcommand{\nfilter}{\texttt{filter}}
\newcommand{\nfilterp}{\texttt{filter'}}
\newcommand{\naqsort}{\texttt{qsort'}}
\newcommand{\nqsort}{\texttt{qsort}}
\newcommand{\nqsortp}{\texttt{qsort'}}
\newcommand{\nnewMod}{\texttt{newMod}}
\newcommand{\nchange}{\texttt{change}}
\newcommand{\npropagate}{\texttt{propagate}}
\newcommand{\ndest}{\texttt{d}}
\newcommand{\ninit}{\texttt{init}}



%% Comment sth out. 
\newcommand{\out}[1] {}
\newcommand{\sthat}{\ensuremath{~|~}}

%% definitions
\newcommand{\defi}[1]{{\bfseries\itshape #1}}


% Code listings.
\newcounter{codeLineCntr}
\newcommand{\codeLine}
 {\refstepcounter{codeLineCntr}{\thecodeLineCntr}}
\newcommand{\codeLineL}[1]
 {\refstepcounter{codeLineCntr}\label{#1}{\thecodeLineCntr}}

\newenvironment{codeListing}
 {\setcounter{codeLineCntr}{0}
%  \fontsize{10}{12}
 % the first one is width the second is height
 \fontsize{9}{11}
  \vspace{-.1in}
  \ttfamily\begin{tabbing}}
  {\end{tabbing}
   \vspace{-.1in}}

\newenvironment{codeListing8}
 {\setcounter{codeLineCntr}{0}
  \fontsize{8}{10}
  \vspace{-.1in}
  \ttfamily
  \begin{tabbing}}
 {\end{tabbing}
 \vspace{-.1in}
}

\newenvironment{codeListing8h}
 {\setcounter{codeLineCntr}{0}
  \fontsize{8.5}{10.5}
  \vspace{-.1in}
  \ttfamily
  \begin{tabbing}}
 {\end{tabbing}
 \vspace{-.1in}
}


\newenvironment{codeListing9}
 {\setcounter{codeLineCntr}{0}
  \fontsize{9}{11}
  \vspace{-.1in}
  \ttfamily
  \begin{tabbing}}
 {\end{tabbing}
 \vspace{-.1in}
}

\newenvironment{codeListing10}
 {\setcounter{codeLineCntr}{0}
  \fontsize{10}{12}
  \vspace{-.1in}
  \ttfamily
  \begin{tabbing}}
 {\end{tabbing}
 \vspace{-.1in}
}


\newenvironment{codeListingNormal}
 {\setcounter{codeLineCntr}{0}
  \vspace{-.1in}
  \ttfamily
  \begin{tabbing}}
 {\end{tabbing}
 \vspace{-.1in}
}

\newcommand{\codeFrame}[1]
{\begin{center}\fbox{\parbox[t]{5in}{#1}}\end{center}
% \vspace*{-.15in}
}

\newcommand{\fixedCodeFrame}[1]
{
\vspace*{-4mm}
\begin{center}
\fbox{
\vspace*{-2mm}
\parbox[t]{0.999\columnwidth}{
\vspace*{-2mm}
#1
}
}\end{center}
\vspace*{-4mm}
}

% Footnote commands.
\newcommand{\footnotenonumber}[1]{{\def\thempfn{}\footnotetext{#1}}}

% Margin notes - use \notesfalse to turn off notes.
\setlength{\marginparwidth}{0.6in}
\reversemarginpar
\newif\ifnotes
\notestrue
\newcommand{\longnote}[1]{
  \ifnotes
    {\medskip\noindent Note:\marginpar[\hfill$\Longrightarrow$]
      {$\Longleftarrow$}{#1}\medskip}
  \fi}
\newcommand{\note}[1]{
  \ifnotes
    {\marginpar{\raggedright{\tiny #1}}}
  \fi}

% Stuff not wanted.
\newcommand{\punt}[1]{}

% Sectioning commands.
\newcommand{\subsec}[1]{\subsection{\boldmath #1 \unboldmath}}
\newcommand{\subheading}[1]{\subsubsection*{#1}}
\newcommand{\subsubheading}[1]{\paragraph*{#1}}

% Reference shorthands.
\newcommand{\spref}[1]{Modified-Store Property~\ref{sp:#1}}
\newcommand{\prefs}[2]{Properties~\ref{p:#1} and~\ref{p:#2}}
\newcommand{\pref}[1]{Property~\ref{p:#1}}


\newcommand{\partref}[1]{Part~\ref{part:#1}}
\newcommand{\chref}[1]{Chapter~\ref{ch:#1}}
\newcommand{\chreftwo}[2]{Chapters \ref{ch:#1} and~\ref{ch:#2}}
\newcommand{\chrefthree}[3]{Chapters \ref{ch:#1}, and~\ref{ch:#2}, and~\ref{ch:#3}}
\newcommand{\secref}[1]{Section~\ref{sec:#1}}
\newcommand{\subsecref}[1]{Subsection~\ref{subsec:#1}}
\newcommand{\secreftwo}[2]{Sections \ref{sec:#1} and~\ref{sec:#2}}
\newcommand{\secrefthree}[3]{Sections \ref{sec:#1},~\ref{sec:#2},~and~\ref{sec:#3}}
\newcommand{\appref}[1]{Appendix~\ref{app:#1}}
\newcommand{\figref}[1]{Figure~\ref{fig:#1}}
\newcommand{\figreftwo}[2]{Figures \ref{fig:#1} and~\ref{fig:#2}}
\newcommand{\figpageref}[1]{page~\pageref{fig:#1}}
\newcommand{\tabref}[1]{Table~\ref{tab:#1}}
\newcommand{\stref}[1]{step~\ref{step:#1}}
\newcommand{\caseref}[1]{case~\ref{case:#1}}
\newcommand{\lineref}[1]{line~\ref{line:#1}}
\newcommand{\linereftwo}[2]{lines \ref{line:#1} and~\ref{line:#2}}
\newcommand{\linerefthree}[3]{lines \ref{line:#1},~\ref{line:#2},~and~\ref{line:#3}}
\newcommand{\linerefrange}[2]{lines \ref{line:#1} through~\ref{line:#2}}
\newcommand{\thmref}[1]{Theorem~\ref{thm:#1}}
\newcommand{\thmreftwo}[2]{Theorems \ref{thm:#1} and~\ref{thm:#2}}
\newcommand{\thmpageref}[1]{page~\pageref{thm:#1}}
\newcommand{\lemref}[1]{Lemma~\ref{lem:#1}}
\newcommand{\lemreftwo}[2]{Lemmas \ref{lem:#1} and~\ref{lem:#2}}
\newcommand{\lemrefthree}[3]{Lemmas \ref{lem:#1},~\ref{lem:#2},~and~\ref{lem:#3}}
\newcommand{\lempageref}[1]{page~\pageref{lem:#1}}
\newcommand{\corref}[1]{Corollary~\ref{cor:#1}}
\newcommand{\defref}[1]{Definition~\ref{def:#1}}
\newcommand{\defreftwo}[2]{Definitions \ref{def:#1} and~\ref{def:#2}}
\newcommand{\defpageref}[1]{page~\pageref{def:#1}}
%\newcommand{\eqref}[1]{Equation~(\ref{eq:#1})}
\newcommand{\eqreftwo}[2]{Equations (\ref{eq:#1}) and~(\ref{eq:#2})}
\newcommand{\eqpageref}[1]{page~\pageref{eq:#1}}
\newcommand{\ineqref}[1]{Inequality~(\ref{ineq:#1})}
\newcommand{\ineqreftwo}[2]{Inequalities (\ref{ineq:#1}) and~(\ref{ineq:#2})}
\newcommand{\ineqpageref}[1]{page~\pageref{ineq:#1}}
\newcommand{\itemref}[1]{Item~\ref{item:#1}}
\newcommand{\itemreftwo}[2]{Item~\ref{item:#1} and~\ref{item:#2}}

% Useful shorthands.
\newcommand{\abs}[1]{\left| #1\right|}
\newcommand{\card}[1]{\left| #1\right|}
\newcommand{\norm}[1]{\left\| #1\right\|}
\newcommand{\floor}[1]{\left\lfloor #1 \right\rfloor}
\newcommand{\ceil}[1]{\left\lceil #1 \right\rceil}
  \renewcommand{\choose}[2]{{{#1}\atopwithdelims(){#2}}}
\newcommand{\ang}[1]{\langle#1\rangle}
\newcommand{\paren}[1]{\left(#1\right)}
\newcommand{\prob}[1]{\Pr\left\{ #1 \right\}}
\newcommand{\expect}[1]{\mathrm{E}\left[ #1 \right]}
\newcommand{\expectsq}[1]{\mathrm{E}^2\left[ #1 \right]}
\newcommand{\variance}[1]{\mathrm{Var}\left[ #1 \right]}
\newcommand{\twodots}{\mathinner{\ldotp\ldotp}}

% Standard number sets.
\newcommand{\reals}{{\mathrm{I}\!\mathrm{R}}}
\newcommand{\integers}{\mathbf{Z}}
\newcommand{\naturals}{{\mathrm{I}\!\mathrm{N}}}
\newcommand{\rationals}{\mathbf{Q}}
\newcommand{\complex}{\mathbf{C}}

% Special styles.
\newcommand{\proc}[1]{\ifmmode\mbox{\textsc{#1}}\else\textsc{#1}\fi}
\newcommand{\procdecl}[1]{
  \proc{#1}\vrule width0pt height0pt depth 7pt \relax}
  \newcommand{\func}[1]{\ifmmode\mathrm{#1}\else\textrm{#1}fi} %
%  Multiple cases.  
\renewcommand{\cases}[1]{\left\{
  \begin{array}{ll}#1\end{array}\right.}
  \newcommand{\cif}[1]{\mbox{if $#1$}} 

%% spacing hacks
\newcommand{\longpage}{\enlargethispage{\baselineskip}}
\newcommand{\shortpage}{\enlargethispage{-\baselineskip}}

\newcommand{\dom}[1]{\mathop{\textrm{dom}}(\ensuremath{#1})}
\newcommand{\loclabel}[1]{\mathop{\textrm{label}}(\ensuremath{#1})}


\newcommand{\machine}{\ensuremath{\mathcal{M}}\xspace}
\newcommand{\programs}{\ensuremath{\mathcal{P}}\xspace}
\newcommand{\program}{\ensuremath{{P}}\xspace}
\newcommand{\inputs}{\ensuremath{\mathcal I}}
\newcommand{\traces}{\ensuremath{\mathcal{T}}\xspace}
\newcommand{\trdist}[2]{\ensuremath{\delta({#1},{#2})}}
\newcommand{\trdistddg}[2]{\ensuremath{\delta_{DDG}({#1},{#2})}}
\newcommand{\trdistmemo}[2]{\ensuremath{\delta_{Memo}({#1},{#2})}}
\newcommand{\strdistmono}[2]{\ensuremath{\delta^s_{Monotone}({#1},{#2})}}
\newcommand{\symdistmono}[2]{\ensuremath{\delta^s_{Monotone}({#1},{#2})}}
\newcommand{\symdist}[2]{\ensuremath{\delta^s_{Monotone}({#1},{#2})}}
\newcommand{\trdistmono}[2]{\ensuremath{\delta_{Monotone}({#1},{#2})}}
\newcommand{\trdistint}[2]{\ensuremath{\delta^{\mbox{min}}({#1},{#2})}}
\newcommand{\trleq}[2]{\ensuremath{\alpha({#1},{#2})}}
\newcommand{\locs}[1]{\ensuremath{\textrm{locs}({#1})}}

\newcommand{\cogddg}[2]{\ensuremath{\mathcal{C}_{DDG}({#1},{#2})}}
\newcommand{\cogmemo}[2]{\ensuremath{\mathcal{C}_{Memo}({#1},{#2})}}
\newcommand{\cogmono}[2]{\ensuremath{\mathcal{C}_{Monotone}({#1},{#2})}}
\newcommand{\shared}[2]{\ensuremath{\mathcal{S}({#1},{#2})}}


%\newcommand{\tag}[1]{\ensuremath{\mbox{\emph{tag}}({#1})}}
\renewcommand{\func}[1]{\ensuremath{\mbox{\emph{fun}}({#1})}}
\newcommand{\args}[1]{\ensuremath{\mbox{\emph{args}}({#1})}}
\newcommand{\reads}[1]{\ensuremath{\mbox{\emph{reads}}({#1})}}
\newcommand{\returns}[1]{\ensuremath{\mbox{\emph{returns}}({#1})}}
\newcommand{\w}[1]{\ensuremath{\mbox{\emph{w}}({#1})}}
\newcommand{\mfinger}{\ensuremath{\Phi}\xspace}


\newcommand{\iadd}{\texttt{add}\xspace}
\newcommand{\iif}{\texttt{if}\xspace}
\newcommand{\ithen}{\texttt{then}\xspace}
\newcommand{\imul}{\texttt{mul}\xspace}
\newcommand{\imove}{\texttt{move}\xspace}
\newcommand{\ialloc}{\texttt{alloc}\xspace}
\newcommand{\iwrite}{\texttt{write}\xspace}
\newcommand{\ichange}{\texttt{change}\xspace}
\newcommand{\iread}{\texttt{read}\xspace}
\newcommand{\ioffset}{\texttt{offset}\xspace}
\newcommand{\icall}{\texttt{call}\xspace}
\newcommand{\ijump}{\texttt{jump}\xspace}
\newcommand{\iccall}{\texttt{ccall}\xspace}
\newcommand{\ireturn}{\texttt{return}\xspace}

\newcommand{\istart}{\texttt{start}\xspace}
\newcommand{\ihalt}{\texttt{halt}\xspace}

\newcommand{\eg}{\textit{e.g.}}
\newcommand{\ie}{\textit{i.e.}}

\newcommand{\structure}{\ensuremath{\mathbf{structure}}}
\newcommand{\functor}{\ensuremath{\mathbf{functor}}}
\newcommand{\struct}[1]{\ensuremath{\mathbf{struct}\{#1\}}}
\newcommand{\inx}{\ensuremath{\mathsf{in}}}
\newcommand{\en}{\ensuremath{\mathbf{end}}}
\newcommand{\local}{\ensuremath{\mathbf{local}}}
\newcommand{\letx}{\ensuremath{\mathsf{let}}}
\newcommand{\letin}[2]{\ensuremath{\mathbf{let}~#1~\mathbf{in}~#2}}
\newcommand{\where}{\ensuremath{\mathbf{where}}}
\newcommand{\sig}[1]{\ensuremath{\mathbf{sig}\{#1\}}}
\newcommand{\type}{\ensuremath{\mathbf{type}}}
\newcommand{\eqtype}{\ensuremath{\mathbf{eqtype}}}
\newcommand{\val}{\ensuremath{\mathsf{val}}}
\newcommand{\sharing}{\ensuremath{\mathbf{sharing}}}
\newcommand{\incl}{\ensuremath{\mathbf{include}}}
\newcommand{\exception}{\ensuremath{\mathbf{exception}}}
\newcommand{\Int}{\ensuremath{\mathbf{int}}}
\newcommand{\signature}{\ensuremath{\mathbf{signature}}}
\newcommand{\open}{\ensuremath{\mathbf{open}}}
\newcommand{\newx}{\ensuremath{\mathsf{new}}}
\newcommand{\getx}{\ensuremath{\mathbf{get}}}
\newcommand{\EP}{\ensuremath{\mathsf{EP}}}
\newcommand{\inj}{\ensuremath{\mathsf{inj}}}
\newcommand{\env}{\ensuremath{\mathsf{env}}}
\newcommand{\an}[1]{\ensuremath{\langle\langle #1 \rangle\rangle}}
\newcommand{\dt}{\ensuremath{\mathbf{datatype}}}
\newcommand{\letins}[2]{\ensuremath{\mathsf{let}~#1~\mathsf{in}~#2}}
\newcommand{\tycon}{\ensuremath{\mathsf{tycon}}}
\newcommand{\sel}[1]{\ensuremath{\mathsf{sel}(#1)}}
\newcommand{\coerce}{\ensuremath{\mathsf{coerce}}}
\newcommand{\lettyc}[2]{\ensuremath{\mathsf{let~tyc}~#1~\mathsf{in}~#2}}
\newcommand{\lettycline}[2]{\ensuremath{\begin{array}{l}
\mathsf{let~tyc}~#1\\\mathsf{in}~#2\end{array}}}
\newcommand{\tyc}{\ensuremath{\mathsf{tyc}}}
\newcommand{\inlet}{\ensuremath{\mathsf{in~let}}}
\newcommand{\inv}{\ensuremath{\mathsf{inv}}}
\newcommand{\entpath}{\ensuremath{\mathsf{entpath}}}
\newcommand{\typ}{\ensuremath{\mathsf{typ}}}

\newcommand{\Uclo}{\ensuremath{\Upsilon_{c}}}
\newcommand{\Uloc}{\ensuremath{\Upsilon_{l}}}

\bibliographystyle{plain}

\begin{document}
	\permission{\copyright ACM, 2008. http://doi.acm.org/10.1145/nnnnnn.nnnnnn}
	%\conferenceinfo{ML'07,} {October 5, 2007, Freiburg, Germany.}
	\CopyrightYear{2008}
	\copyrightdata{978-1-59593-676-9/07/0010}
	
\title{Principles for an Extensible Module System}
\authorinfo{}{University of Chicago}{}
\maketitle

\begin{abstract}
The ML module system has inspired a series of formalism describing the workings and mechanisms of variations. These formalisms are quite varied and run the gamut from Leroy's presentation based on syntactic mechanisms alone, to Harper-Lillibridge and Dreyer-Crary-Harper's type-theoretic approach, to the elaboration semantics approach as represented by the Definition. 	
\end{abstract}

\category{D.3.1}{Programming Languages}{Formal Definitions and Theory}
\category{D.3.3}{Programming Languages}{Language Constructs and Features}[Abstract data types, Modules] 
\category{F.3.3}{Logics and Meanings of Programs}{Studies of Program Constructs}[Type structure]

\terms 
Languages, Theory

\keywords
modularity, translucent sum, singleton types, type theory, elaboration semantics, abstract data types, functors, generativity

{
\section{Outline}
\begin{enumerate}
	\item ML module system is powerful because:
	\begin{enumerate}
		\item functors can be typechecked independent of functor applications
		\item enforces type abstraction by opaque ascription
		\item hierarchical modularity
		\item higher-order functors
		\item type definitions
	\end{enumerate}
	\item Background
	\begin{enumerate}
		\item Modular module\cite{leroy00} provides a syntactic model 
		\begin{enumerate}
			\item Functorized over core language syntax and typechecker
			\item Core language must be aware of paths
			\item functor can only be applied to rooted path (can be extended to anonymous argument module if parameter dependency in the result signature can be reduced away)
			\item Because notion of core language (especially types) is abstract, the system is not easily extensible to richly typed languages. 
			\item Does not support shadowing in core language declarations. Shadowing would fundamentally break syntactic notion of opaque ascription. 
		\end{enumerate}
		\item Definition \cite{mthm97} provides a semantic approach
		\item Harper-Lillibridge\cite{lillibridge94}, Dreyer-Crary-Harper\cite{dhc03} provide a type theoretic model
		\begin{enumerate}
			\item purity
			\item totality/partiality
			\item static vs. dynamic effects; strong and weak sealing
			\item comparability and projectibility
		\end{enumerate}
		\item Harper-Pierce \cite{ATTAPL} provides high-level design principles and issues
		\begin{enumerate}
			\item sharing type constraints cannot always be expanded out as claimed by Pierce-Harper. Symmetric constraints are necessary in the absence of an ability for type definitions to reference their enclosing signatures and thus specifications that come after the type definition in question. 
			\item determinacy versus static/dynamic
			\item first-class modules
			\item In Stone, full signatures are called ``very precise'' versus abstract; he argues that the avoidance problem is one reason why translucent sums do not have full signatures (most precise); also that restricting all programs to Leroy's named form guarantees the existence of ``most-specific'' interfaces. This terminology leads to the term ``natural interface'' -- the most precise interface that can be computed without appealing to strengthening (M-SELF). 
		\end{enumerate}
		\item Treatments of first-class modules
		\begin{enumerate}
			\item Harper-Lillibridge
			\item Russo \cite{russo00}
			\item Dreyer-Crary-Harper
		\end{enumerate}
		\item Leroy \cite{Leroy-generativity} gives a module system where type generativity and SML90-style definitional sharing = path equivalence + A-normalization (for functor applications) + S-normalization (a consolidation of sharing constraints). Leroy shows that a module system with generative datatypes (but no constructors), sharing between type paths, and abstract type specifications can be expressed in terms of a module system with generative datatypes and manifest types. Leroy's simplified module system does not include value specifications and datatype constructors both of which can constrain the order in which specifications must be written in and therefore result in situations where sharing constraints cannot be in general reduced to manifest types. 
	\end{enumerate}
	\item Design space: Propagation of types 
	\begin{enumerate}
		\item Primarily through functor application
		\item Shao fully transparent signature calculus
		\item SML90 - only explicit sharing equations and structure (identity) sharing 
		\item SML93 - plus definitional sharing
		\item SML97 - plus where type and definitional specs; structure identity sharing eliminated
		\item existential types, dependent sums, translucent sums, singleton kinds/signatures, Shao flexroot
	\end{enumerate}
	\item Key problem in modularity: modularizing types and their interpretations
	\begin{enumerate}
		\item Example: Symbol table versus Ord
		\item type class an imperfect solution because limits interpretation to a single instance and the type to only generative nonstructured types (i.e., datatypes)
		\item applicative functors imperfect because they permit too much sharing
		\item constructors of higher kind
		\item relationship to expression problem?
	\end{enumerate}
	\item True higher-order functors (MacQueen-Tofte 94) -- {\bf full transparency}
	\begin{enumerate}
		\item functor parameter type information should be propagated through functor application; in other words, should transparent signature matching semantics carry over to higher-order functor setting?
		\item Shao \cite{shao98} cites optimal propagation of types (ensuring that inlined and separately compiled modules receive the same typing) as a benefit of full transparency
		\item \begin{verbatim}
			signature FPS = sig type t end
			signature FRS = sig type t end
			structure M = struct type t = int end
			functor F(X: FPS): FRS = 
			  struct type t = X.t end
			functor G(functor F(X: FPS):FRS) = 
			  struct structure R = F(M) end
		\end{verbatim}
		\item \verb|t = int| should propagate through the HO functor application to \verb|G(F).R|
		\item type definitions in signatures insufficient because not all parameter functor F's will propagate this type information
		\item MacQueen-Tofte 94 appears to be the only module system that accounts for this class of type sharing
		\item Unfortunately, this feature apparently conflicts with separate compilation as noted by Dreyer-Crary-Harper \cite{dhc03}
		\item Primary criticism of stamp-based operational semantics is the difficulty of extending such a semantics
		\item The MT94 semantics also stratifies the stamp computation in the peculiar way 
		\item Shao \cite{shao99} offers an alternative example for fully transparent higher-order functors\\
		\begin{verbatim}
			signature S = sig type t val x : t end
			funsig FS = fsig (X:S): S
			structure S = struct type t=int val x=1 end
			functor F1(X:S) = struct type t=X.t val x=X.x end
			functor F2(X:S) = struct type t=int val x=1 end
			functor APPS(F:FS) = F(S)
			structure R = 
			  struct structure R1 = APPS(F1)
				 structure R2 = APPS(F2)
				 val res = (R1.x = R2.x)
			  end
		\end{verbatim}
		\item Shao offers a signature language based on gathering all flexible components in a higher-order type constructor that can be applied to obtain the fully transparent signature at a later point. The resultant signature language superficially resembles applicative functors. However, applications in the signature language must be on paths. Consequently, it does not address fully transparency in the general case. 
		\item Shao \cite{shao98} extends MacQueen-Tofte fully transparency modules with support for type definitions, type sharing (normalized into type definitions), and hidden module components. 
	\end{enumerate}
	\item Extending modules to support signature bindings as components
	\begin{enumerate}
		\item In ML modules, structures can be arranged in a hierarchy. This feature enables flexible namespace management. In contrast, signatures cannot be arranged in such a hierarchy. In the ML module system, signatures must be defined at the top-level and can never be enclosed in any other signature or module. For complex hierarchies such the SML/NJ's Control module that contains layers of submodules, the corresponding signature CONTROL and the signatures of the submodules PRINT and ELAB are related only incidentally by occurrence in structure specifications in CONTROL. This shortcoming in the signature language unnecessarily pollutes the signature namespace and complicates browsing through and working with highly nested hierarchies. It would be desirable to permit (transparent) signature specifications within signatures. For added flexibility and perhaps increased expressiveness, it may be useful permit signature definitions within structures and functors. Furthermore, in order for modules to match these signatures enriched with signature specifications, modules must permit corresponding signature definitions. 
		\item Leroy \cite{leroy94} offers an example that introducing signature bindings into structures would add polymorphic modules and F$_\omega$-like type operators. In particular, he offers 
		\begin{verbatim}
			functor(x: sig signature X end) (m{x.X/X})
		\end{verbatim}
		as an encoding of $\Lambda X.m$
		\item Swasey \cite{swasey06} and Leroy \cite{leroy94} both cite Harper Lillibridge's proof of the undecidability of $\lambda^{\rightarrow,\exists,\exists=}$ as reason for their skepticism that such a feature can be added to a module system without breaking type-checking
		\item Harper and Lillibridge's proof establishes that in a type calculus with opaque and binary sums, subtyping is undecidable in the presence of a Forget rule that forgets transparency. The example they use is a subtyping relation on transparent and opaque sums containing a type constructor with a contravariant subtyping rule such as $T\rightarrow \alpha'$
		\item Adding signature components does not necessarily provide parametric polymorphism in the style of System F because functor application uses coercive subtyping 
	\end{enumerate}
	\item Polymorphism and modules
	\begin{enumerate}
		\item Interaction with Hindley-Milner polymorphism in core language
		\item Moscow ML's first-class modules provides first-class polymorphism
		\item Example: polymorphic data structures, continuation monad \cite{kahrs94}
	\end{enumerate}
	\item Claim: Instantiation is an analysis process that can detect cyclic sharing and other behaviors that may result in an unrealizable signature (though certainly not all behaviors)
\end{enumerate}	

The key observation in Leroy's syntactic presentation of a module system is that we can check a sense of type equality by comparing rooted paths that uniquely determine type identity. Unfortunately, this technique suffers from the inability to support core and module language level shadowing of bindings. 
 
True separate compilation poses an interesting challenge in the ML module system. In order to have true separate compilation, the surface signature language must be able to express the full signature of all structures and functors. Even in a module language that only supports first-order functors, this requirement proves to be a problem because the signature language would be unable to express generative types in the body of a functor. Generative types in the body of a functor do not have externally expressible names prior to functor application. 

\begin{figure}
	\tiny
\begin{tabular}{|l|l|l|l|l|l|l|l|}
	\hline
System & higher-order & first-class & sep comp & rec mod & app fct & gen & phase sep\\
	\hline
	HL \cite{lillibridge94} & & \checkmark & & & & \\
	\hline
	Leroy 94 & & & & & & \\
	\hline
	Russo \cite{russo01} & & \checkmark & & \checkmark & & \\
	\hline
	DCH (\cite{dhc03}) & & & & & & \\
	\hline
	RTG (\cite{dreyer05}) & & & & & & \\
	\hline
	$\lambda^{\llparenthesis~\rrparenthesis}$ (\cite{ATTAPL}) & & & & & & \\
	\hline
	$\lambda^S$ (\cite{ATTAPL}) & & N & & N & N & N \\
	\hline
	MT94 (\cite{mt94}) & \checkmark & & & & & \checkmark & \checkmark \\
	\hline
\end{tabular}
\end{figure}
%Would a completely nondependent module calculus work? More to the point, signature specs may depend on module terms, but is there a way to avoid this kind of dependence? 

%!TEX root = ../principles.tex
\begin{figure}
\centering
\fixedCodeFrame{
\small
\[
\setlength{\tabcolsep}{0ex}
\renewcommand{\arraystretch}{1.1}
\begin{array}{rcll}
	d & ::= &\signature~s=\sig{\overline{spec}}\\
	  & ~~\bnfalt& \local~\overline{d}~\inx~\overline{d}~\en\\
	  & ~~\bnfalt&~ld\\
	ld & ::= & \structure~X~=~m\\
	   & ~~\bnfalt&\functor~F\overline{(X:sigexp)}~cnstr = m\\
	   & ~~\bnfalt&\val~x=e~\bnfalt~\type~\overline{\alpha}~t=\tau\\
	   & ~~\bnfalt&\local~\overline{ld}~\inx~\overline{ld}~\en\\
	   & ~~\bnfalt&\open~\overline{q}\\
	m & ::= & q~\bnfalt~\struct{\overline{ld}}~\bnfalt~q~arg~\bnfalt~\letx~\overline{d}~\inx~m~\en\\
	  & ~~\bnfalt~&~m:sigexp~\bnfalt~m:>sigexp\\
	arg & ::= & ( \overline{d} )~\overline{arg}~\bnfalt~(m)~arg~\bnfalt~(m)~\bnfalt~(\overline{d})\\
	cnstr & ::= & :~sigexp~\bnfalt~:>~sigexp\\
	sigexp & ::= & s~\bnfalt~\sig{\overline{spec}}~\bnfalt~sigexp~\where~\type~\overline{\alpha}~q=\tau\\
	spec & ::= & \structure~q:sigexp[=q]\\
	     & ~~\bnfalt &\functor~q~\overline{(X:sigexp)} : sigexp\\
	     & ~~\bnfalt &\type~\overline{\alpha}~t[=\tau]~\bnfalt~\val~x:\tau~\bnfalt~\sharing~\type~p=p\\
	     & ~~\bnfalt &\sharing~p=p~\bnfalt~\exception~exn\\
	     & ~~\bnfalt & \eqtype~\overline{\alpha}~t[=\tau]\\
	\tau & ::= & \tau\rightarrow\tau~\bnfalt~\Int~\bnfalt~t\\
\end{array}
\]
}
The overline notation indicates a vector of the components. 
\caption{Module surface language: Design caveat -- In SML/NJ the AST permits signature declarations within structures. The parser, however, does not support this. The surface language follows the parser. The implementation AST has an Abstract Structure form, but the parser does not seem to ever produce it. }
\label{fig:pureml}
\end{figure}
%!TEX root = ../principles.tex
\begin{figure}
\centering
\fixedCodeFrame{
\small
\[
\setlength{\tabcolsep}{0ex}
\renewcommand{\arraystretch}{1.1}
\begin{array}{ll}
	\infer[\rlabel{local}]{\begin{array}{l}\Gamma\vdash\local~\overline{d_1}~\inx~\overline{d_2}~\en\\
	\Rightarrow(\underline{\local}~\underline{\overline{d_1}}~\inx~\underline{\overline{d_2}}~\en,entdec,\Gamma_2,\Delta)\end{array}}
		{\Gamma\vdash\overline{d_1}\Rightarrow(\underline{\overline{d_1}},entdec_1,\Gamma_1,\Delta_1)\\ 
 \Gamma\oplus\Gamma_1\vdash\overline{d_2}\Rightarrow(\underline{\overline{d_2}},entdec_2,\Gamma_2,\Delta_2)}
\\
\qquad
\\
\infer[\rlabel{type}]{\Gamma\vdash\type~\overline{\alpha}~t=\tau\Rightarrow(,entdec,\Gamma,\Delta)}
{\strut}
\\
\qquad
\\
\infer[\rlabel{sig}]{\Gamma\vdash\sig{\overline{spec}}\Rightarrow(\sig{\overline{spec}},\bullet,\Gamma',\emptyset)}{\strut}
\\
\qquad
\\
\infer[\rlabel{structure}]{\Gamma\vdash\struct{\overline{ld}}\Rightarrow\struct{\overline{ld}}}{}
\end{array}
\]
}
The underline notation denotes the elaborated version of the nonterminal, i.e., the explicitly typed and elaborated variant.
\nocaptionrule
\caption{Static Semantics}
\label{fig:statsem}
\end{figure}
\input{figs/fig-rlzn}
\input{figs/fig-evalent}
%!TEX root = ../principles.tex
\begin{figure}
\centering
\fixedCodeFrame{
\small
~\\[2mm]
\fbox{$\Upsilon\vdash\theta\Downarrow_{fct}\psi$}
\begin{equation}
\infer{\Upsilon\vdash\vec{\rho}\Downarrow_{fct}\Upsilon(\vec{\rho})}
{\strut} 
\label{eq:fctentpath}
\end{equation}

\begin{equation}
\infer{\Upsilon\vdash\lambda\rho.\varphi
\Downarrow_{fct}\langle
  \lambda\rho.\varphi;\Upsilon\rangle}
{\strut}	
\label{eq:fctentlam}
\end{equation}

\begin{equation}
\infer{\Upsilon\vdash\lambda\rho.\Sigma \Downarrow_{fct} \langle
  \lambda\rho.\Sigma; \Upsilon
  \rangle}
{\strut}
\label{eq:fctentformal}
\end{equation}

}
\caption{Functor entity evaluation}
\label{fig:fctenteval}
\end{figure}
 %!TEX root = ../principles.tex
\begin{figure}
\centering

\fixedCodeFrame{
\small
~\\[0.5mm]
\fbox{$\Upsilon\vdash \eta \Downarrow_{decl} \Upsilon'$}
\begin{equation}
	\infer{\Upsilon\vdash\bullet\Downarrow_{decl} \emptyset_{ee}}
        {\strut}
\label{eq:entdecempty}
\end{equation}

\begin{equation}
	 \infer{\Upsilon\vdash\rho=_{str}\varphi,\eta\Downarrow_{decl}[\rho\mapsto R]\Upsilon'}
	{\Upsilon\vdash\varphi\Downarrow_{str}R\qquad 
          \Upsilon[\rho\mapsto R]\vdash\eta\Downarrow_{decl}
          \Upsilon'} 
\label{eq:entdecstr}
\end{equation}
	
\begin{equation}
	\infer{\Upsilon\vdash\rho=_{fct}\theta,\eta\Downarrow_{decl}[\rho\mapsto\psi]\Upsilon'}
	  {\Upsilon\vdash\theta\Downarrow_{fct}\psi\qquad 
            \Upsilon[\rho\mapsto\psi]\vdash\eta\Downarrow_{decl}
            \Upsilon'}
\label{eq:entdecfct}
\end{equation}

\begin{equation}
	\infer{\Upsilon\vdash\rho=_{tyc}\newx(n),\eta\Downarrow_{decl}
          [\rho\mapsto\tau^n]\Upsilon'}
	    {\begin{array}{c}
                \Upsilon[\rho\mapsto\tau^n]\vdash\eta\Downarrow_{decl}\Upsilon'\qquad
            (\tau\textrm{ is fresh in }\Upsilon)
          \end{array}} 
\label{eq:entdecnew}
\end{equation}

\begin{equation}
         \infer{\Upsilon\vdash[\rho=_{def}\mathbb{C}^\lambda],\eta\Downarrow_{decl}
           [\rho\mapsto\mathbb{C}^\lambda]\Upsilon'}
         {\Upsilon[\rho\mapsto\mathbb{C}^\lambda]\vdash\eta\Downarrow_{decl}
           \Upsilon'}
         \label{eq:entdectypedef}
\end{equation}

\begin{equation}
         \infer{\Upsilon\vdash[\rho=_{def}\vec{\rho}],\eta\Downarrow_{decl}
           [\rho\mapsto\Upsilon(\vec{\rho})]\Upsilon'}
         {\Upsilon[\rho\mapsto\Upsilon(\vec{\rho})]\vdash \eta
           \Downarrow_{decl} \Upsilon'}
         \label{eq:entdecalias}
\end{equation}
}
\caption{Entity declaration semantics}
\label{fig:entdecsems}
\end{figure}
%!TEX root = ../principles.tex
\begin{figure}
\centering
\fixedCodeFrame{
\small
\setlength{\tabcolsep}{0ex}
\renewcommand{\arraystretch}{1.1}
~\\[2mm]
\fbox{$\Gamma,\Upsilon,\Sigma\vdash sigexp \Rightarrow_{sig} \Sigma'$}
	
\begin{equation}
\infer{\Gamma,\Upsilon,\Sigma\vdash x \Rightarrow_{sig} \Gamma(x)}
{\strut} 
\label{eq:emptysig}
\end{equation}

\begin{equation}
\infer{\Gamma,\Upsilon,\Sigma\vdash sigexp~\textbf{where type}~p = C^\lambda\Rightarrow_{sig}\mathsf{rebind}(p,\mathbb{C}^\lambda,\Sigma')}
	{\begin{array}{c}
	  \Gamma,\Upsilon,\Sigma\vdash sigexp\Rightarrow_{sig}\Sigma'\qquad
	  \Sigma'(p) = (\rho,n)\\ \Gamma,\Upsilon\vdash C^\lambda \Rightarrow_{tyc} \mathfrak{C}^\lambda\qquad 
          \Gamma,\Upsilon,\Sigma\Sigma'\vdash C^\lambda \searrow^{tyc} \mathfrak{C}^\lambda \\
          \Upsilon\vdash \mathfrak{C}^\lambda \searrow^{tyc} 
          \mathbb{C}^\lambda\qquad |\mathbb{C}^\lambda|=n
	 \end{array}} 
\label{eq:wheretype}
\end{equation}

\begin{equation}
\infer{\Gamma,\Upsilon,\Sigma\vdash \textbf{sig } specs \textbf{ end} \Rightarrow_{sig} \Sigma'}
	{\begin{array}{c}
	\Gamma,\Upsilon,\Sigma\vdash specs \Rightarrow_{specs} \Sigma'
	\end{array}} 
\label{eq:sigspecs}
\end{equation}

$\mathsf{rebind}(p,\mathbb{C}^\lambda,\Sigma)$ replaces the
binding $[x\mapsto (\rho,n)]$ in $\Sigma$ with $[x \mapsto 
\mathbb{C}^\lambda]$ where $p$ ends in $x$. 

\fbox{$\Gamma,\Upsilon,\Sigma\vdash specs \Rightarrow_{specs} \Sigma$}
\begin{equation}
\infer{\Gamma,\Upsilon,\Sigma\vdash \emptyset_{specs} \Rightarrow_{specs} \emptyset_{sig}}
{\strut}
\end{equation}

\begin{equation}
\infer{\Gamma,\Upsilon,\Sigma\vdash spec,specs \Rightarrow_{specs} \Sigma'\Sigma''}
{\Gamma,\Upsilon,\Sigma\vdash spec \Rightarrow_{spec} \Sigma' \qquad \Gamma,\Upsilon,\Sigma\Sigma'\vdash specs \Rightarrow_{specs} \Sigma''}
\label{eq:specs}
\end{equation}

}
\caption{Signature elaboration}
\label{fig:elabsig}
\end{figure}

\begin{figure}
	\centering
	\fixedCodeFrame{
	\small
	~\\[2mm]
	\fbox{$\Gamma, \Upsilon, \Sigma \vdash spec \Rightarrow_{spec} \Sigma'$}
\begin{equation}
\infer{\Gamma,\Upsilon,\Sigma\vdash\textbf{type
                  }\vec{\alpha}~t\Rightarrow_{spec} [t\mapsto (\rho,|\vec{\alpha}|)]}
{(\rho\textrm{ fresh in }\Gamma\textrm{ and
                  }\Upsilon)} 
\end{equation}
% Do we need to extend the static environment during elaboration of
% the subsequent specs? The only reason we might need to is to support
% relativization. I think we have to. 
\begin{equation}
\infer{\Gamma,\Upsilon,\Sigma\vdash\textbf{type }t=C^\lambda\Rightarrow_{spec} [t\mapsto \mathbb{C}^\lambda]}
{\Gamma,\Upsilon\vdash C^\lambda \Rightarrow_{tyc} \mathfrak{C}^\lambda\qquad \Upsilon,\Sigma\vdash \mathfrak{C}^\lambda \searrow^{tyc}
 \mathbb{C}^\lambda}
\label{eq:typedefspec}
\end{equation}

\begin{equation}
              \infer{\Gamma,\Upsilon,\Sigma\vdash\textbf{val }x:T
                 \Rightarrow_{spec} [x\mapsto\mathbb{T}]}
{\Gamma,\Upsilon\vdash T \Rightarrow_{te} \mathfrak{T}\qquad \Upsilon,\Sigma\vdash \mathfrak{T} \searrow^{tyc} \mathbb{T}}
\label{eq:valspec}
\end{equation}

\begin{equation}
\infer{\begin{array}{c}
    \Gamma,\Upsilon,\Sigma\vdash\textbf{structure }x :
                  sigexp
                  \Rightarrow_{spec} [x\mapsto
                  (\rho,\Sigma')]
\end{array}}
		{\begin{array}{c}
                    \Gamma,\Upsilon,\Sigma\vdash sigexp
                    \Rightarrow_{sig}
                    \Sigma'\qquad
                    (\rho~\textrm{fresh in }\Gamma\textrm{ and
                    }\Upsilon) 
              \end{array}}
\end{equation}

\begin{equation}
\infer{\begin{array}{c}
\Gamma,\Upsilon,\Sigma\vdash\textbf{functor }f(X: sigexp_1):sigexp_2\\
   \Rightarrow_{spec} [f\mapsto(\rho,\Pi\rho_x:\Sigma_1.\Sigma_2)]
\end{array}}
		{\begin{array}{c}
\Gamma,\Upsilon,\Sigma\vdash sigexp_1 \Rightarrow_{sig}
\Sigma_1 \qquad
(\rho_x\textrm{ and }\rho~\textrm{fresh in }\Gamma\textrm{ and }\Upsilon )\\
\Gamma,\Upsilon,\Sigma[X\mapsto(\rho_x,\Sigma_1)]\vdash
sigexp_2 \Rightarrow_{sig} \Sigma_2\\
% \qquad \psi=\langle\lambda\rho_x.\Sigma_2 ;
%  \Upsilon\rangle % [4/8/10] Is this the correct closure environment? Can \Sigma_2 mention local entities? Yes, but those are local, therefore, they should be interpreted locally and not by the closure, which only interpret nonlocal entities.  
\end{array}}
\label{eq:fctspec}
\end{equation}

	}
	\caption{Signature spec elaboration}
	\label{fig:elabspec}
\end{figure}
%!TEX root = ../principles.tex
\begin{figure}
\centering
\fixedCodeFrame{
\small
\setlength{\tabcolsep}{0ex}
\renewcommand{\arraystretch}{1.1}
~\\[2mm]
\fbox{$\Gamma,\Upsilon\vdash d^m \Rightarrow_{decl} (\eta, \Gamma', \Upsilon')$}

	\begin{equation} 
          \infer{\Gamma,\Upsilon\vdash \circ
            \Rightarrow_{decl} (\bullet, \emptyset_{se},
            \emptyset_{ee})}{\strut}  
          \label{eq:emptydecl}
        \end{equation}

        \begin{equation}
          \infer{\Gamma,\Upsilon\vdash \mathbf{val}~x=e,d^m
            \Rightarrow_{decl} (\eta, [x\mapsto\mathfrak{T}]\Gamma', \Upsilon')}
          {\Gamma \vdash e \Rightarrow_{core} \mathfrak{T} \qquad
            \Gamma[x\mapsto\mathfrak{T}], \Upsilon \vdash d^m \Rightarrow_{decl}
            (\eta, \Gamma', \Upsilon')}
          \label{eq:valdecl}
        \end{equation}

	\begin{equation} 
          \infer{\begin{array}{l} 
              \Gamma,\Upsilon\vdash \mathbf{type}~t=C^\lambda,d^m
              \Rightarrow_{decl}(\eta,[t\mapsto \mathfrak{C}^\lambda]\Gamma',\Upsilon')
	\end{array}}
	{\begin{array}{c}
            \Gamma,\Upsilon \vdash C^\lambda \Rightarrow_{tyc} \mathfrak{C}^\lambda\qquad
            \Gamma[t\mapsto \mathfrak{C}^\lambda],\Upsilon\vdash
            d^m\Rightarrow_{decl}(\eta,\Gamma',
            \Upsilon')
          \end{array}} 
        \label{eq:typedefdecl}
      \end{equation}

        \begin{equation} 
       \infer{\begin{array}{c}
           \Gamma,\Upsilon\vdash
         \mathbf{datatype}~\vec{\alpha}~t,d^m\\
         \Rightarrow_{decl}([\rho_t =_{tyc} \newx(n)]\eta, [t\mapsto
         \tau^n]\Gamma', [\rho_t\mapsto \tau^n]\Upsilon')
       \end{array}}
{\begin{array}{c}
    n=|\vec{\alpha}|\qquad
    \Gamma[t\mapsto\tau^n],\Upsilon[\rho_t\mapsto \tau^n]\vdash d^m \Rightarrow_{decl} (\eta, \Gamma',
    \Upsilon')\\ (\rho_t\textrm{ and }\tau\textrm{ are fresh})
\end{array}}
      \label{eq:dtdecl}
        \end{equation}

\begin{equation} 
          \infer{\begin{array}{c}
              \Gamma,\Upsilon\vdash \mathbf{structure}~X=strexp,d^m\\
  \Rightarrow_{decl} ([\rho=_{str}\varphi]\eta, [X\mapsto (\rho, M)]\Gamma',
  [\rho\mapsto R]\Upsilon')
\end{array}}
	{\begin{array}{c}
\Gamma,\Upsilon\vdash strexp\Rightarrow_{str} (M, \varphi)\qquad 
M = (\Sigma,R)\qquad (\rho~\textrm{fresh})\\
\Gamma[X\mapsto (\rho, M)],\Upsilon[\rho\mapsto R]\vdash d^m\Rightarrow_{decl}(\eta, \Gamma', \Upsilon')
	\end{array}} 
      \label{eq:strdecl}
\end{equation}


	\begin{equation} 
          \infer{\begin{array}{c}
              \Gamma,\Upsilon\vdash
              \mathbf{functor}~f(X:sigexp)=strexp,d^m \\
              \Rightarrow_{decl} ([\rho=_{fct}\theta]\eta, [f\mapsto(\rho,(\Pi\rho_x:\Sigma_x.\Sigma_{res},\psi))]\Gamma',
              [\rho\mapsto\psi]\Upsilon')
            \end{array}}
	        {\begin{array}{c} 
                    \Gamma,\Upsilon, \emptyset_{sig} \vdash
                    sigexp\Rightarrow_{sig} \Sigma_x\qquad
                    \Upsilon,\emptyset_{ee}\vdash \Sigma_x \uparrow
                    \Upsilon_x \\
                    R_x = \langle \Upsilon_x,\Upsilon \rangle\\
                    \Gamma[X\mapsto(\rho_x, (\Sigma_x,
                   R_x))],\Upsilon[\rho_x\mapsto R_x]\vdash
                    strexp\Rightarrow_{str}((\Sigma_{res},\_),\varphi)\\
                    % \Upsilon_\Delta is out of scope at the
                    % declaration level, so it is dropped. 
        \theta =
        \lambda\rho_x.\varphi\qquad \psi =
        \langle\theta;\Upsilon\rangle\\
	\Gamma [f\mapsto(\rho,(\Pi
        \rho_x:\Sigma_x.\Sigma_{res},\psi))],
        \Upsilon[\rho\mapsto\psi]\vdash
        d^m \Rightarrow_{decl}(\eta,\Gamma',\Upsilon')\\
        % No need for extending entity environment because \rho_F
        % won't be looked up
	(\rho_x,\rho~\textrm{fresh})\\
        %\Gamma, \Upsilon \vdash \emptyset_{se}\gamma \hookrightarrow (M_{ext}, \eta_{ext})
	         \end{array}} 
               \label{eq:fctdecl}
        \end{equation}
}
\vspace{1em}
The resultant $\Upsilon$ must be the local entity environment in order
for the structure expression judgment for struct $d^m$ end to properly
construct a structure realization. 
\caption{Module declaration elaboration}
\label{fig:elabmod}
\end{figure}

\begin{figure}
\centering
\fixedCodeFrame{
\small
~\\[2mm]
\fbox{$\Gamma,\Upsilon\vdash strexp \Rightarrow_{str} (M,\varphi)$}
	\begin{equation} 
\infer{\Gamma,\Upsilon\vdash p \Rightarrow_{str} (M, \vec{\rho})}
	          {\Gamma(p)=(\vec{\rho}, M)} 
\label{eq:strpath}
\end{equation}

	\begin{equation} 
\infer{\Gamma,\Upsilon\vdash \mathbf{struct}~d^m~\mathbf{end}
  \Rightarrow_{str} ( (\Sigma,\langle \Uloc,\Upsilon\rangle),\llparenthesis\eta\rrparenthesis)}
	{\begin{array}{c}
\Gamma,\Upsilon\vdash d^m\Rightarrow_{decl}(\eta,\Gamma', \Uloc)\qquad
\Upsilon \Uloc \vdash\Gamma'\hookrightarrow \Sigma
\end{array}} 
\label{eq:basestr}
\end{equation}

% [4/8/2010] How do local environments work with functor entities? 
\begin{equation} 
\infer{\begin{array}{c}
\Gamma,\Upsilon\vdash p(strexp)\Rightarrow_{str}
((\Sigma_{body},R_{app}),\varphi_{app})
\end{array}}
	{\begin{array}{c}
\Gamma(p) = (\vec{\rho}, (\Pi X:\Sigma_{par}.\Sigma_{body}, \langle\theta; \Upsilon'\rangle))\\
\Gamma,\Upsilon\vdash strexp\Rightarrow_{str}
(M,\varphi)\\ 
\Upsilon\vdash (M,\varphi) : \Sigma_{par} \Rightarrow_{match} (R_{c},\varphi_{c})\\
\varphi_{app} = \vec{\rho}(\varphi_{c})\qquad \Upsilon\vdash \varphi_{app} \Downarrow_{str} R_{app}
\end{array}}
\label{eq:strapp}
\end{equation}

	\begin{equation} 
\infer
	{\Gamma,\Upsilon\vdash \mathbf{let}~d^m~\mathbf{in}~strexp\Rightarrow_{str}(M,\mathbf{let}~\eta_{def}~\mathbf{in}~\varphi)}
	{\begin{array}{c}\Gamma,\Upsilon\vdash d^m\Rightarrow_{decl}(\eta_{def},\Gamma_{def},\Upsilon_{def})\\ \Gamma_{def},\Upsilon_{def}\vdash strexp\Rightarrow_{str}(M, \varphi)
\end{array}} 
\label{eq:letexp}
\end{equation}

\begin{equation} 
\infer{\begin{array}{c}
\Gamma,\Upsilon\vdash strexp : sigexp
 \Rightarrow_{str} ((\Sigma_{spec},R_c),\varphi_c)
\end{array}}
{\begin{array}{c}
	   \Gamma,\Upsilon,\emptyset_{sig}\vdash sigexp \Rightarrow_{sig} \Sigma_{spec} \qquad
	   \Gamma,\Upsilon\vdash strexp \Rightarrow_{str} (M_{u},\varphi_{u})\\
	   \Upsilon\vdash (M_{u},\varphi_u) : \Sigma_{spec} \Rightarrow_{match} (R_c,\varphi_{c})
\end{array}} 
\label{eq:transascription}
\end{equation}
     
 \begin{equation} 
\infer{\begin{array}{c}
\Gamma,\Upsilon \vdash strexp :> sigexp 
\Rightarrow_{str}
   ((\Sigma_{spec},\langle\Upsilon_{spec},\Upsilon\rangle),
   \varphi_{c})
\end{array}}
{\begin{array}{cc}
\Gamma,\Upsilon,\emptyset_{sig}\vdash sigexp
     \Rightarrow_{sig} \Sigma_{spec}\qquad
   \Gamma,\Upsilon\vdash strexp \Rightarrow_{str} (M_u, \varphi_u)\\
  \Upsilon\vdash (M_u,\varphi_u) : \Sigma_{spec}
  \Rightarrow_{match} (R_{c},\varphi_{c})\\
  \Upsilon, \emptyset_{ee} \vdash \Sigma_{spec} \uparrow \Upsilon_{spec}
% Does it matter whether we return the uncoerced or coerced varphi?
% The compiler returns the coerced version.  
\end{array}
}
\label{eq:opaqueascription}
\end{equation}

}
\caption{Structure expression elaboration}
\label{fig:strexpelab}
\end{figure}


%!TEX root = ../principles.tex
\begin{figure}
	\centering
	\fixedCodeFrame{
	\small
	\begin{align}
		& E;\Upsilon\vdash tycexp \Rightarrow_{tyc} t & \notag \\[3mm]
		& E;\Upsilon\vdash & 
	\end{align}	
	}
	\caption{Type constructor elaboration}
	\label{fig:elabtyc}
\end{figure}
%!TEX root = ../principles.tex

% Observation: Don't need the entire static environment as context
% because all symbolic names should have already been reduced
% away. The place where the entire static environment does play a role
% is in relativization of type expressions for values and (derived) tycon
% expressions. 

% There are some fundamental flaws in the rules as given. First, there
% is no base case, for the empty static environment. That would
% establish what the realization part of the resultant full signature
% really is. Second, none of the rules extend the realization part of
% the full signature at this point. Thus, I expect the resultant
% realization to end up empty anyway. 

\begin{figure}
	\centering
	\fixedCodeFrame{
	\small
        ~\\[2mm]
	\fbox{$\Upsilon\vdash \Gamma \hookrightarrow \Sigma $}
               \begin{equation}
                 \infer{\Upsilon\vdash\emptyset_{se}\hookrightarrow
                   \emptyset_{sig}}
                 {\strut}
               \end{equation}

		\begin{equation} 
                  \infer{\Upsilon\vdash
                    [x\mapsto \mathfrak{T}]\Gamma \hookrightarrow
                  [x\mapsto \mathbb{T}]\Sigma_r
                  }
		{\begin{array}{c}
                  \Upsilon\vdash \mathfrak{T} \searrow^{te}
                  \mathbb{T}\qquad
                  \Upsilon\vdash \Gamma \hookrightarrow \Sigma_r
                \end{array}} 
\label{eq:extraval}
            \end{equation}
% The value binding rule doesn't add any entity decl. It merely adds a value spec with a relativized version of value binding's type.

% There are no open tycon bindings in the static environment. All tycons in the static environment are defined or instantiated. 
% Perhaps combine \upharpoonright and C_{ep}() notation together because one is just an extension to type expressions
              %\begin{equation} \infer{\begin{array}{c} C_{ep},C_{elab}\vdash
               % [q\mapsto[=t]]\Gamma\\ \hookrightarrow
                %(([q\mapsto (\rho, \mathsf{arity}(t))]\Sigma, \Upsilon),
                %\eta)\end{array}}
              %{C_{ep},C_{elab}\vdash\Gamma\hookrightarrow((\Sigma,\Upsilon),
               % \eta)\qquad C_{ep}(t)=[\rho]} & \\[5mm]
              \begin{equation} 
\infer{\Upsilon\vdash [t\mapsto \mathfrak{C}^\lambda]\Gamma \hookrightarrow [t\mapsto \mathbb{C}^\lambda]\Sigma_r}
{\Upsilon\vdash \mathfrak{C}^\lambda
  \searrow^{tyc} \mathbb{C}^\lambda\qquad \Upsilon\vdash \Gamma \hookrightarrow
  \Sigma_r}
\label{eq:extratypedef}
\end{equation}

% SML/NJ's representation of tycon entities included type definitions
% (represented by an entity path). In this new semantics, there is no
% need for any form other than new(arity). 
% The question remains, what is the relationship between the scoping
% in the entity environment and in the static environment. That is to
% say, do we expect \Upsilon^{-1}(\tau^n) to be anything other than
% singleton? If so, what is an example?
\begin{equation} 
\infer{\Upsilon\vdash[t\mapsto\tau^n]\Gamma \hookrightarrow [t\mapsto (\rho, n)]\Sigma_r}
{\inv(\Upsilon,\tau^n)=\rho\qquad\Upsilon\vdash
  \Gamma \hookrightarrow\Sigma_r} 
\label{eq:extraatomictyc}
\end{equation}

% The interesting part about producing a tycon spec is how we compute the entity variable. The tycon definition may be local, in which case we can reuse that tycon's (RHS) entity variable. Otherwise, we need to produce a new entity variable and a corresponding entity declaration mapping this new entity variable to a CONSTtyc or a VARtyc for new and nonlocal tycons respectively. 
% What we need is a judgment that produces an entity variable, entity environment, and entity declarations from a static entity.
\begin{equation}
  \infer{\begin{array}{c}
      \Upsilon\vdash[x\mapsto (\rho, (\Sigma_1,R_1))]\Gamma_r\hookrightarrow [x\mapsto
                (\rho,\Sigma_1)]\Sigma_r
              \end{array}}
              {\begin{array}{c}
                  \Upsilon\vdash\Gamma_r\hookrightarrow \Sigma_r
                \end{array}}
\label{eq:extrastr}
            \end{equation}

% The structure binding rule first looks up the entity path for the full signature 
% If the structure has an entity path which is a singleton, then it uses that as entity variable in the structure spec we are producing. 
% If the entity path is not singleton, then the entdec is a structure e_new = ep such that ep is the non-singleton entity path. It is this e_new that is used in the structure spec. 
% Otherwise, no entity path exists yet so we produce a CONSTstr with the realization from the full signature, mapping a e_new to it. 
% In the latter two cases, the entity is not local so we need to add a
% local version of the entity to the entity environment, mapping e_new
% to rlzn.  
              \begin{equation} 
                \infer{\Upsilon\vdash[f\mapsto (\rho, (\Sigma^f_1,\psi))]\Gamma_r \hookrightarrow [f\mapsto (\rho,\Sigma^f_1)]\Sigma_r}
              {\begin{array}{c}
                 \Upsilon\vdash\Gamma_r\hookrightarrow\Sigma_r
              \end{array}} 
\label{eq:extrafct}
              \end{equation}

              % Ignoring signature and functor signature
              % bindings. Since this is the cumulative static
              % environment, it may contain top-level declared
              % signatures and functor signatures. 
             %\begin{equation}
             %\infer{\Upsilon\vdash[x\mapsto \Sigma]\Gamma\hookrightarrow \Sigma'}
             %{\Upsilon\vdash\Gamma\hookrightarrow \Sigma'}
             %\end{equation}

             %\begin{equation}
             %\infer{\Gamma_0,\Upsilon\vdash[x\mapsto\Sigma^f]\Gamma\hookrightarrow (M,\eta)}
             %{\Gamma_0,\Upsilon\vdash\Gamma\hookrightarrow (M,\eta)} 
             %\end{equation}

	}
\caption{Signature extraction}
\label{fig:extractsig}
\end{figure}

\section{Comparison to MacQueen-Tofte}
In MacQueen-Tofte, signatures support higher-order functors by including a functor signature environment, denoted $\Phi$, that maps functor maps to functor signatures. Because functor signature components may depend on specifications that came earlier in the enclosing structure signature, the system introduces a binding $\lambda\rho$ that binds $\rho$, the entity variable representing the entire enclosing structure signature. 

The MT structure signatures require this $\lambda\rho$ binding because they incorporate functor signature environments indexed by functor paths, $\Phi$. Without the $\lambda\rho$-binding, $\Phi$ cannot depend on entities in the enclosing signatures. Structure matching first looks up all $fp$s in $\Phi$ and then attempts to match the static functor with $\Phi[\varphi/\rho]$.

In contrast, in the current language, SML/NJ no longer permits nonlocal forms of sharing of the flavor illustrated in the example in the MT94 paper. Instead, structure definitions can express the same sharing. Signature specifications contain the signatures and realizations for structure definitions ({\it i.e.}, $\rho,\Sigma=\Sigma$ and $\rho,\Sigma=_{\overline{\rho}} \Sigma$) at the point of structure signature matching. These definition structure signatures and realizations are filled in during signature elaboration by looking up the static environment and entity path context.  

\section{Primary and secondary components}
The form of a functor argument is constrained by the functor parameter signature possibly modified by a where type definition. In the parameter signature, there are structure specifications, formal functor specifications, structure/type sharing constraints, and two classes of type specifications. Type specifications may be abstract or definitional. Abstract type specifications that remain abstract after the elaborator resolves all sharing and where type constraints are called flexible or primary components. These primary type components are those essential components that must be kept to maintain the semantics of functor application (i.e., the type application associated with the functor application). The specific function of primary type components is to capture a canonical representative of an equivalence class of abstract types induced by type sharing constraints. Each equivalence class has exactly one primary type component that serves as a representative element. References to all other members of the equivalence class should be redirected to the associated primary type component. The remaining type components are secondary and therefore should be fully derivable from the primary components and externally defined types. Secondary types do not have to be explicitly represented in the parameter signature because all occurrences of these secondary types can be expanded out according to their definitions. 

\begin{lstlisting}
functor F(type s type t type u = s * t
          sharing type t = s) = ...
\end{lstlisting}

In the above example, \lstinline|s| can be primary, representative for the equivalence class containing both \lstinline|s| and \lstinline|t|, and \lstinline|u| is secondary.

\section{Relationship to Harper-Stone Semantics}

\section{A Fully Expressive Signature Language}
The ML signature language permits type definitions that may refer to general type expressions. Type expressions may involve both primitive type constructors such as $\rightarrow$ and type operators. It is the inclusion of type operators that gives the signature language much of its expressiveness. The semantics of type sharing constraints differs significant between SML90 and SML97. Type sharing constraints could be imposed on two type constructors without restriction in SML90. In SML97, the designers partitioned the semantics of type sharing into type definitions which expressed sharing between an abstract type and an arbitrary type expression, and regular type sharing constraints which can only be imposed between two flexible (or primary) types whose names must be in scope. 

A module system that permits both type definitions and type sharing constraints in signatures introduces significant new complexity. For example, whereas in Leroy's \cite{Leroy-generativity} TypModl language, which only permits SML90-style definitional type sharing constraints and no type definitions, type sharing constraints can be ``normalized'' by pushing them up the signature and eliminated by turning them into type definitions, type sharing constraints cannot be eliminated in a language that permits both type definitions and type sharing constraints.

\section{FLINT Compilation}
The goal of FLINT compilation was to enable all type-based optimizations to work across module boundaries by compiling the module calculus into System F. 

%!TEX root = ../principles.tex
\begin{figure}
	\centering
	\fixedCodeFrame{
	\small
	\[
	\setlength{\tabcolsep}{0ex}
	\renewcommand{\arraystretch}{1.1}
	\begin{array}{rcll}
		p_s & ::= & s_i~|~p_s.s_i\\
		p_f & ::= & f_i~|~p_s.f_i\\
		p_t & ::= & t_i~|~p_s.t_i\\
		D & ::= & \epsilon~|~D~D'~|~type~t_i::\kappa_c\\
		  & ~~| & type~t_i::\kappa_c = \mu_c~|~structure~s_i:M_s\\
		  & ~~| & functor~f_i:M_f\\
		M_s & ::= & sig~D~end\\
		M_f & ::= & fsig(s_i:M_s)M'_s\\
		\kappa_c & ::= & \Omega~|~\Omega\rightarrow\kappa_c\\
		\mu_c & ::= & p_t~|~int~|~\mu_c \rightarrow \mu'_c~|~\lambda t_i::\Omega.\mu_c~|~\mu_c[\mu'_c]\\
		d & ::= & \epsilon~|~d d'~|~local~d~in~d'~end~|~type~t_i::\kappa_c = \mu_c\\
		  & ~~| & structure~s_i=m_s~|~functor~f_i=m_f\\
		m_s & ::= & p_s~|~f_i(s_i)~|~(s_i:M_s)~|~m_b\\
		m_b & ::= & struct~d~end\\
		m_f & ::= & p_f~|~funct (s_i:M_s)m_b
	\end{array}
	\]
	}
\caption{Normalized module calculus NRC}
\end{figure}
%!TEX root = ../principles.tex
\begin{figure}
	\centering
	\fixedCodeFrame{
	\small
	\[
	\setlength{\tabcolsep}{0ex}
	\renewcommand{\arraystretch}{1.1}
	\begin{array}{rcll}
		\kappa_t & ::= & \Omega~|~\kappa_t\rightarrow\kappa'_t~|~\{l::\kappa_t,\ldots,l'::\kappa'_t\}\\
		\mu_t & ::= & \alpha~|~Int~|~\mu_t\rightarrow\mu'_t~|~\lambda\alpha::\kappa_t.\mu_t~|~\mu_t[\mu'_t]\\
		      & ~~| & \{l=\mu_t,\ldots,l'=\mu'_t\}~|~\mu_t.l\\
		\sigma_t & ::= & T(\mu_t)~|~\sigma_t\rightarrow\mu'_t~|~\{l:\sigma_t,\ldots,l':\sigma'_t\}~|~\forall\alpha::\kappa_t.\sigma_t\\
		e_t & ::= & x~|~i~|~\lambda x:\sigma_t.e_t~|~@e_t e'_t~|~\Lambda\alpha::\kappa_t.e_t~|~e_t[\mu_t]\\
		    & ~~| & \{l=e_t,\ldots,l'=e'_t\}~|~e_t.l~|~let~d_t~in~e_t]\\
		d_t & ::= & \epsilon~|~(x=e_t);d_t
	\end{array}
	\]
	}
	\caption{F$_\omega$-based target calculus TGC}
\end{figure}
}

\bibliography{modules}
\end{document}
