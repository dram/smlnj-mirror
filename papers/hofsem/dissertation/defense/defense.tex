\documentclass{beamer}
\usepackage{listings,amsmath,../../design/aux/proof}
\usepackage{pgf,pgfarrows,pgfnodes,tikz}

%\newcommand{\structure}{\ensuremath{\mathbf{structure}}}
\newcommand{\functor}{\ensuremath{\mathbf{functor}}}
\newcommand{\struct}[1]{\ensuremath{\mathbf{struct}\{#1\}}}
\newcommand{\inx}{\ensuremath{\mathsf{in}}}
\newcommand{\en}{\ensuremath{\mathbf{end}}}
\newcommand{\local}{\ensuremath{\mathbf{local}}}
\newcommand{\letx}{\ensuremath{\mathsf{let}}}
\newcommand{\letin}[2]{\ensuremath{\mathbf{let}~#1~\mathbf{in}~#2}}
\newcommand{\where}{\ensuremath{\mathbf{where}}}
\newcommand{\sig}[1]{\ensuremath{\mathbf{sig}\{#1\}}}
\newcommand{\type}{\ensuremath{\mathbf{type}}}
\newcommand{\eqtype}{\ensuremath{\mathbf{eqtype}}}
\newcommand{\val}{\ensuremath{\mathsf{val}}}
\newcommand{\sharing}{\ensuremath{\mathbf{sharing}}}
\newcommand{\incl}{\ensuremath{\mathbf{include}}}
\newcommand{\exception}{\ensuremath{\mathbf{exception}}}
\newcommand{\Int}{\ensuremath{\mathbf{int}}}
\newcommand{\signature}{\ensuremath{\mathbf{signature}}}
\newcommand{\open}{\ensuremath{\mathbf{open}}}
\newcommand{\newx}{\ensuremath{\mathsf{new}}}
\newcommand{\getx}{\ensuremath{\mathbf{get}}}
\newcommand{\EP}{\ensuremath{\mathsf{EP}}}
\newcommand{\inj}{\ensuremath{\mathsf{inj}}}
\newcommand{\env}{\ensuremath{\mathsf{env}}}
\newcommand{\an}[1]{\ensuremath{\langle\langle #1 \rangle\rangle}}
\newcommand{\dt}{\ensuremath{\mathbf{datatype}}}
\newcommand{\letins}[2]{\ensuremath{\mathsf{let}~#1~\mathsf{in}~#2}}
\newcommand{\tycon}{\ensuremath{\mathsf{tycon}}}
\newcommand{\sel}[1]{\ensuremath{\mathsf{sel}(#1)}}
\newcommand{\coerce}{\ensuremath{\mathsf{coerce}}}
\newcommand{\lettyc}[2]{\ensuremath{\mathsf{let~tyc}~#1~\mathsf{in}~#2}}
\newcommand{\lettycline}[2]{\ensuremath{\begin{array}{l}
\mathsf{let~tyc}~#1\\\mathsf{in}~#2\end{array}}}
\newcommand{\tyc}{\ensuremath{\mathsf{tyc}}}
\newcommand{\inlet}{\ensuremath{\mathsf{in~let}}}
\newcommand{\inv}{\ensuremath{\mathsf{inv}}}
\newcommand{\entpath}{\ensuremath{\mathsf{entpath}}}
\newcommand{\typ}{\ensuremath{\mathsf{typ}}}

\newcommand{\Uclo}{\ensuremath{\Upsilon_{c}}}
\newcommand{\Uloc}{\ensuremath{\Upsilon_{l}}}

\newcommand{\letx}{\ensuremath{\mathsf{let}}}
\newcommand{\lettycline}[2]{\ensuremath{\begin{array}{l}
\mathsf{let~tyc}~#1\\\mathsf{in}~#2\end{array}}}
\newcommand{\tyc}{\ensuremath{\mathsf{tyc}}}
\newcommand{\inx}{\ensuremath{\mathsf{in}}}
\newcommand{\inlet}{\ensuremath{\mathsf{in~let}}}
\newcommand{\letin}[2]{\ensuremath{\mathsf{let}~#1~\mathsf{in}~#2}}
\newcommand{\newx}{\ensuremath{\mathsf{new}}}
\newcommand{\chk}{\ding{51}}
\newcommand{\ex}{\ding{55}}

\lstset{escapeinside=`',columns=fullflexible}
\setbeamercovered{invisible}
\usetheme{Luebeck}
%\usecolortheme{seagull}
%\useinnertheme{rectangles}
%\useoutertheme{infolines}

\mode<presentation>
\setbeamertemplate{navigation symbols}{}

\lstset{language=ML,basicstyle=\small}

\tikzstyle{every picture}+=[remember picture]
\everymath{\displaystyle}
\tikzstyle{na} = [baseline=-.5ex]
\tikzstyle{arr} = [line width=2pt,->]
\tikzstyle{block} = [rectangle, draw, fill=blue!20, 
    text width=5em, text centered, rounded corners, minimum height=4em]


\title[A True Higher-Order Module System]{A True Higher-Order Module System}
\author{George Kuan}
%\institute{University of Chicago}
\date{Dissertation Defense\\ May 3, 2010}

\begin{document}
	
	\begin{frame}
		\maketitle
	\end{frame}

	\begin{frame}[fragile]
		\frametitle{Higher-Order Modules}
		\begin{itemize}[<+->] 
			\item[]
		\begin{lstlisting}
		signature T = sig type t end
		functor Apply(functor F(X:T):T) (M:T) = F(M)
		\end{lstlisting}
			\item[]
		\begin{lstlisting}
		functor Id(X:sig type t end) = X
		structure N = Apply (functor F=Id) (struct type t = int end)
		\end{lstlisting}
			\item[] \alert<.>{Expect N.t = int}
			\item[]
		\begin{lstlisting}
		functor Const(X:sig type t end) = struct type t = bool end
		structure R = Apply (functor F=Const) (struct type t = int end)
		\end{lstlisting}
			\item[] \alert<.>{Expect R.t = bool}
		\end{itemize}
	\end{frame}
	
	% \begin{frame}[fragile]
	% 	\frametitle{True Higher-Order Semantics (2)}
	% 	\begin{itemize}[<+->]
	% 	\item[]
	% 	\begin{lstlisting}
	% 	functor F() = 
	% 	struct 
	% 	  datatype t = S of int 
	% 	  val m = S 0 
	% 	end	
	% 	\end{lstlisting}
	% 	
	% 	\item[]
	% 	\begin{lstlisting}
	% 	structure M0 = F()
	% 	\end{lstlisting}
	% 	
	% 	\item[]
	% 	\begin{lstlisting}
	% 	structure M1 = F()
	% 	\end{lstlisting}
	% 	
	% 	\item[] \alert{M0.t $\ne$ M1.t}
	% 	\end{itemize}
	% \end{frame}

		\begin{frame}[fragile]
			\frametitle{Approaches (1)}
			\begin{enumerate}[<+->]
				\item Syntactic (Applicative Functors [Leroy 1995])~\\

	functor Apply(functor F(X:T):T) (M:T) \\
	  : sig type t=\tikz[baseline]{\node[fill=blue!20,anchor=base](t1){F(M).t}} end = F(M) 	

				\item[] \hspace{10em}\tikz[na] \node[coordinate] (n1) {}; \alert<.>{can only be treated superficially} 
				\item Semantic [MacQueen-Tofte 1994]~\\

	functor Apply(functor F(X:T):T) (M:T) \\
	~~: sig type t end = F(M) 	\\[3mm]


	Dependence of result type t on F and M is inferred by the compiler

			\end{enumerate}

			\begin{tikzpicture}[overlay]
				\path[->]<2-> (n1) edge [bend left] (t1);
			\end{tikzpicture}
		\end{frame}



		\begin{frame}[fragile]
			\frametitle{Approaches (2)}
			\begin{lstlisting}
			functor F(functor G(X:T):T) = 
			struct 
			  datatype s = S of int 
			  structure M = G(struct type t = s end)
			  type u = M.t
			end	
			\end{lstlisting}

			\uncover<2->{Syntactic approach breaks down.}\\
			\only<3->{A descriptive signature would have to involve static effects and the  actions taken by formal functor G.}

		\end{frame}
			
	\begin{frame}
		\frametitle{True Higher-Order Semantics }
		\begin{block}{Functor Action}
		A \alert{function action} is the way in which a functor computes its output types from its parameter types, namely: 
		\begin{enumerate} 
			\item type generativity 
			\item functor actions of formal functors
		\end{enumerate}
		\end{block}
	\end{frame}
	
	\begin{frame}
		\frametitle{Motivation}
		\begin{block}{Syntactic Approaches}
			All module type information is syntactic
			\begin{enumerate}
				\item Give up non-syntactic module type information
				\item Try to express more module type information syntactically 
			\end{enumerate}
		\end{block}
		\vspace{4mm}
		\begin{block}{Semantic Approach}
			Some module type information is semantic (functor actions)
		\end{block}
	\end{frame}
	
	\begin{frame}[fragile]
		\frametitle{Motivation}
		\begin{enumerate}
			\itemsep=4mm
			\item Restricting to syntactic module types is analogous to restricting a language to $\lambda$-terms where each $\lambda$ is given a really powerful dependent type that computes the result of the $\lambda$
			\item An abstract model of current SML/NJ implementation of higher-order modules
			\item A more detailed and realistic expansion of MacQueen-Tofte~1994 and  Shao~1998 
			\item True higher-order semantics without re-elaboration
		\end{enumerate}
	\end{frame}
		
	\begin{frame}
		\frametitle{Outline}
		\tableofcontents
	\end{frame}
	
	\section{Type of a Structure}

\begin{frame}
	\frametitle{Type of a Structure}
	{\Large Big Question: What is the ``type'' of a structure?}
\end{frame}

\begin{frame}
	\frametitle{Signatures}
structure A = struct\\
~~datatype $\alpha$ t = c of $\alpha$\\
~~structure M = struct datatype t = d val x = c d end\\
end\\[2mm]

\only<1>{Syntactic}\only<2->{{\color{blue}Semantic}} Signature\\[1mm]
sig\\
~~type $\alpha$ t \tikz[na] \node[coordinate] (n0) {};\uncover<2->{{\color{blue}$\rho_0$}}\\
~~structure M \uncover<2->{{\color{blue}$\rho_M$}}: sig~type~ t~\uncover<2->{{\color{blue}$\rho_1$}}~val~x: \only<1>{\tikz[na] \node[coordinate,yshift=3mm] (t0) {};\alert{??}} \only<2->{{\color{blue}\tikz[baseline]{\node[anchor=base](t1) {$\rho_0(\rho_1)$};}}}
%~val~k : \only<1>{$(\alpha, $ t}\only<2->{{\color{blue}\tikz[baseline]{\node[anchor=base] (t2) {$\forall\alpha.\rho_1(\alpha,\rho_0)$}; }}} 
end\\
end\\[3mm]
\only<2>{Because type names can shadow, syntactic names are insufficient}
\only<3>{Need unshadowable \alert{entity variables} (\emph{aka} internal names  [Harper-Lillibridge 94]) and \alert{entity paths} (\emph{e.g.}, $\rho_M\rho_1$)}
\only<4>{\hspace{8em}\tikz{\node (s0) {relativized types};}}
\uncover<5->{
Abbreviated: 
\[\left\{\begin{array}{l}
t : (\rho_0, 1),\\
M : (\rho_M, \{t : (\rho_1, 0),~x : \rho_0(\rho_1) \})\\
\end{array}
\right\}
\]}
\begin{tikzpicture}[overlay]
	\path[arr]<1> (n0) edge [bend left] (t0);
	
	\path[arr]<4> (s0) edge [bend right] (t1);
	%\path[arr]<4> (s0) edge [bend right] (t2);
\end{tikzpicture}

\end{frame}

\begin{frame}
	\frametitle{Type of a Structure}
	functor F() = struct\\
	~~structure M = struct datatype~t end\\
	~~val x : M.t = ...\\
	end\\
	\vspace{1em}
	structure A = F()\\[2em]

	\only<1-2>{What is the module ``type'' for A?}
	\uncover<2->{
	$\left\{\begin{array}{l}
	M:(\rho_M, \{t : (\rho_t, 0)\}),~x : \rho_M\rho_t
	\end{array}
	\right\}
	$\\[1em]}

	\only<+>{A semantic signature is not the complete module ``type''.}
	\uncover<+>{Need an \alert{entity environment} mapping entity variables to static entities (tycons, structures, and functors) $\{ \rho_M\mapsto\{\rho_t\mapsto \tau^0\} \}$\\ $\tau^0$ is a fresh atomic semantic tycon}

\end{frame}

\begin{frame}
	\frametitle{Entity Environments}
	functor F() = struct\\
	~~structure M = struct datatype~t end\\
	~~val x : M.t = ...\\
	end\\[1em]
\begin{itemize}[<+->]
	\item[] structure A = F() \hspace{4em} $\rho_A \mapsto \{ \rho_M\mapsto\{\rho_t\mapsto \tau^{0}_a\} \}$
	\item[] structure B = F() \hspace{4em} $\rho_B \mapsto \{ \rho_M\mapsto\{\rho_t\mapsto \tau^0_b\} \}$
\end{itemize}
\uncover<2>{Each time F is applied, we get 
 a fresh atomic tycon}	
\end{frame}

\begin{frame}
	\frametitle{Full Signature}
	\begin{equation*}
		\tikz[baseline]{\node[fill=blue!20,anchor=base] (n0) {$\left\{\begin{array}{l}
	M:(\rho_M, \{t : (\rho_t, 0)\}),~x : \rho_M\rho_t
	\end{array}
	\right\}$};}
	+ \tikz[baseline]{\node[fill=red!20,anchor=base] (n1) {$\{ \rho_M\mapsto\{\rho_t\mapsto \tau^0\} \}$};}
	\end{equation*}
	
	\begin{tikzpicture}[overlay]
		\node[above of=n0] (t0) {signature (fixed)};
		\node[above of=n1] {realization (volatile)};
	\end{tikzpicture}

\vspace{3em}
\uncover<2->{
	\begin{tikzpicture}[node distance=4cm, auto]
		\node[block] (n2) {M.t};
		\node[block, right of=n2] (n3) {$\rho_M\rho_t$};
		\node[block, right of=n3] (n4) {$\tau^0$};
		
		\path[arr] (n2) edge node {signature} (n3);
		\path[arr] (n3) edge node {entity env} (n4);
	\end{tikzpicture}}
	
	
\end{frame}

\begin{frame}
	\frametitle{Type of a Structure}
	{\Large But what is a functor entity?}
% \begin{itemize}[<+->]
% 	\item[] 
% 	\item[] An entity function $\lambda\rho.\varphi$ such that $\varphi$ is a structure entity expression that evaluates to a structure entity (realization)
% \end{itemize}

\uncover<2>{
\begin{tikzpicture}[auto,node distance=3cm]
	\node (par) {R$_{par}$};
	\node[block,right of=par] (lam) {entity function};
	\node[right of=lam] (res) {R$_{res}$};
	
	\path[arr] (par) edge (lam);
	\path[arr] (lam) edge (res); 
	
\end{tikzpicture}}
\end{frame}

\section{Entity Calculus}

% \begin{frame}
% 	\frametitle{Entity Calculus}
% 	\begin{enumerate}
% 		\item Entity variable ($\rho$): unshadowable variable for static content (tycons, structures, and functors)
% 		\item Entity path ($\rho_0\rho_1\rho_2$): path into structure hierarchy 
% 		\item Entity environment: a finite map from entity variables to tycon, structure, and functor entities
% 	\end{enumerate}
% \end{frame}

\begin{frame}
	\frametitle{Entity Calculus (1)}
{\only<3>{\color{red}}datatype v}\\
functor F(X:sig type t end) = struct\\
~~{\only<2->{\color{blue}}datatype $(\alpha,\beta)$ u}\\
~~{\only<2,4>{\color{red}}type s = (X.t,v) u}\\
end\\[2em]
	\only<2>{A {\bf tycon entity} is either 
	\begin{itemize}
		\item an atomic tycon (\emph{e.g.}, {\only<2>{\color{blue}}$\tau^2_u$})
		\item or normal form semantic tycon (\emph{e.g.}, {\only<2>{\color{red}}$\lambda().\tau^2_u(\tau^0_t,\tau^0_v)$})
	\end{itemize}}
	\only<3>{A {\bf structure entity} $R$ is a pair of entity environments $\langle\{{\color{blue}\rho_u\mapsto\tau^2_u,~\rho_s\mapsto\lambda().\tau^2_u(\tau^0_t,\tau^0_v)}\},~~\{{\color{red}\rho_v\mapsto\tau^0_v}\}\rangle$\\[2mm]
	a local one defining all entities in the structure and a closure environment \\[2mm]
	$\tau^0_t$ is a dummy atomic tycon to stand in for the tycon in the functor argument }
	\uncover<4>{A {\bf functor entity} is a closure: a $\lambda$-expression mapping structure entity to an expression that evaluates to a structure entity $\lambda\rho_x.\{{\color{blue}\rho_u = \newx(2)}, {\color{red}\rho_s = \lambda().\rho_u(\rho_x\rho_t,\rho_v)} \}$ + $\{\rho_v\mapsto \tau^0_v\}$}
\end{frame}

\begin{frame}
	\frametitle{Entity Calculus (2)}
	\begin{itemize}
		\item[] Tycon entity expression:
		\[\zeta ::= \newx(n)~|~\mathbb{C}^\lambda \textrm{(relativized tycons)}\]
		\item[] Structure entity expression:
		\[ \varphi ::= \vec{\rho}~|~\{ \eta \}~|~\theta(\varphi)~|~ \letin{\eta}{\varphi}\]
		
		\item[] Functor entity expression:
		\[ \theta ::= \vec{\rho}~|~\lambda\rho.\varphi~|~\lambda\rho.\Sigma \]
		
		\item[] Entity declaration:
		\[ \eta ::= \circ~|~\rho = \zeta,\eta~|~\rho = \varphi,\eta~|~\rho = \theta,\eta\]
		
	\end{itemize}
\end{frame}

\begin{frame}
\frametitle{Life-Cycle of a Type}

\begin{tikzpicture}[node distance= 4.5cm,auto]
	\uncover<1->{\node[block] (sy) {Syntactic};}
	\uncover<2->{\node[block, right of=sy] (se) {Semantic};}
	\uncover<3->{\node[block, right of=se] (re) {Relativized};}

	\uncover<2->{\node[node distance=2cm,below of=se] (statenv) {static env};}
	\uncover<3->{\node[node distance=1cm,,below of=statenv] {entity env};}
	\uncover<3->{\node[node distance=2cm,below of=re] (semsig) {semantic sig};}
	\uncover<3->{\node[node distance=1cm,below of=semsig]  {entity expr};}
		
	\uncover<2->{\path[arr] (sy) edge node {\only<2>{\color{red}}elaborates} (se);}
	\uncover<3->{\path[arr] ([yshift=-5mm]se.north east) edge node {\only<3>{\color{red}}relativizes} ([yshift=-5mm]re.north west);}
	\uncover<3->{\path[arr] (se) edge [loop above] node {\only<3,4>{\color{red}}evaluates} (se);}
	\uncover<4->{\path[arr] ([yshift=5mm]re.south west) edge node {\only<4>{\color{red}}instantiates} ([yshift=5mm]se.south east);}
	

\end{tikzpicture}
\end{frame}

\begin{frame}[fragile]
	\frametitle{Functor Actions}
	\begin{itemize}[<+->]
		\item[]
	\begin{lstlisting}
	functor Apply(X:sig functor F(Y:sig type t end):sig type t end
	                    structure M: sig type t end
	                end) =
	struct structure R = X.F(X.M) end
	\end{lstlisting}
	\item[]
	\[\lambda\rho_x.\{\rho_r=\rho_x\rho_f(\rho_x\rho_m)\}\]
		\item[]
	\begin{lstlisting}
	functor G() = struct datatype t end	
	\end{lstlisting}
	\item[]
	\[\lambda().\{\rho_t=new(0)\}\]
	\end{itemize}
\end{frame}

\begin{frame}[fragile]
	\frametitle{Full Functor Signature}
	\begin{lstlisting}
	datatype v 
	functor F() = struct datatype t val a : v end	
	\end{lstlisting}
	
	\tikz[baseline]{\node[fill=blue!20,anchor=base] (sem) {Semantic functor signature};} + \tikz[baseline]{\node[fill=red!20,anchor=base] (ent) {Entity function (closure)};}
	
	\begin{tikzpicture}
	\node[below of=sem] {$\Pi().
	\{t : (\rho_t, 0),~a : \rho_v \}$};
	\node[below of=ent] {$\langle\lambda().\{\rho_t = \newx(0)\},~\{\rho_v\mapsto \tau^0\}\rangle$};
	\end{tikzpicture}
\end{frame}

\begin{frame}
	\frametitle{Constructing Full Signatures and Entity Expressions}
	Full signatures and entity expressions are produced by \alert{elaboration} in two interweaving modes:
	
	\begin{tikzpicture}[auto, node distance=4cm]
		\node[block] (cc) {Compilation mode};
		\node[block,right of=cc] (dm) {Direct mode};
		
		\path[arr] (cc.north) edge [bend left] node {entexps (compilation)} (dm.north);
		\path[arr] (dm.south) edge [bend left] node {entenvs (evaluation)} (cc.south);
	\end{tikzpicture}

\end{frame}

	% \begin{frame}[fragile]
	% 	\frametitle{Type System}
	% 
	% 	functor Symbols(functor F(X:T):T) = \\
	% 	struct \\
	% 	~~~datatype t = S of \tikz[baseline]{\node[fill=red!20,anchor=base](t0){int}} \\
	% 	~~~structure M = F(struct type t= \tikz[baseline]{\node[fill=red!20,anchor=base](t2){t}} end)\\
	% 	~~~type u = \tikz[baseline]{\node[fill=red!20,anchor=base](t3){M.t * bool}}\\
	% 	end	\\
	% 
	% 	\begin{enumerate}[<+->]
	% 		\item Syntactic types \tikz[na] \node[coordinate] (n0) {}; 
	% 		\item[] \hspace{15em} \tikz[na] \node[coordinate] (n1) {}; symbolic paths			 
	% 		\item Semantic types
	% 		\begin{enumerate}
	% 			\item no paths: symbolic paths looked up
	% 			\item generated datatypes and opaque tycons manifested (represented as $\tau^n$ where $n$ is arity) 
	% 			\item can be evaluated (by $\beta$-reduction) to a normal form
	% 		\end{enumerate}
	% 		\item Relativized
	% 	\end{enumerate}
	% 	
	% 	\begin{tikzpicture}[overlay]
	% 		\path[->]<1-> (n0) edge [bend right] (t0);
	% 		\path[->]<1-> (n0) edge [bend right] (t2);
	% 		\path[->]<1-> (n0) edge [bend right] (t3);
	% 		\path[->]<2-> (n1) edge [bend left] (t3.south west);
	% 	\end{tikzpicture}
	% \end{frame}
	

	\section{Elaboration}
	
	\begin{frame}
		\frametitle{Structure Elaboration}
		$\Gamma,\Upsilon\vdash strexp \Rightarrow_{str} (M, \varphi)$
		
		\begin{enumerate}[<+->]
			\item $\Gamma$ is the static environment mapping symbols to semantic tycons, full signatures, full functor signatures 
			\item $\Upsilon$ is the entity environment
			\item Full signature $M=\langle \Sigma, R\rangle$ where $\Sigma$ is the semantic signature and $R$ is the structure entity
			\item Structure entity expression $\varphi$: Evaluates to an $R'$ isomorphic to $R$ under the current entity environment $\Upsilon$. 
		\end{enumerate}
	\end{frame}
	
	\begin{frame}
		\frametitle{Functor Application}

		\begin{equation*} 
		\infer{\begin{array}{c}
		\Gamma,\Upsilon\vdash p(strexp)\Rightarrow_{str}
		((\Sigma_{body},R_{app}),\varphi_{app})
		\end{array}}
			{\begin{array}{c}
		\alert<1>{\Gamma(p) = (\vec{\rho}, (\Pi X:\Sigma_{par}.\Sigma_{body}, \langle\theta, \Upsilon'\rangle))}\\
		\alert<2>{\Gamma,\Upsilon\vdash strexp\Rightarrow_{str}
		(M,\varphi)}\\ 
		\alert<3>{\Upsilon\vdash (M,\varphi) : \Sigma_{par} \Rightarrow_{match} (R_{c},\varphi_{c})}\\
		\alert<4>{\varphi_{app} = \theta(\varphi_{c})}\qquad \alert<5>{\Upsilon'\Upsilon\vdash \varphi_{app} \Downarrow_{str} R_{app}}
		\end{array}}
		\label{eq:strapp}
		\end{equation*}
		
	\begin{enumerate}
		\item<1-> Lookup symbolic path p in static environment 
		\item<2-> Elaborate argument
		\item<3-> Coerce argument to formal parameter form
		\item<4-> Form entity expression
		\item<5-> Evaluate entity expression (no re-elaboration of functor)
	\end{enumerate}
	\end{frame}

	\begin{frame}[fragile]
			\frametitle{Signature Matching}

	\[\Upsilon \vdash ((\Sigma_a, R_a), \varphi) : \Sigma_s \Rightarrow_{match} (R_c, \varphi_c)\]
	
	\begin{enumerate}[<+->]
		\item Coerces full signature $(\Sigma_a, R_a)$ to form of spec $\Sigma_s$ and produces a coerced structure entity expression $\varphi_c$ from $\varphi$
		\item Fill in (\emph{i.e.}, instantiate) open tycons with actuals in $R_a$
		\item Verify type definitional specs 
		\item Functor signature matching 
		\item Construct a coercion that rebinds actual $\rho$'s to spec variables\\
		actual: $((\{t:(\rho_t',0)\},~\langle\{\rho_t'\mapsto \tau^0\},\emptyset\}),~~\{ \rho_t'=\newx(0) \})$\\
		spec: $\{t:(\rho_t, 0)\}$\\
		coercion: let $\rho_{raw}=\{ \rho_t'=\newx(0) \}$ in $\{\rho_t = \rho_{raw}\rho_t'\}$  

	\end{enumerate}

		\end{frame}

	\begin{frame}
		\frametitle{Signature Matching (2)}
		\begin{itemize}[<+->]
			\itemsep=2cm
			\item[]
		If $\Upsilon\vdash ((\Sigma_a, R_a), \varphi) : \Sigma_s \Rightarrow_{match} (R_c, \varphi_c)$, then for all $x : s\in \Sigma_s$, there exists $x : s'\in \Sigma_a$ such that $R_c(s) = R_a(s')$.
		\item[]
		If $\Upsilon\vdash ((\Sigma_a, R_a), \varphi) : \Sigma_s \Rightarrow_{match} (R_c, \varphi_c)$, then $\Upsilon\vdash \varphi_c \Downarrow_{str} R'$ such that $R'$ and $R_c$ are isomorphic.
		\end{itemize}
	\end{frame}
		
	\begin{frame}
		\frametitle{Other Elaboration Semantics}
		\begin{enumerate}[<+->]
			\item Base structure: extract a semantic signature from a static environment by \emph{relativizing} types/type expressions (also relevant during signature elaboration)
			\item Functor declaration and opaque ascription: instantiation of formal parameter
		\end{enumerate}
	\end{frame}
	
	\section{Soundness}
	\begin{frame}
		\frametitle{Translation}
		% \begin{tikzpicture}[node distance=3cm, auto]
		% 	\node (absyn) {absyn};
		% 	\node[block,right of=absyn] (elab) {Elaboration};
		% 	\node[block,right of=elab] (trans) {Translation};
		% 	\node[right of=trans] (ome) {System F$_\omega$};
		% 	
		% 	\path[arr] (absyn) edge (elab);
		% 	\path[arr] (elab) edge (trans);
		% 	\path[arr] (trans) edge (ome);
		% \end{tikzpicture}
		
		\begin{enumerate}[<+->]
			\itemsep=4mm
			\item To show soundness, use a translation to System~F$_\omega$
			\item A translation of elaborated module language to a standard System F$_\omega$ enriched with records and new 
			\item Factors structures into static and value parts (phase~separation~[Harper, Mitchell, Moggi 1990])
			\item Constructs type- and value-level coercions (signature~matching)
		\end{enumerate}
	\end{frame}
	
	\begin{frame}[fragile]
		\frametitle{Translation}
		\begin{lstlisting}
		functor F(X:sig type s end) = 
		struct 
		  datatype w 
		  val n : w -> w = fn z : w => z 
		end
		\end{lstlisting}

		\[\lettycline{\widehat{f}=\lambda\widehat{x}::\{s::\Omega\}.\{ w = \newx(0) \}}
		  {\letx~f = \Lambda \widehat{x}::\{s::\Omega\}.\Lambda
		      \widehat{f_{res}}::\{ w :: \Omega \} . \lambda x::\{\}.\{n = \lambda z:(\widehat{f_{res}}.w).z\}\\
		    \inx~\ldots}
		\]\\[3mm]		
		
		$\widehat{\bullet}$ indicates type part of $\bullet$.
	\end{frame}
	
	\begin{frame}
		\frametitle{(Relative) Soundness}
		\begin{tikzpicture}[auto]
			\node (strexp) {strexp};
			\node[right of=strexp,node distance=8cm] (e) {$e_\omega$};
			\node[below of=strexp,node distance=2cm] (M) {M};
			\node[right of=M,node distance=8cm] (t) {t};
			
			\path[arr] (strexp) edge node {translation} (e);
			\path[arr] (e) edge node {F$_\omega$ typing} (t);
			\path[arr] (strexp) edge node {elaboration} (M);
			\path[arr] (M) edge node {type translation} (t);			
		\end{tikzpicture}
		% \begin{block}{Theorem}
		% 	If a structure expression $strexp$ elaborates to a full signature $M$ and $strexp$ translates to an F$_\omega$ expression $e_\omega$, then $e_\omega$ has F$_\omega$ type t and $M$ also translates to t.
		% \end{block} 
		\vspace{4mm}\\
		Proof: Induction on a strengthened version of the above. The proof depends on the correctness of type and value coercion, signature matching, and signature instantiation.
	\end{frame}
	
	\begin{frame}
		\frametitle{Main Ideas}
		\begin{enumerate}
			\itemsep=3mm
			\item Factoring module type information (full signature) into semantic signature and realization
			\item Entity calculus encodes functor actions
			\item Elaboration semantics in compilation and direct modes
			\item Coercive signature matching
		\end{enumerate}
	\end{frame}
	
	\begin{frame}
		\frametitle{Related Work: Syntactic Approach}
		\begin{itemize}
			\item CMU
		\begin{itemize}
			\item Harper-Lillibridge 1994, Harper-Stone 1997
			\item Dreyer-Crary-Harper 2003
			\item Harper-Pierce 2005
			\item Dreyer 2005, 2007
			\item Dreyer-Blume 2007
			\item Dreyer-Rossberg 2009
			\item Rossberg-Russo-Dreyer 2010
		\end{itemize}
			\item Leroy 1994, 1995, 1996, 2000
			\item Biswas 1995, Russo 2000
			\item Shao 1998, 1999
			\item Govereau 2005
			\item Montagu and R\'emy 2009
		\end{itemize}
	\end{frame}
		
	\begin{frame}
		\frametitle{Related Work: Semantic Approach}
		\begin{itemize}
			\itemsep=3mm
			\item MacQueen-Tofte 1994
			\item Cr\'egut and MacQueen 1994
			\item Shao 1998
			\item Kuan and MacQueen 2009
		\end{itemize}	
	\end{frame}
	
	\begin{frame}
		\frametitle{Future Work}
		\begin{enumerate}
			\itemsep=1cm
			\item Relationship to type classes
			\item Exceptions and modules
			\item Type inference and modules
		\end{enumerate}
	\end{frame}
		
	\begin{frame}
		\frametitle{Conclusion}
		\begin{itemize}
			\itemsep=4mm
			\item HO module semantics is analogous to $\beta$-reduction semantics
			\item Module types are and should be semantic
			\item One neither has to give up generative datatypes/functors nor true higher-order semantics for a practical semantics
		\end{itemize}
	\end{frame}
	
	\begin{frame}
		\frametitle{}
		{\Huge Thank You}
	\end{frame}
	
	\begin{frame}
		\frametitle{Syntactic Signature}
		Syntactic signatures are comprised of specs:
	\begin{columns}[t]
		\begin{column}{.7\textwidth}
			\begin{itemize}[<alert@+(1)>]
			\item[] \tikz[na] \node[coordinate] (n0) {}; type ($\alpha$, $\beta$) t 
			\item[] \tikz[na] \node[coordinate] (n1) {}; type s = int 
			\item[] \tikz[na] \node[coordinate] (n2) {}; structure M : sig type u end 
			\item[] \tikz[na] \node[coordinate] (n4) {}; val a : (M.u, s) t 
			\item[] \tikz[na] \node[coordinate] (n3) {}; functor F(X:T) : T 
			\end{itemize}
		\end{column}
		\begin{column}{.3\textwidth}
	\begin{itemize}
		\item[]<2-> \alert<2>{open tycon}
		\item[]<3-> \alert<3>{type definition}
		\item[]<4-> \alert<4>{structure}
		\item[]<5-> \alert<5>{value}
		\item[]<6-> \alert<6>{functor}
	\end{itemize}
		\end{column}
		\end{columns}

		\begin{tikzpicture}[overlay]
			\path[arr]<7> (n1) edge [bend left] (n0);
			\path[arr]<7> (n1) edge [bend right] (n2);

			\path[arr]<8> (n2) edge [bend left] (n0);
			\path[arr]<8> (n2) edge [bend left] (n1);

			\path[arr]<9> (n4) edge [bend left] (n0);
			\path[arr]<9> (n4) edge [bend left] (n1);
			\path[arr]<9> (n4) edge [bend left] (n2);

			\path[arr]<10> (n3) edge [bend left] (n0);
			\path[arr]<10> (n3) edge [bend left] (n1);
			\path[arr]<10> (n3) edge [bend left] (n2);
		\end{tikzpicture}
	\end{frame}
	
		\begin{frame}[fragile]
			\frametitle{Relativization}
	\begin{lstlisting}
	functor f() = 
	struct
	  datatype t = S of int
	  type u = t
	  val x : u = S 1
	end
	\end{lstlisting}
		
		\begin{block}{How to represent u in the signature?}
		\[ \lambda().\{ \rho_t = \newx(0), \rho_u = \rho_t \} \]
		\[ x : \rho_u \] 	
		\end{block}
		\end{frame}
		
	\begin{frame}
		\frametitle{Relativization}
		\begin{enumerate}
			\item Value and definitional tycon bindings must be relativized
			\item Look up first occurrence of atomic tycons in entity environment, the entity path mapping to that occurrence is the canonical entity path
			\item Replace atomic tycons with canonical entity path, entity paths which always point to the current instantiation given the current entity environment
		\end{enumerate}
		
	\end{frame}
		
		\begin{frame}[fragile]
			\frametitle{Signature Extraction}
			\begin{lstlisting}
			struct
			datatype t
			structure A = struct type u = t end
			functor F(X:sig type s end) = struct type v = X.s end
			end
			\end{lstlisting}
			
			\begin{eqnarray*}
			  & & t\mapsto (\rho_0, \tau^0) \\
			  & & A\mapsto \{u\mapsto (\rho_1, \rho_0)\} \\
			  & & F\mapsto (\rho_2, \Pi\rho_3:\{s\mapsto (\rho_4, 0)\}.\{v\mapsto \lambda().\rho_3\rho_4\})
			\end{eqnarray*}
		\end{frame}


		\begin{frame}
			\frametitle{Elaboration}
		  	\begin{itemize}
			    \item<1->    $\Gamma,\Upsilon,\Sigma\vdash sigexp \Rightarrow_{sig} \Sigma'$  signature elaboration
			    \item<2->   $\Gamma,\Upsilon,\Sigma\vdash fsgexp \Rightarrow_{fsg} \Sigma^f$  functor signature elaboration
			    \item<3-> $\Upsilon^{clo},\Upsilon^{lcl}\vdash \Sigma \uparrow \Upsilon^{lcl}$  signature instantiation
			    \item<4-> $\Upsilon\vdash\Gamma\hookrightarrow \Sigma$ signature extraction
			    \item<5-> $\Upsilon\vdash(M,\varphi):\Sigma\Rightarrow_{match} (M_c,\varphi_c)$  signature matching
			    \item<6-> $\Upsilon\vdash F \preceq \Sigma^f \Rightarrow_{fsgmtch} (\psi_c, \theta_c)$ functor signature matching
			    \item<7-> $\Gamma,\Upsilon\vdash d^m \Rightarrow_{decl} (\eta,\Gamma',\Upsilon')$  module declaration elaboration
			    \item<8-> $\Gamma,\Upsilon\vdash strexp \Rightarrow_{str} (M, \varphi)$  structure expression elaboration
			\end{itemize}
			
		\end{frame}
\end{document}