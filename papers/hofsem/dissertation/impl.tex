%!TEX root = main.tex

\chapter{Implementation in SML/NJ}\label{ch:impl}

The semantics in this dissertation departs from the implementation (as
of SML/NJ 110.70) in
SML/NJ in a number of significant ways. Whereas the THO semantics
constructs entity expressions for all module expressions, SML/NJ
distinguishes between expressions based on whether they are inside a
functor. For those expressions outside of a functor, no entity
expression needs to be constructed since all tycons including
datatypes are nonvolatile because they have a single, fixed
instantiation. SML/NJ's distinction between volatile and nonvolatile
contexts (\ie, in functor or outside) is an optimization that skips
the construction of entity expressions during elaboration of all
expressions outside of any functor. 

A major difference is how relativization is performed. SML/NJ uses a
construct called the \emph{entity path context} to memoize the inverse
lookup of the entity environment for the canonical entity path to a
given entity. The entity path context must be threaded through the
entire elaboration process and complicates the elaboration process. 
 
For structure entities, SML/NJ assigns each a stamp to optimize
equality checks. In SML90, this stamp was used to implement structure
sharing, but the post-SML97 versions only use the stamp as a shortcut
during typechecking. Because SML/NJ supports sharing type constraints, signature
instantiation is more involved. It must resolve all sharing, which it
does using an algorithm adapted from the Patterson-Wegman linear unification algorithm \cite{PattersonWegman}.


%%% Local Variables: 
%%% mode: latex
%%% TeX-master: "main"
%%% End: 
