% -*- latex -*-
\documentclass[titlepage,letterpaper]{article}
\usepackage{times}
\usepackage{hyperref}

\marginparwidth0pt\oddsidemargin0pt\evensidemargin0pt\marginparsep0pt
\topmargin0pt\advance\topmargin by-\headheight\advance\topmargin by-\headsep
\textwidth6.7in\textheight9.1in
\columnsep0.25in

\newcommand{\smlmj}{110}
\newcommand{\smlmn}{46}

\author{Matthias Blume \\
Toyota Technological Institute at Chicago}

\title{{\bf NLFFI}\\
A new SML/NJ Foreign-Function Interface \\
{\it\small (for SML/NJ version \smlmj.\smlmn~and later)} \\
User Manual}

\setlength{\parindent}{0pt}
\setlength{\parskip}{6pt plus 3pt minus 2pt}

\newcommand{\nt}[1]{{\it #1}}
\newcommand{\tl}[1]{{\underline{\bf #1}}}
\newcommand{\ttl}[1]{{\underline{\tt #1}}}
\newcommand{\ar}{$\rightarrow$\ }
\newcommand{\vb}{~$|$~}

\begin{document}

\bibliographystyle{alpha}

\maketitle

\pagebreak

\tableofcontents

\pagebreak

%%%%%%%%%%%%%%%%%%%%%%%%%%%%%%%%%%%%%%%%%%%%%%%%%%%%%%%%%%%%%%%%%%%%%%%%%%
\section{Introduction}

Introduce...

%%%%%%%%%%%%%%%%%%%%%%%%%%%%%%%%%%%%%%%%%%%%%%%%%%%%%%%%%%%%%%%%%%%%%%%%%%
\section{The C Library}

The C library...

%%%%%%%%%%%%%%%%%%%%%%%%%%%%%%%%%%%%%%%%%%%%%%%%%%%%%%%%%%%%%%%%%%%%%%%%%%
\section{Translation conventions}

The {\tt ml-nlffigen} tool generates one ML structure for each
exported C definition.  In particular, there is one structure per
external variable, function, {\tt typedef}, {\tt struct}, {\tt union},
and {\tt enum}.
Each generated ML structure contains the ML type and values necessary
to manipulate the corresponding C item.

%-------------------------------------------------------------------------
\subsection{External variables}

An external C variable $v$ of type $t_C$ is represented by an ML
structure {\tt G\_}$v$.  This structure always contains a type {\tt t}
encoding $t_C$ and a value {\tt obj'} providing (``light-weight'')
access to the memory location that $v$ stands for in C.  If $t_C$ is
{\em complete}, then {\tt G\_}$v$ will also contain a value {\tt obj}
(the ``heavy-weight'' equivalent of {\tt obj'}) as well as value {\tt
  typ} holding run-time type information corresponding to $t_C$ (and
{\tt t}).

\paragraph*{Details}

\begin{description}\setlength{\itemsep}{0pt}
\item[{\tt type t}] is the type to be substituted for $\tau$ in {\tt
    ($\tau$, $\zeta$) C.obj} to yield the correct type for ML values
  representing C memory objects of type $t_C$ (i.e., $v$'s type).
  (This assumes a properly instantiated $\zeta$ based on whether or
  not the corresponding object was declared {\tt const}.)
\item[!{\tt val typ}] is the run-time type information corresponding
  to type {\tt t}.  The ML type of {\tt typ} is {\tt t C.T.typ}.  This
  value is not present if $t_C$ is {\em incomplete}.
\item[!{\tt val obj}] is a function that returns the ML-side
  representative of the C object (i.e., the memory location) referred
  to by $v$.  Depending on whether or not $v$ was declared {\tt
    const}, the type of {\tt obj} is either {\tt unit -> (t, C.ro)
    C.obj} or {\tt unit -> (t, C.rw) C.obj}.  The result of {\tt
    obj()} is ``heavy-weight,'' i.e., it implicitly carries run-time
  type information.  This value is not present if $t_C$ is {\em
    incomplete}.
\item[{\tt val obj'}] is analogous to {\tt val obj}, the only
  difference being that its result is ``light-weight,'' i.e., without
  run-time type information.  The type of {\tt val obj'} is
  either {\tt unit -> (t, C.ro) C.obj} or {\tt unit -> (t, C.rw) C.obj}.
\end{description}

\subsubsection*{Examples}

\begin{small}
\begin{center}
\begin{tabular}{c|c}
C declaration & signature of ML-side representation \\ \hline\hline
\begin{minipage}{2in}
\begin{verbatim}
extern int i;
\end{verbatim}
\end{minipage}
&
\begin{minipage}{4in}
\begin{verbatim}

structure G_i : sig
    type t   = C.sint
    val typ  : t C.T.typ
    val obj  : unit -> (t, C.rw) C.obj
    val obj' : unit -> (t, C.rw) C.obj'
end

\end{verbatim}
\end{minipage}
\\ \hline
\begin{minipage}{2in}
\begin{verbatim}
extern const double d;
\end{verbatim}
\end{minipage}
&
\begin{minipage}{4in}
\begin{verbatim}

structure G_d : sig
    type t   = C.double
    val typ  : t C.T.typ
    val obj  : unit -> (t, C.ro) C.obj
    val obj' : unit -> (t, C.ro) C.obj'
end

\end{verbatim}
\end{minipage}
\\ \hline
\begin{minipage}{2in}
\begin{verbatim}
extern struct str s1;
/* str complete */
\end{verbatim}
\end{minipage}
&
\begin{minipage}{4in}
\begin{verbatim}

structure G_s1 : sig
    type t   = (S_str.tag, rw) C.su_obj C.ptr
    val typ  : t C.T.typ
    val obj  : unit -> (t, C.rw) C.obj
    val obj' : unit -> (t, C.rw) C.obj'
end

\end{verbatim}
\end{minipage}
\\ \hline
\begin{minipage}{2in}
\begin{verbatim}
extern struct istr s2;
/* istr incomplete */
\end{verbatim}
\end{minipage}
&
\begin{minipage}{4in}
\begin{verbatim}

structure G_s2 : sig
    type t   = (ST_istr.tag, rw) C.su_obj C.ptr
    val obj' : unit -> (t, C.rw) C.obj'
end

\end{verbatim}
\end{minipage}
\end{tabular}
\end{center}
\end{small}

%-------------------------------------------------------------------------
\subsection{Functions}

An external C function $f$ is represented by an ML structure {\tt
  F\_}$f$.  Each such structure always contains at last three values:
{\tt typ}, {\tt fptr}, and {\tt f'}.  Variable {\tt typ} holds
run-time type information regarding function pointers that share $f$'s
prototype.  The most important part of this information is the code
that implements native C calling conventions for these functions.
Variable {\tt fptr} provides access to a C pointer to $f$.  And {\tt
  f'} is an ML function that dispatches a call of $f$ (through {\tt
  fptr}), using ``light-weight'' types for arguments and results.  If
the result type of $f$ is {\em complete}, then {\tt F\_}$f$ will also
contain a function {\tt f}, using ``heavy-weight'' argument- and
result-types.

\paragraph*{Details}

\begin{description}\setlength{\itemsep}{0pt}
\item[{\tt val typ}] holds run-time type information for pointers to
  functions of the same prototype.  The ML type of {\tt typ} is {\tt
    ($A$ -> $B$) C.fptr C.T.typ} where $A$ and $B$ are types encoding
  $f$'s argument list and result type, respectively.  A
  description of $A$ and $B$ is given below.
\item[{\tt val fptr}] is a function that returns the (heavy-weight)
  function pointer to $f$. The type of {\tt fptr} is {\tt unit -> ($A$
    -> $B$) C.fptr}.  The encodings of argument- and result types in
  $A$ and $B$ is the same as the one used for {\tt typ} (see below).
  Notice that although {\tt fptr} is a heavy-weight value carrying
  run-time type information, pointer arguments within $A$ or $B$ still
  use the light-weight version!
\item[!{\tt val f}] is an ML function that dispatches a call to $f$
  via {\tt fptr}.  For convenience, {\tt f} has built-in conversions
  for arguments (from ML to C) and the result (from C to ML).  For
  example, if $f$ has an argument of type {\tt double}, then {\tt f}
  will take an argument of type {\tt MLRep.Real.real} in its place and
  implicitly convert it to its C equivalent using {\tt
    C.Cvt.c\_double}.  Similarly, if $f$ returns an {\tt unsigned
    int}, then {\tt f} has a result type of {\tt MLRep.Unsigned.word}.
  This is done for all types that have a conversion function in
  {\tt C.Cvt}.
  Pointer values (as well as the object argument used for {\tt
    struct}- or {\tt union}-return values) are taken and returned in
  their heavy-weight versions.  Function {\tt f} will not be generated
  if the return type of $f$ is incomplete.
\item[{\tt val f'}] is the light-weight equivalent to {\tt f}.  a
  light-weight function.  The main difference is that pointer- and
  object-values are passed and returned in their light-weight
  versions.
\end{description}

\subsubsection*{Type encoding rules for {\tt ($A$ -> $B$) C.fptr}}

A C function $f$'s prototype is encoded as an ML type {\tt $A$ ->
  $B$}.  Calls of $f$ from ML take an argument of type $A$ and
produce a result of type $B$.

\begin{itemize}
\item Type $A$ is constructed from a sequence $\langle T_1, \ldots,
  T_k \rangle$ of types.  If that sequence is empty, then {\tt $A =$
    unit}; if the sequence has only one element $T_1$, then $A = T_1$.
  Otherwise $A$ is a tuple type {\tt $T_1$ * $\ldots$ * $T_k$}.
\item If $f$'s result is neither a {\tt struct} nor a {\tt union},
  then $T_1$ encodes the type of $f$'s first argument, $T_2$ that of
  the second, $T_3$ that of the third, and so on.
\item If $f$'s result is some {\tt struct} or some {\tt union}, then
  $T_1$ will be {\tt ($\tau$, C.rw) C.su\_obj'} with $\tau$
  instantiated to the appropriate {\tt struct}- or {\tt union}-tag
  type.  Moreover, we then also have $B = T_1$. $T_2$ encodes the type
  of $f$'s {\em first} argument, $T_3$ that of the second.  (In
  general, $T_{i+1}$ will encode the type of the $i$th argument of
  $f$ in this case.)
\item The encoding of the $i$th argument of $f$ ($T_i$ or $T_{i+1}$
  depending on $f$'s return type) is the light-weight ML equivalent of
  the C type of that argument.
\item An argument of C {\tt struct}- or {\tt union}-type corresponds
  to {\tt ($\tau$, C.ro) C.su\_obj'} with $\tau$ instantiated to the
  appropriate tag type.
\item If $f$'s result type is {\tt void}, then {\tt $B =$ unit}.  If
  the result type is not a {\tt struct}- or {\tt union}-type, then $B$
  is the light-weight ML encoding of that type.  Otherwise $B = T_1$
  (see above).
\end{itemize}

\subsubsection*{Examples}

\begin{small}
\begin{center}
\begin{tabular}{c|c}
C declaration & signature of ML-side representation \\ \hline\hline
{\tt void f1 (void);}
&
\begin{minipage}{4in}
\begin{verbatim}

structure F_f1 : sig
    val typ  : (unit -> unit) C.fptr C.T.typ
    val fptr : unit -> (unit -> unit) C.fptr
    val f    : unit -> unit
    val f'   : unit -> unit
end

\end{verbatim}
\end{minipage}
\\ \hline
{\tt int f2 (void);}
&
\begin{minipage}{4in}
\begin{verbatim}

structure F_f2 : sig
    val typ  : (C.sint -> unit) C.fptr C.T.typ
    val fptr : unit -> (C.sint -> unit) C.fptr
    val f    : MLRep.Signed.int -> unit
    val f'   : MLRep.Signed.int -> unit
end

\end{verbatim}
\end{minipage}
\\ \hline
{\tt void f3 (int);}
&
\begin{minipage}{4in}
\begin{verbatim}

structure F_f3 : sig
    val typ  : (unit -> C.sint) C.fptr C.T.typ
    val fptr : unit -> (unit -> C.sint) C.fptr
    val f    : unit -> MLRep.Signed.int
    val f'   : unit -> MLRep.Signed.int
end

\end{verbatim}
\end{minipage}
\\ \hline
{\tt void f4 (double, struct s*);}
&
\begin{minipage}{4in}
\begin{verbatim}

structure F_f4 : sig
    val typ  : (C.double *
                (ST_s.tag, C.rw) C.su_obj C.ptr'
                -> unit)
                    C.fptr C.T.typ
    val fptr : unit -> (C.double *
                        (ST_s.tag, C.rw) C.su_obj C.ptr'
                        -> unit) C.fptr
    val f    : MLRep.Real.real *
               (ST_s.tag, C.rw) C.su_obj C.ptr
               -> unit
    val f'   : MLRep.Real.real *
               (ST_s.tag, C.rw) C.su_obj C.ptr'
               -> unit
end

\end{verbatim}
\end{minipage}
\end{tabular}
\end{center}
\end{small}

\begin{small}
\begin{center}
\begin{tabular}{c|c}
C declaration & signature of ML-side representation \\ \hline\hline
\begin{minipage}{2in}
\begin{verbatim}
struct s *f5 (float);
/* s incomplete */
\end{verbatim}
\end{minipage}
&
\begin{minipage}{4in}
\begin{verbatim}

structure F_f5 : sig
    val typ  : (C.float
                -> (ST_s.tag, C.rw) C.su_obj C.ptr')
                    C.fptr C.T.typ
    val fptr : unit -> (C.float
                       -> (ST_s.tag, C.rw) C.su_obj C.ptr')
                           C.fptr
    val f'   : MLRep.Real.real ->
               (ST_s.tag, C.rw) C.su_obj C.ptr'
end

\end{verbatim}
\end{minipage}
\\ \hline
\begin{minipage}{2in}
\begin{verbatim}
struct t *f6 (float);
/* t complete */
\end{verbatim}
\end{minipage}
&
\begin{minipage}{4in}
\begin{verbatim}

structure F_f6 : sig
    val typ  : (C.float
                -> (S_t.tag, C.rw) C.su_obj C.ptr')
                    C.fptr C.T.typ
    val fptr : unit -> (C.float
                       -> (S_t.tag, C.rw) C.su_obj C.ptr')
                           C.fptr
    val f    : MLRep.Real.real ->
               (S_t.tag, C.rw) C.su_obj C.ptr
    val f'   : MLRep.Real.real ->
               (S_t.tag, C.rw) C.su_obj C.ptr'
end

\end{verbatim}
\end{minipage}
\\ \hline
\begin{minipage}{2in}
\begin{verbatim}
struct t f7 (int, double);
/* t complete */
\end{verbatim}
\end{minipage}
&
\begin{minipage}{4in}
\begin{verbatim}

structure F_f7 : sig
    val typ  : ((S_t.tag, C.rw) C.su_obj' *
                C.sint * C.double
                -> (S_t.tag, C.rw) C.su_obj')
                    C.fptr C.T.typ
    val fptr : unit -> ((S_t.tag, C.rw) C.su_obj' *
                        C.sint * C.double
                        -> (S_t.tag, C.rw) C.su_obj')
                            C.fptr
    val f    : (S_t.tag, C.rw) C.su_obj *
               MLRep.Signed.int *
               MLRep.Real.real
               -> (S_t.tag, C.rw) C.su_obj
    val f'   : (S_t.tag, C.rw) C.su_obj' *
               MLRep.Signed.int *
               MLRep.Real.real
               -> (S_t.tag, C.rw) C.su_obj'
end

\end{verbatim}
\end{minipage}
\end{tabular}
\end{center}
\end{small}

\subsection{Type definitions ({\tt typedef})}

In C a {\tt typedef} declaration associates a type name $t$ with a
type $t_C$.  On the ML side, $t$ is represented by an ML structure
{\tt T\_$t$}.  This structure contains a type abbreviation {\tt t} for
the ML encoding of $t_C$ and, provided $t_C$ is not {\em incomplete},
a value {\tt typ} of type {\tt t C.T.typ} with run-time type
information regarding $t_C$.

\subsubsection*{Examples}

\begin{small}
\begin{center}
\begin{tabular}{c|c}
C declaration & signature of ML-side representation \\ \hline\hline
\begin{minipage}{2in}
\begin{verbatim}
typedef int t1;
\end{verbatim}
\end{minipage}
&
\begin{minipage}{4in}
\begin{verbatim}

structure T_t1 : sig
    type t   = C.sint
    val typ  : t C.T.typ
end

\end{verbatim}
\end{minipage}
\\ \hline
\begin{minipage}{2in}
\begin{verbatim}
typedef struct s t2;
/* s incomplete */
\end{verbatim}
\end{minipage}
&
\begin{minipage}{4in}
\begin{verbatim}

structure T_t2 : sig
    type t  = ST_s.tag C.su
end

\end{verbatim}
\end{minipage}
\\ \hline
\begin{minipage}{2in}
\begin{verbatim}
typedef struct s *t3;
/* s incomplete */
\end{verbatim}
\end{minipage}
&
\begin{minipage}{4in}
\begin{verbatim}

structure T_t3 : sig
    type t  = (ST_s.tag, C.rw) C.su_obj C.ptr
end

\end{verbatim}
\end{minipage}
\\ \hline
\begin{minipage}{2in}
\begin{verbatim}
typedef struct t t4;
/* t complete */
\end{verbatim}
\end{minipage}
&
\begin{minipage}{4in}
\begin{verbatim}

structure T_t4 : sig
    type t  = ST_t.tag C.su
    val typ : t T.typ
end

\end{verbatim}
\end{minipage}
\end{tabular}
\end{center}
\end{small}

\subsection{{\tt struct} and {\tt union}}

...

\subsection{Enumerations ({\tt enum})}

...

%%%%%%%%%%%%%%%%%%%%%%%%%%%%%%%%%%%%%%%%%%%%%%%%%%%%%%%%%%%%%%%%%%%%%%%%%%
%\appendix
%% -*- latex -*-

\section{CM description file syntax}

\subsection{Lexical Analysis}

The CM parser employs a context-sensitive scanner.  In many cases this
avoids the need for ``escape characters'' or other lexical devices
that would make writing description files cumbersome.  (The downside
of this is that it increases the complexity of both documentation and
implementation.)

The scanner skips all nestable SML-style comments (enclosed with {\bf
(*} and {\bf *)}).

Lines starting with {\bf \#line} may list up to three fields separated
by white space.  The first field is taken as a line number and the
last field (if more than one field is present) as a file name.  The
optional third (middle) field specifies a column number.  A line of
this form resets the scanner's idea about the name of the file that it
is currently processing and about the current position within that
file.  If no file is specified, the default is the current file.  If
no column is specified, the default is the first column of the
(specified) line.  This feature is meant for program-generators or
tools such as {\tt noweb} but is not intended for direct use by
programmers.

The following lexical classes are recognized:

\begin{description}
\item[Namespace specifiers:] {\bf structure}, {\bf signature},
{\bf functor}, or {\bf funsig}.  These keywords are recognized
everywhere.
\item[CM keywords:] {\bf group}, {\bf Group}, {\bf GROUP}, {\bf
library}, {\bf Library}, {\bf LIBRARY}, {\bf is}, {\bf IS}.  These
keywords are recognized everywhere except within ``preprocessor''
lines (lines starting with {\bf \#}) or following one of the namespace
specifiers.
\item[Preprocessor control keywords:] {\bf \#if}, {\bf \#elif}, {\bf
\#else}, {\bf \#endif}, {\bf \#error}.  These keywords are recognized
only at the beginning of the line and indicate the start of a
``preprocessor'' line.  The initial {\bf \#} character may be
separated from the rest of the token by white space (but not by comments).
\item[Preprocessor operator keywords:] {\bf defined}, {\bf div}, {\bf
mod}, {\bf andalso}, {\bf orelse}, {\bf not}.  These keywords are
recognized only when they occur within ``preprocessor'' lines.  Even
within such lines, they are not recognized as keywords when they
directly follow a namespace specifier---in which case they are
considered SML identifiers.
\item[SML identifiers (\nt{mlid}):] Recognized SML identifiers
include all legal identifiers as defined by the SML language
definition. (CM also recognizes some tokens as SML identifiers that
are really keywords according to the SML language definition. However,
this can never cause problems in practice.)  SML identifiers are
recognized only when they directly follow one of the namespace
specifiers.
\item[CM identifiers (\nt{cmid}):] CM identifiers have the same form
as those ML identifiers that are made up solely of letters, decimal
digits, apostrophes, and underscores.  CM identifiers are recognized when they
occur within ``preprocessor'' lines, but not when they directly follow
some namespace specifier.
\item[Numbers (\nt{number}):] Numbers are non-empty sequences of
decimal digits.  Numbers are recognized only within ``preprocessor''
lines.
\item[Preprocessor operators:] The following unary and binary operators are
recognized when they occur within ``preprocessor'' lines: {\tt +},
{\tt -}, {\tt *}, {\tt /}, {\tt \%}, {\tt <>}, {\tt !=}, {\tt <=},
{\tt <}, {\tt >=}, {\tt >}, {\tt ==}, {\tt =}, $\tilde{~}$, {\tt
\&\&}, {\tt ||}, {\tt !}.  Of these, the following (``C-style'')
operators are considered obsolete and trigger a warning
message\footnote{The use of {\tt -} as a unary minus also triggers
this warning.} as long as {\tt CM.Control.warn\_obsolete} is set to
{\tt true}: {\tt /}, {\tt \%}, {\tt !=}, {\tt ==}, {\tt \&\&}, {\tt
||}, {\tt !}.
\item[Standard path names (\nt{stdpn}):] Any non-empty sequence of
upper- and lower-case letters, decimal digits, and characters drawn
from {\tt '\_.;,!\%\&\$+/<=>?@$\tilde{~}$|\#*-\verb|^|} that occurs
outside of ``preprocessor'' lines and is neither a namespace specifier
nor a CM keyword will be recognized as a stardard path name.  Strings
that lexically constitute standard path names are usually---but not
always---interpreted as file names. Sometimes they are simply taken as
literal strings.  When they act as file names, they will be
interpreted according to CM's {\em standard syntax} (see
Section~\ref{sec:basicrules}).  (Member class names, names of
privileges, and many tool optios are also specified as standard path
names even though in these cases no actual file is being named.)
\item[Native path names (\nt{ntvpn}):] A token that has the form of an
SML string is considered a native path name.  The same rules as in SML
regarding escape characters apply.  Like their ``standard''
counterparts, native path names are not always used to actually name
files, but when they are, they use the native file name syntax of the
underlying operating system.
\item[Punctuation:] A colon {\bf :} is recognized as a token
everywhere except within ``preprocessor'' lines. Parentheses {\bf ()}
are recognized everywhere.
\end{description}

\subsection{EBNF for preprocessor expressions}

\noindent{\em Lexical conventions:}\/ Syntax definitions use {\em
Extended Backus-Naur Form} (EBNF).  This means that vertical bars
\vb separate two or more alternatives, curly braces \{\} indicate
zero or more copies of what they enclose (``Kleene-closure''), and
square brackets $[]$ specify zero or one instances of their enclosed
contents.  Round parentheses () are used for grouping.  Non-terminal
symbols appear in \nt{this}\/ typeface; terminal symbols are
\tl{underlined}.

\noindent The following set of rules defines the syntax for CM's
preprocessor expressions (\nt{ppexp}):

\begin{tabular}{rcl}
\nt{aatom}  &\ar& \nt{number} \vb \nt{cmid} \vb \tl{(} \nt{asum} \tl{)} \vb (\ttl{$\tilde{~}$} \vb \ttl{-}) \nt{aatom} \\
\nt{aprod}  &\ar& \{\nt{aatom} (\ttl{*} \vb \tl{div} \vb \tl{mod}) \vb \ttl{/} \vb \ttl{\%} \} \nt{aatom} \\
\nt{asum}   &\ar& \{\nt{aprod} (\ttl{+} \vb \ttl{-})\} \nt{aprod} \\
\\
\nt{ns}     &\ar& \tl{structure} \vb \tl{signature} \vb \tl{functor} \vb \tl{funsig} \\
\nt{mlsym}  &\ar& \nt{ns} \nt{mlid} \\
\nt{query}  &\ar& \tl{defined} \tl{(} \nt{cmid} \tl{)} \vb \tl{defined} \tl{(} \nt{mlsym} \tl{)} \\
\\
\nt{acmp}   &\ar& \nt{asum} (\ttl{<} \vb \ttl{<=} \vb \ttl{>} \vb \ttl{>=} \vb \ttl{=} \vb \ttl{==} \vb \ttl{<>} \vb \ttl{!=}) \nt{asum} \\
\\
\nt{batom}  &\ar& \nt{query} \vb \nt{acmp} \vb (\tl{not} \vb \ttl{!}) \nt{batom} \vb \tl{(} \nt{bdisj} \tl{)} \\
\nt{bcmp}   &\ar& \nt{batom} [(\ttl{=} \vb \ttl{==} \vb \ttl{<>} \vb \ttl{!=}) \nt{batom}] \\
\nt{bconj}  &\ar& \{\nt{bcmp} (\tl{andalso} \vb \ttl{\&\&})\} \nt{bcmp} \\
\nt{bdisj}  &\ar& \{\nt{bconj} (\tl{orelse} \vb \ttl{||})\} \nt{bconj} \\
\\
\nt{ppexp} &\ar& \nt{bdisj}
\end{tabular}

\subsection{EBNF for export lists}

The following set of rules defines the syntax for export lists (\nt{elst}):

\begin{tabular}{rcl}
\nt{guardedexports} &\ar& \{ \nt{export} \} (\tl{\#endif} \vb
\tl{\#else} \{ \nt{export} \} \tl{\#endif} \vb \tl{\#elif} \nt{ppexp}
\nt{guardedexports}) \\
\nt{restline}      &\ar& rest of current line up to next newline character \\
\nt{export}        &\ar& \nt{mlsym} \vb \tl{\#if} \nt{ppexp}
\nt{guardedexports} \vb \tl{\#error} \nt{restline}  \\
\nt{elst}       &\ar& \nt{export} \{ \nt{export} \} \\
\end{tabular}

\subsection{EBNF for tool options}

The following set of rules defines the syntax for tool options
(\nt{toolopts}):

\begin{tabular}{rcl}
\nt{pathname} &\ar& \nt{stdpn} \vb \nt{ntvpn} \\
\nt{toolopts} &\ar& \{ \nt{pathname} [\tl{:} (\tl{(} \nt{toolopts} \tl{)} \vb \nt{pathname})] \}
\end{tabular}

\subsection{EBNF for member lists}

The following set of rules defines the syntax for member lists (\nt{members}):

\begin{tabular}{rcl}
\nt{class}          &\ar& \nt{stdpn} \\
\nt{member}         &\ar& \nt{pathname} [\tl{:} \nt{class}] [\tl{(} \nt{toolopts} \tl{)}] \\
\nt{guardedmembers} &\ar& \nt{members} (\tl{\#endif} \vb \tl{\#else} \nt{members} \tl{\#endif} \vb \tl{\#elif} \nt{ppexp} \nt{guardedmembers}) \\
\nt{members}        &\ar& \{ (\nt{member} \vb \tl{\#if} \nt{ppexp}
\nt{guardedmembers} \vb \tl{\#error} \nt{restline}) \} 
\end{tabular}

\subsection{EBNF for library descriptions}

The following set of rules defines the syntax for library descriptions
(\nt{library}).  Notice that although the syntax used for \nt{version}
is the same as that for \nt{stdpn}, actual version strings will
undergo further analysis according to the rules given in
section~\ref{sec:versions}:

\begin{tabular}{rcl}
\nt{libkw}     &\ar& \tl{library} \vb \tl{Library} \vb \tl{LIBRARY} \\
\nt{version}   &\ar& \nt{stdpn} \\
\nt{privilege} &\ar& \nt{stdpn} \\
\nt{lprivspec} &\ar& \{ \nt{privilege} \vb \tl{(} \{ \nt{privilege} \} \tl{)} \} \\
\nt{library}  &\ar& [\nt{lprivspec}] \nt{libkw} [[\tl{(} \nt{version} \tl{)}] \nt{elst}] \nt{iskw} \nt{members}
\end{tabular}

\subsection{EBNF for library component descriptions (group descriptions)}

The main differences between group- and library-syntax can be
summarized as follows:

\begin{itemize}\setlength{\itemsep}{0pt}
\item Groups use keyword \tl{group} instead of \tl{library}.
\item Groups may have an empty export list.
\item Groups cannot wrap privileges, i.e., names of privileges (in
front of the \tl{group} keyword) never appear within parentheses.
\item Groups have no version.
\item Groups have an optional owner.
\end{itemize}

\noindent The following set of rules defines the syntax for library
component (group) descriptions (\nt{group}):

\begin{tabular}{rcl}
\nt{groupkw}   &\ar& \tl{group} \vb \tl{Group} \vb \tl{GROUP} \\
\nt{owner}     &\ar& \nt{pathname} \\
\nt{gprivspec} &\ar& \{ \nt{privilege} \} \\
\nt{group}     &\ar& [\nt{gprivspec}] \nt{groupkw} [\tl{(} \nt{owner} \tl{)}] [\nt{elst}] (\tl{is} \vb \tl{IS}) \nt{members}
\end{tabular}


\bibliography{blume,appel,ml}

\end{document}
