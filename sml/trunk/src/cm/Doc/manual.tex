\documentclass[titlepage,letterpaper]{article}
\usepackage{times}
\usepackage{epsfig}

\marginparwidth0pt\oddsidemargin0pt\evensidemargin0pt\marginparsep0pt
\topmargin0pt\advance\topmargin by-\headheight\advance\topmargin by-\headsep
\textwidth6.7in\textheight9.1in
\columnsep0.25in

\newcommand{\smlmj}{110}
\newcommand{\smlmn}{29}

\author{Matthias Blume \\
Research Institute for Mathematical Sciences \\
Kyoto University}

\title{{\bf CM}\\
The SML/NJ Compilation and Library Manager \\
{\it\small (for SML/NJ version \smlmj.\smlmn~and later)} \\
User Manual}

\setlength{\parindent}{0pt}
\setlength{\parskip}{6pt plus 3pt minus 2pt}

\newcommand{\nt}[1]{{\it #1}}
\newcommand{\tl}[1]{{\underline{\bf #1}}}
\newcommand{\ttl}[1]{{\underline{\tt #1}}}
\newcommand{\ar}{$\rightarrow$\ }
\newcommand{\vb}{~$|$~}

\begin{document}

\bibliographystyle{alpha}

\maketitle

\pagebreak

\tableofcontents

\pagebreak

\section{Introduction}

This manual describes a new implementation of CM, the ``Compilation
and Library Manager'' for Standard ML of New Jersey (SML/NJ).  Like its
previous version, CM is in charge of managing separate compilation and
facilitates access to stable libraries.

Programming projects that use CM are typically composed of separate
{\em libraries}.  Libraries are collections of ML compilation units
and themselves can be internally sub-structured using CM's notion of
{\em library components}.

CM offers the following features to the programmer:

\begin{itemize}
\item separate compilation and type-safe linking~\cite{appel94:sepcomp}
\item hierarchical modularity~\cite{blume:appel:cm99}
\item automatic dependency analysis~\cite{blume:depend99}
\item optimization of the compilation process via {\em
cutoff}-recompilation techniques~\cite{tichy94}
\item management of program libraries, distinguishing between libraries
that are {\em under development} and libraries that are {\em stable}
\item operating-system independent file naming (with optional escape
to native file names)
\item adaptability to changing environments using the {\em path anchor}
facility
\item an extensible set of auxiliary {\em tools} that lets CM
seemlessly interoperate with other program generators, source-control
systems, literate-programming facilities, or shell-scripts
\item checked version numbers on libraries
\item (still rudimentary) support for access control
\item access to libraries from the interactive toplevel loop
\item sharing of code that is common to several programs or code that
is common to both user programs and SML/NJ itself
\item management of (the sharing of) link-time state
\item support for parallel and distributed compilation
\item an API to access all these facilities at SML/NJ's interactive
prompt
\end{itemize}

CM puts emphasis on {\em working with libraries}.  This contrasts with
previous compilation managers for SML/NJ where the focus was on
compilation management while libraries were added as an afterthought.

\section{The CM model}

\subsection{Basic philosophy}

The venerable {\bf make} of Unix~\cite{feldman79} is {\em
target-oriented}: one starts with a main goal (target) and applies
production rules (with associated actions such as invoking a compiler)
until no more rules are applicable. The leaves of the resulting
derivation tree can be seen as defining the set of source files that
are used to make the main target.

CM, on the other hand, is largely {\em source-oriented}: Whereas with
{\bf make} one specifies the tree and let the program derive the
leaves, with CM one specifies the leaves and let the program derive
the tree.  Thus, the programmer writes down a list of sources, and CM
will calculate and then execute a series of steps to make the
corresponding program.  In {\bf make} terminology, this resulting
program acts as the ``main goal'', but it does not need to be
explicitly named.  In fact, since there typically is no corresponding
single file system object, a ``natural'' name for it does not even
exist.

For simple projects it is literally true that all the programmer has
to do is to tell CM about the set of sources: a description file lists
little more than the names of participating ML source files. Larger
projects typically benefit from a hierarchical structuring.  This can
be achieved by grouping ML source files into separate libraries and
library components.  Dependencies between such libraries have to be
specified explicitly and must form an acyclic directed graph (DAG).

CM's own semantics, particularly its dependency analysis, interact
with the ML language in such a way that for any well-formed project
there will be exactly one possible interpretation as far as static
semantics are concerned.  Only well-formed projects are accepted by
CM; projects that are not well-formed will cause error messages.

\subsection{Description files}

Technically, a CM library is a (possibly empty) collection of ML
source files and may also contain references to other libraries.  Each
library comes with an explicit export interface which lists all
toplevel-defined symbols of the library that shall be exported to its
clients.  A library is described by the contents of its {\em
description file}.\footnote{The description file may also contain
references to input files for {\em tools} like {\tt ml-lex} or {\tt
ml-yacc} that produce ML source files.  See section~\ref{sec:tools}.}

\noindent Example:

\begin{verbatim}
Library
    signature BAR
    structure Foo
is
    bar.sig
    foo.sml
    helper.sml

    $/basis.cm
    $/smlnj-lib.cm
\end{verbatim}

This library exports two definitions, one for a structure named {\tt
Foo} and one for a signature named {\tt BAR}.  The specification for
such exports appear between the keywords {\tt Library} and {\tt is}.
The {\em members} of the library are specified after the keyword {\tt
is}.  Here we have three ML source files ({\tt bar.sig}, {\tt
foo.sml}, and {\tt helper.sml}) as well as references to two external
libraries ({\tt \$/basis.cm} and {\tt \$/smlnj-lib.cm}).  The entry
{\tt \$/basis.cm} typically denotes the description file for the {\it
Standard ML Basis Library}~\cite{reppy99:basis}; most programs will
want to list it in their own description file(s).  The other library
in this example ({\tt \$/smlnj-lib.cm}) is a library of data
structures and algorithms that comes bundled with SML/NJ.

\subsection{Members of a library}

Members of a library do not have to be listed in any particular order
since CM will automatically calculate the dependency graph.  Some
minor restrictions on the source language are necessary to make this
work:
\begin{enumerate}
\item All top-level definitions must be {\em module} definitions
(structures, signatures, functors, or functor signatures).  In other
words, there can be no top-level type-, value-, or infix-definitions.
\item For a given symbol, there can be at most one ML source file per
library (or---more correctly---one file per library component; see
Section~\ref{sec:groups}) that defines the symbol at top level.
\item If more than one sub-library or sub-group is exporting the same
symbol, then the definition (i.e., the ML source file that actually
defines the symbol) must be identical in all cases.
\label{rule:diamond}
\item The use of ML's {\bf open} construct is not permitted at the top
level of ML files compiled by CM.  (The use is still ok at the
interactive top level.)
\end{enumerate}

Note that these rules do not require the exports of sub-groups or
sub-libraries to be distinct from the exports of ML source files in
the current library or group.  If an ML source file re-defines an
imported name, then the disambiguating rule is that the definition
from the ML source takes precedence over the definition imported from
the group or library.

Rule~\ref{rule:diamond} may come as a bit of a surprise considering
that each ML source file can be a member of at most one group or
library (see section~\ref{sec:multioccur}).  However, it is indeed
possible for two libraries to (re-)export the ``same'' definition
provided they both import that definition from a third library.  For
example, let us assume that {\tt a.cm} exports a structure {\tt X}
which was defined in {\tt x.sml}---one of {\tt a.cm}'s members.  Now,
if both {\tt b.cm} and {\tt c.cm} re-export that same structure {\tt
X} after importing it from {\tt a.cm}, it is legal for a fourth
library {\tt d.cm} to import from both {\tt b.cm} and {\tt c.cm}.

The full syntax for library description files also includes provisions
for a simple ``conditional compilation'' facility (see
Section~\ref{sec:preproc}), for access control (see
Section~\ref{sec:access}), and it accepts ML-style nestable comments
delimited by \verb|(*| and \verb|*)|.

\subsection{Name visibility}

In general, all definitions exported from members (i.e., ML source
files, subgroups and sublibraries) of a library are visible in all ML
source files of that library.  The source code in those source files
can refer to them directly without further qualification.  Here,
``exported'' means either a top-level definition within an ML source
file or a definition listed in a sublibrary's export list.

If a library is structured into library components using {\em groups}
(see Section~\ref{sec:groups}), then---as far as name visibility is
concerned---each component (group) is treated like a separate library.

Cyclic dependencies among libraries, library components, or ML source
files within a library are detected and flagged as errors.

\subsection{Groups}
\label{sec:groups}

CM's group model eliminates a whole class of potential naming problems
by providing control over name spaces for program linkage.  The group
model in full generality sometimes requires bindings to be renamed at
the time of import. As has been described
separately~\cite{blume:appel:cm99}, in the case of ML this can also be
achieved using ``administative'' libaries, which is why CM can get
away with not providing more direct support for renaming.

However, under CM, the term ``library'' does not only mean namespace
management (as it would from the point of view of the pure group
model) but also refers to actual file system objects (e.g., CM
description files and stable library files).  It would be inconvenient
if name resolution problems would result in a proliferation of
additional library files.  Therefore, CM also provides the notion of
groups (or: ``library components'').  Name resolution for groups works
like name resolution for entire libraries, but grouping is entirely
internal to each library.

When a library is {\em stabilized} (via {\tt CM.stabilize} -- see
Section~\ref{sec:api}), the entire library is compiled to a single
file (hence groups do not result in separate stable files).

During development, each group has its own description file which will
be referred to by the surrounding library or by other groups of that
library. The syntax of group description files is the same as that of
library description files with the following exceptions:

\begin{itemize}
\item The initial keyword {\tt Library} is replaced with {\tt Group}.
It is followed by the name of the surrounding library's description
file in parentheses.
\item The export list can be left empty, in which case CM will provide
a default export list: all exports from ML source files plus all
exports from subcomponents of the component.  (Note that this does not
include the exports of other libraries.)
\item There are some small restrictions on access control
specifications (see Section~\ref{sec:access}).
\end{itemize}

As an example, let us assume that
{\tt foo-utils.cm} contains the following text:

%note: emacs gets temporarily confused by the single dollar
\begin{verbatim}
Group (foo-lib.cm)
is
    set-util.sml
    map-util.sml
    $/basis.cm
\end{verbatim}

This description defines group {\tt foo-utils.cm} to have the
following properties:

\begin{itemize}
\item it is a component of library {\tt foo-lib.cm} (meaning that only
foo-lib.cm itself or other groups thereof may list {\tt foo-utils.cm} as one
of their members)
\item {\tt set-utils.sml} and {\tt map-util.sml} are ML source files
belonging to this component
\item exports from the Standard Basis Library are available when
compiling these ML source files
\item since the export list has been left blank, the only (implicitly
specified) exports of this component are the top-level definitions in
its ML source files
\end{itemize}

With this, the library description file {\tt foo-lib.cm} could list
{\tt foo-utils.cm} as one of its members:

\begin{verbatim}
Library
    signature FOO
    structure Foo
is
    foo.sig
    foo.sml
    foo-utils.cm
    $/basis.cm
\end{verbatim}
%note: emacs should be sufficiently un-confused again by now

No harm is done if {\tt foo-lib.cm} does not actually mention {\tt
foo-utils.cm}.  In this case it could be that\linebreak {\tt
foo-utils.cm} is mentioned indirectly via a chain of other components
of {\tt foo-lib.cm}.  The other possibility is that it is not
mentioned at all (in which case CM would never know about it, so it
cannot complain).

\subsection{Multiple occurences of the same member}
\label{sec:multioccur}

The following rules apply to multiple occurences of the same ML source
file, the same library, or the same group within a program:

\begin{itemize}
\item Within the same description file, each member can be specified
at most once.
\item Libraries can be referred to freely from as many other groups or
libraries as the programmer desires.
\item A group cannot be used from outside the uniquely defined library
(as specified in its description file) of which it is a component.
However, within that library it can be referred to from arbitrarily
many other groups.
\item The same ML source file cannot appear more than once.  If an ML
source file is to be referred to by multiple clients, it must first be
``wrapped'' into a library (or---if all references are from within the
same library---a group).
\end{itemize}

\subsection{Top-level groups}

Mainly to facilitate some superficial backward-compatibility, CM also
allows groups to appear at top level, i.e., outside of any library.
Such groups must omit the parenthetical library specification and then
cannot also be used within libraries. One could think of the top level
itself as a ``virtual unnamed library'' whose components are these
top-level groups.

\section{Naming objects in the file system}

\subsection{Motivation}

File naming has been an area notorious for its problems and was the
cause of most of the gripes from CM's users.  With this in mind, CM
now takes a different approach to file name resolution.

The main difficulty lies in the fact that files or even whole
directories may move after CM has already partially (but not fully)
processed them.  For example, this happens when the {\em autoloader}
(see Section~\ref{sec:autoload}) has been invoked and the session
(including CM's internal state) is then frozen (i.e., saved to a file)
via {\tt SMLofNJ.exportML}.  The new CM is now able to resume such a
session even when operating in a different environment, perhaps on a
different machine with different file system mounted, or a different
location of the SML/NJ installation.

To make this possible, CM provides a configurable mechanism for
locating file system objects.  Moreover, it invokes this mechanism
always as late as possible and is prepared to re-invoke it after the
configuration changes.

\subsection{Basic rules}
\label{sec:basicrules}

CM uses its own ``standard'' syntax for pathnames which for the most
part happens to be the same as the one used by most Unix-like systems:
\begin{itemize}
\item Path name components are separated by ``{\bf /}''.
\item Special components ``{\bf .}'' and ``{\bf ..}'' denote {\em
current} and {\em previous} directory, respectively.
\item Paths beginning
with ``{\bf /}'' are considered {\em absolute}.
\item Other paths are {\em relative} unless they start with ``{\bf \$}''.
\end{itemize}
\noindent There is an important third form of standard paths: {\em
anchored} paths.  Anchored paths always start with ``{\bf \$}''.

Since this standard syntax does not cover system-specific aspects such
as volume names, it is also possible to revert to ``native'' syntax by
enclosing a path name in double-quotes.  Of course, description files
that use path names in native syntax are not portable across operating
systems.

\begin{description}
\item[Absolute pathnames] are resolved in the usual manner
specific to the operating system.  However, it is advisable to avoid
absolute pathnames because they are certain to ``break'' if the
corresponding file moves to a different location.
\item[Relative pathnames that occur in some CM description file] whose
name is {\it path}{\tt /}{\it file}{\tt .cm} will be resolved relative
to {\it path}, i.e., relative to the directory that contains the
description file.
\item[Relative pathnames that have been entered interactively,] for
example as an argument to one of CM's interface functions,
will be resolved in the OS-specific manner, i.e., relative to the
current working directory.  However, CM will internally represent the
name in such a way that it remembers the corresponding working
directory.  Should the working directory change during an ongoing CM
session while there still is a reference to the name, then CM will
switch its mode of operation and prepend the path of the original
working directory. As a result, two names specified using identical
strings but at different times when different working directories were
in effect will be kept distinct and continue to refer to the file
system locations that they referred to when they were first seen.
\item[Anchored paths] consist of an anchor name (of non-zero length)
and a non-empty list of additional arcs.  The name is enclosed by
the path's leading {\bf \$} on the left and the path's first {\bf /}
on the right.  The list of arcs follows the first {\bf /}.  As with
all standard paths, the arcs themselves are also separated by {\bf /}.
An error is signalled if the anchor name is not known to CM.
If $a$ is a know anchor name currently bound to some directory name
$d$, then the standard path {\tt \$}$a${\tt /}$p$ (where $p$ is a list
of arcs) refers to $d${\tt /}$p$.  The frequently occuring case where
$a$ coincides with the first arc of $p$ can be abbreviated as {\tt
\$/}$p$.
\end{description}

\subsection{Anchor environments}
\label{sec:anchor:env}

Anchor names are resolved in the {\em anchor environment} that is in
effect at the time the anchor is read.

The basis for all anchor environments is the {\em root environment}.
Conceptually, the root environments is a fixed mapping that binds
every possible anchor to a mutable location.  The location can store a
native directory name or can be marked ``undefined''.  Most locations
initially start out undefined.  The contents of each location is
configurable (see Section~\ref{sec:anchor:config}).

At the time a CM description file $a${\tt .cm} refers to another
library's or library component's description file $b${\tt .cm}, it can
augment the current anchor environment with new bindings.  The new
bindings are in effect while $b${\tt .cm} (including any description
files {\it it}\/ mentions!) is being processed.  If a new binding
binds an anchor name that was already bound in the current
environment\footnote{which is technically always the case given our
explanation of the root environment}, then the old binding is being
hidden.  The effect is scoping for anchor names.

Using CM's {\em tool parameter} mechanism (see
Section~\ref{sec:toolparam}), a new binding is specified as a pair of
anchor name and anchor value.  The value has the form of another path
name (standard or native). Example:

\begin{verbatim}
  a.cm (bind:(anchor:lib value:$mystuff/a-lib)
        bind:(anchor:support value:$lib)
        bind:(anchor:utils value:/home/bob/stuff/ML/utils))
\end{verbatim}

As shown in this example, it is perfectly legal for the specification
of the value to involve the use of another anchor.  That anchor will
be resolved in the original anchor environment. Thus, a path anchored
at {\tt \$lib} in {\tt a.cm} will be resolved using the binding for
{\tt \$mystuff} that is currently in effect.  The point here is that a
re-configuration of the root environment that affects {\tt \$mystuff}
now also affects how {\tt \$lib} is resolved as it occurs within {\tt
a.cm}.

The list of {\tt bind}-directives is processed ``in parallel,'' which
means that {\tt \$support} is {\em not} being bound to\linebreak {\tt
\$mystuff/a-lib/asupport} but will refer to the original meaning of
{\tt \$lib}.

The example also demonstrates that {\tt value}-paths can be single
anchors. In other words, the restriction that there has to be at least
one arc after the anchor does not apply here. This makes it possible
to ``rename'' anchors, or, to put it more precisely, for one anchor
name to be established as an ``alias'' for another anchor name.

\subsection{Anchor configuration}
\label{sec:anchor:config}

Anchor configuration is concerned with the values that are stored in
the root anchor environment.  At startup time, the root environment is
initialized by reading two configuration files: an
installation-specific one and a user-specific one.  After that, the
contents of root locations can be maintained using CM's interface
functions {\tt CM.Anchor.anchor} and {\tt CM.Anchor.reset} (see
Section~\ref{sec:api}).

Although there is a hard-wired default for the installation-specific
configuration file\footnote{which happens to be {\tt
/usr/lib/smlnj-pathconfig}}, this default in rarely being used.
Instead, in a typical installation of SML/NJ the default will be a
file $r${\tt /lib/pathconfig} where $r$ is the {\it root} directory
into which SML/NJ had been installed.  (The installation procedure
establishes this new default by setting the environment variable {\tt
CM\_PATHCONFIG\_DEFAULT} at the time it produces the heap image for
the interactive system.)  The user can specify a new location at
startup time using the environment variable {\tt CM\_PATHCONFIG}.

The default location of the user-specific configuration file is {\tt
.smlnj-pathconfig} in the user's home directory (which must be given
by the {\tt HOME} environment variable).  At startup time, this
default can be overridden by a fixed location which must be given as
the value of the environment variable {\tt CM\_LOCAL\_PATHCONFIG}.

The syntax of all configuration files is identical.  Lines are
processed from top to bottom. White space divides lines into tokens.
\begin{itemize}
\item A line with exactly two tokens associates an anchor (the first
token) with a directory in native syntax (the second token).  Neither
anchor nor directory name may contain white space and the anchor
should not contain a {\bf /}.  If the directory name is a relative
name, then it will be expanded by prepending the name of the directory
that contains the configuration file.
\item A line containing exactly one token that is the name of an
anchor cancels any existing association of that anchor with a
directory.
\item A line with a single token that consists of a single minus sign
{\bf -} cancels all existing anchors.  This typically makes sense only
at the beginning of the user-specific configuration file and
erases any settings that were made by the installation-specific
configuration file.
\item Lines with no token (i.e., empty lines) will be silently ignored.
\item Any other line is considered malformed and will cause a warning
but will otherwise be ignored.
\end{itemize}

\section{Using CM}

\subsection{Structure CM}
\label{sec:api}

Functions that control CM's operation are accessible as members of a
structure named {\tt CM}.  This structure itself is exported from a
library called {\tt \$smlnj/cm/full.cm} (or, alternatively, {\tt
\$smlnj/cm.cm}).  This library is pre-registered for auto-loading at
the interactive top level.

Other libraries can exploit CM's functionality simply by putting a
{\tt \$smlnj/cm/full.cm} entry into their own description file.
Section~\ref{sec:dynlink} shows one interesting use of this feature.

Here is a description of all members:

\subsubsection*{Compiling}

Two main activities when using CM are to compile ML source code and to
build stable libraries:

\begin{verbatim}
  val recomp : string -> bool
  val stabilize : bool -> string -> bool
\end{verbatim}

{\tt CM.recomp} takes the name of a program's ``root'' description
file and compiles or recompiles all ML source files that are necessary
to provide definitions for the root library's export list.  ({\em
Note:} The difference to {\tt CM.make} is that no linking takes
place.)

{\tt CM.stabilize} takes a boolean flag and then the name of a library
and {\em stabilizes} this library.  A library is stabilized by writing
all information pertaining to it, including all of its library
components (i.e., subgroups), into a single file.  Sublibraries do not
become part of the stabilized library; CM records stub entries for them.
When a stabilized library is used in other programs, all members of
the library are guaranteed to be up-to-date; no dependency analysis
work and no recompilation work will be necessary.  If the boolean flag
is {\tt false}, then all sublibraries of the library must already be
stable.  If the flag is {\tt true}, then CM will recursively stabilize
all libraries reachable from the given root.

After a library has been stabilized it can be used even if none of its
original sources---including the description file---are present.

The boolean result of {\tt CM.recomp} and {\tt CM.stabilize} indicates
success or failure of the operation ({\tt true} = success).

\subsubsection*{Linking}

In SML/NJ, linking means executing top-level code (i.e., module
creation and initialization code) of each compilation unit.  The
resulting bindings can then be registered at the interactive top
level.

\begin{verbatim}
  val make : string -> bool
  val autoload : string -> bool
\end{verbatim}

{\tt CM.make} first acts like {\tt CM.recomp}.  If the
(re-)compilation is successful, then it proceeds by linking all
modules that require linking.  Provided there are no link-time errors,
it finally introduces new bindings at top level.

During the course of the same {\tt CM.make}, the code of each
compilation module that is reachable from the root will be executed at
most once.  Code in units that are marked as {\it private} (see
Section~\ref{sec:sharing}) will be executed exactly once.  Code in
other units will be executed only if the unit has been recompiled
since it was executed last time or if it depends on another
compilation unit whose code has been executed since.

In effect, different invocations of {\tt CM.make} (and {\tt
CM.autoload}) will share dynamic state created at link time as much as
possible unless the compilation units in question have been explicitly
marked private.

{\tt CM.autoload} acts like {\tt CM.make}, only ``lazily''. See
Section~\ref{sec:autoload} for more information.

As before, the result of {\tt CM.make} indicates success or failure of
the operation.  The result of {\tt CM.autoload} indicates success or
failure of the {\em registration}.  (It does not know yet whether
loading will actually succeed.)

\subsubsection*{Registers}

Several internal registers control the operation of CM.  A register of
type $T$ is accessible via a variable of type $T$ {\tt controller},
i.e., a pair of {\tt get} and {\tt set} functions.\footnote{The type
constructor {\tt controller} is defined as part of {\tt structure
CM}.}  Any invocation of the corresponding {\tt get} function reads
the current value of the register.  An invocation of the {\tt set}
function replaces the current value with the argument given to {\tt
set}.

Controllers are members of {\tt CM.Control}, a sub-structure of
structure {\tt CM}.

\begin{verbatim}
  type 'a controller = { get: unit -> 'a, set: 'a -> unit }
  structure Control : sig
    val verbose : bool controller
    val debug : bool controller
    val keep_going : bool controller
    val parse_caching : int controller
    val warn_obsolete : bool controller
    val conserve_memory : bool controller
  end
\end{verbatim}

{\tt CM.Control.verbose} can be used to turn off CM's progress
messages.  The default is {\em true} and can be overriden at startup
time by the environment variable {\tt CM\_VERBOSE}.

In the case of a compile-time error {\tt CM.Contol.keep\_going}
instructs the {\tt CM.recomp} phase to continue working on parts of
the dependency graph that are not related to the error.  (This does
not work for outright syntax errors because a correct parse is needed
before CM can construct the dependency graph.)  The default is {\em
false}, meaning ``quit on first error'', and can be overriden at
startup by the environment variable {\tt CM\_KEEP\_GOING}.

{\tt CM.Control.parse\_caching} sets a limit on how many parse trees
are cached in main memory.  In certain cases CM must parse source
files in order to be able to calculate the dependency graph.  Later,
the same files may need to be compiled, in which case an existing
parse tree saves the time to parse the file again.  Keeping parse
trees can be expensive in terms of memory usage.  Moreover, CM makes
special efforts to avoid re-parsing files in the first place unless
they have actually been modified.  Therefore, it may not make much
sense to set this value very high.  The default is {\em 100} and can
be overriden at startup time by the environment variable {\tt
CM\_PARSE\_CACHING}.

This version of CM uses an ML-inspired syntax for expressions in its
conditional compilation subsystem (see Section~\ref{sec:preproc}).
However, for the time being it will accept most of the original
C-inspired expressions but produces a warning for each occurrence of
an old-style operator. {\tt CM.Control.warn\_obsolete} can be used to
turn these warnings off. The default is {\em true}, meaning ``warnings
are issued'', and can be overriden at startup time by the environment
variable {\tt CM\_WARN\_OBSOLETE}.

{\tt CM.Control.debug} can be used to turn on debug mode.  This
currently has the effect of dumping a trace of the master-slave
protocol for parallel and distributed compilation (see
Section~\ref{sec:parmake}) to TextIO.stdOut. The default is {\em
false} and can be overriden at startup time by the environment
variable {\tt CM\_DEBUG}.

Using {\tt CM.Control.conserve\_memory}, CM can be told to be slightly
more conservative with its use of main memory at the expense of
occasionally incurring additional input from stable library files.
This does not save very much and, therefore, is normally turned off.
The default ({\em false}) can be overridden at startup by the
environment variable {\tt CM\_CONSERVE\_MEMORY}.

\subsubsection*{Path anchors}

Structure {\tt CM} also provides functions to explicitly manipulate
the path anchor configuration.  These functions are members of
structure {\tt CM.Anchor}.

\begin{verbatim}
  structure Anchor : sig
    val anchor : string -> string option controller
    val reset : unit -> unit
  end
\end{verbatim}

{\tt CM.Anchor.anchor} returns a pair of {\tt get} and {\tt set}
functions that can be used to query and modify the status of the named
anchor.  Note that the {\tt get}-{\tt set}-pair operates over type
{\tt string option}; a value of {\tt NONE} means that the anchor is
currently not bound (or, in the case of {\tt set}, that it is being
cancelled).  The (optional) string given to {\tt set} must be a
directory name in native syntax ({\em without} trailing arc separator,
e.g., {\bf /} in Unix).  If it is specified as a relative path name,
then it will be expanded by prepending the name of the current working
directory.

{\tt CM.Anchor.reset} erases the entire existing path configuration.
After a call of this function has completed, all root environment
locations are marked as being ``undefined''.

\subsubsection*{Setting CM variables}

CM variables are used by the conditional compilation system (see
Section~\ref{sec:cmvars}).  Some of these variables are predefined,
but the user can add new ones and alter or remove those that already
exist.

\begin{verbatim}
  val symval : string -> int option controller
\end{verbatim}

Function {\tt CM.symval} returns a {\tt get}-{\tt set}-pair for the
symbol whose name string was specified as the argument.  Note that the
{\tt get}-{\tt set}-pair operates over type {\tt int option}; a value
of {\tt NONE} means that the variable is not defined.

\noindent Examples:
\begin{verbatim}
#get (CM.symval "X") ();       (* query value of X *)
#set (CM.symval "Y") (SOME 1); (* set Y to 1 *)
#set (CM.symval "Z") NONE;     (* remove definition for Z *)
\end{verbatim}

Some care is necessary as {\tt CM.symval} does not check whether the
syntax of the argument string is valid.  (However, the worst thing
that could happen is that a variable defined via {\tt CM.symval} is
not accessible\footnote{from within CM's description files} because
there is no legal syntax to name it.)

\subsubsection*{Library registry}
\label{sec:libreg}

To be able to share associated data structures such as symbol tables
and dependency graphs, CM maintains an internal registry of all stable
libraries that it has encountered during an ongoing interactive
session.  The {\tt CM.Library} sub-structure of structure {\tt CM}
provides access to this registry.

\begin{verbatim}
  structure Library : sig
    type lib
    val known : unit -> lib list
    val descr : lib -> string
    val osstring : lib -> string
    val dismiss : lib -> unit
    val unshare : lib -> unit
  end
\end{verbatim}

{\tt CM.Library.known}, when called, produces a list of currently
known stable libraries.  Each such library is represented by an
element of the abstract data type {\tt CM.Library.lib}.

{\tt CM.Library.descr} extracts a string describing the location of
the CM description file associated with the given library.  The syntax
of this string is almost the same as that being used by CM's
master-slave protocol (see section~\ref{sec:pathencode}).

{\tt CM.Library.osstring} produces a string denoting the given
library's description file using the underlying operating system's
native pathname syntax.  In other words, the result of a call to {\tt
CM.Library.osstring} is suitable as an argument to {\tt
TextIO.openIn}.

{\tt CM.Library.dismiss} is used to remove a stable library from CM's
internal registry.  Although removing a library from the registry may
recover considerable amounts of main memory, doing so also eliminates
any chance of sharing the associated data structures with later
references to the same library.  Therefore, it is not always in the
interest of memory-conscious users to use this feature.

While dependency graphs and symbol tables need to be reloaded when a
previously dismissed library is referenced again, the sharing of
link-time state created by this library is {\em not} affected.
(Link-time state is independently maintained in a separate data
structure.  See the discussion of {\tt CM.unshare} below.)

{\tt CM.Library.unshare} is used to remove a stable library from CM's
internal registry, and---at the same time---to inhibit future sharing
with its existing link-time state.  Any future references to this
library will see newly created state (which will then be properly
shared again).  ({\bf Warning:} {\it This feature is not the preferred
way of creating unshared state; use functors for that.  However, it
can come in handy when two different (and perhaps incompatible)
versions of the same library are supposed to coexist---especially if
one of the two versions is used by SML/NJ itself.  Normally, only
programmers working on SML/NJ's compiler are expected to be using this
facility.})

\subsubsection*{Internal state}

For CM to work correctly, it must maintain an up-to-date picture of
the state of the surrounding world (as far as that state affects CM's
operation).  Most of the time, this happens automatically and should be
transparent to the user.  However, occasionally it may become
necessary to intervene expliticly.

Access to CM's internal state is facilitated by members of the {\tt
CM.State} structure.

\begin{verbatim}
  structure State : sig
    val pending : unit -> string list
    val synchronize : unit -> unit
    val reset : unit -> unit
  end
\end{verbatim}

{\tt CM.State.pending} produces a list of strings, each string naming
one of the symbols that are currently registered (i.e., ``virtually
bound'') but not yet resolved by the autoloading mechanism.

{\tt CM.State.synchronize} updates tables internal to CM to reflect
changes in the file system.  In particular, this will be necessary
when the association of file names to ``file IDs'' (in Unix: inode
numbers) changes during an ongoing session.  In practice, the need for
this tends to be rare.

{\tt CM.State.reset} completely erases all internal state in CM.  To
do this is not very advisable since it will also break the association
with pre-loaded libraries.  It may be a useful tool for determining
the amount of space taken up by the internal state, though.

\subsubsection*{Compile servers}

On Unix-like systems, CM supports parallel compilation.  For computers
connected using a LAN, this can be extended to distributed compilation
using a network file system and the operating system's ``rsh''
facility.  For a detailed discussion, see Section~\ref{sec:parmake}.

Sub-structure {\tt CM.Server} provides access to and manipulation of
compile servers.  Each attached server is represented by a value of
type {\tt CM.Server.server}.

\begin{verbatim}
  structure Server : sig
    type server
    val start : { name: string,
                  cmd: string * string list,
                  pathtrans: (string -> string) option,
                  pref: int } -> server option
    val stop : server -> unit
    val kill : server -> unit
    val name : server -> string
  end
\end{verbatim}

CM is put into ``parallel'' mode by attaching at least one compile
server.  Compile servers are attached using invocations of {\tt
CM.Server.start}.  The function takes the name of the server (as an
arbitrary but unique string) ({\tt name}), the Unix command used to
start the server in a form suitable as an argument to {\tt
Unix.execute} ({\tt cmd}), an optional ``path transformation
function'' for converting local path names to remote pathnames ({\tt
pathtrans}), and a numeric ``preference'' value that is used to choose
servers at times when more than one is idle ({\tt pref}).  The
optional result is the handle representing the successfully attached
server.

An existing server can be shut down and detached using {\tt
CM.Server.stop} or {\tt CM.Server.kill}.  The argument in either case
must be the result of an earlier call to {\tt CM.Server.start}.
Function {\tt CM.Server.stop} uses CM's master-slave protocol to
instruct the server to shut down gracefully.  Only if this fails it
may become necessary to use {\tt CM.Server.kill}, which will send a
Unix TERM signal to destroy the server.

Given a server handle, function {\tt CM.Server.name} returns the
string that was originally given to the call of\linebreak {\tt
CM.Server.start} used to created the server.

\subsubsection*{Plug-ins}

As an alternative to {\tt CM.make} or {\tt CM.autoload}, where the
main purpose is to subsequently be able to access the library from
interactively entered code, one can instruct CM to load libraries
``for effect''.

\begin{verbatim}
  val load_plugin : string -> bool
\end{verbatim}

Function {\tt CM.load\_plugin} acts exactly like {\tt CM.make} except
that even in the case of success no new symbols will be bound in the
interactive top-level environment.  That means that link-time
side-effects will be visible, but none of the exported definitions
become available.  This mechanism can be used for ``plug-in'' modules:
a core library provides hooks where additional functionality can be
registered later via side-effects; extensions to this core are
implemented as additional libraries which, when loaded, register
themselves with those hooks.  By using {\tt CM.load\_plugin} instead
of {\tt CM.make}, one can avoid polluting the interactive top-level
environment with spurious exports of the extension module.

CM itself uses plug-in modules in its member-class subsystem (see
section~\ref{sec:classes}).  This makes it possible to add new classes
and tools very easily without having to reconfigure or recompile CM,
not to mention modify its source code.

\subsubsection*{Building stand-alone programs}

CM can be used to build stand-alone programs. In fact SML/NJ
itself---including CM---is an example of this.  (The interactive
system cannot rely on an existing compilation manager when starting
up.)

A stand-alone program is constructed by the runtime system from
existing binfiles or members of existing stable libraries.  CM must
prepare those binfiles or libraries together with a list that
describes them to the runtime system.

\begin{verbatim}
  val mk_standalone : bool option -> string -> string list option
\end{verbatim}

Depending on the optional boolean argument, function {\tt
CM.mk\_standalone} first acts like either {\tt CM.recomp} or {\tt
CM.stabilize}.  {\tt NONE} means {\tt CM.recomp}, and {\tt (SOME $r$)}
means {\tt CM.stabilize $r$}.  After recompilation (or stabilization)
is successful, {\tt CM.mk\_standalone} constructs a topologically
sorted list of strings that, when written to a file, can be passed to the
runtime system in order to perform stand-alone linkage of the given
program. Upon failure, {\tt CM.mk\_standalone} returns {\tt NONE}.

\paragraph*{\bf ml-build:} The programmer should normally have no need
to invoke {\tt CM.mk\_standalone} directly.  Instead, SML/NJ provides
a command {\tt ml-build} which does all the work.  To be able to use
{\tt ml-build}, one must implement a library exporting a structure
that has some function suitable to be an argument to {\tt
SMLofNJ.exportFn}.  Suppose the library is called {\tt myproglib.cm}, the
structure is called {\tt MyProg}, and the function is called {\tt
MyProg.main}.  If one wishes to produce a heap image file {\tt myprog}
one simply has to invoke the following command:

\begin{verbatim}
ml-build myproglib.cm MyProg.main myprog
\end{verbatim}

\subsubsection*{Finding all sources}

The {\tt CM.sources} function can be used to find the names of all
source files that a given library depends on.  It returns the names of
all files involved with the exception of skeleton files and binfiles
(see Section~\ref{sec:files}).  Stable libraries are represented by
their library file; their description file or consitutent members are
{\em not} listed.

Normally, the function reports actual file names as used for accessing
the file system.  For (stable) library files this behavior can be
inconvenient because these names depend on architecture and operating
system.  For this reason, {\tt CM.sources} accepts an optional pair of
strings that then will be used in place of the architecture- and
OS-specific part of these names.

\begin{verbatim}
  val sources :
    { arch: string, os: string } option ->
    string ->
    { file: string, class: string, derived: bool } list option
\end{verbatim}

In case there was some error analyzing the specified library or group,
{\tt CM.sources} returns {\tt NONE}.  Otherwise the result is a list
of records, each carrying a file name, the corresponding class, and
information about whether or not the source was created by some tool.

Examples:

\begin{description}
\item[generating ``make'' dependencies:]
To generate dependency information usable by Unix' {\tt make} command,
one would be interested in all files that were not derived by some
tool application.  Moreover, one would probably like to use shell
variables instead of concrete architecture- and OS-names:
\begin{verbatim}
  Option.map (List.filter (not o #derived))
    (CM.sources (SOME { arch = "$ARCH", os = "$OPSYS" })
         "foo.cm");
\end{verbatim}
\item[finding all {\tt noweb} sources:]
To find all {\tt noweb} sources (see Section~\ref{sec:builtin-tools}),
e.g., to be able to run the document preparation program {\tt noweave}
on them, one can simply look for entries of the {\tt noweb} class.
Here, one would probably want to include derived sources:
\begin{verbatim}
  Option.map (List.filter (fn x => #class x = "noweb"))
    (CM.sources NONE "foo.cm");
\end{verbatim}
\end{description}

\subsection{The autoloader}
\label{sec:autoload}

From the user's point of view, a call to {\tt CM.autoload} acts very
much like the corresponding call to {\tt CM.make} because the same
bindings that {\tt CM.make} would introduce into the top-level
enviroment are also introduced by {\tt CM.autoload}.  However, most
work will be deferred until some code that is entered later refers to
one or more of these bindings.  Only then will CM go and perform just
the minimal work necessary to provide the actual definitions.

The autoloader plays a central role for the interactive system.
Unlike in earlier versions, it cannot be turned off since it provides
many of the standard pre-defined top-level bindings.

The autoloader is a convenient mechanism for virtually ``loading'' an
entire library without incurring an undue increase in memory
consumption for library modules that are not actually being used.

\subsection{Sharing of state}
\label{sec:sharing}

Whenever it is legal to do so, CM lets multiple invocations of {\tt
CM.make} or {\tt CM.autoload} share dynamic state created by link-time
effects.  Of course, sharing is not possible (and hence not ``legal'')
if the compilation unit in question has recently been recompiled or
depends on another compilation unit whose code has recently been
re-executed.  The programmer can explicitly mark certain ML files as
{\em shared}, in which case CM will issue a warning whenever the
unit's code has to be re-executed.

State created by compilation units marked as {\em private} is never
shared across multiple calls to {\tt CM.make} or {\tt CM.autoload}.
To understand this behavior it is useful to introduce the notion of a
{\em traversal}.  A traversal is the process of traversing the
dependency graph on behalf of {\tt CM.make} or {\tt CM.autoload}.
Several traversals can be executed interleaved with each other because
a {\tt CM.autoload} traversal normally stays suspended and is
performed incrementally driven by input from the interactive top level
loop.

As far as sharing is concerned, the rule is that during one traversal
each compilation unit will be executed at most once.  This means that
the same ``program'' will not see multiple instantiations of the same
compilation unit (where ``program'' refers to the code managed by one
call to {\tt CM.make} or {\tt CM.autoload}).  Each compilation unit
will be linked at most once during a traversal and private state
will not be confused with private state of other traversals that might
be active at the same time.

% Need a good example here.

\subsubsection*{Sharing annotations}

ML source files in CM description files can be specified as being {\em
private} or {\em shared}.  This is done by adding a {\em tool
parameter} specification for the file in the library- or group
description file (see Section~\ref{sec:classes}). To mark an ML file
as {\em private}, follow the file name with the word {\tt private} in
parentheses.  For {\em shared} ML files, replace {\tt private} with
{\tt shared}.

An ML source file that is not annotated will typically be treated as
{\em shared} unless it statically depends on some other {\em private}
source.  It is an error, checked by CM, for a {\em shared} source to
depend on a {\em private} source.

\subsubsection*{Sharing with the interactive system}

The SML/NJ interactive system, which includes the compiler, is itself
created by linking modules from various libraries. Some of these
libraries can also be used in user programs.  Examples are the
Standard ML Basis Library {\tt \$/basis.cm}, the SML/NJ library {\tt
\$/smlnj-lib.cm}, and the ML-Yacc library {\tt \$/ml-yacc-lib.cm}.

If a module from a library is used by both the interactive system and
a user program running under control of the interactive system, then
CM will let them share code and dynamic state.  Moreover, the affected
portion of the library will never have to be ``relinked''.

\section{Version numbers}
\label{sec:versions}

A CM library can carry a version number.  Version numbers are
specified in parentheses after the keyword {\tt Library} as non-empty
dot-separated sequences of non-negative integers.  Example:

\begin{verbatim}
Library (1.4.1.4.2.1.3.5)
        structure Sqrt2
is
        sqrt2.sml
\end{verbatim}

\subsection{How versions are compared}

Version numbers are compared lexicographically, dot-separated
component by dot-separated component, from left to right.  The
components themselves are compared numerically.

\subsection{Version checking}

An importing library or library component can specify which version of
the imported library it would like to see.  See the discussion is
section~\ref{sec:toolparam} for how this is done.  Where a version
number is requested, an error is signalled if one of the following is
true:

\begin{itemize}
\item the imported library does not carry a version number
\item the imported library's version number is smaller than the
one requested
\item the imported library's version number has a first component
(known as the ``major'' version number) that is greater than the one
requested
\end{itemize}

A warning (but no error) is issued if the imported library has the
same major version but the version as a whole is greater than the one
requested.

Note: {\it Version numbers should be incremented on every change to a
library.  The major version number should be increased on every change
that is not backward-compatible.}

\section{Member classes and tools}
\label{sec:classes}

Most members of groups and libraries are either plain ML files or
other description files.  However, it is possible to incorporate other
types of files---as long as their contents can in some way be expanded
into ML code or CM descriptions.  The expansion is carried out by CM's
{\it tools} facility.

CM maintains an internal registry of {\em classes} and associated {\em
rules}.  Each class represents the set of source files that its
corresponding rule is applicable to.  For example, the class {\tt
mlyacc} is responsible for files that contain input for the parser
generator ML-Yacc~\cite{tarditi90:yacc}.  The rule for {\tt mlyacc}
takes care of expanding an ML-Yacc specifications {\tt foo.grm} by
invoking the auxiliary program {\tt ml-yacc}.  The resulting ML files
{\tt foo.grm.sig} and {\tt foo.grm.sml} are then used as if their
names had directly been specified in place of {\tt foo.grm}.

CM knows a small number of built-in classes.  In many situations these
classes will be sufficient, but in more complicated cases it may be
worthwhile to add a new class.  Since class rules are programmed in
ML, adding a class is not as simple a matter as writing a rule for
{\sc Unix}' {\tt make} program~\cite{feldman79}.  Of course,
using ML has also advantages because it keeps CM extremely flexible in
what rules can do.  Moreover, it is not necessary to learn yet another
``little language'' in order to be able to program CM's tool facility.

When looking at the member of a description file, CM determines which
tool to use by looking at clues like the file name suffix.  However,
it is also possible to specify the class of a member explicitly.  For
this, the member name is followed by a colon {\bf :} and the name of
the member class.  All class names are case-insensitive.

In addition to genuine tool classes, there are two member classes
that refer to facilities internal to CM: {\tt sml} is the class of
ordinary ML source files and {\tt cm} is the class of CM library or
group description files.

CM automatically classifies files with a {\tt .sml} suffix, a {\tt
.sig} suffix, or a {\tt .fun} suffix as ML-source, file names ending
in {\tt .cm} as CM descriptions.\footnote{Suffixes that are not known
and for which no plugin module can be found are treated as ML source
code.  However, as new tools are added there is no guarantee that
this behavior will be preserved in future versions of CM.}

\subsection{Tool parameters}
\label{sec:toolparam}

In many cases the name of the member that caused a rule to be invoked
is the only input to that rule.  However, rules can be written in such
a way that they take additional parameters.  Those parameters, if
present, must be specified in the CM description file between
parentheses following the name of the member and the optional member
class.

CM's core mechanism parses these tool options and breaks them up into
a list of items, where each item is either a filename (i.e., {\em
looks} like a filename) or a named list of sub-options.  However, CM
itself does not interpret the result but passes it on to the tool's
rule function.  It is in each rule's own responsibility to assign
meaning to its options.

\subsubsection*{Parameters for class {\tt sml}}

The {\tt sml} class accepts two optional parameters.  One is the {\em
sharing annotation} that was explained earlier (see
Section~\ref{sec:sharing}).  The sharing annotation must be one of the
two strings {\tt shared} and {\tt private}.  If {\tt shared} is
specified, then dynamic state created by the compilation unit at
link-time must be shared across invocations of {\tt CM.make} or {\tt
CM.autoload}.  The {\tt private} annotation, on the other hand, means
that dynamic state cannot be shared across such calls to {\tt CM.make}
or {\tt CM.autoload}.

The other possible parameter for class {\tt sml} is a sub-option
list labeled {\tt setup} and can be used to specify code that will be
executed just before and just after the compiler is invoked for the
ML source file.  Code to be executed before compilation is labeled
{\tt pre}, code to be executed after compilation is complete is
labeled {\tt post}; either part is optional.  Executable code itself
is specified using strings that contain ML source text.

For example, if one wishes to disable warning messages for a specific
source file {\tt poorlywritten.sml} (but not for others), then one
could write:

\begin{verbatim}
  poorlywritten.sml (setup:(pre: "local open Compiler.Control\n\
                                 \   in val w = !printWarnings before\n\
                                 \              printWarnings := false\n\
                                 \  end;"
                            post:"Compiler.Control.printWarnings := w;"))
\end{verbatim}

\noindent Note that neither the pre- nor the post-section will be
executed if the ML file does not need to be compiled.

The pre-section is compiled and executed in the current
toplevel-environment while the post-section uses the
toplevel-environment augmented with definitions from the pre-section.
After the ML file has been compiled and the post-section (if present)
has completed execution, definitions made by either section will be
erased.  This means that setup code for other files {\em cannot} refer
to them, and neither can code that in the future might be entered at
top level.

\subsubsection*{Parameters for class {\tt cm}}

The {\tt cm} class understands two kinds of parameters.  The first is
a named parameter labeled by the string {\tt version}.  It must have
the format of a version number.  CM will interpret this as a version
request, thereby insuring that the imported library is not too old or
too new. (See section~\ref{sec:versions} for more on this topic.)

All named sub-option lists (for any class) are specified by a name
string followed by a colon {\bf :} and a parenthesized list of other
tool options.  If the list contains precisely one element, the
parentheses may be omitted.  Example:

\begin{verbatim}
    euler.cm (version:2.71828)
    pi.cm    (version:3.14159)
\end{verbatim}

Normally, CM looks for stable library files in directory
{\tt CM/}{\it arch}{\tt -}{\it os} (see section~\ref{sec:files}).
However, if an explicit version has been requested, it will first try
directory {\tt CM/}{\it version}{\tt /}{\it arch}{\tt -}{\it os}
before looking at the default location.  This way it is possible to
keep several versions of the same library in the file system.

However, CM normally does {\em not} permit the simultaneous use of
multiple versions of the same library in one session.  The
disambiguating rule is that the version that gets loaded first
``wins''; subsequent attempts to load different versions result in
warnings or errors.  (See the discussion of {\tt CM.unshare} in
section~\ref{sec:libreg} for how to to circumvent this restriction.)

The second kind of parameter understood by {\tt cm} is a named
parameter labeled by the string {\tt bind} (see
Section~\ref{sec:anchor:env}).  It can occur arbitrarily many times
and each occurence must be a suboption-list of the form {\tt
(anchor:$a$ value:$v$)}.  The set of {\tt bind}-parameters augments
the current anchor environment to form the environment that is used
while processing the contents of the named CM description file.

\subsection{Built-in tools}
\label{sec:builtin-tools}

\subsubsection*{The ML-Yacc tool}

The ML-Yacc tool is responsible for files that are input to the
ML-Yacc parser generator.  Its class name is {\tt mlyacc}.  Recognized
file name suffixes are {\tt .grm} and {\tt .y}.  For a source file
$f$, the tool produces two targets $f${\tt .sig} and $f${\tt .sml},
both of which are always treated as ML source files.  The {\tt mlyacc}
class accepts two optional tool parameters labeled {\tt sigoptions}
and {\tt smloptions}.  They specify tool options to be passed on to
the generated {\tt .sig}- and {\tt .sml}-files, respectively.
Example\footnote{Since the generated {\tt .sig}-file contains nothing
more than an ML signature definition, it is typically not very useful
to pass any options to it.}:

\begin{verbatim}
    lang.grm (sigoptions:(setup:(pre:"print \"compiling lang.grm.sig\\n\";"))
              smloptions:(private))
\end{verbatim}

The tool invokes the {\tt ml-yacc} command if the targets are
``outdated''.  A target is outdated if it is missing or older than the
source.  Unless anchored using the path anchor mechanism (see
Section~\ref{sec:anchor:env}), the command {\tt ml-yacc} will be located
using the operating system's path search mechanism (e.g., the {\tt
\$PATH} environment variable).

\subsubsection*{ML-Lex}

The ML-Lex tool governs files that are input to the ML-Lex lexical
analyzer generator~\cite{appel89:lex}.  Its class name is {\tt mllex}.
Recognized file name suffixes are {\tt .lex} and {\tt .l}.  For a
source file $f$, the tool produces one targets $f${\tt .sml} which
will always be treated as ML source code.  Tool parameters are passed
on without change to that file.

The tool invokes the {\tt ml-lex} command if the target is outdated
(just like in the case of ML-Yacc).  Unless anchored using the path
anchor mechanism (see Section~\ref{sec:anchor:env}), the command {\tt
ml-lex} will be located using the operating system's path search
mechanism (e.g., the {\tt \$PATH} environment variable).

\subsubsection*{ML-Burg}

The ML-Burg tool deals with files that are input to the ML-Burg
code-generater generator~\cite{mlburg93}.  Its class name is {\tt
mlburg}.  The only recognized file name suffix is {\tt .burg}.  For a
source file $f${\tt .burg}, the tool produces one targets $f${\tt
.sml} which will always be treated as ML source code.  Any tool
parameters are passed on without change to the target.

The tool invokes the {\tt ml-burg} command if the target is outdated.
Unless anchored using the path anchor mechanism (see
Section~\ref{sec:anchor:env}), the command {\tt ml-lex} will be located
using the operating system's path search mechanism (e.g., the {\tt
\$PATH} environment variable).

\subsubsection*{Shell}

The Shell tool can be used to specify arbitrary shell commands to be
invoked on behalf of a given file.  The name of the class is {\tt
shell}.  There are no recognized file name suffixes.  This means that
in order to use the shell tool one must always specify the {\tt shell}
member class explicitly.

The rule for the {\tt shell} class relies on tool parameters.  The
parameter list must be given in parentheses and follow the {\tt shell}
class specification.

Consider the following example:

\begin{verbatim}
  foo.pp : shell (target:foo.sml options:(shared)
                        /lib/cpp -P -Dbar=baz %s %t)
\end{verbatim}

This member specification says that file {\tt foo.sml} can be obtained
from {\tt foo.pp} by running it through the C preprocessor {\tt cpp}.
The fact that the target file is given as a tool parameter implies
that the member itself is the source.  The named parameter {\tt
options} lists the tool parameters to be used for that target. (In the
example, the parentheses around {\tt shared} are optional because it
is the only element of the list.) The command line itself is given by
the remaining non-keyword parameters.  Here, a single {\bf \%s} is
replaced by the source file name, and a single {\bf \%t} is replaced
by the target file name; any other string beginning with {\bf \%} is
shortened by its first character.

In the specification one can swap the positions of source and target
(i.e., let the member name be the target) by using a {\tt source}
parameter:

\begin{verbatim}
  foo.sml : shell (source:foo.pp options:shared
                         /lib/cpp -P -Dbar=baz %s %t)
\end{verbatim}

Exactly one of the {\tt source} and {\tt target} parameters must be
specified; the other one is taken to be the member name itself.  The
target class can be given by writing a {\tt class} parameter whose
single sub-option must be the desired class name.

The usual distinction between native and standard filename syntax
applies to any given {\tt source} or {\tt target} parameters.

For example, if one were working on a Win32 system and the target file
is supposed to be in the root directory on volume {\tt D:},
then one must use native syntax to write it.  One way of doing this
would be:

\begin{verbatim}
  "D:\\foo.sml" : shell (source : foo.pp options : shared
                               cpp -P -Dbar=baz %s %t)
\end{verbatim}

\noindent As a result, {\tt foo.sml} is interpreted using native
syntax while {\tt foo.pp} uses standard conventions (although in this
case it does not make a difference).  Had we used the {\tt target}
version from above, one would have to write:

\begin{verbatim}
  foo.pp : shell (target : "D:\\foo.sml" options : shared
                                 cpp -P -Dbar=baz %s %t)
\end{verbatim}

The shell tool invokes its command whenever the target is outdated
with respect to the source.

\subsubsection*{Make}

The Make tool (class {\tt make}) can (almost) be seen as a specialized
version of the Shell tool.  It has no source and one target (the
member itself) which is always considered outdated.  As with the Shell
tool, it is possible to specify target class and parameters using the
{\tt class} and {\tt options} keyword parameters.

The tool invokes the shell command {\tt make} on the target.  Unless
anchored using the path anchor mechanism~\ref{sec:anchor:env}, the
command will be located using the operating system's path search
mechanism (e.g., the {\tt \$PATH} environment variable).

Any parameters other than the {\tt class} and {\tt options}
specifications must be plain strings and are given as additional
command line arguments to {\tt make}.  The target name is always the
last command line argument.

Example:

\begin{verbatim}
  bar-grm : make (class:mlyacc -f bar-grm.mk)
\end{verbatim}

Here, file {\tt bar-grm} is generated (and kept up-to-date) by
invoking the command:
\begin{verbatim}
make -f bar-grm.mk bar-grm
\end{verbatim}
\noindent The target file is then treated as input for {\tt ml-yacc}.

Cascading Shell- and Make-tools is easily possible.  Here is an
example that first uses Make to build {\tt bar.pp} and then filters
the contens of {\tt bar.pp} through the C preprocessor to arrive at
{\tt bar.sml}:

\begin{verbatim}
  bar.pp : make (class:shell
                     options:(target:bar.sml cpp -Dbar=baz %s %t)
                 -f bar-pp.mk)
\end{verbatim}

\subsubsection*{Noweb}

The {\tt noweb} class handles sources written for Ramsey's {\it noweb}
literate programming facility~\cite{ramsey:simplified}.  Files ending
with suffix {\tt .nw} are automatically recognized as belonging to
this class.

The list of targets that are to be extracted from a noweb file must be
specified using tool options.  A target can then have a variety of its
own options.  Each target is specified by a separate tool option
labelled {\tt target}.  The option usually has the form of a
sub-option list.  Recognized sub-options are:

\begin{description}
\item[name] the name of the target
\item[root] the (optional) root tag for the target (given to the {\tt
-R} command line switch for the {\tt notangle} command); if {\tt root}
is missing, {\tt name} is used instead
\item[class] the (optional) class of the target
\item[options] (optional) options for the tool that handles the
target's class
\item[lineformat] a string that will be passed to the {\tt -L} command
line option of {\tt notangle}
\end{description}

Example:

\begin{verbatim}
   project.nw (target:(name:main.sml options:(private))
               target:(name:grammar class:mlyacc witness:grammar.wtn)
               target:(name:parse.sml))
\end{verbatim}

In place of the sub-option list there can be a single string option
which will be used for {\tt name} or even an unnamed parameter (i.e.,
without the {\tt target} label).  If no targets are specified, the
tool will assume two default targets by stripping the {\tt .nw}
suffix (if present) from the source name and adding {\tt .sig} as well
as {\tt .sml}.

The following four examples are all equivalent:

\begin{verbatim}
   foo.nw (target:(name:foo.sig) target:(name:foo.sml))
   foo.nw (target:foo.sig target:foo.sml)
   foo.nw (foo.sig foo.sml)
   foo.nw
\end{verbatim}

If {\tt lineformat} is missing, then a default based on the target
class is used.  Currently only the {\tt sml} and {\tt cm} classes are
known to CM; other classes can be added or removed by using the {\tt
NowebTool.lineNumbering} controller function exported from library
{\tt \$/noweb-tool.cm}:

\begin{verbatim}
  val lineNumbering: string -> { get: unit -> string option,
                                 set: string option -> unit }
\end{verbatim}

The {\tt noweb} class accepts two other parameter besides {\tt
target}:

\begin{description}
\item[subdir] specifies a sub-option that is used to specify a
directory where derived files (i.e., target files and witness files as
far as they have been specified using relative path names) are
created.  If the {\tt subdir} option is missing, its value defaults to
{\tt NW}.
\item[witness] specifies an auxiliary derived file whose time stamp is
used by CM to avoid recompiling extracted files whose contents have
not changed.  If {\tt witness} has not been specified, then CM uses
time stamps on extracted files directly to determine whether {\tt
notangle} needs to be run.  Thus, with no witness, any change to the
master file causes time stamps on all extracted files to be updated as
well.  If a witness was specified, then CM will write over extracted
files, causing their time stamps to change, only if their contents
have also changed.  The {\tt subdir} specification also applies to the
name of the witness file.
\end{description}

Example:

\begin{verbatim}
   foo.nw (subdir:NOWEBFILES
           witness:foo.wtn
           target:(name:main.sml))
\end{verbatim}

Here, the files named {\tt main.sml} and {\tt foo.wtn} will be
created as
\begin{verbatim}
   NOWEBFILES/main.sml
   NOWEBFILES/foo.wtn
\end{verbatim}
\noindent while without the {\tt subdir}-option it would have been
\begin{verbatim}
   NW/main.sml
   NW/foo.wtn
\end{verbatim}
\noindent To avoid the creation of such a sub-directory, one can use
the {\em current arc} ``{\bf .}'' and write:
\begin{verbatim}
   foo.nw (subdir:.
           witness:foo.wtn
           target:(name:main.sml))
\end{verbatim}

\section{Conditional compilation}
\label{sec:preproc}

In its description files, CM offers a simple conditional compilation
facility inspired by the preprocessor for the C language~\cite{k&r2}.
However, it is not really a {\it pre}-processor, and the syntax of the
controlling expressions is borrowed from SML.

Sequences of members can be guarded by {\tt \#if}-{\tt \#endif}
brackets with optional {\tt \#elif} and {\tt \#else} lines in between.
The same guarding syntax can also be used to conditionalize the export
list.  {\tt \#if}-, {\tt \#elif}-, {\tt \#else}-, and {\tt
\#endif}-lines must start in the first column and always
extend to the end of the current line.  {\tt \#if} and {\tt \#elif}
must be followed by a boolean expression.

Boolean expressions can be formed by comparing arithmetic expressions
(using operators {\tt <}, {\tt <=}, {\tt =}, {\tt >=}, {\tt >}, or
{\tt <>}), by logically combining two other boolean expressions (using
operators {\tt andalso}, {\tt orelse}, {\tt =}, or {\tt <>}, by
querying the existence of a CM symbol definition, or by querying the
existence of an exported ML definition.

Arithmetic expressions can be numbers or references to CM symbols, or
can be formed from other arithmetic expressions using operators {\tt
+}, {\tt -} (subtraction), \verb|*|, {\tt div}, {\tt mod}, or $\tilde{~}$
(unary minus).  All arithmetic is done on signed integers.

Any expression (arithmetic or boolean) can be surrounded by
parentheses to enforce precedence.

\subsection{CM variables}
\label{sec:cmvars}

CM provides a number of ``variables'' (names that stand for certain
integers). These variables may appear in expressions of the
conditional-compilation facility. The exact set of variables provided
depends on SML/NJ version number, machine architecture, and
operating system.  A reference to a CM variable is considered an
arithmetic expression. If the variable is not defined, then it
evaluates to 0.  The expression {\tt defined}($v$) is a boolean
expression that yields true if and only if $v$ is a defined CM
variable.

The names of CM variables are formed starting with a letter followed
by zero or more occurences of letters, decimal digits, apostrophes, or
underscores.

The following variables will be defined and bound to 1:
\begin{itemize}
\item depending on the operating system: {\tt OPSYS\_UNIX}, {\tt
OPSYS\_WIN32}, {\tt OPSYS\_MACOS}, {\tt OPSYS\_OS2}, or {\tt
OPSYS\_BEOS}
\item depending on processor architecture: {\tt ARCH\_SPARC}, {\tt
ARCH\_ALPHA32}, {\tt ARCH\_MIPS}, {\tt ARCH\_X86}, {\tt ARCH\_HPPA},
{\tt ARCH\_RS6000}, or {\tt ARCH\_PPC}
\item depending on the processor's endianness: {\tt BIG\_ENDIAN} or
{\tt LITTLE\_ENDIAN}
\item depending on the native word size of the implementation: {\tt
SIZE\_32} or {\tt SIZE\_64}
\item the symbol {\tt NEW\_CM}
\end{itemize}

Furthermore, the symbol {\tt SMLNJ\_VERSION} will be bound to the
major version number of SML/NJ (i.e., the number before the first dot)
and {\tt SMLNJ\_MINOR\_VERSION} will be bound to the system's minor
version number (i.e., the number after the first dot).

Using the {\tt CM.symval} interface one can define additional
variables or modify existing ones.

\subsection{Querying exported definitions}

An expression of the form {\tt defined}($n$ $s$), where $s$ is an ML
symbol and $n$ is an ML namespace specifier, is a boolean expression
that yields true if and only if any member included before this test
exports a definition under this name.  Therefore, order among members
matters after all (but it remains unrelated to the problem of
determining static dependencies)!  The namespace specifier must be one
of: {\tt structure}, {\tt signature}, {\tt functor}, or {\tt funsig}.

If the query takes place in the ``exports'' section of a description
file, then it yields true if {\em any} of the included members exports
the named symbol.

\noindent Example:

\begin{verbatim}
Library
  structure Foo
#if defined(structure Bar)
  structure Bar
#endif
is
#if SMLNJ_VERSION > 110
  new-foo.sml
#else
  old-foo.sml
#endif
#if defined(structure Bar)
  bar-client.sml
#else
  no-bar-so-far.sml
#endif
\end{verbatim}

Here, the file {\tt bar-client.sml} gets included if {\tt
SMLNJ\_VERSION} is greater than 110 and {\tt new-foo.sml} exports a
structure {\tt Bar} {\em or} if {\tt SMLNJ\_VERSION <= 110} and {\tt
old-foo.sml} exports structure {\tt Bar}.  Otherwise\linebreak {\tt
no-bar-so-far.sml} gets included instead.  In addition, the export of
structure {\tt Bar} is guarded by its own existence.  (Structure {\tt
Bar} could also be defined by {\tt no-bar-so-far.sml} in
which case it would get exported regardless of the outcome of the
other {\tt defined} test.)

\subsection{Explicit errors}

A pseudo-member of the form {\tt \#error $\ldots$}, which---like other
{\tt \#}-items---starts in the first column and extends to the end of
the line, causes an explicit error message to be printed unless it
gets excluded by the conditional compilation logic.  The error message
is given by the remainder of the line after the word {\tt error}.

\section{Access control}
\label{sec:access}

The basic idea behind CM's access control is the following: In their
description files, groups and libraries can specify a list of
{\em privileges} that the client must have in order to be able to use them.
Privileges at this level are just names (strings) and must be written
in front of the initial keyword {\tt Library} or {\tt Group}.  If one
group or library imports from another group or library, then
privileges (or rather: privilege requirements) are being inherited.
In effect, to be able to use a program, one must have all privileges
for all its libraries, sub-libraries and library components,
components of sub-libraries, and so on.

Of course, this alone would not yet be satisfactory.  The main service
of the access control system is that it can let a client use an
``unsafe'' library ``safely''.  For example, a library {\tt LSafe.cm}
could ``wrap'' all the unsafe operations in {\tt LUnsafe.cm} with
enough error checking that they become safe.  Therefore, a user of
{\tt LSafe.cm} should not also be required to possess the privileges
that would be required if one were to use {\tt LUnsafe.cm} directly.

In CM's access control model it is possible for a library to ``wrap''
privileges.  If a privilege $P$ has been wrapped, then the user of the
library does not need to have privilege $P$ even though the library is
using another library that requires privilege $P$.  In essence, the
library acts as a ``proxy'' who provides the necessary credentials for
privilege $P$ to the sub-library.

Of course, not everybody can be allowed to establish a library with
such a ``wrapped'' privilege $P$.  The programmer who does that should at
least herself have privilege P (but perhaps better, she should have
{\em permission to wrap $P$}---a stronger requirement).

In CM, wrapping a privilege is done by specifying the name of that
privilege within parenthesis.  The wrapping becomes effective once the
library gets stabilized via {\tt CM.stabilize}.  The (not yet
implemented) enforcement mechanism must ensure that anyone who
stabilizes a library that wraps $P$ has permission to wrap $P$.

Note that privileges cannot be wrapped at the level of CM groups.

Access control is a new feature. At the moment, only the basic
mechanisms are implemented, but there is no enforcement.  In other
words, everybody is assumed to have every possible privilege.  CM
merely reports which privileges ``would have been required''.

\section{The pervasive environment}

The {\em pervasive environment} can be thought of as a compilation
unit that all compilation units implicitly depend upon.  The pervasive
enviroment exports all non-modular bindings (types, values, infix
operators, overloaded symbols) that are mandated by the specification
for the Standard ML Basis Library~\cite{reppy99:basis}.  (All other
bindings of the Basis Library are exported by {\tt \$/basis.cm} which is
a genuine CM library.)

The pervasive environment is the only place where CM conveys
non-modular bindings from one compilation unit to another, and its
definition is fixed.

\section{Files}
\label{sec:files}

CM uses three kinds of files to store derived information during and
between sessions:

\begin{enumerate}
\item {\it Skeleton files} are used to store a highly abbreviated
version of each ML source file's abstract syntax tree---just barely
sufficient to drive CM's dependency analysis.  Skeleton files are much
smaller and easier to read than actual ML source code.  Therefore, the
existence of valid skeleton files makes CM a lot faster because
usually most parsing operations can be avoided that way.
\item {\it Binfiles} are the SML/NJ equivalent of object files.  They
contain executable code and a symbol table for the associated ML
source file.
\item {\it Library files} (sometimes called: {\em stablefiles}) contain
dependency graph, executable code, and symbol tables for an entire CM
library including all of its components (groups).  Other libraries
used by a stable library are not included in full.  Instead,
references to those libraries are recorded using their (preferably
anchored) pathnames.
\end{enumerate}

Normally, all these files are stored in a subdirectory of directory
{\tt CM}. {\tt CM} itself is a subdirectory of the directory where the
original ML source file or---in the case of library files---the
original CM description file is located.

Skeleton files are machine- and operating system-independent.
Therefore, they are always placed into the same directory {\tt
CM/SKEL}. Parsing (for the purpose of dependency analysis) will be
done only once even if the same file system is accessible from
machines of different type.

Binfiles and library files contain executable code and other
information that is potentially system- and architecture-dependent.
Therefore, they are stored under {\tt CM/}{\it arch}{\tt -}{\it os}
where {\it arch} is a string indicating the type of the current
CPU architecture and {\it os} a string denoting the current operating
system type.

Library files are a bit of an exception in the sense that they do not
require any source files or any other derived files of the same
library to exist.  As a consequence, the location of such a library
file is best described as being relative to ``the location of the
original CM description file if that description file still existed''.
(Of course, nothing precludes the CM description file from actually
existing, but in the presence of a corresponding library file CM will
not take any notice.)

{\em Note:} As discussed in section~\ref{sec:toolparam}, CM sometimes
looks for library files in
{\tt CM/}{\it version}{\tt /}{\it arch}{\tt -}{\it os}.
However, library files are never {\em created} there by CM.  If
several versions of the same library are to be provided, an
administrator must arrange the directory hierarchy accordingly ``by
hand''.

\subsection{Time stamps}

For skeleton files and binfiles, CM uses file system time stamps
(i.e., modification time) to determine whether a file has become
outdated.  The rule is that in order to be considered ``up-to-date''
the time stamp on skeleton file and binfile has to be exactly the
same\footnote{CM explicitly sets the time stamp to be the same.} as
the one on the ML source file.  This guarantees that all changes to a
source will be noticed.\footnote{except for the pathological case where
two different versions of the same source file have exactly the same
time stamp}

CM also uses time stamps to decide whether tools such as ML-Yacc or
ML-Lex need to be run (see Section~\ref{sec:tools}).  However, the
difference is that a file is considered outdated if it is older than
its source.  Some care on the programmers side is necessary since this
scheme does not allow CM to detect the situation where a source file
gets replaced by an older version of itself.

\section{Tools}
\label{sec:tools}

CM's tool set is extensible: new tools can be added by writing a few
lines of ML code.  The necessary hooks for this are provided by a
structure {\tt Tools} which is exported by the {\tt \$smlnj/cm/tools.cm}
library.

If the tool is implemented as a ``typical'' shell command, then all
that needs to be done is a single call to:

\begin{verbatim}
Tools.registerStdShellCmdTool
\end{verbatim}

For example, suppose you have made a
new, improved version of ML-Yacc (``New-ML-Yacc'') and want to
register it under a class called {\tt nmlyacc}.  Here is what you
write:

\begin{verbatim}
val _ = Tools.registerStdShellCmdTool
  { tool = "New-ML-Yacc",
    class = "nmlyacc",
    suffixes = ["ngrm", "ny"],
    cmdStdPath = "new-ml-yacc",
    template = NONE,
    extensionStyle =
        Tools.EXTEND [("sig", SOME "sml", fn _ => NONE),
                      ("sml", SOME "sml", fn x => x)],
    dflopts = [] }
\end{verbatim}

\begin{sloppy}
This code can either be packaged as a CM library or entered at the
interactive top level after loading the {\tt \$smlnj/cm/ tools.cm}
library via {\tt CM.make} or {\tt CM.load\_plugin}.  ({\tt
CM.autoload} is not enough because of its lazy nature which prevents
the required side-effects to occur.)
\end{sloppy}

In our example, the shell command name for our tool is {\tt
new-ml-yacc}.  When looking for this command in the file system, CM
first tries to treat it as a path anchor (see
section~\ref{sec:anchor:env}).  For example, suppose {\tt new-ml-yacc} is
mapped to {\tt /bin}.  In this case the command to be
invoked would be {\tt /bin/new-ml-yacc}.  If path anchor resolution
fails, then the command name will be used as-is.  Normally this
causes the shell's path search mechanism to be used as a fallback.

{\tt Tools.registerStdShellCmdTool} creates the class and installs the
tool for it.  The arguments must be specified as follows:

\begin{description}
\item[tool] a descriptive name of the tool (used in error messages);
type: {\tt string}
\item[class] the name of the class; the string must not contain
upper-case letters; type: {\tt string}
\item[suffixes] a list of file name suffixes that let CM automatically
recognize files of the class; type: {\tt string list}
\item[cmdStdPath] the command string from above; type: {\tt string}
\item[template] an optional string that describes how the command line
is to be constructed from pieces; \\
The string is taken verbatim except for embedded \% format specifiers:
  \begin{description}\setlength{\itemsep}{0pt}
  \item[\%c] the command name (i.e., the elaboration of {\tt cmdStdPath})
  \item[\%s] the source file name in native pathname syntax
  \item[\%$n$t] the $n$-th target file in native pathname syntax; \\
    ($n$ is specified as a decimal number, counting starts at $1$, and
    each target file name is constructed from the corresponding {\tt
    extensionStyle} entry; if $n$ is $0$ (or missing), then all
    targets---separated by single spaces---are inserted;
    if $n$ is not in the range between $0$ and the number of available
    targets, then {\bf \%$n$t} expands into itself) 
  \item[\%$n$o] the $n$-th tool parameter; \\
    (named sub-option parameters are ignored;
     $n$ is specified as a decimal number, counting starts at $1$;
     if $n$ is $0$ (or missing), then all options---separated by single
     spaces---are inserted;
     if $n$ is not in the range between $0$ and the number of available
     options, then {\bf \%$n$o} expands into itself) 
  \item[\%$x$] the character $x$ (where $x$ is neither {\bf c}, nor
    {\bf s}, {\bf t}, or {\bf o})
  \end{description}
If no template string is given, then it defaults to {\tt "\%c \%s"}.
\item[extensionStyle] a specification of how the names of files
generated by the tool relate to the name of the tool input file;
type: {\tt Tools.extensionStyle}. \\
Currently, there are two possible cases:
\begin{enumerate}
\item ``{\tt Tools.EXTEND} $l$'' says that if the tool source file is
{\it file} then for each suffix {\it sfx} in {\tt (map \#1 $l$)} there
will be one tool output file named {\it file}{\tt .}{\it sfx}.  The
list $l$ consists of triplets where the first component specifies the
suffix string, the second component optionally specifies the
member class name of the corresponding derived file, and the
third component is a function to calculate tool options for the 
target from those of the source. (Argument and result type of these
functions is {\tt Tools.toolopts option}.)
\item ``{\tt Tools.REPLACE }$(l_1, l_2)$'' specifies that given the
base name {\it base} there will be one tool output file {\it base}{\tt
.}{\it sfx} for each suffix {\it sfx} in {\tt (map \#1 $l_2$)}.  Here,
{\it base} is determined by the following rule: If the name of the
tool input file has a suffix that occurs in $l_1$, then {\it base} is
the name without that suffix.  Otherwise the whole file name is taken
as {\it base} (just like in the case of {\tt Tools.EXTEND}).  As with
{\tt Tools.EXTEND}, the second components of the elements of $l_2$ can
optionally specify the member class name of the corresponding derived
file, and the third component maps source options to target options.
\end{enumerate}
\item[dflopts] a list of tool options which is used for
substituting {\bf \%$n$o} fields in {\tt template} (see above) if no
options were specified.  (Note that the value of {\tt dflopts} is never
passed to the option mappers in {\tt Tools.EXTEND} or {\tt
Tools. REPLACE}.)  Type: {\tt Tools.toolopts}.
\end{description}

Less common kinds of rules can also be defined using the generic
interface {\tt Tools.registerClass}.

\subsection{Plug-in Tools}

If CM comes across a member class name $c$ that it does not know
about, then it tries to load a plugin module named {\tt \$}$c${\tt
-tool.cm} or {\tt ./}$c${\tt -tool.cm}.  If it sees a file whose name
ends in suffix $s$ for which no explicit member class has been
specified in the CM description file and for which automatic
member classification fails, then it tries to load a plugin module
named {\tt \$}$s${\tt -ext.cm} or {\tt ./}$s${\tt -ext.cm}.  The
so-loaded module can then register the required tool which enables CM
to successfully deal with the previously unknown member.

This mechanism makes it possible for new tools to be added by simply
placing appropriately-named plug-in libraries in such a way that CM
can find them.  This can be done in one of two ways:

\begin{enumerate}
\item For general-purpose tools that are installed in some central
place, corresponding tool description files {\tt \$}$c${\tt -tool.cm}
and {\tt \$}$s${\tt -ext.cm} should be registered using the path
anchor mechanism.  If this is done, actual description files can be
placed in arbitrary locations.
\item For special-purpose tools that are part of a specific program
and for which there is no need for central installation, one should
simply put the tool description files into the same directory as the
one that contains their ``client'' description file.
\end{enumerate}

\section{Parallel and distributed compilation}
\label{sec:parmake}

To speed up recompilation of large projects with many ML source files,
CM can exploit parallelism that is inherent in the dependency graph.
Currently, the only kind of operating system for which this is
implemented is Unix ({\tt OPSYS\_UNIX}), where separate processes are
used.  From there, one can distribute the work across a network of
machines by taking advantage of the network file system and the
``rsh'' facility.

To perform parallel compilations, one must attach ``compile servers''
to CM.  This is done using function\linebreak {\tt CM.Server.start}
with the following signature:

\begin{verbatim}
  structure Server : sig
    type server
    val start : { name: string,
                  cmd: string * string list,
                  pathtrans: (string -> string) option,
                  pref: int } -> server option
  end
\end{verbatim}

Here, {\tt name} is a string uniquely identifying the server and {\tt
cmd} is a value suitable as argument to {\tt Unix.execute}.

The program to be specified by {\tt cmd} should be another instance of
CM---running in ``slave mode''.  To start CM in slave mode, start {\tt
sml} with a single command-line argument of {\tt @CMslave}.  For
example, if you have installed in /path/to/smlnj/bin/sml, then a
server process on the local machine could be started by

\begin{verbatim}
  CM.Server.start { name = "A", pathtrans = NONE, pref = 0,
                    cmd = ("/path/to/smlnj/bin/sml",
                           ["@CMslave"]) };
\end{verbatim}

To run a process on a remote machine, e.g., ``thatmachine'', as
compute server, one can use ``rsh''.\footnote{On certain systems it
may be necessary to wrap {\tt rsh} into a script that protects rsh
from interrupt signals.}  Unfortunately, at the moment it
is necessary to specify the full path to ``rsh'' because {\tt
Unix.execute} (and therefore {\tt CM.Server.start}) does not perform
a {\tt PATH} search. The remote machine
must share the file system with the local machine, for example via NFS.

\begin{verbatim}
  CM.Server.start { name = "thatmachine",
                    pathtrans = NONE, pref = 0,
                    cmd = ("/usr/ucb/rsh",
                           ["thatmachine",
                            "/path/to/smlnj/bin/sml",
                            "@CMslave"]) };
\end{verbatim}

You can start as many servers as you want, but they all must have
different names.  If you attach any servers at all, then you should
attach at least two (unless you want to attach one that runs on a
machine vastly more powerful than your local one).  Local servers make
sense on multi-CPU machines: start as many servers as there are CPUs.
Parallel make is most effective on multiprocessor machines because
network latencies can have a severely limiting effect on what can be
gained in the distributed case.
(Be careful, though.  Since there is no memory-sharing to speak of
between separate instances of {\tt sml}, you should be sure to check
that your machine has enough main memory.)

If servers on machines of different power are attached, one can give
some preference to faster ones by setting the {\tt pref} value higher.
(But since the {\tt pref} value is consulted only in the rare case
that more than one server is idle, this will rarely lead to vastly
better throughput.) All attached servers must use the same
architecture-OS combination as the controlling machine.

In parallel mode, the master process itself normally does not compile
anything.  Therefore, if you want to utilize the master's CPU for
compilation, you should start a compile server on the same machine
that the master runs on (even if it is a uniprocessor machine).

The {\tt pathtrans} argument is used when connecting to a machine with
a different file-system layout.  For local servers, it can safely be
left at {\tt NONE}.  The ``path transformation'' function is used to
translate local path names to their remote counterparts.  This can be
a bit tricky to get right, especially if the machines use automounters
or similar devices.  The {\tt pathtrans} functions consumes and
produces names in CM's internal ``protocol encoding'' (see
Section~\ref{sec:pathencode}).

Once servers have been attached, one can invoke functions like
{\tt CM.recomp}, {\tt CM.make}, and {\tt CM.stabilize}.  They should
work the way the always do, but during compilation they will take
advantage of parallelism.

When CM is interrupted using Control-C (or such), one will sometimes
experience a certain delay if servers are currently attached and busy.
This is because the interrupt-handling code will wait for the servers
to finish what they are currently doing and bring them back to an
``idle'' state first.

\subsection{Pathname protocol encoding}
\label{sec:pathencode}

A path encoded by CM's master-slave protocol encoding does not only
specify which file a path refers to but also, in some sense, specifies
why CM constructed this path in the first place.  For example, the
encoding {\tt a/b/c.cm:d/e.sml} represents the file {\tt a/b/d/e.sml}
but also tells us that it was constructed by putting {\tt d/e.sml}
into the context of description file {\tt a/b/c.cm}.  Thus, an encoded
path name consists of one or more colon-separated ({\bf :}) sections,
and each section consists of slash-separated ({\bf /}) arcs.  To find
out what actual file a path refers to, it is necessary to erase all
arcs that precede colons.

The first section is special because it also specifies whether the
whole path was relative or absolute, or whether it was an anchored
path.

\begin{description}
\item[Anchored paths] start with a dollar-symbol {\bf \$}.  The name
of the anchor is the string between this leading dollar-symbol and the
first occurence of a slash {\bf /} within the first section.  The
remaining arcs of the first section are interpreted relative to the
current value of the anchor.
\item[Absolute paths] start either with a percent-sign {\bf \%} or a
slash {\bf /}.  The canonical form is the one with the percent-sign:
it specifies the volume name between the {\bf \%} and the first slash.
The common case where the volume name is empty (i.e, {\em always} on
Unix systems), the path starts with {\bf /}.
\item[Relative paths] are all other paths.
\end{description}

Encoded path names never contain white space.  Moreover, the encoding
for path arcs, volume names, or anchor names does not contain special
characters such as {\bf /}, {\bf \$}, {\bf \%}, {\bf :}, {\bf
\verb|\|}, {\bf (}, and {\bf )}.  Instead, should white space or
special characters occur in the non-encoded name, then they will be
encoded using the escape-sequence \verb|\ddd| where {\tt ddd} is the
decimal value of the respective character's ordinal number (i.e, the
result of applying {\tt Char.ord}).

The so-called {\em current} arc is encoded as {\bf .}, the {\em
parent} arc uses {\bf ..} as its representation.  On some operating
systems it can happen that although an arc is either {\tt .} or {\tt
..}, it still does not actually refer to the current or the parent
arc.  In such a case, CM will encode the dots in these names using the
\verb|\ddd| method, too.

When issuing progress messages, CM shows path names in a form that is
almost the same as the protocol encoding.  The only difference is that
arcs that precede colon-sign {\bf :} are enclosed within parentheses
to emphasize that they are ``not really there''.  The same form is
also used by {\tt CM.Library.descr}.

\subsection{Parallel bootstrap compilation}

The bootstrap compiler\footnote{otherwise not mentioned in this
document} with its main function {\tt CMB.make} and the corresponding
cross-compilation variants of the bootstrap compiler will also use any
attached compile servers.  If one intends to exclusively use the
bootstrap compiler, one can even attach servers that run on machines
with different architecture or operating system.

Since the master-slave protocol is fairly simple, it cannot handle
complicated scenarios such as the one necessary for compiling the
``init group'' (i.e., the small set of files necessary for setting up
the ``pervasive'' environment) during {\tt CMB.make}.  Therefore, this
will always be done locally by the master process.

\section{Example: Dynamic linking}
\label{sec:dynlink}

Autoloading is convenient and avoids wasted memory for modules that
should be available at the interactive prompt but have not actually
been used so far.  However, sometimes one wants to be even more
aggressive and save the space needed for a function until---at
runtime---that function is actually being dynamically invoked.

CM does not provide immediate support for this kind of {\em dynamic
linking}, but it is quite simple to achieve the effect by carefully
arranging some helper libraries and associated stub code.

Consider the following module:
\begin{verbatim}
structure F = struct
    fun f (x: int): int =
        G.g x + H.h (2 * x + 1)
end
\end{verbatim}

Let us further assume that the implementations of structures {\tt G}
and {\tt H} are rather large so that it would be worthwhile to avoid
loading the code for {\tt G} and {\tt H} until {\tt F.f} is called
with some actual argument.  Of course, if {\tt F} were bigger, then we
also want to avoid loading {\tt F} itself.

To achieve this goal, we first define a {\em hook} module which will
be the place where the actual implementation of our function will be
registered once it has been loaded.  This hook module is then wrapped
into a hook library.  Thus, we have {\tt f-hook.cm}:
\begin{verbatim}
Library
        structure F_Hook
is
        f-hook.sml
\end{verbatim}

and {\tt f-hook.sml}:

\begin{verbatim}
structure F_Hook = struct
    local
        fun placeholder (i: int) : int =
            raise Fail "F_Hook.f: unitinialized"
        val r = ref placeholder
    in
        fun init f = r := f
        fun f x = !r x
    end
end
\end{verbatim}

The hook module provides a reference cell into which a function of
type equal to {\tt F.f} can be installed.  Here we have chosen to hide
the actual reference cell behind a {\bf local} construct.  Accessor
functions are provided to install something into the hook
({\tt init}) and to invoke the so-installed value ({\tt f}).

With this preparation we can write the implementation module {\tt f-impl.sml}
in such a way that not only does it provide the actual
code but also installs itself into the hook:
\begin{verbatim}
structure F_Impl = struct
    local
        fun f (x: int): int =
            G.g x + H.h (2 * x + 1)
    in
        val _ = F_Hook.init f
    end
end
\end{verbatim}
\noindent The implementation module is wrapped into its implementation
library {\tt f-impl.cm}:
\begin{verbatim}
Library
        structure F_Impl
is
        f-impl.sml
        f-hook.cm
        g.cm       (* imports G *)
        h.cm       (* imports H *)
\end{verbatim}
\noindent Note that {\tt f-impl.cm} must mention {\tt f-hook.cm} for
{\tt f-impl.sml} to be able to access structure {\tt F\_Hook}.

Finally, we replace the original contents of {\tt f.sml} with a stub
module that defines structure {\tt F}:
\begin{verbatim}
structure F = struct
    local
        val initialized = ref false
    in
        fun f x =
            (if !initialized then ()
              else if CM.make "f-impl.cm" then initialized := true
              else raise Fail "dynamic linkage for F.f failed";
             F_Hook.f x)
    end
end
\end{verbatim}
\noindent The trick here is to explicitly invoke {\tt CM.make} the
first time {\tt F.f} is called.  This will then cause {\tt f-impl.cm}
(and therefore {\tt g.cm} and also {\tt h.cm}) to be loaded and the
``real'' implementation of {\tt F.f} to be registered with the hook
module from where it will then be available to this and future calls
of {\tt F.f}.

For the new {\tt f.sml} to be compiled successfully it must be placed
into a library {\tt f.cm} that mentions {\tt f-hook.cm} and {\tt
\$smlnj/cm/full.cm}.  As we have seen, {\tt f-hook.cm} exports {\tt
F\_Hook.f} and {\tt \$smlnj/cm/full.cm} is needed because it exports
{\tt CM.make}:

\begin{verbatim}
Library
        structure F
is
        f.sml
        f-hook.cm
        $smlnj/cm.cm (* or $smlnj/cm/full.cm *)
\end{verbatim}

\noindent{\bf Beware!}  This solution makes use of {\tt \$smlnj/cm.cm}
which in turn requires the SML/NJ compiler to be present.  Therefore,
is worthwhile only for really large program modules where the benefits
of their absence are not outweighed be the need for the compiler.

\section{Some history}

Although its programming model is more general, CM's implementation is
closely tied to the Standard ML programming language~\cite{milner97}
and its SML/NJ implementation~\cite{appel91:sml}.

The current version is preceded by several other compilation managers.
Of those, the most recent went by the same name
``CM''~\cite{blume95:cm}, while earlier ones were known as IRM ({\it
Incremental Recompilation Manager})~\cite{harper94:irm} and SC (for
{\it Separate Compilation})~\cite{harper-lee-pfenning-rollins-CM}.  CM
owes many ideas to SC and IRM.

Separate compilation in the SML/NJ system heavily relies on mechanisms
for converting static environments (i.e., the compiler's symbol
tables) into linear byte stream suitable for storage on
disks~\cite{appel94:sepcomp}.  However, unlike all its predecessors,
the current implementation of CM is integrated into the main compiler
and no longer relies on the {\em Visible Compiler} interface.

\pagebreak

\appendix

\section{CM description file syntax}

\subsection{Lexical Analysis}

The CM parser employs a context-sensitive scanner.  In many cases this
avoids the need for ``escape characters'' or other lexical devices
that would make writing description files cumbersome.  On the other
hand, it increases the complexity of both documentation and implementation.

The scanner skips all nestable SML-style comments (enclosed with {\bf
(*} and {\bf *)}).

Lines starting with {\bf \#line} may list up to three fields separated
by white space.  The first field is taken as a line number and the
last field (if more than one field is present) as a file name.  The
optional third (middle) field specifies a column number.  A line of
this form resets the scanner's idea about the name of the file that it
is currently processing and about the current position within that
file.  If no file is specified, the default is the current file.  If
no column is specified, the default is the first column of the
(specified) line.  This feature is meant for program-generators or
tools such as {\tt noweb} but is not intended for direct use by
programmers.

The following lexical classes are recognized:

\begin{description}
\item[Namespace specifiers:] {\bf structure}, {\bf signature},
{\bf functor}, or {\bf funsig}.  These keywords are recognized
everywhere.
\item[CM keywords:] {\bf group}, {\bf Group}, {\bf GROUP}, {\bf
library}, {\bf Library}, {\bf LIBRARY}, {\bf is}, {\bf IS}.  These
keywords are recognized everywhere except within ``preprocessor''
lines (lines starting with {\bf \#}) or following one of the namespace
specifiers.
\item[Preprocessor control keywords:] {\bf \#if}, {\bf \#elif}, {\bf
\#else}, {\bf \#endif}, {\bf \#error}.  These keywords are recognized
only at the beginning of the line and indicate the start of a
``preprocessor'' line.  The initial {\bf \#} character may be
separated from the rest of the token by white space (but not by comments).
\item[Preprocessor operator keywords:] {\bf defined}, {\bf div}, {\bf
mod}, {\bf andalso}, {\bf orelse}, {\bf not}.  These keywords are
recognized only when they occur within ``preprocessor'' lines.  Even
within such lines, they are not recognized as keywords when they
directly follow a namespace specifier---in which case they are
considered SML identifiers.
\item[SML identifiers (\nt{mlid}):] Recognized SML identifiers
include all legal identifiers as defined by the SML language
definition. (CM also recognizes some tokens as SML identifiers that
are really keywords according to the SML language definiten. However,
this can never cause problems in practice.)  SML identifiers are
recognized only when they directly follow one of the namespace
specifiers.
\item[CM identifiers (\nt{cmid}):] CM identifiers have the same form
as those ML identifiers that are made up solely of letters, decimal
digits, apostrophes, and underscores.  CM identifiers are recognized when they
occur within ``preprocessor'' lines, but not when they directly follow
some namespace specifier.
\item[Numbers (\nt{number}):] Numbers are non-empty sequences of
decimal digits.  Numbers are recognized only within ``preprocessor''
lines.
\item[Preprocessor operators:] The following unary and binary operators are
recognized when they occur within ``preprocessor'' lines: {\tt +},
{\tt -}, {\tt *}, {\tt /}, {\tt \%}, {\tt <>}, {\tt !=}, {\tt <=},
{\tt <}, {\tt >=}, {\tt >}, {\tt ==}, {\tt =}, $\tilde{~}$, {\tt
\&\&}, {\tt ||}, {\tt !}.  Of these, the following (``C-style'')
operators are considered obsolete and trigger a warning
message\footnote{The use of {\tt -} as a unary minus also triggers
this warning.} as long as {\tt CM.Control.warn\_obsolete} is set to
{\tt true}: {\tt /}, {\tt \%}, {\tt !=}, {\tt ==}, {\tt \&\&}, {\tt
||}, {\tt !}.
\item[Standard path names (\nt{stdpn}):] Any non-empty sequence of
upper- and lower-case letters, decimal digits, and characters drawn
from {\tt '\_.;,!\%\&\$+/<=>?@$\tilde{~}$|\#*-\verb|^|} that occurs
outside of ``preprocessor'' lines and is neither a namespace specifier
nor a CM keyword will be recognized as a stardard path name.  Strings
that lexically constitute standard path names are usually---but not
always---interpreted as file names. Sometimes they are simply taken as
literal strings.  When they act as file names, they will be
interpreted according to CM's {\em standard syntax} (see
Section~\ref{sec:basicrules}).  (Member class names, names of
privileges, and many tool optios are also specified as standard path
names even though in these cases no actual file is being named.)
\item[Native path names (\nt{ntvpn}):] A token that has the form of an
SML string is considered a native path name.  The same rules as in SML
regarding escape characters apply.  Like their ``standard''
counterparts, native path names are not always used to actually name
files, but when they are, they use the native file name syntax of the
underlying operating system.
\item[Punctuation:] A colon {\bf :} is recognized as a token
everywhere except within ``preprocessor'' lines. Parentheses {\bf ()}
are recognized everywhere.
\end{description}

\subsection{EBNF for preprocessor expressions}

\noindent{\em Lexical conventions:}\/ Syntax definitions use {\em
Extended Backus-Naur Form} (EBNF).  This means that vertical bars
\vb separate two or more alternatives, curly braces \{\} indicate
zero or more copies of what they enclose (``Kleene-closure''), and
square brackets $[]$ specify zero or one instances of their enclosed
contents.  Round parentheses () are used for grouping.  Non-terminal
symbols appear in \nt{this}\/ typeface; terminal symbols are
\tl{underlined}.

\noindent The following set of rules defines the syntax for CM's
preprocessor expressions (\nt{ppexp}):

\begin{tabular}{rcl}
\nt{aatom}  &\ar& \nt{number} \vb \nt{cmid} \vb \tl{(} \nt{asum} \tl{)} \vb (\ttl{$\tilde{~}$} \vb \ttl{-}) \nt{aatom} \\
\nt{aprod}  &\ar& \{\nt{aatom} (\ttl{*} \vb \tl{div} \vb \tl{mod}) \vb \ttl{/} \vb \ttl{\%} \} \nt{aatom} \\
\nt{asum}   &\ar& \{\nt{aprod} (\ttl{+} \vb \ttl{-})\} \nt{aprod} \\
\\
\nt{ns}     &\ar& \tl{structure} \vb \tl{signature} \vb \tl{functor} \vb \tl{funsig} \\
\nt{mlsym}  &\ar& \nt{ns} \nt{mlid} \\
\nt{query}  &\ar& \tl{defined} \tl{(} \nt{cmid} \tl{)} \vb \tl{defined} \tl{(} \nt{mlsym} \tl{)} \\
\\
\nt{acmp}   &\ar& \nt{asum} (\ttl{<} \vb \ttl{<=} \vb \ttl{>} \vb \ttl{>=} \vb \ttl{=} \vb \ttl{==} \vb \ttl{<>} \vb \ttl{!=}) \nt{asum} \\
\\
\nt{batom}  &\ar& \nt{query} \vb \nt{acmp} \vb (\tl{not} \vb \ttl{!}) \nt{batom} \vb \tl{(} \nt{bdisj} \tl{)} \\
\nt{bcmp}   &\ar& \nt{batom} [(\ttl{=} \vb \ttl{==} \vb \ttl{<>} \vb \ttl{!=}) \nt{batom}] \\
\nt{bconj}  &\ar& \{\nt{bcmp} (\tl{andalso} \vb \ttl{\&\&})\} \nt{bcmp} \\
\nt{bdisj}  &\ar& \{\nt{bconj} (\tl{orelse} \vb \ttl{||})\} \nt{bconj} \\
\\
\nt{ppexp} &\ar& \nt{bdisj}
\end{tabular}

\subsection{EBNF for export lists}

The following set of rules defines the syntax for export lists (\nt{elst}):

\begin{tabular}{rcl}
\nt{guardedexport} &\ar& \{ \nt{export} \} (\tl{\#endif} \vb
\tl{\#else} \{ \nt{export} \} \tl{\#endif} \vb \tl{\#elif} \nt{ppexp}
\nt{guardedexports}) \\
\nt{restline}      &\ar& rest of current line up to next newline character \\
\nt{export}        &\ar& \nt{mlsym} \vb \tl{\#if} \nt{ppexp}
\nt{guardedexports} \vb \tl{\#error} \nt{restline}  \\
\nt{elst}       &\ar& \nt{export} \{ \nt{export} \} \\
\end{tabular}

\subsection{EBNF for tool options}

The following set of rules defines the syntax for tool options
(\nt{toolopts}):

\begin{tabular}{rcl}
\nt{pathname} &\ar& \nt{stdpn} \vb \nt{ntvpn} \\
\nt{toolopts} &\ar& \{ \nt{pathname} [\tl{:} (\tl{(} \nt{toolopts} \tl{)} \vb \nt{pathname})] \}
\end{tabular}

\subsection{EBNF for member lists}

The following set of rules defines the syntax for member lists (\nt{members}):

\begin{tabular}{rcl}
\nt{class}          &\ar& \nt{stdpn} \\
\nt{member}         &\ar& \nt{pathname} [\tl{:} \nt{class}] [\tl{(} \nt{toolopts} \tl{)}] \\
\nt{guardedmembers} &\ar& \nt{members} (\tl{\#endif} \vb \tl{\#else} \nt{members} \tl{\#endif} \vb \tl{\#elif} \nt{ppexp} \nt{guardedmembers}) \\
\nt{members}        &\ar& \{ (\nt{member} \vb \tl{\#if} \nt{ppexp}
\nt{guardedmembers} \vb \tl{\#error} \nt{restline}) \} 
\end{tabular}

\subsection{EBNF for library descriptions}

The following set of rules defines the syntax for library descriptions
(\nt{library}).  Notice that although the syntax used for \nt{version}
is the same as that for \nt{stdpn}, actual version strings will
undergo further analysis according to the rules given in
section~\ref{sec:versions}:

\begin{tabular}{rcl}
\nt{libkw}     &\ar& \tl{library} \vb \tl{Library} \vb \tl{LIBRARY} \\
\nt{version}   &\ar& \nt{stdpn} \\
\nt{privilege} &\ar& \nt{stdpn} \\
\nt{lprivspec} &\ar& \{ \nt{privilege} \vb \tl{(} \{ \nt{privilege} \} \tl{)} \} \\
\nt{library}   &\ar& [\nt{lprivspec}] \nt{libkw} [\tl{(} \nt{version} \tl{)}] \nt{elst} (\tl{is} \vb \tl{IS}) \nt{members}
\end{tabular}

\subsection{EBNF for library component descriptions (group descriptions)}

The main differences between group- and library-syntax can be
summarized as follows:

\begin{itemize}\setlength{\itemsep}{0pt}
\item Groups use keyword \tl{group} instead of \tl{library}.
\item Groups may have an empty export list.
\item Groups cannot wrap privileges, i.e., names of privileges (in
front of the \tl{group} keyword) never appear within parentheses.
\item Groups have no version.
\item Groups have an optional owner.
\end{itemize}

\noindent The following set of rules defines the syntax for library
component (group) descriptions (\nt{group}):

\begin{tabular}{rcl}
\nt{groupkw}   &\ar& \tl{group} \vb \tl{Group} \vb \tl{GROUP} \\
\nt{owner}     &\ar& \nt{pathname} \\
\nt{gprivspec} &\ar& \{ \nt{privilege} \} \\
\nt{group}     &\ar& [\nt{gprivspec}] \nt{groupkw} [\tl{(} \nt{owner} \tl{)}] [\nt{elst}] (\tl{is} \vb \tl{IS}) \nt{members}
\end{tabular}

\section{Full signature of {\tt structure CM}}

Structure {\tt CM} serves as the compilation manager's user interface
and also constitutes the major part of the API.  The structure is the
(only) export of library {\tt \$smlnj/cm.cm}.  The standard
installation procedure of SML/NJ registers this library for
autoloading at the interactive top level.

\begin{verbatim}
signature CM = sig

    val autoload : string -> bool
    val make : string -> bool
    val recomp : string -> bool
    val stabilize : bool -> string -> bool

    type 'a controller = { get : unit -> 'a, set : 'a -> unit }

    structure Anchor : sig
        val anchor : string -> string option controller
        val reset : unit -> unit
    end

    structure Control : sig
        val keep_going : bool controller
        val verbose : bool controller
        val parse_caching : int controller
        val warn_obsolete : bool controller
        val debug : bool controller
        val conserve_memory : bool controller
    end

    structure Library : sig
        type lib
        val known : unit -> lib list
        val descr : lib -> string
        val osstring : lib -> string
        val dismiss : lib -> unit
        val unshare : lib -> unit
    end

    structure State : sig
        val synchronize : unit -> unit
        val reset : unit -> unit
        val pending : unit -> string list
    end

    structure Server : sig
        type server
        val start : { cmd : string * string list,
                      name : string,
                      pathtrans : (string -> string) option,
                      pref : int } -> server option
        val stop : server -> unit
        val kill : server -> unit
        val name : server -> string
    end

    val sources :
        { arch: string, os: string } option ->
        string -> { file: string, class: string, derived: bool } list option

    val symval : string -> int option controller
    val load_plugin : string -> bool

    val mk_standalone : bool option -> string -> string list option
end

structure CM : CM
\end{verbatim}

\section{Listing of all pre-defined CM identifiers}

\begin{center}
\begin{tabular}{l||c|c|c|c|c|c|c}
   & Alpha32 & HP-PA & PowerPC & PowerPC & Sparc & IA32 & IA32 \\
   & Unix & Unix & MACOS & Unix & Unix & Unix & Win32 \\
\hline \hline
{\tt ARCH\_ALPHA32}  & 1 & & & & & & \\
{\tt ARCH\_HPPA}     & & 1 & & & & & \\
{\tt ARCH\_PPC}      & & & 1 & 1 & & & \\
{\tt ARCH\_SPARC}    & & & & & 1 & & \\
{\tt ARCH\_X86}      & & & & & & 1 & 1 \\
{\tt OPSYS\_UNIX}    & 1 & 1 & & 1 & 1 & 1 & \\
{\tt OPSYS\_MACOS}   & & & 1 & & & & \\
{\tt OPSYS\_WIN32}   & & & & & & & 1 \\
{\tt BIG\_ENDIAN}    & & & & & 1 & & \\
{\tt LITTLE\_ENDIAN} & 1 & 1 & 1 & 1 & & 1 & 1 \\
{\tt SIZE\_32}       & 1 & 1 & 1 & 1 & 1 & 1 & 1 \\
{\tt NEW\_CM}        & 1 & 1 & 1 & 1 & 1 & 1 & 1 \\
{\tt SMLNJ\_VERSION} & \smlmj & \smlmj & \smlmj & \smlmj & \smlmj & \smlmj & \smlmj \\
{\tt SMLNJ\_MINOR\_VERSION} & \smlmn & \smlmn & \smlmn & \smlmn & \smlmn & \smlmn & \smlmn
\end{tabular}
\end{center}

\section{Listing of all CM-specific environment variables}

Most control parameters that affect CM's operation can be adjusted
using environment variables $v_s$ at startup time, i.e, when the {\tt
sml} command is invoked.  Each such parameter has a default setting.
Default settings are determined at bootstrap time, i.e., the time when
the heap image for SML/NJ's interactive system is
built.\footnote{Normally this is the same as installation time, but
for SML/NJ compiler there is also a {\tt makeml} script for the
purpose of bootstrapping.}  At bootstrap time, it is possible to
adjust defaults by using a different set of environment variables
$v_b$.  If neither $v_s$ nor $v_b$ were set, a hard-wired fallback
value will be used.

The rule for constructing (the names of) $v_s$ and $v_b$ is the
following: For each adjustable parameter $x$ there is a {\em name
stem}.  If the stem for $x$ is $s$, then $v_s = \mbox{\tt CM\_}s$ and
$v_b = v_s\mbox{\tt \_DEFAULT}$.

Since the normal installation procedure for SML/NJ sets some of the
$v_b$ variables at bootstrap time, there are two columns with default
values in the following table.  The value labeled {\em fallback} is
the one that would have been used had there been no environment
variable at bootrap time, the one labeled {\em default} is the one the
user will actually see.

To save space, the table lists the stem but not the names for its
associated (longer) $v_s$ and $v_b$.  For example, since the the table
shows {\tt VERBOSE} in the row for {\tt CM.Control.verbose}, CM's
per-session verbosity can be adjusted using {\tt CM\_VERBOSE} and the
boot-time default can be set using {\tt CM\_VERBOSE\_DEFAULT}.

\begin{center}
\begin{small}
\begin{tabular}{@{}l||c|c|c|c|p{1.5in}@{}}
{\tt CM.Control.}$c$ & stem & type & fallback & default & default's meaning \\
\hline \hline
{\tt verbose} & {\tt VERBOSE} & {\tt bool} & {\tt true} & same & issue
progess messages \\
{\tt debug} & {\tt DEBUG} & {\tt bool} & {\tt false} & same & do not
issue debug messages \\
{\tt keep\_going} & {\tt KEEP\_GOING} & {\tt bool} & {\tt false} &
same & quit on first error \\
(none) & {\tt PATHCONFIG} & {\tt string} & see below & see below &
standard library directory of SML/NJ installation \\
{\tt parse\_caching} & {\tt PARSE\_CACHING} & {\tt int} & {\tt 100} &
same & at most 100 parse trees will be cached in main memory \\
(none) & {\tt LOCAL\_PATHCONFIG} & {\tt string} & see below & same &
user-specific path configuration file \\
{\tt warn\_obsolete} & {\tt WARN\_OBSOLETE} & {\tt bool} & {\tt true}
& same & issue warnings about obsolete C-style operators in
description files \\
{\tt conserve\_memory} & {\tt CONSERVE\_MEMORY} & {\tt bool} & {\tt
false} & same & avoid repeated I/O operations by keeping certain
information in main memory
\end{tabular}
\end{small}
\end{center}

The fallback for {\tt PATHCONFIG} is {\tt /usr/lib/smlnj-pathconfig},
but the standard installation overrides this and uses {\tt
\$INSTALLDIR/lib/pathconfig} (where {\tt \$INSTALLDIR} is the SML/NJ
installation directory) instead.

The default for the ``local'' path configuration file is {\tt
.smlnj-pathconfig}. This file is located in the user's home directory
(given by the environment variable {\tt \$HOME}).

\section{Listing of all class names and their tools}

\begin{center}
\begin{tabular}{c|l|c|l}
class & file contents & tool & file name suffixes \\
\hline\hline
sml & ML source code & built-in & {\tt .sig}, {\tt .sml}, {\tt .fun} \\
cm  & CM description file & built-in & {\tt .cm} \\
mlyacc & ML-Yacc grammar & ml-yacc & {\tt .grm}, {\tt .y} \\
mllex & ML-Lex specification & ml-lex & {\tt .lex}, {\tt .l} \\
mlburg & ML-Burg specification & ml-burg & {\tt .burg} \\
noweb & literate program & noweb & {\tt .nw} \\
make & makefile & make & \\
shell & arbitrary & shell command & 
\end{tabular}
\end{center}

\section{Available libraries}

Compiler and interactive system of SML/NJ consist of several hundred
individual compilation units.  Like modules of application programs,
these compilation units are also organized using CM libraries.  

Some of the libraries that make up SML/NJ are actually the same ones
that application programmers are likely to use, others exist for
organizational purposes only.  There are ``plugin'' libraries---mainly
for the CM ``tools'' subsystem---that will be automatically loaded on
demand, and libraries such as {\tt \$smlnj/cmb.cm} can be used to
obtain access to functionality that by default is not present.

\subsection{Libraries for general programming}

Libraries listed in the following table provide a broad palette of
general-purpose programming tools\footnote{Recall that anchored paths
of the form {\tt \$$/x[/\cdots]$} act as an abbreviation for {\tt
\$$x/x[/\cdots]$}.}:

\begin{center}
\begin{tabular}{p{2.3in}||p{2.8in}|c|c}
name & description & installed & loaded \\
\hline\hline
{\tt \$/basis.cm} & Standard Basis Library & always & auto \\
\hline\hline
{\tt \$/ml-yacc-lib.cm} & ML-Yacc library & always & no \\
\hline\hline
{\tt \$/smlnj-lib.cm} & SML/NJ general-purpose utility library &
always & no \\
\hline
{\tt \$/unix-lib.cm} & SML/NJ Unix programming utility library &
optional & no \\
\hline
{\tt \$/inet-lib.cm} & SML/NJ internet programming utility library &
optional & no \\
\hline
{\tt \$/regexp-lib.cm} & SML/NJ regular expression library & optional
& no \\
\hline
{\tt \$/reactive-lib.cm} & SML/NJ reactive programming library &
optional & no \\
\hline
{\tt \$/pp-lib.cm} & SML/NJ pretty-printing library & always & no \\
\hline
{\tt \$/html-lib.cm} & SML/NJ HTML handling library & always & no
\end{tabular}
\end{center}

\subsection{Libraries for controlling SML/NJ's operation}

The following table lists those libraries that provide access to the
so-called {\em visible compiler} infrastructure and to the compilation
manager API.

\begin{center}
\begin{tabular}{p{2.3in}||p{2.5in}|c|c}
name & description & installed & loaded \\
\hline\hline
{\tt \$smlnj/compiler.cm} \newline
{\tt \$smlnj/compiler/current.cm} & visible compiler for current
architecture & always & auto \\
\hline\hline
{\tt \$smlnj/cm.cm} \newline
{\tt \$smlnj/cm/full.cm} & compilation manager & always & auto \\
\hline
{\tt \$smlnj/cm/tools.cm} & API for extending CM with new tools &
always & no \\
\hline\hline
{\tt \$/mllex-tool.cm} & plugin library for class {\tt mllex} & always
& on demand \\
\hline
{\tt \$/lex-ext.cm} & plugin library for extension {\tt .lex} & always
& on demand \\
\hline
{\tt \$/mlyacc-tool.cm} & plugin library for class {\tt mlyacc} &
always & on demand \\
\hline
{\tt \$/grm-ext.cm} & plugin library for extension {\tt .grm} & always
& on demand \\
\hline
{\tt \$/mlburg-tool.cm} & plugin library for class {\tt mlburg} &
always & on demand \\
\hline
{\tt \$/burg-ext.cm} & plugin library for extension {\tt .burg} &
always & on demand \\
\hline
{\tt \$/noweb-tool.cm} & plugin library for class {\tt noweb} & always
& on demand \\
\hline
{\tt \$/nw-ext.cm} & plugin library for extension {\tt .nw} & always &
on demand \\
\hline
{\tt \$/make-tool.cm} & plugin library for class {\tt make} & always &
on demand \\
\hline
{\tt \$/shell-tool.cm} & plugin library for class {\tt shell} & always
& on demand \\
\end{tabular}
\end{center}

\subsection{Libraries for SML/NJ compiler hackers}

The following table lists libraries that provide access to the SML/NJ
{\em bootstrap compiler}.  The bootstrap compiler is a derivative of
the compilation manager.  In addition to being able to recompile
SML/NJ for the ``host'' system there are also cross-compilers that
can target all of SML/NJ's supported platforms.

\begin{center}
\begin{tabular}{p{2.3in}||p{2.8in}|c|c}
name & description & installed & loaded \\
\hline\hline
{\tt \$smlnj/cmb.cm} \newline
{\tt \$smlnj/cmb/current.cm} & bootstrap compiler for current
architecture and OS & always & no \\
\hline\hline
{\tt \$smlnj/cmb/alpha32-unix.cm} & bootstrap compiler for Alpha/Unix
systems & always & no \\
\hline
{\tt \$smlnj/cmb/hppa-unix.cm} & bootstrap compiler for HP-PA/Unix
systems & always & no \\
\hline
{\tt \$smlnj/cmb/ppc-macos.cm} & bootstrap compiler for PowerPC/Unix
systems & always & no \\
\hline
{\tt \$smlnj/cmb/ppc-unix.cm} & bootstrap compiler for PowerPC/MacOS
systems & always & no \\
\hline
{\tt \$smlnj/cmb/sparc-unix.cm} & bootstrap compiler for Sparc/Unix
systems & always & no \\
\hline
{\tt \$smlnj/cmb/x86-unix.cm} & bootstrap compiler for IA32/Unix
systems & always & no \\
\hline
{\tt \$smlnj/cmb/x86-win32.cm} & bootstrap compiler for IA32/Win32
systems & always & no \\
\hline\hline
{\tt \$smlnj/compiler/alpha32.cm} & visible compiler for
Alpha-specific cross-compiler & always & no \\
\hline
{\tt \$smlnj/compiler/hppa.cm} & visible compiler for
HP-PA-specific cross-compiler & always & no \\
\hline
{\tt \$smlnj/compiler/ppc.cm} & visible compiler for
PowerPC-specific cross-compiler & always & no \\
\hline
{\tt \$smlnj/compiler/sparc.cm} & visible compiler for
Sparc-specific cross-compiler & always & no \\
\hline
{\tt \$smlnj/compiler/x86.cm} & visible compiler for
IA32-specific cross-compiler & always & no \\
\hline
{\tt \$smlnj/compiler/all.cm} & visible compilers for all
architecture-specific cross-compilers and all cross-compilation
bootstrap compilers & always & no \\
\end{tabular}
\end{center}

\subsection{Internal libraries}

For completeness, here is the list of other libraries that are part of
SML/NJ's implementation:

\begin{center}
\begin{tabular}{p{2.8in}||p{2.3in}|c|c}
name & description & installed & loaded \\
\hline\hline
{\tt \$MLRISC/Lib.cm} & utility library for MLRISC backend & always &
no \\
\hline 
{\tt \$MLRISC/Control.cm} & control facilities for MLRISC backend &
always & no \\
\hline
{\tt \$MLRISC/MLRISC.cm} & architecture-neutral core of MLRISC backend
& always & no \\
\hline
{\tt \$MLRISC/ALPHA.cm} & Alpha-specific MLRISC backend & always & no \\
\hline
{\tt \$MLRISC/HPPA.cm} & HP-PA-specific MLRISC backend & always & no \\
\hline
{\tt \$MLRISC/PPC.cm} & PowerPC-specific MLRISC backend & always & no \\
\hline
{\tt \$MLRISC/SPARC.cm} & Sparc-specific MLRISC backend & always & no \\
\hline 
{\tt \$MLRISC/IA32.cm} & IA32-specific MLRISC backend & always & no \\
\hline\hline
{\tt \$/comp-lib.cm} & utility library for compiler & always & no \\
\hline
{\tt \$smlnj/viscomp/core.cm} & architecture-neutral core of compiler
& always & no \\
\hline
{\tt \$smlnj/viscomp/alpha32.cm} & Alpha-specific part of compiler &
always & no \\
\hline
{\tt \$smlnj/viscomp/hppa.cm} & HP-PA-specific part of compiler &
always & no \\
\hline
{\tt \$smlnj/viscomp/ppc.cm} & PowerPC-specific part of compiler &
always & no \\
\hline
{\tt \$smlnj/viscomp/sparc.cm} & Sparc-specific part of compiler &
always & no \\
\hline
{\tt \$smlnj/viscomp/x86.cm} & IA32-specific part of compiler & always
& no \\
\hline \hline
{\tt \$smlnj/init/init.cmi} & initial ``glue''; implementation of
pervasive environment & always & no \\
\hline \hline
{\tt \$smlnj/internal/cm-lib.cm} & implementation of compilation
manager (not yet specialized to specific backends) & always & no \\
\hline
{\tt \$smlnj/internal/host-compiler-0.cm} & selection of host-specific
visible compiler and specialization of compilation manager & always &
no \\
\hline
{\tt \$smlnj/internal/intsys.cm} & root library implementing the
interactive system and glueing all the other parts together & always &
no
\end{tabular}
\end{center}


\pagebreak

\bibliography{blume,appel,ml}

\end{document}
