\section{MLTree Utilities} 

The \MLRISC{} system contains numerous utilities for working with
MLTree datatypes.  Some of the following utilizes are also useful for clients
use:
\begin{description}
  \item[MLTreeUtils] implements basic hashing, equality and pretty
printing functions,
  \item[MLTreeFold] implements a fold function over the MLTree datatypes,  
  \item[MLTreeRewrite] implements a generic rewriting engine,
  \item[MLTreeSimplify] implements a simplifier that performs algebraic
simplification and constant folding.
\end{description}
\subsubsection{Hashing, Equality, Pretty Printing}

The functor \mlrischref{mltree/mltree-utils.sml}{MLTreeUtils} provides
the basic utilities for hashing an MLTree term, comparing two
MLTree terms for equality and pretty printing.  The hashing and comparision
functions are useful for building hash tables using MLTree datatype as keys.
The signature of the functor is:
\begin{SML}
signature \mlrischref{mltree/mltree-utils.sig}{MLTREE_UTILS} =
sig
   structure T : MLTREE 

   (*
    * Hashing
    *)
   val hashStm   : T.stm -> word
   val hashRexp  : T.rexp -> word
   val hashFexp  : T.fexp -> word
   val hashCCexp : T.ccexp -> word

   (*
    * Equality
    *)
   val eqStm     : T.stm * T.stm -> bool
   val eqRexp    : T.rexp * T.rexp -> bool
   val eqFexp    : T.fexp * T.fexp -> bool
   val eqCCexp   : T.ccexp * T.ccexp -> bool
   val eqMlriscs : T.mlrisc list * T.mlrisc list -> bool

   (*
    * Pretty printing 
    *)
   val show : (string list * string list) -> T.printer

   val stmToString   : T.stm -> string
   val rexpToString  : T.rexp -> string
   val fexpToString  : T.fexp -> string
   val ccexpToString : T.ccexp -> string

end
functor \mlrischref{mltree/mltree-utils.sml}{MLTreeUtils} 
  (structure T : MLTREE
   (* Hashing extensions *)
   val hashSext  : T.hasher -> T.sext -> word
   val hashRext  : T.hasher -> T.rext -> word
   val hashFext  : T.hasher -> T.fext -> word
   val hashCCext : T.hasher -> T.ccext -> word

   (* Equality extensions *)
   val eqSext  : T.equality -> T.sext * T.sext -> bool
   val eqRext  : T.equality -> T.rext * T.rext -> bool
   val eqFext  : T.equality -> T.fext * T.fext -> bool
   val eqCCext : T.equality -> T.ccext * T.ccext -> bool

   (* Pretty printing extensions *)
   val showSext  : T.printer -> T.sext -> string
   val showRext  : T.printer -> T.ty * T.rext -> string
   val showFext  : T.printer -> T.fty * T.fext -> string
   val showCCext : T.printer -> T.ty * T.ccext -> string
  ) : MLTREE_UTILS =
\end{SML} 

The types \sml{hasher}, \sml{equality},
and \sml{printer} represent functions for hashing,
equality and pretty printing.   These are defined as:
\begin{SML} 
   type hasher =
      \{stm    : T.stm -> word,
       rexp   : T.rexp -> word,
       fexp   : T.fexp -> word,
       ccexp  : T.ccexp -> word
      \}    

   type equality =
      \{ stm    : T.stm * T.stm -> bool,
        rexp   : T.rexp * T.rexp -> bool,
        fexp   : T.fexp * T.fexp -> bool,
        ccexp  : T.ccexp * T.ccexp -> bool
      \} 
   type printer =
      \{ stm    : T.stm -> string,
        rexp   : T.rexp -> string,
        fexp   : T.fexp -> string,
        ccexp  : T.ccexp -> string,
        dstReg : T.ty * T.var -> string,
        srcReg : T.ty * T.var -> string
      \}
\end{SML}

For example, to instantiate a \sml{Utils} module for our \sml{DSPMLTree},
we can write:
\begin{SML}
   structure U = MLTreeUtils
     (structure T = DSPMLTree
      fun hashSext \{stm, rexp, fexp, ccexp\} (FOR(i, a, b, s)) =
           Word.fromIntX i + rexp a + rexp b + stm s
      and hashRext \{stm, rexp, fexp, ccexp\} e =
          (case e of
             SUM(i,a,b,c) => Word.fromIntX i + rexp a + rexp b + rexp c
           | SADD(a,b) => rexp a + rexp b
           | SSUB(a,b) => 0w12 + rexp a + rexp b
           | SMUL(a,b) => 0w123 + rexp a + rexp b
           | SDIV(a,b) => 0w1245 + rexp a + rexp b
          )
      fun hashFext _ _ = 0w0
      fun hashCCext _ _ = 0w0
      fun eqSext \{stm, rexp, fexp, ccexp\} 
        (FOR(i, a, b, s), FOR(i', a', b', s')) =
           i=i' andalso rexp(a,a') andalso rexp(b,b') andalso stm(s,s')
      fun eqRext \{stm, rexp, fexp, ccexp\} (e,e') =
       (case (e,e') of
          (SUM(i,a,b,c),SUM(i',a',b',c')) => 
            i=i' andalso rexp(a,a') andalso rexp(b,b') andalso stm(c,c')
        | (SADD(a,b),SADD(a',b')) => rexp(a,a') andalso rexp(b,b')
        | (SSUB(a,b),SSUB(a',b')) => rexp(a,a') andalso rexp(b,b')
        | (SMUL(a,b),SMUL(a',b')) => rexp(a,a') andalso rexp(b,b')
        | (SDIV(a,b),SDIV(a',b')) => rexp(a,a') andalso rexp(b,b')
        | _ => false
       )
      fun eqFext _ _ = true
      fun eqCCext _ _ = true

      fun showSext \{stm, rexp, fexp, ccexp, dstReg, srcReg\}  
            (FOR(i, a, b, s)) =
          "for("^dstReg i^":="^rexp a^".."^rexp b^")"^stm s
      fun ty t = "."^Int.toString t
      fun showRext \{stm, rexp, fexp, ccexp, dstReg, srcReg\} e = 
           (case (t,e) of
             SUM(i,a,b,c) => 
              "sum"^ty t^"("^dstReg i^":="^rexp a^".."^rexp b^")"^rexp c
           | SADD(a,b) => "sadd"^ty t^"("rexp a^","^rexp b^")"
           | SSUB(a,b) => "ssub"^ty t^"("rexp a^","^rexp b^")"
           | SMUL(a,b) => "smul"^ty t^"("rexp a^","^rexp b^")"
           | SDIV(a,b) => "sdiv"^ty t^"("rexp a^","^rexp b^")"
           )
      fun showFext _ _ = ""
      fun showCCext _ _ = ""
     )
\end{SML}

\subsubsection{MLTree Fold}
The functor \mlrischref{mltree/mltree-fold.sml}{MLTreeFold}
provides the basic functionality for implementing various forms of
aggregation function over the MLTree datatypes.  Its signature is
\begin{SML}
signature \mlrischref{mltree/mltree-fold.sig}{MLTREE_FOLD} =
sig
   structure T : MLTREE

   val fold : 'b folder -> 'b folder
end
functor \mlrischref{mltree/mltree-fold.sml}{MLTreeFold}
  (structure T : MLTREE
   (* Extension mechnism *)
   val sext  : 'b T.folder -> T.sext * 'b -> 'b
   val rext  : 'b T.folder -> T.ty * T.rext * 'b -> 'b
   val fext  : 'b T.folder -> T.fty * T.fext * 'b -> 'b
   val ccext : 'b T.folder -> T.ty * T.ccext * 'b -> 'b
  ) : MLTREE_FOLD =
\end{SML}
The type \newtype{folder} is defined as:
\begin{SML}
   type 'b folder =
       \{ stm   : T.stm * 'b -> 'b,
         rexp  : T.rexp * 'b -> 'b,
         fexp  : T.fexp * 'b -> 'b, 
         ccexp : T.ccexp * 'b -> 'b
       \}
\end{SML}


\subsubsection{MLTree Rewriting}

The functor \mlrischref{mltree/mltree-rewrite.sml}{MLTreeRewrite}
implements a generic term rewriting engine which is useful for performing
various transformations on MLTree terms. Its signature is
\begin{SML}
signature \mlrischref{mltree/mltree-rewrite.sig}{MLTREE_REWRITE} =
sig
   structure T : MLTREE

  val rewrite : 
       (* User supplied transformations *)
       \{ rexp  : (T.rexp -> T.rexp) -> (T.rexp -> T.rexp), 
         fexp  : (T.fexp -> T.fexp) -> (T.fexp -> T.fexp),
         ccexp : (T.ccexp -> T.ccexp) -> (T.ccexp -> T.ccexp),
         stm   : (T.stm -> T.stm) -> (T.stm -> T.stm)
       \} -> T.rewriters
end
functor \mlrischref{mltre/mltree-rewrite.sml}{MLTreeRewrite}
  (structure T : MLTREE
   (* Extension *)
   val sext : T.rewriter -> T.sext -> T.sext
   val rext : T.rewriter -> T.rext -> T.rext
   val fext : T.rewriter -> T.fext -> T.fext
   val ccext : T.rewriter -> T.ccext -> T.ccext
  ) : MLTREE_REWRITE =
\end{SML}

The type \newtype{rewriter} is defined in signature
\mlrischref{mltree/mltree.sig}{MLTREE} as:
\begin{SML}
   type rewriter = 
       \{ stm   : T.stm -> T.stm,
         rexp  : T.rexp -> T.rexp,
         fexp  : T.fexp -> T.fexp,
         ccexp : T.ccexp -> T.ccexp
       \} 
\end{SML}
 
\subsubsection{MLTree Simplifier}

The functor \mlrischref{mltree/mltree-simplify.sml}{MLTreeSimplify}
implements algebraic simplification and constant folding for MLTree.
Its signature is:
\begin{SML}
signature \mlrischref{mltree/mltree-simplify.sig}{MLTREE_SIMPLIFIER} =
sig

   structure T : MLTREE

   val simplify  :
       { addressWidth : int } -> T.simplifier
   
end
functor \mlrischref{mltree/mltree-simplify.sml}{MLTreeSimplifier}
  (structure T : MLTREE
   (* Extension *)
   val sext : T.rewriter -> T.sext -> T.sext
   val rext : T.rewriter -> T.rext -> T.rext
   val fext : T.rewriter -> T.fext -> T.fext
   val ccext : T.rewriter -> T.ccext -> T.ccext
  ) : MLTREE_SIMPLIFIER =
\end{SML}

Where type \newdef{simplifier} is defined in signature 
\mlrischref{mltree/mltree.sig}{MLTREE} as:
\begin{SML}
   type simplifier =
       \{ stm   : T.stm -> T.stm,
         rexp  : T.rexp -> T.rexp,
         fexp  : T.fexp -> T.fexp,
         ccexp : T.ccexp -> T.ccexp
       \}
\end{SML}


