% -*- latex -*-

\section{Some history}

Although its programming model is more general, CM's implementation is
closely tied to the Standard ML programming language~\cite{milner97}
and its SML/NJ implementation~\cite{appel91:sml}.

The current version is preceded by several other compilation managers.
Of those, the most recent went by the same name
``CM''~\cite{blume95:cm}, while earlier ones were known as IRM ({\it
Incremental Recompilation Manager})~\cite{harper94:irm} and SC (for
{\it Separate Compilation})~\cite{harper-lee-pfenning-rollins-CM}.  CM
owes many ideas to SC and IRM.

Separate compilation in the SML/NJ system heavily relies on mechanisms
for converting static environments (i.e., the compiler's symbol
tables) into linear byte stream suitable for storage on
disks~\cite{appel94:sepcomp}.  However, unlike all its predecessors,
the current implementation of CM is integrated into the main compiler
and no longer relies on the {\em Visible Compiler} interface.
