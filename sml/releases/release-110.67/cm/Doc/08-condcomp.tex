% -*- latex -*-

\section{Conditional compilation}
\label{sec:preproc}

In its description files, CM offers a simple conditional compilation
facility inspired by the preprocessor for the C language~\cite{k&r2}.
However, it is not really a {\it pre}-processor, and the syntax of the
controlling expressions is borrowed from SML.

Sequences of members can be guarded by {\tt \#if}-{\tt \#endif}
brackets with optional {\tt \#elif} and {\tt \#else} lines in between.
The same guarding syntax can also be used to conditionalize the export
list.  {\tt \#if}-, {\tt \#elif}-, {\tt \#else}-, and {\tt
\#endif}-lines must start in the first column and always
extend to the end of the current line.  {\tt \#if} and {\tt \#elif}
must be followed by a boolean expression.

Boolean expressions can be formed by comparing arithmetic expressions
(using operators {\tt <}, {\tt <=}, {\tt =}, {\tt >=}, {\tt >}, or
{\tt <>}), by logically combining two other boolean expressions (using
operators {\tt andalso}, {\tt orelse}, {\tt =}, or {\tt <>}, by
querying the existence of a CM symbol definition, or by querying the
existence of an exported ML definition.

Arithmetic expressions can be numbers or references to CM symbols, or
can be formed from other arithmetic expressions using operators {\tt
+}, {\tt -} (subtraction), \verb|*|, {\tt div}, {\tt mod}, or $\tilde{~}$
(unary minus).  All arithmetic is done on signed integers.

Any expression (arithmetic or boolean) can be surrounded by
parentheses to enforce precedence.

\subsection{CM variables}
\label{sec:cmvars}

CM provides a number of ``variables'' (names that stand for certain
integers). These variables may appear in expressions of the
conditional-compilation facility. The exact set of variables provided
depends on SML/NJ version number, machine architecture, and
operating system.  A reference to a CM variable is considered an
arithmetic expression. If the variable is not defined, then it
evaluates to 0.  The expression {\tt defined}($v$) is a boolean
expression that yields true if and only if $v$ is a defined CM
variable.

The names of CM variables are formed starting with a letter followed
by zero or more occurences of letters, decimal digits, apostrophes, or
underscores.

The following variables will be defined and bound to 1:
\begin{itemize}
\item depending on the operating system: \\
{\tt OPSYS\_UNIX}, {\tt OPSYS\_WIN32}, {\tt OPSYS\_MACOS}, {\tt
OPSYS\_OS2}, or {\tt OPSYS\_BEOS}
\item depending on processor architecture: \\
{\tt ARCH\_SPARC}, {\tt ARCH\_ALPHA}, {\tt ARCH\_MIPS}, {\tt
ARCH\_X86}, {\tt ARCH\_HPPA}, {\tt ARCH\_RS6000}, or {\tt ARCH\_PPC}
\item depending on the processor's endianness:
{\tt BIG\_ENDIAN} or {\tt LITTLE\_ENDIAN}
\item depending on the native word size of the implementation:
{\tt SIZE\_32} or {\tt SIZE\_64}
\item the symbol {\tt NEW\_CM}
\end{itemize}

Furthermore, the symbol {\tt SMLNJ\_VERSION} will be bound to the
major version number of SML/NJ (i.e., the number before the first dot)
and {\tt SMLNJ\_MINOR\_VERSION} will be bound to the system's minor
version number (i.e., the number after the first dot).

Using the {\tt CM.symval} interface one can define additional
variables or modify existing ones.

\subsection{Querying exported definitions}

An expression of the form {\tt defined}($n$ $s$), where $s$ is an ML
symbol and $n$ is an ML namespace specifier, is a boolean expression
that yields true if and only if any member included before this test
exports a definition under this name.  Therefore, order among members
matters after all (but it remains unrelated to the problem of
determining static dependencies)!  The namespace specifier must be one
of: {\tt structure}, {\tt signature}, {\tt functor}, or {\tt funsig}.

If the query takes place in the ``exports'' section of a description
file, then it yields true if {\em any} of the included members exports
the named symbol.

\noindent Example:

\begin{verbatim}
  Library
      structure Foo
  #if defined(structure Bar)
      structure Bar
  #endif
  is
  #if SMLNJ_VERSION > 110
      new-foo.sml
  #else
      old-foo.sml
  #endif
  #if defined(structure Bar)
      bar-client.sml
  #else
      no-bar-so-far.sml
  #endif
\end{verbatim}

Here, the file {\tt bar-client.sml} gets included if {\tt
SMLNJ\_VERSION} is greater than 110 and {\tt new-foo.sml} exports a
structure {\tt Bar} {\em or} if {\tt SMLNJ\_VERSION <= 110} and {\tt
old-foo.sml} exports structure {\tt Bar}.  Otherwise file {\tt
no-bar-so-far.sml} gets included instead.  In addition, the export of
structure {\tt Bar} is guarded by its own existence.  (Structure {\tt
Bar} could also be defined by {\tt no-bar-so-far.sml} in
which case it would get exported regardless of the outcome of the
other {\tt defined} test.)

\subsection{Explicit errors}

A pseudo-member of the form {\tt \#error $\ldots$}, which---like other
{\tt \#}-items---starts in the first column and extends to the end of
the line, causes an explicit error message to be printed unless it
gets excluded by the conditional compilation logic.  The error message
is given by the remainder of the line after the word {\tt error}.
