% -*- latex -*-

\section{Introduction}

This manual describes a new implementation of CM, the ``Compilation
and Library Manager'' for Standard ML of New Jersey (SML/NJ).  Like
its previous incarnation, CM is in charge of managing separate
compilation and facilitates access to stable libraries.

Most programming projects that use CM are composed of separate {\em
libraries}.  Libraries are collections of ML compilation units.  These
collections themselves can be internally sub-structured using CM's
notion of {\em library components}.

CM offers the following features to the programmer:

\begin{itemize}
\item separate compilation and type-safe linking~\cite{appel94:sepcomp}
\item hierarchical modularity~\cite{blume:appel:cm99}
\item automatic dependency analysis~\cite{blume:depend99}
\item optimization of the compilation process via {\em
cutoff}-recompilation techniques~\cite{tichy94}
\item management of program libraries, distinguishing between libraries
that are {\em under development} and libraries that are {\em stable}
\item operating-system independent file naming (with optional escape
to native file names)
\item adaptability to changing environments using the {\em path anchor}
facility
\item an extensible set of auxiliary {\em tools} that lets CM
seemlessly interoperate with other program generators, source-control
systems, literate-programming facilities, or shell-scripts
\item checked version numbers on libraries
\item (still rudimentary) support for access control
\item access to libraries from the interactive toplevel loop
\item sharing of code that is common to several programs or code that
is common to both user programs and SML/NJ itself
\item management of (the sharing of) link-time state
\item conditional compilation (at compilation unit granularity)
\item support for parallel and distributed compilation
\item facilities for generating stand-alone ML programs
\item a mechanism for deriving dependency information for use
by other compilation managers (e.g., Unix' {\bf make})
\item an API to access all these facilities at SML/NJ's interactive
prompt or from user programs
\end{itemize}

CM puts emphasis on {\em working with libraries}.  This contrasts with
previous compilation managers for SML/NJ where the focus was on
compilation management while libraries were added as an afterthought.

