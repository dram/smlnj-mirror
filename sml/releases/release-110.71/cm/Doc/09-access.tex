% -*- latex -*-

\section{Access control}
\label{sec:access}

The basic idea behind CM's access control is the following: In their
description files, groups and libraries can specify a list of
{\em privileges} that the client must have in order to be able to use them.
Privileges at this level are just names (strings) and must be written
in front of the initial keyword {\tt Library} or {\tt Group}.  If one
group or library imports from another group or library, then
privileges (or rather: privilege requirements) are being inherited.
In effect, to be able to use a program, one must have all privileges
for all its libraries, sub-libraries and library components,
components of sub-libraries, and so on.

Of course, this alone would not yet be satisfactory.  The main service
of the access control system is that it can let a client use an
``unsafe'' library ``safely''.  For example, a library {\tt LSafe.cm}
could ``wrap'' all the unsafe operations in {\tt LUnsafe.cm} with
enough error checking that they become safe.  Therefore, a user of
{\tt LSafe.cm} should not also be required to possess the privileges
that would be required if one were to use {\tt LUnsafe.cm} directly.

In CM's access control model it is possible for a library to ``wrap''
privileges.  If a privilege $P$ has been wrapped, then the user of the
library does not need to have privilege $P$ even though the library is
using another library that requires privilege $P$.  In essence, the
library acts as a go-between who provides the necessary credentials
for privilege $P$ to the sub-library.

Of course, not everybody can be allowed to establish a library with
such a ``wrapped'' privilege $P$.  The programmer who does that should at
least herself have privilege P (but perhaps better, she should have
{\em permission to wrap $P$}---a stronger requirement).

In CM, wrapping a privilege is done by specifying the name of that
privilege within parenthesis.  The wrapping becomes effective once the
library gets stabilized via {\tt CM.stabilize}.  The (not yet
implemented) enforcement mechanism must ensure that anyone who
stabilizes a library that wraps $P$ has permission to wrap $P$.

Note that privileges cannot be wrapped at the level of CM groups.

Access control is a new feature. At the moment, only the basic
mechanisms are implemented, but there is no enforcement.  In other
words, everybody is assumed to have every possible privilege.  CM
merely reports which privileges ``would have been required''.
