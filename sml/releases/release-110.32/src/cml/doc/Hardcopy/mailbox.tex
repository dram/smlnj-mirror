\section{The \texttt{Mailbox} structure}


The \texttt{Mailbox} structure provides buffered asynchronous channels, which we call mailboxes.\- 
\begin{synopsis}
\item {\textbf{signature}} MAILBOX\label{sig-MAILBOX}
\item {\STRUCTURE} Mailbox: MAILBOX\label{str-Mailbox}
\end{synopsis}

\begin{interface}
\item {\index{mbox@mbox!MAILBOX@MAILBOX}}{\textbf{type}} \(\alpha\)~mbox
\item {\index{mailbox@mailbox!MAILBOX@MAILBOX}}{\textbf{val}} {\tt mailbox: \(\mathrm{unit}\rightarrow \alpha\:\mathrm{mbox}\)}

\item {\index{sameMailbox@sameMailbox!MAILBOX@MAILBOX}}{\textbf{val}} {\tt sameMailbox: \((\alpha\:\mathrm{mbox}\:\verb,*,\:\alpha\:\mathrm{mbox})\rightarrow \mathrm{bool}\)}

\item {\index{send@send!MAILBOX@MAILBOX}}{\textbf{val}} {\tt send: \((\alpha\:\mathrm{mbox}\:\verb,*,\:\alpha)\rightarrow \mathrm{unit}\)}

\item {\index{recv@recv!MAILBOX@MAILBOX}}{\textbf{val}} {\tt recv: \(\alpha\:\mathrm{mbox}\rightarrow \alpha\)}

\item {\index{recvEvt@recvEvt!MAILBOX@MAILBOX}}{\textbf{val}} {\tt recvEvt: \(\alpha\:\mathrm{mbox}\rightarrow \alpha\:\mathrm{event}\)}

\item {\index{recvPoll@recvPoll!MAILBOX@MAILBOX}}{\textbf{val}} {\tt recvPoll: \(\alpha\:\mathrm{mbox}\rightarrow \alpha\:\mathrm{option}\)}

\end{interface}

\begin{descr}
\item {\index{mbox@mbox!MAILBOX@MAILBOX}}{\textbf{type}} \(\alpha\)~mbox\label{ty-MAILBOX.mbox}

\begin{speccomment}
\item 

	This is the type constructor for a mailbox.\- 	A mailbox is an unbounded, buffered communication channel.\-     \end{speccomment}
\item {\index{mailbox@mailbox!MAILBOX@MAILBOX}}{\textbf{val}} {\tt mailbox: \(\mathrm{unit}\rightarrow \alpha\:\mathrm{mbox}\)}\label{val-MAILBOX.mailbox}


\begin{speccomment}
\item {\tt mailbox ()         } 
creates a new mailbox.\-     \end{speccomment}
\item {\index{sameMailbox@sameMailbox!MAILBOX@MAILBOX}}{\textbf{val}} {\tt sameMailbox: \((\alpha\:\mathrm{mbox}\:\verb,*,\:\alpha\:\mathrm{mbox})\rightarrow \mathrm{bool}\)}\label{val-MAILBOX.sameMailbox}


\begin{speccomment}
\item {\tt same\-Mailbox (\textit{mb1}, \textit{mb2})         } 
returns {\tt true}, if \textit{mb1} and \textit{mb2} are the same mailbox.\-     \end{speccomment}
\item {\index{send@send!MAILBOX@MAILBOX}}{\textbf{val}} {\tt send: \((\alpha\:\mathrm{mbox}\:\verb,*,\:\alpha)\rightarrow \mathrm{unit}\)}\label{val-MAILBOX.send}


\begin{speccomment}
\item {\tt send (\textit{mb}, \textit{msg})         } 
sends the message \textit{msg} to the mailbox \textit{mb}.\- 	Note that unlike \texttt{CML.\-send}\marginref{send}{val-CML.send}{valCMLsend}, this is a non-blocking 	operation.\-     \end{speccomment}
\item {\index{recv@recv!MAILBOX@MAILBOX}}{\textbf{val}} {\tt recv: \(\alpha\:\mathrm{mbox}\rightarrow \alpha\)}\label{val-MAILBOX.recv}


\begin{speccomment}
\item {\tt recv \textit{mb}         } 
receive the next message from the mailbox \textit{mb}.\- 	If, the mailbox is empty, then this blocks the calling 	thread until there is a message available.\-     \end{speccomment}
\item {\index{recvEvt@recvEvt!MAILBOX@MAILBOX}}{\textbf{val}} {\tt recvEvt: \(\alpha\:\mathrm{mbox}\rightarrow \alpha\:\mathrm{event}\)}\label{val-MAILBOX.recvEvt}


\begin{speccomment}
\item {\tt recv\-Evt \textit{mb}         } 
returns the event value that represents the \texttt{recv} operation 	on \textit{mb}.\-     \end{speccomment}
\item {\index{recvPoll@recvPoll!MAILBOX@MAILBOX}}{\textbf{val}} {\tt recvPoll: \(\alpha\:\mathrm{mbox}\rightarrow \alpha\:\mathrm{option}\)}\label{val-MAILBOX.recvPoll}


\begin{speccomment}
\item 

        This is the non-blocking version of \texttt{recv}.\- 	If the corresponding blocking form would block (because the mailbox is 	empty), then this returns 	\texttt{NONE}, otherwise it returns 	\texttt{SOME} of the received 	message.\-\end{speccomment}
\end{descr}


Note that mailbox buffers are unbounded, which means that there is no flow control to prevent a producer from greatly outstriping a consumer, and thus exhausting memory.\- In situations where there is no natural limit to the rate of \texttt{send}\marginref{send}{val-MAILBOX.send}{valMAILBOXsend} operations, it is recommended that the synchronous channels from the \texttt{CML} structure be used instead.\-
\begin{seealso}
\texttt{CML}
\end{seealso}
