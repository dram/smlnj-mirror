% -*- latex -*-

\section{File naming---the role of path anchors}

Under normal operation of CM, the mapping from path names to the files
they denote is supposed to be a fixed one.  The path anchor mechanism
is merely a means of configuring this mapping according to the
actual filesystem layout.

The bootstrap compiler and the associated boot procedure, however, use
path anchors extensively for other purposes.  In particular, the
above-mentioned mapping will vary over time.

The ``casual'' compiler hacker does not actually need to remember all
the details because these details are mostly taken care of
automatically by {\tt CMB} and its various scripts ({\tt makeml}, {\tt
testml}, {\tt installml}).  But it is useful to have a rough idea of
what r\^{o}les are being played by various files and directories that
are being used or created.

\subsection{Anchoring requirement}

During bootstrap compilation it is necessary that all pathnames be
either anchored or relative to another anchored pathname.  This rule
guarantees that every path name can be mapped to different locations
in the filesystem at different times.  (Technically, this restriction
is necessary only for library description files.)

\subsection{Different anchor mappings at different times}

We can distinguish between four different anchor configurations which
will be in effect at different times:

\begin{description}
\item[{\bf compilation}:]
At the time {\tt CMB.make} runs, the anchor configuration is taken
from file {\tt pathconfig}.\footnote{All relative pathnames shown here
are relative to {\tt src/system}.}  This configuration maps anchors to
their respective directories in the actual source tree.  At the same
time the policy which determines the names of binfiles
and stablefiles is modified in such a way that all binfiles will be
created in directory {\tt $u$.bin.{\it arch}-{\it os}} and all
stablefiles will be created in {\tt $u$.boot.{\it arch}-{\it os}}.
In particular, if {\tt \$$a$} is the root anchor name controlling
the path leading to an ML source file
{\tt \$$a$/$\cdots$/{\it file}.sml},
then the corresponding binfile will be created as
{\tt $u$.bin.{\it arch}-{\it os}/$a$/$\cdots$/CM/{\it arch}-{\it os}/{\it file}.sml}.
Similarly, the stablefile for the library description
{\tt \$$a$/$\cdots$/{\it file}.cm} controlled by root anchor {\tt \$$a$}
gets written to
{\tt $u$.boot.{\it arch}-{\it os}/$a$/$\cdots$/CM/{\it arch}-{\it os}/{\it file}.cm}.
(Notice how the anchor name {\tt \$$a$} is being incorporated into the
resulting pathnames.)
\item[{\bf boot}:]
At bootstrap time (e.g., when the {\tt makeml} script is invoked), CM
scans the contents of {\tt $u$.boot.{\it arch}-{\it os}}, and for
each directory name $a$ it finds there it maps the anchor {\tt \$$a$} to that
directory.  The result is that the location for library description
{\tt \$$a$/$\cdots$/{\it file}.cm}
is considered to be
{\tt $u$.bin.{\it arch}-{\it os}/$a$/$\cdots$/{\it file}.cm}, which
means that---under the {\em usual} rules of CM---
the corresponding stablefile is going to be
{\tt $u$.bin.{\it arch}-{\it os}/$a$/$\cdots$/CM/{\it arch}-{\it
os}/{\it file}.cm}.  This is precisely where {\tt CMB.make} created
it.  (Of course, the libary description file itself does not exist,
but this is not a problem because the stablefile does.)

The resulting mapping for anchors is the one that is being saved to the
newly generated heap image.
\item[{\bf test}:] After generating a new heap image {\tt $v$.{\it
arch}-{\it osname}}\footnote{As mentioned earlier, we distinguish
between {\it os} which describes the {\em kind} of operating system
and {\it osname} which is the name of a particular operating system.
For example, on a Linux machine {\it osname} is {\tt linux} while {\it
os} is still {\tt unix}.}, the {\tt makeml} script will copy the
hierarchy of directories under {\tt $u$.boot.{\it arch}-{\it os}} to a
new directory {\tt $v$.lib} and generate a path configuration file
{\tt $v$.lib/pathconfig} for it.  Files in this hierarchy will be
re-created as hard links.  The prefix $v$ can be chosen at the time
{\tt makeml} is invoked (see section~\ref{script:makeml})---just like
$u$ can be chosen at the time {\tt CMB.make} is invoked.  The default
for $v$ is {\tt sml}.  Copying the directory prevents any future {\tt
CMB.make} from clobbering the libraries that belong to the newly
generated heap image.

Running the {\tt testml} script will start the runtime system,
instruct it to load heap image {\tt $v$.{\it arch}-{\it osname}},
and arrange for the path configuration file {\tt $v$.lib/pathconfig}
to take effect.  This will cause any anchor $a$ to be resolved within
the {\tt $v$.lib} hierarchy.  Thus, the new system can be tested in a
way that is completely independent from any existing installation.
(Again, $v$ can be specified as a parameter to {\tt testml}.)
\item[{\bf install}:]
When testing has been satisfactory, the new system can be installed
permamently, replacing the old heap image and its old libraries with
their newly created counterparts.  To do so, one should run the {\tt
installml} script.  It will move the heap image
{\tt $v$.{\it arch}-{\it osname}} to
{\tt ../../bin/.heap/sml.{\it arch}-{\it osname}}
and all stablefiles from {\tt $v$.lib} to their usual place
under {\tt ../../lib}.  It then proceeds to edit
{\tt ../../lib/pathconfig} (the main path configuration file of the
installation) to reflect the new situation.
(Library files in {\tt ../../lib} that have nothing to do with the
bootstrap process will remain untouched.)
\end{description}
