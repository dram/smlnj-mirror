% -*- latex -*-
\documentclass[titlepage,letterpaper]{article}
\usepackage{latexsym}
\usepackage{times}
%\usepackage{hyperref}

\newcommand{\gentool}{{\tt ml-nlffigen}}

\marginparwidth0pt\oddsidemargin0pt\evensidemargin0pt\marginparsep0pt
\topmargin0pt\advance\topmargin by-\headheight\advance\topmargin by-\headsep
\textwidth6.7in\textheight9.1in
\columnsep0.25in

\newcommand{\smlmj}{110}
\newcommand{\smlmn}{46}

\author{Matthias Blume \\
Toyota Technological Institute at Chicago}

\title{{\bf NLFFI}\\
A new SML/NJ Foreign-Function Interface \\
{\it\small (for SML/NJ version \smlmj.\smlmn~and later)} \\
User Manual}

\setlength{\parindent}{0pt}
\setlength{\parskip}{6pt plus 3pt minus 2pt}

\newcommand{\nt}[1]{{\it #1}}
\newcommand{\tl}[1]{{\underline{\bf #1}}}
\newcommand{\ttl}[1]{{\underline{\tt #1}}}
\newcommand{\ar}{$\rightarrow$\ }
\newcommand{\vb}{~$|$~}

\begin{document}

\bibliographystyle{alpha}

\maketitle

\pagebreak

\tableofcontents

\pagebreak

%%%%%%%%%%%%%%%%%%%%%%%%%%%%%%%%%%%%%%%%%%%%%%%%%%%%%%%%%%%%%%%%%%%%%%%%%%
\section{Introduction}

Introduce...

%%%%%%%%%%%%%%%%%%%%%%%%%%%%%%%%%%%%%%%%%%%%%%%%%%%%%%%%%%%%%%%%%%%%%%%%%%
\section{The C Library}

The C library...

%%%%%%%%%%%%%%%%%%%%%%%%%%%%%%%%%%%%%%%%%%%%%%%%%%%%%%%%%%%%%%%%%%%%%%%%%%
\section{Translation conventions}

The {\gentool} tool generates one ML structure for each
exported C definition.  In particular, there is one structure per
external variable, function, {\tt typedef}, {\tt struct}, {\tt union},
and {\tt enum}.
Each generated ML structure contains the ML type and values necessary
to manipulate the corresponding C item.

%-------------------------------------------------------------------------
\subsection{External variables}

An external C variable $v$ of type $t_C$ is represented by an ML
structure {\tt G\_}$v$.  This structure always contains a type {\tt t}
encoding $t_C$ and a value {\tt obj'} providing (``light-weight'')
access to the memory location that $v$ stands for in C.  If $t_C$ is
{\em complete}, then {\tt G\_}$v$ will also contain a value {\tt obj}
(the ``heavy-weight'' equivalent of {\tt obj'}) as well as value {\tt
  typ} holding run-time type information corresponding to $t_C$ (and
{\tt t}).

\paragraph*{Details}

\begin{description}\setlength{\itemsep}{0pt}
\item[{\tt type t}] is the type to be substituted for $\tau$ in {\tt
    ($\tau$, $\zeta$) C.obj} to yield the correct type for ML values
  representing C memory objects of type $t_C$ (i.e., $v$'s type).
  (This assumes a properly instantiated $\zeta$ based on whether or
  not the corresponding object was declared {\tt const}.)
\item[!{\tt val typ}] is the run-time type information corresponding
  to type {\tt t}.  The ML type of {\tt typ} is {\tt t C.T.typ}.  This
  value is not present if $t_C$ is {\em incomplete}.
\item[!{\tt val obj}] is a function that returns the ML-side
  representative of the C object (i.e., the memory location) referred
  to by $v$.  Depending on whether or not $v$ was declared {\tt
    const}, the type of {\tt obj} is either {\tt unit -> (t, C.ro)
    C.obj} or {\tt unit -> (t, C.rw) C.obj}.  The result of {\tt
    obj()} is ``heavy-weight,'' i.e., it implicitly carries run-time
  type information.  This value is not present if $t_C$ is {\em
    incomplete}.
\item[{\tt val obj'}] is analogous to {\tt val obj}, the only
  difference being that its result is ``light-weight,'' i.e., without
  run-time type information.  The type of {\tt val obj'} is
  either {\tt unit -> (t, C.ro) C.obj} or {\tt unit -> (t, C.rw) C.obj}.
\end{description}

(Elements that are subject to omission due to incompleteness of types
are marked with an exclamation mark(!).)

\subsubsection*{Examples}

\begin{small}
\begin{center}
\begin{tabular}{c|c}
C declaration & signature of ML-side representation \\ \hline\hline
\begin{minipage}{2in}
\begin{verbatim}
extern int i;
\end{verbatim}
\end{minipage}
&
\begin{minipage}{4in}
\begin{verbatim}

structure G_i : sig
    type t   = C.sint
    val typ  : t C.T.typ
    val obj  : unit -> (t, C.rw) C.obj
    val obj' : unit -> (t, C.rw) C.obj'
end

\end{verbatim}
\end{minipage}
\\ \hline
\begin{minipage}{2in}
\begin{verbatim}
extern const double d;
\end{verbatim}
\end{minipage}
&
\begin{minipage}{4in}
\begin{verbatim}

structure G_d : sig
    type t   = C.double
    val typ  : t C.T.typ
    val obj  : unit -> (t, C.ro) C.obj
    val obj' : unit -> (t, C.ro) C.obj'
end

\end{verbatim}
\end{minipage}
\\ \hline
\begin{minipage}{2in}
\begin{verbatim}
extern struct str s1;
/* str complete */
\end{verbatim}
\end{minipage}
&
\begin{minipage}{4in}
\begin{verbatim}

structure G_s1 : sig
    type t   = (S_str.tag C.su, rw) C.obj C.ptr
    val typ  : t C.T.typ
    val obj  : unit -> (t, C.rw) C.obj
    val obj' : unit -> (t, C.rw) C.obj'
end

\end{verbatim}
\end{minipage}
\\ \hline
\begin{minipage}{2in}
\begin{verbatim}
extern struct istr s2;
/* istr incomplete */
\end{verbatim}
\end{minipage}
&
\begin{minipage}{4in}
\begin{verbatim}

structure G_s2 : sig
    type t   = (ST_istr.tag C.su, rw) C.obj C.ptr
    val obj' : unit -> (t, C.rw) C.obj'
end

\end{verbatim}
\end{minipage}
\end{tabular}
\end{center}
\end{small}

%-------------------------------------------------------------------------
\subsection{Functions}

An external C function $f$ is represented by an ML structure {\tt
  F\_}$f$.  Each such structure always contains at last three values:
{\tt typ}, {\tt fptr}, and {\tt f'}.  Variable {\tt typ} holds
run-time type information regarding function pointers that share $f$'s
prototype.  The most important part of this information is the code
that implements native C calling conventions for these functions.
Variable {\tt fptr} provides access to a C pointer to $f$.  And {\tt
  f'} is an ML function that dispatches a call of $f$ (through {\tt
  fptr}), using ``light-weight'' types for arguments and results.  If
the result type of $f$ is {\em complete}, then {\tt F\_}$f$ will also
contain a function {\tt f}, using ``heavy-weight'' argument- and
result-types.

\paragraph*{Details}

\begin{description}\setlength{\itemsep}{0pt}
\item[{\tt val typ}] holds run-time type information for pointers to
  functions of the same prototype.  The ML type of {\tt typ} is {\tt
    ($A$ -> $B$) C.fptr C.T.typ} where $A$ and $B$ are types encoding
  $f$'s argument list and result type, respectively.  A
  description of $A$ and $B$ is given below.
\item[{\tt val fptr}] is a function that returns the (heavy-weight)
  function pointer to $f$. The type of {\tt fptr} is {\tt unit -> ($A$
    -> $B$) C.fptr}.  The encodings of argument- and result types in
  $A$ and $B$ is the same as the one used for {\tt typ} (see below).
  Notice that although {\tt fptr} is a heavy-weight value carrying
  run-time type information, pointer arguments within $A$ or $B$ still
  use the light-weight version!
\item[!{\tt val f}] is an ML function that dispatches a call to $f$
  via {\tt fptr}.  For convenience, {\tt f} has built-in conversions
  for arguments (from ML to C) and the result (from C to ML).  For
  example, if $f$ has an argument of type {\tt double}, then {\tt f}
  will take an argument of type {\tt MLRep.Real.real} in its place and
  implicitly convert it to its C equivalent using {\tt
    C.Cvt.c\_double}.  Similarly, if $f$ returns an {\tt unsigned
    int}, then {\tt f} has a result type of {\tt MLRep.Unsigned.word}.
  This is done for all types that have a conversion function in
  {\tt C.Cvt}.
  Pointer values (as well as the object argument used for {\tt
    struct}- or {\tt union}-return values) are taken and returned in
  their heavy-weight versions.  Function {\tt f} will not be generated
  if the return type of $f$ is incomplete.
\item[{\tt val f'}] is the light-weight equivalent to {\tt f}.  a
  light-weight function.  The main difference is that pointer- and
  object-values are passed and returned in their light-weight
  versions.
\end{description}

\subsubsection*{Type encoding rules for {\tt ($A$ -> $B$) C.fptr}}

A C function $f$'s prototype is encoded as an ML type {\tt $A$ ->
  $B$}.  Calls of $f$ from ML take an argument of type $A$ and
produce a result of type $B$.

\begin{itemize}
\item Type $A$ is constructed from a sequence $\langle T_1, \ldots,
  T_k \rangle$ of types.  If that sequence is empty, then {\tt $A =$
    unit}; if the sequence has only one element $T_1$, then $A = T_1$.
  Otherwise $A$ is a tuple type {\tt $T_1$ * $\ldots$ * $T_k$}.
\item If $f$'s result is neither a {\tt struct} nor a {\tt union},
  then $T_1$ encodes the type of $f$'s first argument, $T_2$ that of
  the second, $T_3$ that of the third, and so on.
\item If $f$'s result is some {\tt struct} or some {\tt union}, then
  $T_1$ will be {\tt ($\tau$ C.su, C.rw) C.obj'} with $\tau$
  instantiated to the appropriate {\tt struct}- or {\tt union}-tag
  type.  Moreover, we then also have $B = T_1$. $T_2$ encodes the type
  of $f$'s {\em first} argument, $T_3$ that of the second.  (In
  general, $T_{i+1}$ will encode the type of the $i$th argument of
  $f$ in this case.)
\item The encoding of the $i$th argument of $f$ ($T_i$ or $T_{i+1}$
  depending on $f$'s return type) is the light-weight ML equivalent of
  the C type of that argument.
\item An argument of C {\tt struct}- or {\tt union}-type corresponds
  to {\tt ($\tau$ C.su, C.ro) C.obj'} with $\tau$ instantiated to the
  appropriate tag type.
\item If $f$'s result type is {\tt void}, then {\tt $B =$ unit}.  If
  the result type is not a {\tt struct}- or {\tt union}-type, then $B$
  is the light-weight ML encoding of that type.  Otherwise $B = T_1$
  (see above).
\end{itemize}

\subsubsection*{Examples}

\begin{small}
\begin{center}
\begin{tabular}{c|c}
C declaration & signature of ML-side representation \\ \hline\hline
{\tt void f1 (void);}
&
\begin{minipage}{4in}
\begin{verbatim}

structure F_f1 : sig
    val typ  : (unit -> unit) C.fptr C.T.typ
    val fptr : unit -> (unit -> unit) C.fptr
    val f    : unit -> unit
    val f'   : unit -> unit
end

\end{verbatim}
\end{minipage}
\\ \hline
{\tt int f2 (void);}
&
\begin{minipage}{4in}
\begin{verbatim}

structure F_f2 : sig
    val typ  : (C.sint -> unit) C.fptr C.T.typ
    val fptr : unit -> (C.sint -> unit) C.fptr
    val f    : MLRep.Signed.int -> unit
    val f'   : MLRep.Signed.int -> unit
end

\end{verbatim}
\end{minipage}
\\ \hline
{\tt void f3 (int);}
&
\begin{minipage}{4in}
\begin{verbatim}

structure F_f3 : sig
    val typ  : (unit -> C.sint) C.fptr C.T.typ
    val fptr : unit -> (unit -> C.sint) C.fptr
    val f    : unit -> MLRep.Signed.int
    val f'   : unit -> MLRep.Signed.int
end

\end{verbatim}
\end{minipage}
\\ \hline
{\tt void f4 (double, struct s*);}
&
\begin{minipage}{4in}
\begin{verbatim}

structure F_f4 : sig
    val typ  : (C.double *
                (ST_s.tag C.su, C.rw) C.obj C.ptr'
                -> unit)
                    C.fptr C.T.typ
    val fptr : unit -> (C.double *
                        (ST_s.tag C.su, C.rw) C.obj C.ptr'
                        -> unit) C.fptr
    val f    : MLRep.Real.real *
               (ST_s.tag C.su, C.rw) C.obj C.ptr
               -> unit
    val f'   : MLRep.Real.real *
               (ST_s.tag C.su, C.rw) C.obj C.ptr'
               -> unit
end

\end{verbatim}
\end{minipage}
\end{tabular}
\end{center}
\end{small}

\begin{small}
\begin{center}
\begin{tabular}{c|c}
C declaration & signature of ML-side representation \\ \hline\hline
\begin{minipage}{2in}
\begin{verbatim}
struct s *f5 (float);
/* s incomplete */
\end{verbatim}
\end{minipage}
&
\begin{minipage}{4in}
\begin{verbatim}

structure F_f5 : sig
    val typ  : (C.float
                -> (ST_s.tag C.su, C.rw) C.obj C.ptr')
                    C.fptr C.T.typ
    val fptr : unit -> (C.float
                       -> (ST_s.tag C.su, C.rw) C.obj C.ptr')
                           C.fptr
    val f'   : MLRep.Real.real ->
               (ST_s.tag C.su, C.rw) C.obj C.ptr'
end

\end{verbatim}
\end{minipage}
\\ \hline
\begin{minipage}{2in}
\begin{verbatim}
struct t *f6 (float);
/* t complete */
\end{verbatim}
\end{minipage}
&
\begin{minipage}{4in}
\begin{verbatim}

structure F_f6 : sig
    val typ  : (C.float
                -> (S_t.tag C.su, C.rw) C.obj C.ptr')
                    C.fptr C.T.typ
    val fptr : unit -> (C.float
                       -> (S_t.tag C.su, C.rw) C.obj C.ptr')
                           C.fptr
    val f    : MLRep.Real.real ->
               (S_t.tag C.su, C.rw) C.obj C.ptr
    val f'   : MLRep.Real.real ->
               (S_t.tag C.su, C.rw) C.obj C.ptr'
end

\end{verbatim}
\end{minipage}
\\ \hline
\begin{minipage}{2in}
\begin{verbatim}
struct t f7 (int, double);
/* t complete */
\end{verbatim}
\end{minipage}
&
\begin{minipage}{4in}
\begin{verbatim}

structure F_f7 : sig
    val typ  : ((S_t.tag C.su, C.rw) C.obj' *
                C.sint * C.double
                -> (S_t.tag C.su, C.rw) C.obj')
                    C.fptr C.T.typ
    val fptr : unit -> ((S_t.tag C.su, C.rw) C.obj' *
                        C.sint * C.double
                        -> (S_t.tag C.su, C.rw) C.obj')
                            C.fptr
    val f    : (S_t.tag C.su, C.rw) C.obj *
               MLRep.Signed.int *
               MLRep.Real.real
               -> (S_t.tag C.su, C.rw) C.obj
    val f'   : (S_t.tag C.su, C.rw) C.obj' *
               MLRep.Signed.int *
               MLRep.Real.real
               -> (S_t.tag C.su, C.rw) C.obj'
end

\end{verbatim}
\end{minipage}
\end{tabular}
\end{center}
\end{small}

\subsection{Type definitions ({\tt typedef})}

In C a {\tt typedef} declaration associates a type name $t$ with a
type $t_C$.  On the ML side, $t$ is represented by an ML structure
{\tt T\_$t$}.  This structure contains a type abbreviation {\tt t} for
the ML encoding of $t_C$ and, provided $t_C$ is not {\em incomplete},
a value {\tt typ} of type {\tt t C.T.typ} with run-time type
information regarding $t_C$.

\subsubsection*{Examples}

\begin{small}
\begin{center}
\begin{tabular}{c|c}
C declaration & signature of ML-side representation \\ \hline\hline
\begin{minipage}{2in}
\begin{verbatim}
typedef int t1;
\end{verbatim}
\end{minipage}
&
\begin{minipage}{4in}
\begin{verbatim}

structure T_t1 : sig
    type t   = C.sint
    val typ  : t C.T.typ
end

\end{verbatim}
\end{minipage}
\\ \hline
\begin{minipage}{2in}
\begin{verbatim}
typedef struct s t2;
/* s incomplete */
\end{verbatim}
\end{minipage}
&
\begin{minipage}{4in}
\begin{verbatim}

structure T_t2 : sig
    type t  = ST_s.tag C.su
end

\end{verbatim}
\end{minipage}
\\ \hline
\begin{minipage}{2in}
\begin{verbatim}
typedef struct s *t3;
/* s incomplete */
\end{verbatim}
\end{minipage}
&
\begin{minipage}{4in}
\begin{verbatim}

structure T_t3 : sig
    type t  = (ST_s.tag C.su, C.rw) C.obj C.ptr
end

\end{verbatim}
\end{minipage}
\\ \hline
\begin{minipage}{2in}
\begin{verbatim}
typedef struct t t4;
/* t complete */
\end{verbatim}
\end{minipage}
&
\begin{minipage}{4in}
\begin{verbatim}

structure T_t4 : sig
    type t  = ST_t.tag C.su
    val typ : t T.typ
end

\end{verbatim}
\end{minipage}
\end{tabular}
\end{center}
\end{small}

\subsection{{\tt struct} and {\tt union}}
 
The type identity of a named C {\tt struct} (or {\tt union}) is
provided by a unique ML {\em tag} type.  There is a 1-1 correspondence
between C tag names $t$ for {\tt struct}s on one side and ML tag types
$s_t$ on the other.  An analogous correspondence exists between C tag
names $t$ for {\tt union}s and ML tag types $u_t$.  Notice that these
correspondences are {\em independent of the actual declaration} of the
C {\tt struct} or {\tt union} in question.

A C type of the form {\tt struct $t$} is represented in ML as {\tt
  $s_t$ C.su}, a type of the form {\tt union $t$} as {\tt $u_t$ C.su}.
For example, this means that a heavy-weight non-constant memory object
of C type {\tt struct $t$} has ML type {\tt ($s_t$ C.su, C.rw) C.obj}
which can be abbreviated to {\tt ($s_t$ C.su, C.rw) C.obj}.

All ML types {\tt ($\tau$ C.su, $\zeta$) C.obj} are originally
completely abstract: they does not come with any operations that could
be applied to their values.  In C, the operations to be applied to a
{\tt struct}- or {\tt union}-value is field selection.  Field
selection {\em does} depend on the actual C declaration, so it is
{\gentool}'s job to generate a set of ML-side field-accessors that
correspond to field-access operations in C.

Each field is represented by a function mapping a memory object of the
{\tt struct}- or {\tt union}-type to an object of the respective field
type.  Let {\tt int i;} and {\tt const double d;} be fields of some
{\tt struct t} and let {\tt tag} be the ML tag type corresponding to
{\tt t}.  Here are the types of the (heavy-weight) access functions
for {\tt i} and {\tt d}:

\begin{small}
\begin{center}
\begin{tabular}{l@{~~~~$\leadsto$~~~~}l}
{\tt int i;} &
  {\tt val f\_i : (tag C.su, 'c) C.obj -> (C.sint, 'c) C.obj} \\
{\tt const double d;} &
  {\tt val f\_d : (tag C.su, 'c) C.obj -> (C.double, C.ro) C.obj}
\end{tabular}
\end{center}
\end{small}

\noindent Notice how each field access function is polymorphic in the
{\tt const} property of the argument object.  For fields declared {\tt
  const}, the result always uses {\tt C.ro} while for ordinary fields
the argument's type is used---reflecting the idea that a field is
considered writable if it has not been declared {\tt const} and, at
the same time, the enclosing {\tt struct} or {\tt union} is writable.

\subsubsection*{Incomplete declarations}

If the {\tt struct} or {\tt union} is incomplete (i.e., if only its
tag $t$ is known), then {\gentool} will merely generate an ML structure
(called {\tt ST\_$t$} for {\tt struct} and {\tt UT\_$t$} for {\tt
  union}) with a single type {\tt tag} that is an abbreviation for the
library-defined type that corresponds to tag $t$.

\subsubsection*{Complete declarations}

If the {\tt struct} or {\tt union} with tag $t$ is complete, then
{\gentool} will generate an ML structure (called {\tt S\_$t$} for {\tt
  struct} and {\tt U\_$t$} for {\tt union}) which contains at least:
\begin{description}\setlength{\itemsep}{0pt}
\item[{\tt type tag}] --- an abbreviation for the library-defined type
  that corresponds to $t$
\item[{\tt val size}] --- a value representing information about the
  size of memory objects of this {\tt struct}- or {\tt union}-type.
  The ML type of {\tt size} is {\tt tag C.su C.S.size}.
\item[{\tt val typ}] --- a value representing run-time type
  information corresponding to this {\tt struct}- or {\tt union}-type.
  The ML type of {\tt typ} is {\tt tag C.su C.T.typ}.
\end{description}

\subsubsection*{Fields}

In addition to {\tt type tag}, {\tt val size}, and {\tt val typ}, the
{\gentool} tool will generate a small set of structure elements for
each field $f$ of the {\tt struct} or {\tt union}.  Let $t_f$ be the
type of $f$:

\begin{description}\setlength{\itemsep}{0pt}
\item[{\tt type t\_f\_$f$}] is an abbreviation for the ML encoding of $t_f$.
\item[!{\tt val typ\_f\_$f$}] holds runtime type information regarding
  $t_f$.  If $t_f$ is incomplete, then {\tt typ\_f\_$f$} is omitted.
\item[!{\tt val f\_$f$}] is the heavy-weight access function for $f$.
  It maps a value of type {\tt (tag C.su, $\zeta$) C.obj} to a value
  of type {\tt (t\_f\_$f$, ${\zeta}_f$) C.obj} and is polymorphic in
  $\zeta$.  If $f$ was declared {\tt const}, then {\tt ${\zeta}_f =$
    C.ro}.  Otherwise ${\zeta}_f = \zeta$.  If $t_f$ is incomplete,
  then {\tt f\_$f$} is omitted.
\item[{\tt val f\_$f$'}] is the light-weight access function for $f$.
  It maps a value of type {\tt (tag C.su, $\zeta$) C.obj'} to a value
  of type {\tt (t\_f\_$f$, ${\zeta}_f$) C.obj'} and is polymorphic in
  $\zeta$.  If $f$ was declared {\tt const}, then {\tt ${\zeta}_f =$
    C.ro}.  Otherwise ${\zeta}_f = \zeta$.
\end{description}

\subsubsection*{Bitfields}

If $f$ is a bitfield, then two access functions are generated:

\begin{description}\setlength{\itemsep}{0pt}
\item[{\tt val f\_$f$}] is the heavy-weight access function, mapping
  values of type {\tt (tag C.su, $\zeta$) C.obj} to either {\tt
    ${\zeta}_f$ C.sbf} or {\tt ${\zeta}_f$ C.ubf}, depending on
  whether the type of $f$ is {\tt signed} or {\tt unsigned}.  The
  function is polymorphic in $\zeta$.  If $f$ was declared {\tt
    const}, then {\tt ${\zeta}_f =$ C.ro}.  Otherwise, ${\zeta}_f =
  \zeta$.
\item[{\tt val f\_$f$'}] is the light-weight access function, mapping
  values of type {\tt (tag C.su, $\zeta$) C.obj'} to either {\tt
    ${\zeta}_f$ C.sbf} or {\tt ${\zeta}_f$ C.ubf}, using the same
  conventions as those used for {\tt f\_$f$}.
\end{description}

\subsubsection*{Example}

\begin{small}
\begin{center}
\begin{tabular}{c|c}
C declaration & signature of ML-side representation \\ \hline\hline
\begin{minipage}{2in}
\begin{verbatim}
struct t {
  int i;
  const double d;
  struct t *nx;
    /* complete */
  struct s *ms;
    /* incomplete */
  const int f : 2;
  unsigned g : 3;
};
\end{verbatim}
\end{minipage}
&
\begin{minipage}{4in}
\begin{verbatim}

structure S_t : sig
  type tag = ...
  val size : tag C.su C.S.size
  val typ : tag C.su C.T.typ

  type t_f_i = C.T.sint
  val typ_f_i : t_f_i C.T.typ
  val f_i  : (tag C.su, 'c) obj  -> (t_f_i, 'c) C.obj
  val f_i' : (tag C.su, 'c) obj' -> (t_f_i, 'c) C.obj'

  type t_f_d = C.T.double
  val typ_f_d : t_f_d C.T.typ
  val f_d  : (tag C.su, 'c) obj  -> (t_f_d, C.ro) C.obj
  val f_d' : (tag C.su, 'c) obj' -> (t_f_d, C.ro) C.obj'

  type t_f_nx = (tag C.su, C.rw) C.obj C.ptr
  val typ_f_nx : t_f_nx C.T.typ
  val f_nx  : (tag C.su, 'c) obj  -> (t_f_nx, 'c) C.obj
  val f_nx' : (tag C.su, 'c) obj' -> (t_f_nx, 'c) C.obj'

  type t_f_ms = (ST_s.tag C.su, C.rw) C.obj C.ptr
  val f_ms' : (tag C.su, 'c) obj' -> (t_f_ms, 'c) C.obj'

  val f_f  : (tag C.su, 'c) C.obj  -> C.ro C.sbf
  val f_f' : (tag C.su, 'c) C.obj' -> C.ro C.sbf

  val f_g  : (tag C.su, 'c) C.obj  -> 'c C.ubf
  val f_g' : (tag C.su, 'c) C.obj' -> 'c C.ubf
end

\end{verbatim}
\end{minipage}
\end{tabular}
\end{center}
\end{small}

\subsubsection*{Unnamed {\tt struct}s or {\tt union}s}

Each occurrence of an unnamed {\tt struct} or {\tt union} in C has its
own type identity.  The {\gentool} tool models this by artificially
generating a unique tag for each such occurrence.  The tags are chosen
in such a way that they cannot clash with real tag names that might
occur elsewhere in the C code.  After choosing a fresh tag $t$,
{\gentool} produces ML code according to the same rules that it uses
when $t$ is a real tag explicitly present in the C code.

Here are the rules for generating tags:

\begin{itemize}\setlength{\itemsep}{0pt}
\item If the {\tt struct}- or {\tt union}-declaration occurs at top
  level, i.e., not within the context of a {\tt typedef} or another
  {\tt struct}- or {\tt union}-declaration, the generated tag consists
  of a sequence of decimal digits and can be read as a non-negative
  number.
\item If the immediate context of the unnamed {\tt struct} or {\tt
    union} is a {\tt typedef} for a type name $t$, then the generated
  tag will be {\tt '$t$}.
\item The tag of an unnamed {\tt struct} or {\tt union} is another
  (named or unnamed) {\tt struct} or {\tt union} with (real or
  generated) tag $t$ is chosen to be {\tt $t$'$n$} where $n$ is a
  fresh sequence of decimal digits that can be read as a non-negative
  number.
\end{itemize}

\subsubsection*{Examples}

\begin{small}
\begin{center}
\begin{tabular}{c|c}
C declaration & signature of ML-side representation \\ \hline\hline
\begin{minipage}{2in}
\begin{verbatim}
struct {
  int i;
};
\end{verbatim}
\end{minipage}
&
\begin{minipage}{4in}
\begin{verbatim}

structure S_0 : sig
  type tag = ...
  val size : tag C.su C.S.size
  val typ : tag C.su C.T.typ

  type t_f_i = C.T.sint
  val typ_f_i : t_f_i C.T.typ
  val f_i  : (tag C.su, 'c) obj  -> (t_f_i, 'c) C.obj
  val f_i' : (tag C.su, 'c) obj' -> (t_f_i, 'c) C.obj'
end

\end{verbatim}
\end{minipage}  
\\ \hline
\begin{minipage}{2in}
\begin{verbatim}
typedef struct {
  int j;
} s;
\end{verbatim}
\end{minipage}
&
\begin{minipage}{4in}
\begin{verbatim}

structure S_'s : sig
  type tag = ...
  val size : tag C.su C.S.size
  val typ : tag C.su C.T.typ

  type t_f_j = C.T.sint
  val typ_f_j : t_f_j C.T.typ
  val f_j  : (tag C.su, 'c) obj  -> (t_f_j, 'c) C.obj
  val f_j' : (tag C.su, 'c) obj' -> (t_f_j, 'c) C.obj'
end

\end{verbatim}
\end{minipage}  
\\ \hline
\begin{minipage}{2in}
\begin{verbatim}
struct s {
  struct {
    int j;
  } x;
};
\end{verbatim}
\end{minipage}
&
\begin{minipage}{4in}
\begin{verbatim}

structure S_s'0 : sig
    type tag = ...
    val size : tag C.su C.S.size
    val typ : tag C.su C.T.typ

    type t_f_j = C.sint
    val typ_f_j : t_f_j C.T.typ
    val f_j  : (tag C.su, 'c) C.obj  -> (t_f_j, 'c) C.obj
    val f_j' : (tag C.su, 'c) C.obj' -> (t_f_j, 'c) C.obj'
end

structure S_s : sig
    type tag = ...
    val size : tag C.su C.S.size
    val typ : tag C.su C.T.typ

    type t_f_x = S_s'0.tag C.su
    val typ_f_x : t_f_x C.T.typ
    val f_x  : (tag C.su, 'c) C.obj  -> (t_f_x, 'c) C.obj
    val f_x' : (tag C.su, 'c) C.obj' -> (t_f_x, 'c) C.obj'
end

\end{verbatim}
\end{minipage}  
\end{tabular}
\end{center}
\end{small}

\subsection{Enumerations ({\tt enum})}

A C enumeration of constants $c_1, c_2, \ldots, c_k$ declared via {\tt
  enum} is represented by $k$ ML values of a chosen ML representation
type.  By default, that type is {\tt MLRep.Signed.int}, i.e., the same
type that also represents the C type {\tt int}.  A command line switch
({\tt -enum-constructors} or {\tt -ec}) to {\gentool} can change this
behavior in such a way that whenever possible the representation type
for an enumeration becomes an ML datatype, thus making it possible to
perform pattern-matching on constants.  The representation type cannot be a
datatype if two or more {\tt enum} constants share the same value as in:

\begin{verbatim}
  enum ab { A = 12, B = 12 };
\end{verbatim}

\subsubsection*{Complete enumerations}

Let $t$ be the tag of the C {\tt enum} declaration, and let
$c_1,\ldots,c_k$ be its set of constants.  The ML-side representative
of such a declaration is a structure {\tt E\_$t$} which contains $10+k$
elements, the first 10 being:

\begin{description}\setlength{\itemsep}{0pt}
\item[{\tt type tag}] The ML-side encoding of type {\tt enum $t$} is
  {\tt tag C.enum}.  Values of this type are abstract.  They can be
  converted to and from concrete integer values of type {\tt
    MLRep.Signed.int} using {\tt C.Cvt.c2i\_enum} and {\tt
    C.Cvt.i2c\_enum}, respectively.  Like in the case of {\tt struct}
  or {\tt union}, type {\tt tag} is an abbreviation for the
  pre-defined type that uniquely corresponds to the tag name $t$.
\item[{\tt type mlrep}] This is the type of concrete ML-side values
  representing the $c_1,\ldots,c_k$.  This type is not the same as
  {\tt tag C.enum} and defaults to {\tt MLRep.Signed.int}.  As
  mentioned above, by specifying the {\tt -enum-constructors} or {\tt
    -ec} command-line flag one can force {\gentool} to generate a
  datatype definition for type {\tt mlrep}.
\item[{\tt val m2i}] This is a function for converting {\tt mlrep}
  values to values of type {\tt MLRep.Signed.int}.  If the former is
  the same type as the latter (see above), then {\tt m2i} is the
  identity function.  Otherwise {\gentool} generates explicit code to
  map each {\tt mlrep} constructor to an integer value.
\item[{\tt val i2m}] This is the inverse of {\tt m2i}.  If {\tt mlrep}
  is a datatype, then {\tt m2i} will raise exception {\tt Domain} when
  the argument does not correspond to one of the constructors.
\item[{\tt val c}] Function {\tt c} converts values of type {\tt
    mlrep} to values of type {\tt tag C.enum}.  It is merely a
  composition of {\tt C.Cvt.i2c\_enum} and {\tt m2i}.
\item[{\tt val ml}] Function {\tt ml} is the composition of {\tt i2m}
  and {\tt C.Cvt.c2i\_enum} and converts values of type {\tt tag
    C.enum} to values of type {\tt mlrep}.  It can raise exception
  {\tt Domain} if the C type system had been subverted (which is
  always a real possibility).
\item[{\tt val get}] Function {\tt get} fetches a value of type {\tt
    mlrep} from a memory object of type {\tt (tag C.enum, $\zeta$)
    C.obj}.  It is a composition of {\tt i2m} and {\tt C.Get.enum}.
\item[{\tt val get'}] Function {\tt get}' fetches a value of type {\tt
    mlrep} from a memory object of type {\tt (tag C.enum, $\zeta$)
    C.obj'}.  It is a composition of {\tt i2m} and {\tt C.Get.enum'}.
\item[{\tt val set}] Function {\tt set} stores a value of type {\tt
    mlrep} into a memory object of type {\tt (tag C.enum, C.rw)
    C.obj}.  It is a composition of {\tt m2i} and {\tt C.Set.enum}.
\item[{\tt val set'}] Function {\tt set'} stores a value of type {\tt
    mlrep} into a memory object of type {\tt (tag C.enum, C.rw)
    C.obj'}.  It is a composition of {\tt m2i} and {\tt C.Set.enum'}.
\end{description}

Each of the remaining $k$ elements corresponds to one of the
enumeration constants $c_i$.  Concretely, the element generated for
$c_i$ is {\tt val e\_$c_i$} and has type {\tt mlrep}.  If {\tt mlrep}
is a datatype, then the {\tt e\_$c_i$} are constructors which can be
used in ML patterns.

\subsubsection*{Examples}

\begin{small}
\begin{center}
\begin{tabular}{c|c}
C declaration & signature of ML-side representation \\ \hline\hline
\begin{minipage}{2in}
\begin{verbatim}
enum e { A, B, C };
/* default treatment */
\end{verbatim}
\end{minipage}
&
\begin{minipage}{4in}
\begin{verbatim}

structure E_e : sig
    type tag = ...
    type mlrep = MLRep.Signed.int
    val e_A  : mlrep  (* = 0 *)
    val e_B  : mlrep  (* = 1 *)
    val e_C  : mlrep  (* = 2 *)
    val m2i  : mlrep -> MLRep.Signed.int
    val i2m  : MLRep.Signed.int -> mlrep
    val c    : mlrep -> tag C.enum
    val ml   : tag C.enum -> mlrep
    val get  : (tag C.enum, 'c) C.obj  -> mlrep
    val get' : (tag C.enum, 'c) C.obj' -> mlrep
    val set  : (tag C.enum, C.rw) C.obj  * mlrep -> unit
    val set' : (tag C.enum, C.rw) C.obj' * mlrep -> unit
end

\end{verbatim}
\end{minipage}
\\ \hline
\begin{minipage}{2in}
\begin{verbatim}
enum e { A, B, C };
/* -enum-constructors */
\end{verbatim}
\end{minipage}
&
\begin{minipage}{4in}
\begin{verbatim}

structure E_e : sig
    type tag = ...
    datatype mlrep = e_A | e_B | e_C
    val m2i  : mlrep -> MLRep.Signed.int
    val i2m  : MLRep.Signed.int -> mlrep
    val c    : mlrep -> tag C.enum
    val ml   : tag C.enum -> mlrep
    val get  : (tag C.enum, 'c) C.obj  -> mlrep
    val get' : (tag C.enum, 'c) C.obj' -> mlrep
    val set  : (tag C.enum, C.rw) C.obj  * mlrep -> unit
    val set' : (tag C.enum, C.rw) C.obj' * mlrep -> unit
end

\end{verbatim}
\end{minipage}
\\ \hline
\begin{minipage}{2in}
\begin{verbatim}
enum e { A = 0, B = 1,
         C = 0 };
/* with or without
 *  -enum-constructors */
\end{verbatim}
\end{minipage}
&
\begin{minipage}{4in}
\begin{verbatim}

structure E_e : sig
    type tag = ...
    type mlrep = MLRep.Signed.int
    val e_A  : mlrep  (* = 0 *)
    val e_B  : mlrep  (* = 1 *)
    val e_C  : mlrep  (* = 0 *)
    val m2i  : mlrep -> MLRep.Signed.int
    val i2m  : MLRep.Signed.int -> mlrep
    val c    : mlrep -> tag C.enum
    val ml   : tag C.enum -> mlrep
    val get  : (tag C.enum, 'c) C.obj  -> mlrep
    val get' : (tag C.enum, 'c) C.obj' -> mlrep
    val set  : (tag C.enum, C.rw) C.obj  * mlrep -> unit
    val set' : (tag C.enum, C.rw) C.obj' * mlrep -> unit
end

\end{verbatim}
\end{minipage}
\end{tabular}
\end{center}
\end{small}


\subsubsection*{Incomplete enumerations}

If the enumeration is incomplete, i.e., if only its tag $t$ is known,
then no structure {\tt E\_$t$} is generated.  Instead, a structure
{\tt ET\_$t$} takes its place which merely contains the type {\tt tag}
as described above.

\subsubsection*{Unnamed enumerations}

Anonymous enumerations ({\tt enum}s without a tag) are handled in a
way that is very similar to the treatment of unnamed {\tt struct}s and
{\tt union}s.  In particular, the rules for assigning a generated tag
are the same if the {\tt enum} occurs in the context of a {\tt
  typedef} or another {\tt struct} or {\tt union}.

However, by default all constants in unnamed top-level {\tt enum}s get
collected into one single virtual enumeration whose tag is {\tt '}
(apostrophe).  If this is not desired, then the command line flag {\tt
  -nocollect} turns this off and lets {\gentool} fall back to the
exact same rules that are used for unnamed top-level {\tt struct}s and
{\tt union}s: a fresh ``numeric'' tag gets generated for each such
{\tt enum}.

\subsubsection*{Examples for collected unnamed enumerations}

\begin{small}
\begin{center}
\begin{tabular}{c|c}
C declaration & signature of ML-side representation \\ \hline\hline
\begin{minipage}{2in}
\begin{verbatim}
enum { A, B };
enum { C, D };
/* with or without
 *  -enum-constructors */
\end{verbatim}
\end{minipage}
&
\begin{minipage}{4in}
\begin{verbatim}

structure E_' : sig
    type tag = ...
    type mlrep = MLRep.Signed.int
    val e_A  : mlrep  (* = 0 *)
    val e_B  : mlrep  (* = 1 *)
    val e_C  : mlrep  (* = 0 *)
    val e_D  : mlrep  (* = 1 *)
    ...
end

\end{verbatim}
\end{minipage}
\\ \hline
\begin{minipage}{2in}
\begin{verbatim}
enum { A, B };
enum { C = 2, D };
/* -enum-constructors */
\end{verbatim}
\end{minipage}
&
\begin{minipage}{4in}
\begin{verbatim}

structure E_' : sig
    type tag = ...
    datatype mlrep = e_A | e_B | e_C | e_D
    ...
end

\end{verbatim}
\end{minipage}
\end{tabular}
\end{center}
\end{small}

%%%%%%%%%%%%%%%%%%%%%%%%%%%%%%%%%%%%%%%%%%%%%%%%%%%%%%%%%%%%%%%%%%%%%%%%%%
%\appendix
%% -*- latex -*-

\section{CM description file syntax}

\subsection{Lexical Analysis}

The CM parser employs a context-sensitive scanner.  In many cases this
avoids the need for ``escape characters'' or other lexical devices
that would make writing description files cumbersome.  (The downside
of this is that it increases the complexity of both documentation and
implementation.)

The scanner skips all nestable SML-style comments (enclosed with {\bf
(*} and {\bf *)}).

Lines starting with {\bf \#line} may list up to three fields separated
by white space.  The first field is taken as a line number and the
last field (if more than one field is present) as a file name.  The
optional third (middle) field specifies a column number.  A line of
this form resets the scanner's idea about the name of the file that it
is currently processing and about the current position within that
file.  If no file is specified, the default is the current file.  If
no column is specified, the default is the first column of the
(specified) line.  This feature is meant for program-generators or
tools such as {\tt noweb} but is not intended for direct use by
programmers.

The following lexical classes are recognized:

\begin{description}
\item[Namespace specifiers:] {\bf structure}, {\bf signature},
{\bf functor}, or {\bf funsig}.  These keywords are recognized
everywhere.
\item[CM keywords:] {\bf group}, {\bf Group}, {\bf GROUP}, {\bf
library}, {\bf Library}, {\bf LIBRARY}, {\bf is}, {\bf IS}.  These
keywords are recognized everywhere except within ``preprocessor''
lines (lines starting with {\bf \#}) or following one of the namespace
specifiers.
\item[Preprocessor control keywords:] {\bf \#if}, {\bf \#elif}, {\bf
\#else}, {\bf \#endif}, {\bf \#error}.  These keywords are recognized
only at the beginning of the line and indicate the start of a
``preprocessor'' line.  The initial {\bf \#} character may be
separated from the rest of the token by white space (but not by comments).
\item[Preprocessor operator keywords:] {\bf defined}, {\bf div}, {\bf
mod}, {\bf andalso}, {\bf orelse}, {\bf not}.  These keywords are
recognized only when they occur within ``preprocessor'' lines.  Even
within such lines, they are not recognized as keywords when they
directly follow a namespace specifier---in which case they are
considered SML identifiers.
\item[SML identifiers (\nt{mlid}):] Recognized SML identifiers
include all legal identifiers as defined by the SML language
definition. (CM also recognizes some tokens as SML identifiers that
are really keywords according to the SML language definition. However,
this can never cause problems in practice.)  SML identifiers are
recognized only when they directly follow one of the namespace
specifiers.
\item[CM identifiers (\nt{cmid}):] CM identifiers have the same form
as those ML identifiers that are made up solely of letters, decimal
digits, apostrophes, and underscores.  CM identifiers are recognized when they
occur within ``preprocessor'' lines, but not when they directly follow
some namespace specifier.
\item[Numbers (\nt{number}):] Numbers are non-empty sequences of
decimal digits.  Numbers are recognized only within ``preprocessor''
lines.
\item[Preprocessor operators:] The following unary and binary operators are
recognized when they occur within ``preprocessor'' lines: {\tt +},
{\tt -}, {\tt *}, {\tt /}, {\tt \%}, {\tt <>}, {\tt !=}, {\tt <=},
{\tt <}, {\tt >=}, {\tt >}, {\tt ==}, {\tt =}, $\tilde{~}$, {\tt
\&\&}, {\tt ||}, {\tt !}.  Of these, the following (``C-style'')
operators are considered obsolete and trigger a warning
message\footnote{The use of {\tt -} as a unary minus also triggers
this warning.} as long as {\tt CM.Control.warn\_obsolete} is set to
{\tt true}: {\tt /}, {\tt \%}, {\tt !=}, {\tt ==}, {\tt \&\&}, {\tt
||}, {\tt !}.
\item[Standard path names (\nt{stdpn}):] Any non-empty sequence of
upper- and lower-case letters, decimal digits, and characters drawn
from {\tt '\_.;,!\%\&\$+/<=>?@$\tilde{~}$|\#*-\verb|^|} that occurs
outside of ``preprocessor'' lines and is neither a namespace specifier
nor a CM keyword will be recognized as a stardard path name.  Strings
that lexically constitute standard path names are usually---but not
always---interpreted as file names. Sometimes they are simply taken as
literal strings.  When they act as file names, they will be
interpreted according to CM's {\em standard syntax} (see
Section~\ref{sec:basicrules}).  (Member class names, names of
privileges, and many tool optios are also specified as standard path
names even though in these cases no actual file is being named.)
\item[Native path names (\nt{ntvpn}):] A token that has the form of an
SML string is considered a native path name.  The same rules as in SML
regarding escape characters apply.  Like their ``standard''
counterparts, native path names are not always used to actually name
files, but when they are, they use the native file name syntax of the
underlying operating system.
\item[Punctuation:] A colon {\bf :} is recognized as a token
everywhere except within ``preprocessor'' lines. Parentheses {\bf ()}
are recognized everywhere.
\end{description}

\subsection{EBNF for preprocessor expressions}

\noindent{\em Lexical conventions:}\/ Syntax definitions use {\em
Extended Backus-Naur Form} (EBNF).  This means that vertical bars
\vb separate two or more alternatives, curly braces \{\} indicate
zero or more copies of what they enclose (``Kleene-closure''), and
square brackets $[]$ specify zero or one instances of their enclosed
contents.  Round parentheses () are used for grouping.  Non-terminal
symbols appear in \nt{this}\/ typeface; terminal symbols are
\tl{underlined}.

\noindent The following set of rules defines the syntax for CM's
preprocessor expressions (\nt{ppexp}):

\begin{tabular}{rcl}
\nt{aatom}  &\ar& \nt{number} \vb \nt{cmid} \vb \tl{(} \nt{asum} \tl{)} \vb (\ttl{$\tilde{~}$} \vb \ttl{-}) \nt{aatom} \\
\nt{aprod}  &\ar& \{\nt{aatom} (\ttl{*} \vb \tl{div} \vb \tl{mod}) \vb \ttl{/} \vb \ttl{\%} \} \nt{aatom} \\
\nt{asum}   &\ar& \{\nt{aprod} (\ttl{+} \vb \ttl{-})\} \nt{aprod} \\
\\
\nt{ns}     &\ar& \tl{structure} \vb \tl{signature} \vb \tl{functor} \vb \tl{funsig} \\
\nt{mlsym}  &\ar& \nt{ns} \nt{mlid} \\
\nt{query}  &\ar& \tl{defined} \tl{(} \nt{cmid} \tl{)} \vb \tl{defined} \tl{(} \nt{mlsym} \tl{)} \\
\\
\nt{acmp}   &\ar& \nt{asum} (\ttl{<} \vb \ttl{<=} \vb \ttl{>} \vb \ttl{>=} \vb \ttl{=} \vb \ttl{==} \vb \ttl{<>} \vb \ttl{!=}) \nt{asum} \\
\\
\nt{batom}  &\ar& \nt{query} \vb \nt{acmp} \vb (\tl{not} \vb \ttl{!}) \nt{batom} \vb \tl{(} \nt{bdisj} \tl{)} \\
\nt{bcmp}   &\ar& \nt{batom} [(\ttl{=} \vb \ttl{==} \vb \ttl{<>} \vb \ttl{!=}) \nt{batom}] \\
\nt{bconj}  &\ar& \{\nt{bcmp} (\tl{andalso} \vb \ttl{\&\&})\} \nt{bcmp} \\
\nt{bdisj}  &\ar& \{\nt{bconj} (\tl{orelse} \vb \ttl{||})\} \nt{bconj} \\
\\
\nt{ppexp} &\ar& \nt{bdisj}
\end{tabular}

\subsection{EBNF for export lists}

The following set of rules defines the syntax for export lists (\nt{elst}):

\begin{tabular}{rcl}
\nt{guardedexports} &\ar& \{ \nt{export} \} (\tl{\#endif} \vb
\tl{\#else} \{ \nt{export} \} \tl{\#endif} \vb \tl{\#elif} \nt{ppexp}
\nt{guardedexports}) \\
\nt{restline}      &\ar& rest of current line up to next newline character \\
\nt{export}        &\ar& \nt{mlsym} \vb \tl{\#if} \nt{ppexp}
\nt{guardedexports} \vb \tl{\#error} \nt{restline}  \\
\nt{elst}       &\ar& \nt{export} \{ \nt{export} \} \\
\end{tabular}

\subsection{EBNF for tool options}

The following set of rules defines the syntax for tool options
(\nt{toolopts}):

\begin{tabular}{rcl}
\nt{pathname} &\ar& \nt{stdpn} \vb \nt{ntvpn} \\
\nt{toolopts} &\ar& \{ \nt{pathname} [\tl{:} (\tl{(} \nt{toolopts} \tl{)} \vb \nt{pathname})] \}
\end{tabular}

\subsection{EBNF for member lists}

The following set of rules defines the syntax for member lists (\nt{members}):

\begin{tabular}{rcl}
\nt{class}          &\ar& \nt{stdpn} \\
\nt{member}         &\ar& \nt{pathname} [\tl{:} \nt{class}] [\tl{(} \nt{toolopts} \tl{)}] \\
\nt{guardedmembers} &\ar& \nt{members} (\tl{\#endif} \vb \tl{\#else} \nt{members} \tl{\#endif} \vb \tl{\#elif} \nt{ppexp} \nt{guardedmembers}) \\
\nt{members}        &\ar& \{ (\nt{member} \vb \tl{\#if} \nt{ppexp}
\nt{guardedmembers} \vb \tl{\#error} \nt{restline}) \} 
\end{tabular}

\subsection{EBNF for library descriptions}

The following set of rules defines the syntax for library descriptions
(\nt{library}).  Notice that although the syntax used for \nt{version}
is the same as that for \nt{stdpn}, actual version strings will
undergo further analysis according to the rules given in
section~\ref{sec:versions}:

\begin{tabular}{rcl}
\nt{libkw}     &\ar& \tl{library} \vb \tl{Library} \vb \tl{LIBRARY} \\
\nt{version}   &\ar& \nt{stdpn} \\
\nt{privilege} &\ar& \nt{stdpn} \\
\nt{lprivspec} &\ar& \{ \nt{privilege} \vb \tl{(} \{ \nt{privilege} \} \tl{)} \} \\
\nt{library}  &\ar& [\nt{lprivspec}] \nt{libkw} [[\tl{(} \nt{version} \tl{)}] \nt{elst}] \nt{iskw} \nt{members}
\end{tabular}

\subsection{EBNF for library component descriptions (group descriptions)}

The main differences between group- and library-syntax can be
summarized as follows:

\begin{itemize}\setlength{\itemsep}{0pt}
\item Groups use keyword \tl{group} instead of \tl{library}.
\item Groups may have an empty export list.
\item Groups cannot wrap privileges, i.e., names of privileges (in
front of the \tl{group} keyword) never appear within parentheses.
\item Groups have no version.
\item Groups have an optional owner.
\end{itemize}

\noindent The following set of rules defines the syntax for library
component (group) descriptions (\nt{group}):

\begin{tabular}{rcl}
\nt{groupkw}   &\ar& \tl{group} \vb \tl{Group} \vb \tl{GROUP} \\
\nt{owner}     &\ar& \nt{pathname} \\
\nt{gprivspec} &\ar& \{ \nt{privilege} \} \\
\nt{group}     &\ar& [\nt{gprivspec}] \nt{groupkw} [\tl{(} \nt{owner} \tl{)}] [\nt{elst}] (\tl{is} \vb \tl{IS}) \nt{members}
\end{tabular}


\bibliography{blume,appel,ml}

\end{document}
