\documentclass{article} 
\usepackage{mltex}
\usepackage{wrapfig}
\usepackage{float}
\usepackage{alltt}
%\usepackage{floatfig}
\usepackage{fancyheadings}
%\usepackage{draftcopy}
%\usepackage{bookman}
\usepackage{utopia}
%\usepackage{times}
%\usepackage{ncntrsbk}
%\usepackage{palatino}

   \setlength{\textwidth}{6.5in}
   \setlength{\evensidemargin}{0in}
   \setlength{\oddsidemargin}{0in}
   \setlength{\textheight}{8in}
   \setlength{\topmargin}{-0.5in}

   \pagestyle{fancyplain}
   %\addtolength{\headwidth}{\marginparsep}
   %\addtolength{\headwidth}{\marginparwidth}

   \newcommand{\edge}[1]{\rightarrow_{#1}}
   \newcommand{\union}{\cup}
   \newcommand{\Union}{\bigcup}
   \newcommand{\overrides}{overrides}
   \newcommand{\defas}{\stackrel{\rm as}{=}}

   \renewcommand{\sectionmark}[1]{\markright{\thesection\ #1}}
   \renewcommand{\subsectionmark}[1]{\markright{\thesubsection\ #1}}
   \newcommand{\Term}[1]{\mbox{\it #1}}
   \lhead[\fancyplain{}{\bfseries\thepage}]%
         {\fancyplain{}{\bfseries\rightmark}}
   \rhead[\fancyplain{}{\bfseries\leftmark}]%
         {\fancyplain{}{\bfseries\thepage}}
   \cfoot{}

   \newenvironment{Figure}{\begin{figure}[htbp]}{\end{figure}}

\begin{document}
   \title{\bf \LARGE MLRISC \\ \large A framework for retargetable and optimizing compiler back ends}  
   \author{\begin{tabular}{c}
            Lal George \\ \\
            Bell Laboratories \\
            600--700 Mountain Ave. \\
            Murray Hill, NJ 07974--0636. \\
            {\tt george@research.bell-labs.com}
            \end{tabular}
          \and 
          \begin{tabular}{c}
            Allen Leung \\ \\
            New York University \\
            719 Broadway, Rm. 708 \\ 
            New York, NY 10003. \\
            {\tt leunga@cs.nyu.edu}
           \end{tabular}
        }

   \date{\today}
   \bibliographystyle{alpha}

   \maketitle

   \begin{abstract}
Writing native code generators for modern processors is a significant
investment.  Unfortunately it is difficult
to reuse this investment for other architectures, and even more
difficult to reuse for other source language compilers.   MLRISC is
a customizable optimizing back-end written in
\externhref{http://cm.bell-labs.com/cm/cs/what/smlnj/sml.html}{Standard ML}
and has been successfully retargeted to multiple architectures.
MLRISC deals elegantly with the special requirements imposed by the
execution model of different high-level, typed languages, by allowing
many components of the system to be customized to fit the source language
semantics and runtime system requirements.
   \end{abstract}
   \tableofcontents
   \newpage

\majorsection{MLRISC}
\section{MLRISC}
  \begin{center} 
    \begin{Bold}
     A framework for retargetable and optimizing compiler back ends 
    \end{Bold}
  \end{center}
\begin{center}
  \begin{tabular}{cc} 
    \begin{address}
      \href{mailto:george@research.bell-labs.com}{Lal George} 
    \end{address} &
    \begin{address}
      \href{mailto:leunga@cs.nyu.edu}{ Allen Leung}
    \end{address} \\
       Bell Labs & New York University \\
  \end{tabular}   
\end{center}

\begin{center}
\image{MLRISC logo}{pictures/png/uncol.png}{align="middle"}

\begin{Italics}
   \href{contributors.html}{Contributors}
\end{Italics}
\end{center}

Writing native code generators for modern processors is a significant
investment.  Unfortunately it is difficult
to reuse this investment for other architectures, and even more
difficult to reuse for other source language compilers.   MLRISC is
a customizable optimizing back-end written in
\externhref{http://cm.bell-labs.com/cm/cs/what/smlnj/sml.html}{Standard ML}
and has been successfully retargeted to multiple architectures.
MLRISC deals elegantly with the special requirements imposed by the
execution model of different high-level, typed languages, by allowing
many components of the system to be customized to fit the source language
semantics and runtime system requirements.

The \begin{color}{#aa0000}Overview\end{color} pages on the left provide 
an introduction the MLRISC system, mostly from the client's perspective,  
while the \begin{color}{#aa0000}System\end{color}
pages give a more detailed look at the 
innards, and are of interest to MLRISC hackers.   As usual, development of
the system has outpaced the documentation process substantally; thus
the latter part of the document is incomplete but it may still be useful. 

These pages are also available in 
\href{../latex/mlrisc.ps}{tech report} form.

\section{Contributors}
 \subsubsection{Past}
   \begin{itemize} 
    \item Florent Guillame (INRIA)
    \item George C. Necula (CMU)
    \item Ken Cline (CMU)
    \item Andrew Bernard (CMU)
    \item Dino Oliva (NEC)
   \end{itemize} 

\subsubsection{Present}
  \begin{itemize}
    \item Allen Leung (NYU)
    \item Fermin Reig (University of Glasgow)
  \end{itemize}

\section{Requirements}
   The most up-to-date MLRISC system requires 
   \externhref{http://cm.bell-labs.com/cm/cs/what/smlnj/index.html}{Standard ML of New Jersey} version 110.0.3 or later. 

\section{How to Obtain MLRISC}

There are a few ways to obtain the MLRISC system.
\begin{enumerate}
\item
An old version of MLRISC is available from
\externhref{http://cm.bell-labs.com/cm/cs/what/smlnj/doc/MLRISC/quick-tour/index.html}{this link}.   
This version is stable but very out-dated, and does
not contain the most up-to-date features. 
\item
New experimental versions are available from the 
\externhref{http://cm.bell-labs.com/cm/cs/what/smlnj/software.html}{SML/NJ software page} as part of the SML/NJ compiler releases.  
These versions are relative stable, but
do not include the entire MLRISC source tree.
\item \href{mailto:leunga@cs.nyu.edu}{Allen} 
keeps an up-to-date version of MLRISC at NYU for private use.
This version includes everything but is under constant changes, so beware!
To access the CVS repository, set your \sml{CVSROOT} environment variable 
to
\begin{verbatim}
   :pserver:mlrisc@react-ilp.cs.nyu.edu:/home/leunga/mlrisc
\end{verbatim}
and checkout the repository using
\begin{verbatim}
   cvs co MLRISC++
\end{verbatim}
The password to use is \sml{mlrisc}.
\item
Generally speaking, you can get the latest version of MLRISC by asking
\href{mailto:george@research.bell-labs.com}{Lal}.
\end{enumerate}
MLRISC is \newdef{free, open source} software, and is released under the
\href{http://cm.bell-labs.com/cm/cs/what/smlnj/license.html}{SML/NJ license}.

\majorsection{Overview}
\section{Problem Statement}

    Writing a native code generator for any language is a significant
    investment, especially for todays modern processors with require extensive
    compiler support to achieve high performance.  The algorithms that must
    be used to generate high quality code are complex, sometimes quite
    delicate, and require substantial infrastructure.

    \image{Retargeting compiler}{pictures/png/uncol2.png}{align=right}
    A specific architecture has a
    relatively short life time in relation to the time taken to build
    the code generator, and one quickly needs the ability to retarget
    to new versions of the architecture, or to different target
    architectures. This is by no means an open problem. There are many
    compilers today that target multiple architectures, however the
    quality of code varies. For example, 
    \begin{color}{red}\begin{Italics}lcc\end{Italics}\end{color} 
    by Chris Fraser and David Hansen does
    no back end optimizations; 
    \begin{color}{red}\begin{Italics}gcc\end{Italics}\end{color} 
    from the Free Software Foundation does extensive peephole and simple
    data flow optimizations, and falls short on advanced superscalar
    optimizations; and finally the 
    \begin{color}{red}\begin{Italics}IMPACT\end{Italics}\end{color} 
    compiler done by the Impact group at the
    University of Illinois specializes in more advanced superscalar
    and predicated architectures. 

    \br{clear=right}
    
    \image{UNCOL?}{pictures/png/uncol.png}{align=left} Assuming
    the retargeting issue is solved, one would like to use all the
    developed infrastructure for multiple source languages. This
    problem is far from solved; even though \italics{gcc} has been used
    for multiple languages like Ada, Pascal, and Modula III, each of
    these have similiar execution models or were forced to adopt C
    conventions.  \italics{gcc} cannot be used directly for languages
    such as Lisp, Smalltalk, Haskell, or ML that have radically
    different execution models and special requirements to support
    advanced language features.
 
   

\section{Contributions}
    The optimizations provided by MLRISC are at a similar level to
    those performed by the Impact compiler; several target back ends
    exist (Dec Alpha, HPPA, Sparc, x86, and PPC); but more importantly, the
    framework has been demonstrated in \href{systems.html}{real use} 
    for languages with radically different execution models.  These include:
   
   \begin{center}
   \begin{tabular}{|c|c|} \hline 
       Compiler & Association \\ \hline
       \begin{color}{#005500}SML/NJ\end{color} & Bell Labs and Princeton\\\hline
       \begin{color}{#005500}TIL\end{color} & CMU \\ \hline
       \begin{color}{#005500}Tiger\end{color} &  Princeton \\ \hline
       \begin{color}{#005500}C--\end{color} & OGI \\ \hline
       \begin{color}{#005500}SML/Regions\end{color} & DIKU \\ \hline
       \begin{color}{#005500}Moby\end{color} &  Bell Labs \\ \hline
   \end{tabular}
   \end{center}
    
    The strength of MLRISC lies in the ability to easily create high
    quality code generator for each of these systems. For example:
    
   \begin{description} 
      \item[Tiger:] Has an execution
      model very similar to C with stack allocated activation frames,
      and also maintains static and dynamic chains to support lexical
      scoping.

      \item[TIL:] Is similar to C in its
      use of activation frames, however it uses a 
      \emph{typed intermediate language} that 
       supports \emph{almost tag-free}
      garbage collection.  This has severe implications on the
      interaction of spilling and garbage collection. The set of live
      variables and their locations, be it registers or frame slots,
      is recorded in a trace table for a specific program point. When
      spilling occurs, it is necessary to adjust some of these trace
      tables to reflect the new locations of live variables.

      \item[SML/NJ:] Has no runtime
      stack, but stores all execution context in a garbage collected
      heap. This arrangement imposes special requirements for spilling
      registers. SML/NJ also does \emph{dynamic linking} --- that is
      to say, no use is made of a conventional linker, but machine
      code is generated directly and linked into the interactive
      environment, dynamically.
 
      \item[C--:] Is a C-like portable assembly
      language used as an intermediate language for high level typed language,
      and provides direct compilation support for exceptions and 
      precise garbage collection.  In addition, it allows 
      interoperability with C function calls.  
\end{description}

  It is not uncommon for any of these systems to store special global
  values in dedicated registers, and use their own parameter passing
  and callee-save conventions. In any language that supports garbage
  collection, there are also the issues of generating gc type maps,
  and gc-safety in aggressive optimizations.  MLRISC deals with all these
  important issues by allowing customization of many aspects of the system.

\section{MLRISC Based Compiler}
    A traditional compiler will typically consist of a 
    \begin{color}{#dd0000}lex/yacc\end{color} based front end, an 
    \begin{color}{#DD0000}optimization\end{color} 
    phase that is repeatedly invoked
    over some intermediate representation, and finally a 
    \begin{color}{#DD0000}back end\end{color} 
    code generation phase. The intermediate
    representation is usually at a level of detail appropriate to the
    optimization being performed, and may be far removed from the
    native instructions of the target architecture. The back end
    proceeds by translating the intermediate representation into
    instructions and registers for an abstract machine that is much
    closer to the target architecture. Retargetting is then achieved
    by mapping the registers and instructions of the abstract machine
    to registers and instructions of the target architecture.

    \br{clear=left}

    \image{MLRISC based compiler}{pictures/png/compiler-2.png}{align=right}

  An MLRISC based compiler, on the other hand, translates the
  intermediate representation into MLRISC instructions and it is the
  MLRISC instructions that get mapped onto instructions of the target
  architecture. Another possibility is to translate the front end
  abstract machine instructions instead of the intermediate
  representation.  Once MLRISC instructions have been generated,
  nearly all aspects of high quality code generation come for free. A
  long story would be cut short if MLRISC were just another abstract
  machine.

  \begin{color}{#580000} The key idea behind MLRISC is that there is no
  single MLRISC instruction set or intermediate program
  representation, \end{color} but the MLRISC intermediate representation
  is specialized to the needs of the front end source language being
  compiled. The specialization does not stop there, but the:
   \begin{itemize}
  \item  \begin{color}{#005500}target instruction set\end{color},
  \item  \begin{color}{#005500}flowgraph\end{color}, and
  \item  \begin{color}{#005500}entire optimization suite\end{color}
  \end{itemize}

  are specialized to the needs of the front end. The ability to
  consistently specialize each of these to create a back end for a
  specific language, summarizes the characteristics of MLRISC that
  distinguishes it from other retargetable backends.

  \begin{color}{#580000} It is important to emphasize that little
  optimizations performed on the MLRISC intermediate
  representation. \end{color} Most optimizations are done on a flowgraph of
  target machine instructions, to enable optimizations that take advantage
  of the characteristics of each architectural.
  The MLRISC intermediate representation is just used as a stepping 
  stone to get to the flowgraph.

\section{MLRISC Intermediate Representation}
   The MLRISC intermediate language is called 
   \newdef{MLTREE} At the lowest level, the core of MLTREE is a 
    \italics{Register Transfer Language (RTL)} 
   but represented in tree form. The tree
   form makes it convenient to use tree pattern matching tools like
   BURG (where appropriate) to do target instruction selection. Thus a
   tree such as: 

   \begin{SML}  
  MV(32, t, 
     ADDT(32, MULT(32, REG(32, b), REG(32, b)),
              MULT(32, MULT(REG(32, a), LI(4)), REG(32, c))))
   \end{SML}

   computes \sml{t := b*b + 4*a*c} to 32-bit precision. 
   The nodes \sml{ADDT} and
   \sml{MULT} are the trapping form of addition and multiplication,
   and \sml{LI} is used for integer constants. An infinite number
   of registers are assumed by the model, however depending on the
   target machine the first \sml{0..K} registers map onto the first
   \sml{K} registers on the target machine. Everything else is
   assumed to be a pseudo-register. The \sml{REG} node is used to
   indicate a  general purpose register. 

   
   The core MLTREE language makes no assumptions about instructions or
   calling convections of the target architecture. Trees can be
   created and combined in almost any form, with certain meaningless
   trees such as \sml{LOAD(32, FLOAD(64, LI 0))} being forbidden by the
   MLTREE type structure.

    Such pure trees are nice but inadequate in real compilers. One
   needs to be able to propagate front end specific information, such
   as frame sizes and frame offsets where the actual values are only
   available after register allocation and spilling. One could add
   support for frames in MLRISC, however this becomes a slippery slope
   because some compilers (e.g. SML/NJ) do not have a conventional
   notion of frames --- indeed there is no runtime stack in the
   execution of SML/NJ. A frame organization for one person may not
   meet the needs for another, and so on.  In MLRISC, the special
   requirements of different compilers is communicated into the MLTREE
   language, and subsequently into the optimizations phases, by
   specializing the MLTREE data structure with client specific
   information. There are currently \emph{five} dimensions over
   which one could specialize the MLTREE language.

  \begin{description} 
    \item[Constants] Constants are an
    abstraction for integer literals whose value is known after
    certain phases of code generation. Frame sizes and offsets are an
    example.  
    \image{MLRISC intermediate representation}{pictures/png/mlrisc-ir.png}{align=right}
    \item[Regions] While the data
    dependencies between arithmetic operations is implicit in the
    instruction, the data dependencies between memory operations is
    not. Regions are an abstract view of memory that make this
    dependence explicit and is specially useful for instruction
    reordering. 

    \item[Pseudo-ops] Pseudo-ops are
    intended to correspond to pseudo-op directives provided by native
    assemblers to lay out data, jump tables, and perform alignment.

    \item[Annotations]
    \href{annotations.html}{Annotations} are used
    for injecting semantics and other program information from the front-end 
    into the backend.  For example, a probability annotation can be
    attached to a branch instruction.  Similarly, line number annotations
    can be attached to basic blocks to aid debugging.   
    In many language implementations function local variables are
    spilled to activation frames on the stack. Spill slots contribute
    to the size of a function's frame. When an instruction produces a
    spill, we may need to update the frame associated to that
    instruction (increase the size of its spilling area). The frame
    for the current function can be injected in an annotation, which
    can be later examined by the spill callback during register allocation. 

     Annotations are
    implemented as an universal type and can be arbitrarily extended.
    Individual annotations can be associated
    with compiler objects of varying granularity, 
    from compilation units, to regions, basic blocks, flow edges,
    and down to the instructions.


    \item[User Defined Extensions]
    In the most extreme case, the basic constructors defined in the MLTREE
    language may be inadequate for the task at hand.  
    MLTREE allows the client to arbitrarily extend
    the set of statements and expressions to more closely match the
    source language and the target architecture(s). 
    
     For example, when using MLRISC for the backend of a DSP compiler 
     it may be useful to extend the set of MLRISC operators to include 
     fix point and saturated arithmetic.  
     Similarly, when developing a language for loop parallelization, it may
     be useful to extend the MLTREE language with higher-level loop 
     constructs.
  \end{description} 

\subsection{Examples}
   
   In the SML/NJ compiler, an encoding of a list of registers
   is passed to the garbage collector as the roots of live
   variables. This encoding cannot be computed until register
   allocation has been performed, therefore the integer literal
   encoding is represented as an abstract 
   \href{constants.html}{constant}.

    Again, in the SML/NJ compiler, most stores are for initializing 
   records in the allocation space, therefore representing every slot in
   the allocation space as a unique region allows one to commute
   most store instructions. Similarly, most loads are from
   \emph{immutable} records, and a simple analysis marks these are
   being accesses to \emph{read-only} memory. Read-only memory is
   characterized as having multiple \emph{uses} but no
   \emph{definitions}.

    In the TIL compiler, a \emph{trace table} is generated for
   every call site that records the set of live variables, their
   location (register or stack offset), and the type associated with
   the variable. This table is integrated into the program using the
   abstract pseudo-op mechanism. An interesting aspect of these tables
   is that they may need adjustment based on the results of register
   spilling.

    The more convention use of the psuedo-op abstraction is to
   propagate function prologue and epilogue information.

    The constants abstraction are created by a tree node called
   \sml{CONST}. In the SML/NJ compiler, the tree that communicates
   garbage collection information looks like:

\begin{verbatim}
   MV(32, maskReg, CONST{r110,r200,r300,r400 ...})
\end{verbatim}

  where \sml{maskReg} is a dedicated register. On the DEC Alpha,
  this would get translated to:

\begin{verbatim}
   LDA maskReg, {encode(r110,r200,r300,r400, ...)}
\end{verbatim}

   which indicates that the alpha instruction set (and optimization
   suite) know about these types of values. Further, after
   register allocation, the \sml{LDA} instruction may not be
   sufficient as the encoding may result in a value that is too large
   as an operand to \sml{LDA}. Two instructions may ultimately be
   required to load the encoding into the \sml{maskReg}
   register. This expansion is done during 
   \href{span-dep.html}{span-dependency resolution}.

    All these examples are intended to indicate that one
   intermediate representation and optimization suite does not fit
   all, but that the intermediate representation and optimization
   suite needs to be specialized to the needs of the client.

\section{MLRisc Generation}
  Every compiler will eventually compile down to an abstract machine
  that it believes will execute source programs efficiently. The
  abstract machine will typically consists of abstract machine
  registers and instructions, one or more stacks, and parameter
  passing conventions.  The hope is that all this will map down
  efficiently onto the target machine. Indeed, the abstract machine
  should be reasonably close to architectures that are envisioned as
  possible targets. Several step need to be followed in the generation
  of MLRisc.

  \begin{enumerate}
   \item The first step in generating target machine code is to define
   the MLRisc intermediate representation after it has been
   appropriately specialized. The interfaces that describe the
   dimensions of specialization are quite simple. Depending on the
   compiler, these may be target dependent; for example, in the SML/NJ
   compiler, the encoding of registers used to indicate the roots of
   garbage collection depend on how the runtime system decodes the
   information.

  \item The only real connection between the MLRisc intermediate
  representation and the target machine is that the first
  $0..K-1$ MLRisc registers map onto the first $K$
  physical registers on the target machine. Thus some mapping of
  dedicated abstract machine registers to physical target registers is
  required. It is not always necessary to map abstract machine
  registers to physical machine registers. For example, on
  architectures like the x86 with few registers, some abstract machine
  registers may be mapped to fixed memory locations. Thus an abstract
  machine register like the \sml{maskReg} may have something like:
\begin{SML}
  LOAD(32, LABEL maskRegLab) 
\end{SML}
spliced instead.

  \item The unit of compilation is called a 
   \href{cluster.html}{cluster} which
  is the smallest unit for inter-procedural optimizations. A cluster
  will typically consist of several entry points that may call each
  other, as well as call local functions in the module. For maximum
  flexibility, the parameter passing convention for local functions
  should be specialized by the \href{mlrisc-ra.html}{register allocator}.

   Once the MLRisc trees for a cluster have been built, they must
  be converted into target assembly or machine code. This is done by
  building up a function (\newdef{codegen}) that
  glues together optimizations modules that have been specialized. For
  example, the target instruction set must be specialized to hold the
  MLRisc constants; the flowgraph must be specialized to carry these
  instructions as well as the MLRisc pseudo-ops; the optimization
  modules must know about several front end constraints such as how to
  spill registers.
  \end{enumerate}

   If the module that translates the abstract machine instructions
  into MLRisc instructions has been appropriately parameterized, then
  it can be reused for multiple target architectures. For high level
  languages it is better to generate MLRisc instructions from the high
  level intermediate form used by the front end of the compiler.

\section{Back End Optimizations}

  Once MLRisc trees have been generated, they are passed into a module
  that generates a flowgraph of target machine instructions. Again,
  this module and all subsequent optimization phases have been
  specialized to the front end.  
  \image{Back end optimizations}{pictures/png/optimization.png}{align=right} 
   Nearly all
  instruction selection modules provided by MLRISC use a simple tree
  pattern matching algorithm rather than the more heavy weight BURG
  tools --- including the x86 \begin{color}{#580000} It is important to
  emphasis that all optimizations are performed on the flowgraph of
  target machine instructions and \emph{not} MLRisc
  immediate IR. \end{color} There is complete flexibility in the order,
  and nature of the optimizations performed. 

 \section{Register Allocation}
  All the optimization modules are written in a generic fashion but
  parameterized over architecture and client information. The Standard
  ML module system is a central mechanism to the design and
  organization of MLRISC. Parameterized modules in Standard ML are
provided by \newdef{functors}, that takes the
specification of input modules and produces a module that matches some
output specification. In particular, SML/NJ modules are 
\emph{higher order}, which means that a functor can yield functors as a
result. I will use register allocation as an example.

\image{Back end optimizations}{pictures/png/hof-1.png}{align=left}

  The register allocator is written has a higher order functor which
  when applied to suitable arguments produces an integer or floating
  point register allocator. The figure is simplifed because the output
  functor is not restricted to integer and floating point allocators
  but could also be other types of allocators, for example, condition
  code.  The integer and floating point register allocators are
  functors that only take \emph{client specific} parameters as
  input, whereas the higher-order takes architectural parameters as
  input. The client specific parameters include:
\begin{SML}
  nFreeRegs : int
  dedicated : int list
  spill : ..
  reload : ..
\end{SML}

where:
 \begin{description} 
    \item[\sml{nFreeRegs}] is the number of free registers or
    essentially the number of colors available for coloring the
    interference graph.

    \item[\sml{dedicated}] is the list of dedicated registers. It
    is useful to exclude these from the graph-color process to reduce
    the size of the data structures created.

    \item[\sml{spill/reload}] are functions that describe how to
    spill and reload registers that need to be spilled or reloaded in
    an instruction. These two functions are perhaps the most
    complicated pieces of information that need to be supplied by a
    client of MLRISC.
\end{description} 

  The architecture specific parameters supplied to the higher-order
  functor include:
\begin{SML}
  firstPseudoReg : int
  maxPseudoR : unit -> int
  defUse : instruction -> (int list * int list)
\end{SML}

  where: 
\begin{description}  
    \item[\sml{firstPseudoR}] is an integer representing the first
    pseudo register. Any register below this value is a physical
    register.
 
    \item[\sml{maxPseudoR}] is a function that returns an
    integer indicating the number of the highest pseudo-register that
    has been used in the program. This number is useful in estimating
    the intial size of various tables.

    \item[\sml{defUse}] is a function that returns the
    registers defined and used by an instruction.
\end{description}

  These parameters are largely self explanatory, however, there are
  addition architectural parameters that relate to the internal
  representation of instructions that would be ugly to explain. For
  example there is the need for a module that does liveness analysis
  over the register class that is being allocated. This type of
  complexity can be shielded from a user.  For the DEC Alpha the
  situation is as shown in the figure:

  \image{Back end optimizations}{pictures/png/hof-2.png}{align=center}

  The client only sees the functors on the right, to which only client
  specific information need be provided. There is the illusion of a
  dedicated DEC Alpha integer and floating point register
  allocator. There are several advantages to this:
  \begin{itemize}
    \item The architectural parameters that are implementation specific
do not need to be explained to a user, and are supplied by someone
that intimately understands the port to the target architecture. 

     \item The number of parameters that a client supplies is
reduced.

     \item The parameters that the client supplies is restricted to
things that concern the front end. 
  \end{itemize}

\section{Machine Description}
\subsection{Overview}

  \newdef{MDGen} is a simple tool for generating 
various modules in the MLRISC customizable code generator
directly from machine descriptions.   These descriptions 
contain architectural information such as:
\begin{enumerate}
    \item How the the register file(s) are organized.   
    \item How instructions are encoded in machine code: MLRISC uses
this information to generate machine instructions directly into a byte stream.
Directly machine code generation is used in the SML/NJ compiler.
    \item How instructions are pretty printed in assembly: this is used
for debugging and also for assembly output for other non-SML/NJ backends.
    \item How instructions are internally represented in MLRISC. 
   \item Other information needed for performing optimizations, which
        include:
   \begin{enumerate}
     \item The register transfer list (RTL) that defines the 
           operational semantics of the instruction.
     \item Delay slot mechanisms.
     \item Information for performing span dependency resolution.
     \item Pipeline and reservation table characteristics.
   \end{enumerate}
\end{enumerate}

Currently, item 5 is not ready for prime time.

\subsubsection{Why MDGen?}
MLRISC manipulates all instruction sets via a set of abstract
interfaces, which allows the programmer to arbitrarily choose an
instruction representation that is most convenient for a particular 
architecture.  However, various functions that manipulate
this representation must be provided by the instruction set's programmer.  
As the number and complexities of each optimizations grow, and as
the number of architectures increases, the functions
for manipulating the instructions become more numerous and complex.
In order to keep the effort of developing and maintaining
an instruction set manageable,
the MDGen tool is developed to (partially) automate this task.
 
\subsubsection{Syntax}

   MDGen's machine descriptions are written in a syntax that is very
much like that of 
\externhref{http://cm.bell-labs.com/cm/cs/what/smlnj/sml.html}{Standard ML}. 
Most core SML constructs are recognized.  
In addition, new declaration forms specific to MDGen are 
used to specify architectural information.

\paragraph{Reserved Words}
   All SML keywords are reserved words in MDGen.
   In addition, the following keywords are also reserved:

\begin{verbatim}
   always architecture assembly at backwards big bits branching called
   candidate cell cells cellset debug delayslot dependent endian field
   fields formats forwards instruction internal little locations lowercase
   name never nodelayslot nullified opcode ordering padded pipeline predicated
   register rtl signed span storage superscalar unsigned uppercase 
   verbatim version vliw when
\end{verbatim}

   Two kinds are quotations marks are also reserved:   
\begin{SML}
   [[ ]]
   `` ''
\end{SML}

   The first \sml{[[ ]]} is for describing semantics.  The
second \sml{`` ''} is for describing assembly syntax.

\paragraph{Syntactic Sugar}

   MDGen recognizes the following syntactic sugar.
\begin{description}
\item[Record abbreviations]
Record expressions such as \sml{{x=x,y=y,z=z}} 
can be simplified to just \sml{{x,y,z}}.
\item[Binary literals]
Literals in binary can be written with the prefix \sml{0b} (for integer types)
or \sml{0wb} (for word types).   For example, \sml{0wb101111} is the same 
as \sml{0wx2f} and \sml{0w79}.
\item[Bit slices]
   A bit slice, which extracts a range of bits from a word, can be written
using an \sml{at} expression.  For example, \sml{w at [16..18]} 
means the same thing as \verb|Word32.andb(Word32.>>(w, 0w16),0w7)|, i.e.
it extracts bit 16 to 18 from \sml{w}.  
The least significant bit the zeroth bit. 

In general, we can write:
\begin{SML}
  w at [range1, range2, ..., rangen]
\end{SML}
to extract a sequence of slices from $w$ and concatenate them together.
For example, the expression
\begin{SML}
   0wxabcd at [0..3, 4..7, 8..11, 12..15]
\end{SML}
swap the 4 nybbles from the 16-bit word, and evaluates to \sml{0wxdcba}.

\item[Signature]
Signature declarations of the form
\begin{SML}
   val x y z : int -> int
\end{SML}
can be used as a shorthand for the more verbose:
\begin{SML}
   val x : int -> int
   val y : int -> int
   val z : int -> int
\end{SML}
\end{description}

\subsubsection{Elaboration Semantics}

   Unfortunately, there is no complete formal semantics of how
an MDGen specification elaborates.  
   But generally speaking, a machine description is a just a structure 
(in the SML sense).   Different components of this structure describe 
different aspects of the architecture.

\paragraph{Syntactic Overloading}
In general, the syntactic overloading are used heavily in MDGen.
There are three types of definitions:
\begin{itemize}
 \item Definitions that defines properties of the instruction set.
 \item Definitions of functions and terms that are in the RTL meta-language.
The syntax of MDGen's RTL language is borrowed heavily from Lambda-RTL, 
which in turns is borrowed heavily from SML.
 \item Definitions of functions and types that are to be included in the
output generated by the MDGen tool.  These are usually auxiliary
helper functions and definitions.
\end{itemize}
In general, entities of type 2, when appearing in other context, are
properly meta-quoted in the semantics quotations \sml{[[ ]]}.

\subsubsection{Basic Structure of A Machine Description}

   The machine description for an architecture are defined via
an \sml{architecture} declaration, which has the following general
form.

\begin{SML}
architecture name =
struct
   \Term{architecture type declaration}
   \Term{endianess declaration}
   \Term{storage class declarations}
   \Term{locations declarations}
   \Term{assembly case declarations}
   \Term{delayslot declaration}
   \Term{instruction machine encoding format declarations}
   \Term{nested structure declarations}
   \Term{instruction definition}
end
\end{SML}

\subsection{Describing the Architecture} 

\subsubsection{Architecture type}
  Architecture type declaration specifies whether the architecture is
a superscalar or a VLIW/EPIC machine.  Currently, this information is
ignored.

\begin{SML}
   \Term{architecture type declaration} ::= superscalar | vliw
\end{SML}

\subsubsection{Storage class}

Storage class declarations specify various information about the
registers in the architecture.  For example, the Alpha has 32 general
purpose registers and 32 floating point registers.  In addition, MLRISC
requires that each architecture specifies a (pseudo) register 
type\footnote{Called cellkind in MLRISC.} for 
holding condition codes (\sml{CC}). 
To specify these information in MDGen, we can say:

\begin{SML}
   storage
     GP "r" = 32 cells of 64 bits in cellset called "register"
                assembly as (fn (30,_) => "$sp"
                              | (r,_)  => "$"^Int.toString r
                            )
   | FP "f" = 32 cells of 64 bits in cellset called "floating point register"
                assembly as (fn (f,_) => "$f"^Int.toString f)
   | CC "cc" = cells of 64 bits in cellset GP called "condition code register"
                assembly as "cc"
\end{SML}

\begin{itemize}
  \item There are 32 64-bit general purpose registers,
32 64-bit floating point registers, while \sml{CC} is not a 
real register type. 
  \item Cellsets
are used by MLRISC for annotating liveness information in the program.
  The clause \sml{in cellset} states that register type \sml{GP} 
and \sml{FP} are allotted their own components in the cellset,    
while the register type \sml{CC} are put
in the same cellset component as \sml{GP}.
  \item The clause \sml{assembly as} specifies
   how each register is to be pretty printed.  On the Alpha, general 
   purpose register are pretty printed with prefix \sml{$}, while
   floating point registers are pretty printed with the prefix \sml{$f}. 
   A special case is made for register 30, which is the stack pointer, and 
   is pretty printing as \sml{$sp}.  Pseudo condition code registers
   are pretty printed with the prefix \sml{cc}.   
\end{itemize}

\subsubsection{Locations}

  Special locations in the register files can be declared using the
\sml{locations} declarations.  On the Alpha, GPR
30 is the stack pointer, GPR 28 and floating point register 30
are used as the assembly temporaries.  This special constants
can be defined as follows:

\begin{SML}
   locations
       stackptrR = $GP[30]
   and asmTmpR   = $GP[28]
   and fasmTmp   = $FP[30]
\end{SML}

\subsection{Specifying the Machine Encoding}
\subsubsection{Endianess}

The endianess declaration specifies whether the machine is little
endian or big endian so that the correct machine instruction encoding 
functions can be generated.  The general syntax of this is:

\begin{SML}
   \Term{endianess declaration} ::= little endian | big endian
\end{SML}

The Alpha is little endian, so we just say: 
\begin{SML}
    little endian
\end{SML}

\subsubsection{Defining New Instruction Formats}

   How instructions are encoded are specified using 
\sml{instruction format} declarations.  An instruction format declaration
has the following syntax:
\begin{SML}
  \Term{instruction machine encoding format declarations} ::=
     instruction formats n bits 
       \Term{format}1
     | \Term{format}2
     | \Term{format}3
     | ...
     | \Term{format}n-1
     | \Term{format}n
\end{SML}

Each encoding format can be a primitive format, or a derived format.

\paragraph{Primitive formats}

A primitive format is simply specified by giving it a name and specifying
the position, names and types of its fields.   This is usually the same
way it is described in a architectural reference manual.


Here is how we specify some of the (32 bit) primitive instruction formats 
used in the Alpha. 
\begin{SML}
   instruction formats 32 bits
     Memory\{opc:6, ra:5, rb:GP 5, disp: signed 16\} 
   | Jump\{opc:6=0wx1a,ra:GP 5,rb:GP 5,h:2,disp:int signed 14\}  
   | Memory_fun\{opc:6, ra:GP 5, rb:GP 5, func:16\}     
   | Branch\{opc:branch 6, ra:GP 5, disp:signed 21\}         
   | Fbranch\{opc:fbranch 6, ra:FP 5, disp:signed 21\}        
   | Operate0\{opc:6,ra:GP 5,rb:GP 5,sbz:13..15=0,_:1=0,func:5..11,rc:GP 5\}
   | Operate1\{opc:6,ra:GP 5,lit:signed 13..20,_:1=1,func:5..11,rc:GP 5\} 
\end{SML}

For example, the format \sml{Memory}
\begin{SML}
     Memory\{opc:6, ra:5, rb:GP 5, disp: signed 16\} 
\end{SML}
has a 6-bit opcode field, a 5-bit \sml{ra} field, a 5-bit \sml{rb}
field which always hold a general purpose register, and a 16-bit 
sign-extended displacement field.  The field to the left is positioned 
at the most significant bits, while the field to the right is positioned
at the least.  The widths of these fields must add up to 32 bits.


Similarly, the format \sml{Jump}
\begin{SML}
  Jump{opc:6=0wx1a,ra:GP 5,rb:GP 5,h:2,disp:int signed 14}  
\end{SML}
contains a 6-bit opcode field which always hold the constant \sml{0x1a},
two 5-bit fields \sml{ra} and \sml{rb} which are of type \sml{GP},
and a 14-bit sign-extended field of type integer.

  Each field in a primitive format has one of 5 forms:
\begin{SML}
   \Term{name} : \Term{position} 
   \Term{name} : \Term{position} = \Term{value} 
   \Term{name} : \Term{type} \Term{position} 
   \Term{name} : \Term{type} \Term{position} = \Term{value} 
   _           : \Term{position} = \Term{value} 
\end{SML}
where \Term{position} is either a width, or a bits range 
$n$\sml{..}$m$,
with an optional \sml{signed} prefix.  The last form, with a wild card
for the field name, can be used to specify an anonymous field that
always has a fixed value.  


  By default, a field has type \sml{Word32.word}.  If a type $T$ 
is specified, then the function \sml{emit_}$T$ is implicitly called
to convert the type into the appropriate encoding.   The function 
\sml{emit_}$T$ are generated automatically by MDGen if it is a cellkind
defined by the \sml{storage} class declaration, or if it is a primitive
type such as integer or boolean.  
There are also other ways to automatically generate this function
(more on this later.)

  For example, the format \sml{Operate1}
\begin{SML}
   Operate1\{opc:6,ra:GP 5,lit:signed 13..20,_:1=1,func:5..11,rc:GP 5\} 
\end{SML}
states that bits 26 to 31 are allocated to field \sml{opc}, 
bits 21 to 25 are allocated to field \sml{ra}, which is of type 
\sml{GP}, bits 13 to 20 are allocated to field \sml{lit}, bit 12
is a single bit of value 1, etc.


MDGen generates a function for each primitive format declaration of
the same name that can be used for emitting the instruction.  
In the case of the Alpha, the following functions are generated:
\begin{SML}
   val Memory : \{opc:Word32.word, ra:Word32.word, 
                 rb:int, disp:Word32.word\} -> unit
   val Jump   : \{ra:int, rb:int, disp:Word32.word\} -> unit
   val Operate1 : \{opc:Word32.word, ra:int, lit:Word32.word,
                   func:Word32.word, rc:int\} -> unit
\end{SML}

\paragraph{Derived formats}

   Derived formats are simply instruction formats that are defined
in terms of other formats.  On the alpha, we have a \sml{Operate}
format that simplifies to either \sml{Operate0} or \sml{Operate1},
depending on whether the second argument is a literal or a register.  
\begin{SML}
   Operate\{opc,ra,rb,func,rc\} =
     (case rb of
       I.REGop rb => Operate0\{opc,ra,rb,func,rc\}
     | I.IMMop i  => Operate1\{opc,ra,lit=itow i,func,rc\}
     | I.HILABop le => Operate1\{opc,ra,lit=High{le=le},func,rc\}
     | I.LOLABop le => Operate1\{opc,ra,lit=Low{le=le},func,rc\}
     | I.LABop le => Operate1\{opc,ra,lit=itow(LabelExp.valueOf le),func,rc\}
     )
\end{SML}

\subsubsection{Generating Encoding Functions}

   In MLRISC, we represent an instruction as a set of ML datatypes.
Some of these datatypes represent specific fields or 
opcodes of the instructions.
MDGen lets us to associate a machine encoding to each datatype constructor
directly in the specification, and automatically generates an 
encoding function for these datatypes.

There are two different ways of specifying an encoding.  The first way
is just to write the machine encoding directly next the constructor.
Here's an example directly from the Alpha description:
\begin{SML}
   structure Instruction =
   struct
      datatype branch! =  (* table C-2 *)
         BR   0x30
                   | BSR 0x34
                              | BLBC 0x38
       | BEQ  0x39 | BLT 0x3a | BLE  0x3b
       | BLBS 0x3c | BNE 0x3d | BGE  0x3e
       | BGT  0x3f

      datatype fbranch! = (* table C-2 *)
                     FBEQ 0x31 | FBLT 0x32
       | FBLE 0x33             | FBNE 0x35
       | FBGE 0x36 | FBGT 0x37

      ...
   end
\end{SML}

The datatypes \sml{branch} and \sml{fbranch} represent specific
branch opcodes for integer branches \sml{BRANCH}, or floating point
branches \sml{FBRANCH}.  On the Alpha, instruction \sml{BR} is encoded
with an opcode of \sml{0x30}, instruction \sml{BSR} is encoded 
as \sml{0x34} etc.  MDGen automatically generates two functions
\begin{SML}
    val emit_branch : branch -> Word32.word
    val emit_fbranch : branch -> Word32.word
\end{SML}
that perform this encoding.    

In the specification for the instruction set, we state that the
\sml{BRANCH} instruction should be encoded using format \sml{Branch},
while the \sml{FBRANCH} instruction should be encoded using
format \sml{Fbranch}.
\begin{SML}
   structure MC =
   struct
      (* Auxiliary function for computing the displacement of a label *)
      fun disp ... = ...
      ...
   end

   ...

   instruction
     ...

   | BRANCH of branch * $GP * Label.label
     Branch\{opc=branch,ra=GP,disp=disp label\}

   | FBRANCH of fbranch * $FP * Label.label
     Fbranch\{opc=fbranch,ra=FP,disp=disp label\}

   | ...
\end{SML}

Since the primitive instructions formats \sml{Branch} and \sml{FBranch}
are defined with branch and fbranch as the type in the opcode field
\begin{SML}
   | Branch\{opc:branch 6, ra:GP 5, disp:signed 21\}          
   | Fbranch\{opc:fbranch 6, ra:FP 5, disp:signed 21\}       
\end{SML}
the functions \sml{emit_branch} and \sml{emit_fbranch} are implicitly
called.

 
Another way to specify an encoding is to specify a range, as 
in the following example:
\begin{SML}
   datatype fload[0x20..0x23]! = LDF | LDG | LDS | LDT

   datatype fstore[0x24..0x27]! = STF | STG | STS | STT
\end{SML}

This states that \sml{LDF} should be assigned the encoding \sml{0x20},
\sml{LDG} the encoding \sml{0x21} etc.  This form is useful for 
specifying a consecutive range.

\subsubsection{Encoding Variable Length Instructions}

   Most architectures nowadays have fixed length encodings for instructions.  
There are some notatable exceptions, however.  
The Intel x86 architecture uses a legacy
variable length encoding.   Modern RISC machines developed for 
embedded systems may utilize space-reduction compression schemes in their
instruction sets.  Finally, VLIW machines usually have some form
of NOP compression scheme for compacting issue packets. 

\subsection{Specifying the Assembly Formats}

\subsubsection{Assembly Case Declaration}  

  The assembly case declaration specifies whether the assembly should be
emitted in lower case, upper case, or verbatim.  If either lower case
or upper case is specified, all literal strings are converted to the 
appropriate case.  The general syntax of this declaration is:

\begin{SML}
   \Term{assembly case declaration} ::=
      lowercase assembly
    | uppercase assembly
    | verbatim  assembly
\end{SML}

\subsubsection{Assembly Annotations}

   Assembly output are specified in the assembly meta quotations
\sml{`` ''}, or string quotations \sml{" "}.   
For example, here is a fragment from the Alpha description:

\begin{SML}
   instruction
     ...
   | LOAD of \{ldOp:load, r: $GP, b: $GP, d:operand, mem:Region.region\}
     ``<ldOp>\t<r>, <d>()<mem>''

   | STORE of \{stOp:store, r: $GP, b: $GP, d:operand, mem:Region.region\}
     ``<stOp>\t<r>, <d>()<mem>''

   | BRANCH of branch * $GP * Label.label
     ``<branch>\t<GP>, <label>''

   | FBRANCH of fbranch * $FP * Label.label
     ``<fbranch>\t<FP>, <label>''

   | CMOVE of \{oper:cmove, ra: $GP, rb:operand, rc: $GP\}
     ``<oper>\t<ra>, <rb>, <rc>''

   | FOPERATE of \{oper:foperate, fa: $FP, fb: $FP, fc: $FP\}
     ``<oper>\t<fa>, <fb>, <fc>''

   | ...
\end{SML}

   All characters within the quotations \sml{`` ''} have the same 
interpretation as in the string quotation \sml{" "}, except when
they are delimited by the \newdef{backquotes}
\verb|< >|.
Here's how the backquote is interpreted:
\begin{itemize}
\item If it is \verb|<|$x$\verb.>. and $x$ is a variable name of type $t$,
  and if an assembly function of type $t$ is defined, then it will be invoked
  to convert $x$ to the appropriate text.
\item If it is \verb|<|$x$\verb.>. and $x$ is a variable name of type $t$,
  and if an assembly function of type $t$ is NOT defined, 
  then the function \sml{emit_}$x$ will be called to pretty print $x$.
 \item If it is \verb|<|$e$\verb.>. where $e$ is a general expression, then
  it will be used directly. 
\end{itemize}

\subsubsection{Generating Assembly Functions}
   Similar to machine encodings, we can attach assembly annotations to
datatype definitions and let MDGen generate the assembly functions for us.  
Annotations take two forms, explicit or implicit.
Explicit annotations are enclosed within assembly quotations \sml{`` ''}.

 
For example, on the Alpha the datatype \sml{operand} is used to represent
an integer operand.  This datatype is defined as follows:
\begin{SML}
   datatype operand =
       REGop of $GP                     ``<GP>'' 
     | IMMop of int                     ``<int>''
     | HILABop of LabelExp.labexp       ``hi(<labexp>)''
     | LOLABop of LabelExp.labexp       ``lo(<labexp>)''
     | LABop of LabelExp.labexp         ``<labexp>''
     | CONSTop of Constant.const        ``<const>''
\end{SML}
Basicaly this states that \sml{REGop r} should be pretty printed 
as \sml{$r}, \sml{IMMop i} 
as \sml{i}, \sml{HILABexp le} 
as \sml{hi(le)},
etc.
 
Implicit assembly annotations are specified by simply attaching 
an exclamation mark at the end of the datatype name.  This states
that the assembly output is the same as the name of the datatype 
constructor\footnote{But appropriately modified by the assembly case 
declaration.}. For example,
the datatype \sml{operate} is a listing of all integer opcodes 
used in MLRISC.  
\begin{SML}
   datatype operate! = (* table C-5 *)
       ADDL  (0wx10,0wx00)                       | ADDQ (0wx10,0wx20)
                           | CMPBGE(0wx10,0wx0f) | CMPEQ (0wx10,0wx2d)
     | CMPLE (0wx10,0wx6d) | CMPLT (0wx10,0wx4d) | CMPULE (0wx10,0wx3d)
     | CMPULT(0wx10,0wx1d) | SUBL  (0wx10,0wx09)
     | SUBQ  (0wx10,0wx29)
     | S4ADDL(0wx10,0wx02) | S4ADDQ (0wx10,0wx22) | S4SUBL (0wx10,0wx0b)
     | S4SUBQ(0wx10,0wx2b) | S8ADDL (0wx10,0wx12) | S8ADDQ (0wx10,0wx32)
     | S8SUBL(0wx10,0wx1b) | S8SUBQ (0wx10,0wx3b)

     | AND   (0wx11,0wx00) | BIC    (0wx11,0wx08) | BIS (0wx11,0wx20)
                                                  | EQV (0wx11,0wx48)
     | ORNOT (0wx11,0wx28) | XOR    (0wx11,0wx40)

     | EXTBL (0wx12,0wx06) | EXTLH  (0wx12,0wx6a) | EXTLL(0wx12,0wx26)
     | EXTQH (0wx12,0wx7a) | EXTQL  (0wx12,0wx36) | EXTWH(0wx12,0wx5a)
     | EXTWL (0wx12,0wx16) | INSBL  (0wx12,0wx0b) | INSLH(0wx12,0wx67)
     | INSLL (0wx12,0wx2b) | INSQH  (0wx12,0wx77) | INSQL(0wx12,0wx3b)
     | INSWH (0wx12,0wx57) | INSWL  (0wx12,0wx1b) | MSKBL(0wx12,0wx02)
     | MSKLH (0wx12,0wx62) | MSKLL  (0wx12,0wx22) | MSKQH(0wx12,0wx72)
     | MSKQL (0wx12,0wx32) | MSKWH  (0wx12,0wx52) | MSKWL(0wx12,0wx12)
     | SLL   (0wx12,0wx39) | SRA    (0wx12,0wx3c) | SRL  (0wx12,0wx34)
     | ZAP   (0wx12,0wx30) | ZAPNOT (0wx12,0wx31)
     | MULL  (0wx13,0wx00)                        | MULQ (0wx13,0wx20)
                           | UMULH  (0wx13,0wx30)
     | SGNXL "addl" (0wx10,0wx00) (* same as ADDL *)
\end{SML}
This definitions states that \sml{ADDL} should be pretty printed
as \sml{addl}, \sml{ADDQ} as \sml{addq}, etc.  However, the opcode 
\sml{SGNXL} is pretty printed as \sml{addl} since it has been explicitly
overridden.

\subsection{Defining the Instruction Set}

How the instruction set is represented is declared using the
\sml{instruction} declaration.  For example, here's how the Alpha instruction
set is defined:
\begin{SML}
  instruction 
     DEFFREG of $FP
   | LDA of \{r: $GP, b: $GP, d:operand\}	
   | LDAH of \{r: $GP, b: $GP, d:operand\} 
   | LOAD of \{ldOp:load, r: $GP, b: $GP, d:operand, mem:Region.region\}
   | STORE of \{stOp:store, r: $GP, b: $GP, d:operand, mem:Region.region\}
   | FLOAD of \{ldOp:fload, r: $FP, b: $GP, d:operand, mem:Region.region\}
   | FSTORE of \{stOp:fstore, r: $FP, b: $GP, d:operand, mem:Region.region\}
   | JMPL of \{r: $GP, b: $GP, d:int\} * Label.label list
   | JSR of \{r: $GP, b: $GP, d:int\} * C.cellset * C.cellset * Region.region
   | RET of \{r: $GP, b: $GP, d:int\} 
   | BRANCH of branch * $GP * Label.label   
   | FBRANCH of fbranch * $FP * Label.label  
   | ...
\end{SML}

The \sml{instruction} declaration defines a datatype and specifies
that this datatype is used to represent the instruction set.  Generally
speaking, the instruction set's designer has complete freedom in how the
datatype is structured, but there are a few simple rules that she should
follow:
\begin{itemize}
  \item If a field represents a register, it should be typed
with the appropriate storage types \sml{$GP}, 
\sml{$FP}, etc.~instead 
of \sml{int}.   MDGen will treat its value in the correct manner; for
example, during assembly emission a field declared type \sml{int} is
printed as an integer, while a field declared type \sml{$GP} is displayed
as a general purpose register.
  \item MDGen recognizes the following special 
types: \sml{label}, \sml{labexp}, \sml{region}, and \sml{cellset}.
\end{itemize}

\subsection{Specifying Instruction Semantics}
   MLRISC performs all optimizations at 
the granulariy of individual instructions,
specialized to the architecture at hand.  Many
optimizations are possible only if the ``semantics'' of the 
instructions set to are properly specified.  MDGen contains a 
\emph{register transfer language} (RTL) sub-language that let us to describe
instruction semantics in a modular and succinct manner.  
 
   The semantics of this RTL sub-language has been borrowed heavily from
Norman Ramsey's and Jack Davidson's Lambda RTL.  There
are a few main differences, however: 
\begin{itemize}
  \item The syntax of our RTL language 
        is closer to that of ML than Lambda RTL.  
  \item Our RTL language, like that of MDGen, is tied closely to MLRISC.
\end{itemize}

\subsection{How to Run the Tool}

\subsection{Machine Description}
Here are some machine descriptions in varing degree of completion.

\begin{itemize}
 \item \mlrischref{sparc/sparc.md}{Sparc} 
 \item \mlrischref{hppa/hppa.md}{Hppa} 
 \item \mlrischref{alpha/alpha.md}{Alpha}
 \item \mlrischref{ppc/ppc.md}{PowerPC} 
 \item \mlrischref{X86/X86.md}{X86} 
\end{itemize}

\subsection{ Syntax Highlighting Macros }

\begin{itemize}
  \item \href{md.vim}{For vim 5.3}
\end{itemize}

\section{Garbage Collection Safety}
\subsection{Motivation}
   High level languages such as SML make use of garbage collectors
to reclaim unused storage at runtime.   For generality, I assume that
a precise, compacting garbage collector is used.  In general, 
low-level optimizations that reorder instructions 
pass \newdef{gc safepoints}, when applied naively, 
are not safe.  In general, two general classes of safety issues can be identified:
\begin{description}
 \item[derived values] 
A derived value $x$ is a value that are
dependent on the addresses of one of more heap allocated objects
$a1,a2,a3,\ldots$ and/or the recent branch history.
When these allocated objects $a1,a2,a3,\ldots$
are moved by the garbage collector, $x$
has to be adjusted accordingly.  

For example, inductive variable elimination may transformed an array
indexing into a running pointer to the middle of an array object.
Such running pointer is a derived value and is dependent on the 
starting address of the array. 

The main difficulty in handling a derived value $x$ 
during garbage collection is that sometimes it is impossible or 
counter-productive to recompute from $a1,a2,a3,\ldots$.
For example, when the recent branch history is unknown, or when the
precise relationship between $x$ and $a1,a2,a3,\ldots$ cannot
be inferred from context.  
We call these \newdef{unrecoverable} derived values.  
  \item[incomplete allocation]
   If heap allocation is performed inlined, then code motion may 
render some allocation incomplete at a gc safepoint.  In general, incomplete
allocation has to be completed, or rolled backed and then reexecuted
after garbage collection, when the source language semantics allow it.
\end{description}

Typically, two gc safepoints cannot be separated by an unbounded
number of allocations, which implies that in general, optimizations that move
instructions between basic blocks are unsafe when naively applied,
which greatly limits the class of optimizations in such an environment
to trivial basic block level optimizations. 
framework is a necessity.


\subsection{Safety Framework}
  MLRISC contains a gc-safety framework 
for performing aggressive machine level optimizations, including SSA-based
scalar optimizations, global instruction scheduling, and global
register allocation.  Unlike previous work in this area, phases that
perform optimizations and phases that maintain and update 
garbage collection information are completely separate, and the optimizer
is constructed in a fully modular manner.  In particular,
gc-types and safety constraints 
are \emph{parameterizable} 
by the source language semantics, the object representation, 
and the target architectures.  

This framework has the following overall structure:
\begin{description}
\item[Garbage collection invariants annotation]
The front-end client is responsible for annotating each 
value in the program with a \newdef{gc type}, which is 
used to specify the abstract object representation, 
and the constraints on code motion that may be applied to such a value.
The front-end uses an architecture independent \href{mltree.html}{RTL} 
language for representing the program, thus this annotation 
phase is portable between target architectures. 
\item[GC constraints propagation]
    After instruction selection, gc constraint are propagated throughout
the machine level program representation.  Again, for portability, gc typing
rules are specified in terms of the \href{mltree.html}{ RTL } of
the machine instructions.  In this phase, unsafe code motions which
exposes unrecoverable derived values to gc safepoints are automatically 
identified.   (Pseudo) control dependence and anti-control dependence 
constraints are then added the  program representation to prohibit all
gc-unsafe code motions.
\item[Machine level optimizations]
    After constraints propagation, traditional 
machine level optimizations such as
SSA optimizations and/or global scheduling are applied, without regard
to gc information.  This is safe because 
all gc safety constraints have been translated into the appropriate 
code motion constraints. 
\item[GC type propagation and gc code generation]
    GC type inference is performed when all optimizations
have been performed.  GC safepoints are then
identified and the root sets are determined.  In addition, compensation
code are inserted at gc points for repairing recoverable derived values.
\end{description}
\subsection{Concurrency Safety}
 In the presence of \newdef{concurrency}, i.e. multiple threads
of control that communicate via a shared heap, the above framework
will have to slightly extended.  As in before, we assume that
context switching can only occur at well-defined 
\emph{safepoints}.
The crucial aspect is that values that are live at safepoints must be
classified as \newdef{local} or \newdef{global}.
Local values are only observable from
the local thread, while global values are potentially observable and mutable
from other threads.  The invariants to maintain are as follows:
\begin{itemize}
 \item Only local and recoverable derived values may be live at a safepoint,  
 \item Only local and recoverable allocation may be incomplete at a safepoint
\end{itemize}

\section{System Integration}
  In a heavily parameterized system like this, one very quickly ends up
  with a large number of modules and dependencies making it very
  easy to mix things up in the wrong way.  
  \image{module dependencies}{pictures/png/sharing1.png}{align=center} 
  \br{clear=left} 
   For example, MLRisc is parameterised over pseudo-ops,
  constants, and regions. An instruction set must be parameterized
  over constants so that instructions that carry immediate operands
  can also carry these abstract constants. Instructions must also be
  parameterized over regions so that memory operations can be
  appropriately annotated. Finally, the flowgraph module must be
  parameterized over instructions it carries in basic blocks and
  pseudo-ops that describe data layout and alignment constraints.

  \image{sharing constraints}{pictures/png/sharing2.png}{align=right}
  \br{clear=left}
  In integrating a system that involves these modules, it must be the
  case that they were created with the same base modules. That is to
  say the pseudo-ops in flowgraphs must be the same abstraction that
  was used to define the MLRisc intermediate
  representation. Alternatively, we want 
  \begin{color}{#ff0000}sharing constraints\end{color} 
  that assert that identity of modules used to
  specialize other modules. In Standard ML, this is a complete
  non-issue. A single line that says exactly that is all that is
  needed to maintain consistency, and the module system does the rest
  to ensure that the final system is built correctly.

  \image{Back end optimizations}{pictures/png/sharing3.png}{align=left}
  \br{clear=right}
  In certain cases one wants to write a specific module for a
  particular architecture. For instance it may be desirable to collapse
  trap barriers on the DEC Alpha where it is legal to do so. The
  INSTRUCTIONS interface is abstract with no built-in knowledge of 
  trap barriers as not all architectures have them.
  Further the DEC Alpha has fairly unique trap barrier semantics,
  that one may want to write an optimization module specific and
  dedicated to the Alpha instruction set and architecture, and forget
  about writing anything generic. In this case, the Standard ML module
  system allows one to say that a specific abstraction actually is or
  matches a more detailed interface. That is to say the INSTRUCTION
  interface is really the DEC Alpha instruction set.

\section{Optimizations}

  MLRISC assumes that all high level optimizations (target
  independent) have already been performed. This includes things like
  inlining, array dependence analysis, and array bounds check
  elimination.  The target dependent optimizations that remain include
  register allocation, scheduling and traditional optimizations to
  support scheduling. 

\subsection{Register allocation}
  
  MLRISC includes a state-of-the-art graph-coloring based register
  allocator that has an aggressive algorithm for copy-propagation. The
  latter guarantees to eliminate copy instructions without introducing
  spills.

   Spills in the register allocator are under the control of the
client via call-backs to the front end. Where to spill registers and
the associated information that must be maintained is client specific
and varies with the compiler. 

\subsection{Scheduling for Superscalar Architectures}
  Several algorithms for acyclic global scheduling are provided. These
include:

  \begin{itemize}
    \item Superblock,
    \item a variant of Bernstein/Rodeh, and
    \item Percolation based scheduling.
  \end{itemize}

These algorithms tend to be quite complex and require a large number
of support data structures and analysis. These include data structures
such as:

  \begin{itemize}
    \item dominator/post dominator trees,
    \item loop nesting tree,
    \item control dependency graphs, and 
    \item data dependency graphs.
 \end{itemize}

Support analysis and optimization include:

  \begin{itemize}
    \item constant propagation,
    \item global value numbering,
    \item global code motion, and
    \item loop invariant hoisting.
  \end{itemize}

\subsection{VLIW Compilation}
  MLRISC also contains a framework for the compilation of 
predicated VLIW architectures.
Currently, the following algorithms have been implemented.
  \begin{itemize} 
    \item hyperblock formation
    \item hyperblock scheduling
    \item modulo scheduling
  \end{itemize}

\section{Graphical Interface}
  All the major data structures and intermediate program states can be
  viewed graphically using 
    \externhref{http://www.Informatik.Uni-Bremen.DE/~davinci/}{\begin{color}{red}daVinci\end{color}} and
    \externhref{http://www.cs.uni-sb.de/RW/users/sander/html/gsvcg1.html}{\begin{color}{red}vcg\end{color}}
  The following screen dumps are intended to represent the range of
  possibilities. Graphical tools like these are an indispensible
  debugging aid. Each of the dumps below were taken when generating
  code for the \begin{color}{red}mandelbrot\end{color} on the HPPA
  architecture. It will be necessary to make netscape fill the size of
  the screen to view these easily. Even though some of these graphs
  look quite complex, daVinci has several \emph{navigational} modes
  that allow walking to successors, or predecessors, or navigating
  through a scaled down map of the graph. The navigational view is
shown as another window, and the view into the graph that is being
displayed is usually outlined in \begin{color}{blue}blue\end{color}.

  \begin{description}
   \item[\href{graphics/mandelbrot-opt.gif}{Control Flowgraph after Optimization:}] Each basic block is shown with its dynamic profile and
    code before and after a specific optimization. This view
    saves having to pour through pages of assembly code listings -- 
    a tedious and frustrating activity.
   \item[\href{graphics/mandelbrot-ssa.gif}{SSA form:}]
     The generated flow graph is converted to SSA form which
makes many code improvement optimizations easy and efficient.
   \item[\href{graphics/mandelbrot-ddg.gif}{Data Dependency Graph}]
         A graphical view of the data dependency graph and the various
kinds of dependencies decorating the edges, provides a useful clue to
why instructions got rearranged the way they did. The navigational
view helps to control the complexity in the display.
  \end{description}

\section{Line Counts}

  \begin{Table}{|l|r|r|}{align=right} \hline
                                               & SML/NJ & MLRISC \\ \hline
      \begin{color}{#00aa00}Generic\end{color} & 3,023 & 6,814 \\
      \begin{color}{#00aa00}Hppa\end{color}    &  725  & 2,285 \\
      \begin{color}{#00aa00}Alpha\end{color}   &  614  & 2,316 \\ \hline
     TOTAL & 4,362 & 11,415 \\ \hline
  \end{Table} 
  The table shows the number of lines involved in a basic MLRISC code
  generator for SML/NJ that only does graph coloring register
  allocation. The SML/NJ column shows the number of lines specific to
  SML/N and the MLRISC column shows the number of lines specific to
  MLRISC. The \begin{color}{#00aa00}Generic\end{color} shows the
  number of lines that are target independent for both SML/NJ and
  MLRISC. The \begin{color}{#00aa00}Hppa\end{color} and 
  \begin{color}{#00aa00}Alpha\end{color} shows the number of lines that are
  target dependent for both the HP Hppa and DEC Alpha targets.

  The bulk of the \sml{3,023} generic to SML/NJ is involved in the
  generation of MLRisc trees. Once this is done the incremental cost
  of adding a target is between \sml{600} to \sml{700} lines.

  The MLRISC column shows that the bulk of MLRISC is quite generic and
a client is saved from writing \sml{11,415} lines of code.

  \begin{Table}{|l|r|r|}{align=left} \hline
                & SML/NJ & MLRISC \\ \hline
   \begin{color}{#00aa00}Generic\end{color} & 121 + 3,023 & 15,686 + 6,814\\
   \begin{color}{#00aa00}Hppa\end{color}    & 32 + 725    & 920 + 2,285 \\
   \begin{color}{#00aa00}Alpha\end{color}   & 614         & 2,316 \\ \hline
    TOTAL & 153 + 4,362 & 16,606 + 11,415 \\ \hline
  \end{Table}
  If one were to include the preliminary numbers for global acyclic
  scheduling in the above table, we find that the incremental cost
  required by the client is quite small -- approximately \sml{153}
  lines of which \sml{121} are generic. However, the scheduling
infra structure is quite large, a lot of it being quite generic. 

\br{left=clear}

\section{Systems Using MLRISC}
Currently these are the systems that are known to be using MLRISC.
\begin{itemize}
\item \externhref{http://cm.bell-labs.com/cm/cs/what/smlnj/index.html}{SML/NJ},
a Standard ML compiler.  
\item \externhref{http://www.dcs.gla.ac.uk/~reig/c--/index.html}{C--},
a portable assembly language.
\item \externhref{http://www.cs.bu.edu/groups/church/}{The Church Project}:
compilation with flow types.
\item \externhref{http://compiler.kaist.ac.kr/projects/lgic}{The LGIC Project}:
a compiler for the CHILL language, targeting PowerPC.
\item \externhref{http://www.cs.bell-labs.com/who/jhr/moby/index.html}{The Moby Language}
\end{itemize}

\begin{small}
Please send additions to \href{mailto:leunga@cs.nyu.edu}{Allen Leung} 
\end{small}

\section{Future Work}
\subsection{Short Term}

\begin{description}    
\item[Detailed user manual:]
    A detailed user manual describing the interfaces, algorithms, 
    and examples on how to put together code generators.
\item[Support for GC:]
      There is a strong interaction
     with support for GC and global code motion. MLRISC aims at
     providing a generic framework for code generators, and finding
     the right level of information to support GC and global code
     motion is an issue. I think we have several solutions to address
     this that need more evaluation.
\item[Other architectures:] There is the need to port
     to other architectures like the MIPS, and the IA-64. 
\end{description}
\hr
\subsection{Long Term}
\begin{description}
 \item[Predicated VLIW compilation:] Currently, the framework
for predicated VLIW architectures compilation
is incomplete, and contain only one back end (C6)
\item[Other compilers:] I would really like to see some
major compiler effort bootstrapped with an MLRISC backend.
\item[Verification] It is extremely difficult to
debug errors in modules that perform aggressive code
reorganizations. Ideas from formal methods such as typed assembly
language (TAL) or Proof Carrying Code (PCC) are worth investigating.
\end{description}


\majorsection{System}
\section{Architecture of MLRISC}

\subsection{Core Components}

  The core components of MLRISC allow the client to quickly construct 
an backend for various architectures.  These components include:
\begin{itemize}
  \item The \href{mltree.html}{MLTREE} language, 
       which is a RTL-like intermediate language
       that is used by the client
       to communicate to the MLRISC system.  A client is
       responsible for writing the module that generates MLTREE from
       the source program representation.
  \item \href{instrsel.html}{Instruction selection modules}, 
      which generates target machine 
       instructions from MLTREE.
  \item The \href{ra.html}{Register Allocator},
       which performs register allocation.
  \item \href{asm.html}{Assemblers}, which emits assembly code.
\end{itemize}

For systems that require direct machine code generation, the following
modules are included:
\begin{itemize}
  \item \href{span-dep.html}{Span dependency resolution} 
       modules, which compute addresses    
       from symbolic addresses,
       fill delay slots, and expand instructions that are 
       \newdef{span dependent}
  \item \href{mc.html}{Machine code emitters}, 
        which emit executable machine code into a binary stream.
\end{itemize}

\subsection{Optimization Modules}

In addition, MLRISC has been enhanced to support various types of
machine level optimizations.  These include:

\begin{itemize}
  \item Core optimizations, which includes
       various types of control flow transformation, 
       and architectural specific peephole optimizations. 
  \item SSA based scalar optimizations
  \item ILP optimizations for superscalars
  \item ILP optimizations for VLIW/EPIC architectures
  \item GC safety analysis
\end{itemize}

\subsection{Basic Concepts}

  Basic concepts in MLRISC are: 
\begin{itemize}
    \item \href{instructions.html}{Instructions} --
    the instruction set of the target architecture.
    \item \href{cells.html}{Cells} -- which describes registers,
memory and other mutable resources in the machine.
    \item \href{regions.html}{Regions} -- a client defined
   abstract type used to represent aliasing information available from
the front-end.
    \item \href{constants.html}{Constants} -- a client defined
   place holder used to represent constants whose values are unknown 
   in the front-end. 
    \item \href{pseudo-ops.html}{Pseudo Ops} -- a client defined
      
    \item \href{annotations.html}{Annotations} -- this is
   a generic mechinism for propagating information in the MLRISC sstem.
   The client may attach arbitrary annotation of various granularity 
   to MLRISC's program representation,
   which can then be propagated to later phases.
   These can be information related to profiling frequency, dependence, 
   comments, and/or types.
   The same mechanism is also used to propagate 
   analysis information one optimization phase to 
   another.
    \item \href{streams.html}{Instruction Streams} -- an abstraction
   for describing a stream of instructions.  Instruction streams are
   used to connect modules such as instruction selection,  assembler, 
   machine code emitter, and 
   control flow graph builder.
   \item \href{regmap.html}{Regmap} -- a mapping between registers
     names.  MLRISC register allocators represent the result of register
   allocation as a regmap.
   \item \href{labels.html}{Labels} -- a type representing
symbolic labels.
   \item \href{labelexp.html}{Label Expressions} -- a type representing
     constant expressions
    involving symbolic labels.
\end{itemize}

\subsection{How Things Are Fit Together}

  MLRISC uses two different program representations, clusters and MLRISC IR.
\begin{itemize}
  \item \href{cluster.html}{Cluster} is light-weight representation
that is used when only the most basic optimizations are required.
  \item \href{mlrisc-ir.html}{MLRISC IR} is more heavy-weight
   representation that is built from the 
    \href{graphs.html}{MLRISC graph library} and the
    \href{compiler-graphs.html}{MLRISC compiler graph library}.
   MLRISC IR allows more complex transformations and analysis of the
   program graph.
\end{itemize}
Conversion modules between the two representations are provided.

In general MLRISC optimization phases are transformations applied on one
of these representations.  Optimizations may be chained together to form
a compiler backend.  For example, a minimal backend consists of
\begin{itemize}
  \item the instruction selection module, which translates 
\href{mltree.html}{MLTree} into target instructions,
  \item the flowgraph builder, which conversts a stream of target instructions
   into a cluster,
  \item the register allocator, which performs register allocation, and
  \item the assembly code emitter, which generates assembly output
\end{itemize}

\section{The MLTREE Language}

\newdef{MLTree} is the 
register transfer language used in the MLRISC system.
It serves two important purposes:
\image{MLTree}{pictures/png/mlrisc-ir.png}{align=right}
\begin{enumerate}
\item As an intermediate representation for a compiler front-end 
  to talk to the MLRISC system,
\item As specifications for instruction semantics
\end{enumerate}
The latter is needed for optimizations which require precise knowledge of such;
for example, algebraic simplification and constant folding.

MLTree is a low-level \newdef{typed} language: 
all operations are typed by its width or precision.  
Operations on floating point, integer, and condition code 
are also segregated, to prevent accidental misuse. 
MLTree is also \emph{tree-oriented} so that it is possible to write efficient
MLTree transformation routines that uses SML pattern matching.

Here are a few examples of MLTree statements.
\begin{SML}
   MV(32,t,
      ADDT(32,
        MULT(32,REG(32,b),REG(32,b)),
        MULT(32,
          MULT(32,LI(4),REG(32,a)),REG(32,c))))
\end{SML}
computes \sml{t := b*b + 4*a*c}, all in 32-bit precision and overflow
trap enabled; while
\begin{SML}
   MV(32,t,
      ADD(32,
        CVTI2I(32,SIGN_EXTEND,8,
          LOAD(8,
            ADD(32,REG(32,a),REG(32,i))))))
\end{SML}
loads the byte in address \sml{a+i} and sign extend it to a 32-bit
value. 

The statement
\begin{SML}
   IF([],CMP(64,GE,REG(64,a),LI 0),
         MV(64, t, REG(64, a)),
         MV(64, t, NEG(64, REG(64, a)))
     )
\end{SML}
in more traditional form means:
\begin{verbatim}
   if a >= 0 then 
      t := a
   else
      t := -a
\end{verbatim} 
This example can be also expressed in a few different ways: 
\begin{enumerate}
   \item With the conditional move construct described in 
Section~\ref{sec:cond-move}:
     \begin{SML}
    MV(64, t, 
       COND(CMP(64, GE, REG(64, a)), 
            REG(64, a), 
            NEG(64, REG(64, a))))
     \end{SML}
  \item With explicit branching using the conditional branch
construct \verb|BCC|:
    \begin{SML}
     MV(64, t, REG(64, a));
     BCC([], CMP(64, GE, REG(64, a)), L1);
     MV(64, t, NEG(64, REG(64, a)));
     DEFINE L1;
    \end{SML}
\end{enumerate}
\subsection{The Definitions}

MLTree is defined in the signature \mlrischref{mltree/mltree.sig}{\sml{MLTREE}}
and the functor \mlrischref{mltree/mltree.sml}{\sml{MLTreeF}}

The functor \sml{MLTreeF} is parameterized in terms of
the label expression type, the client supplied region datatype,
the instruction stream type, and the client defined MLTree extensions.
\begin{SML}
  functor MLTreeF
    (structure LabelExp : \href{labelexp.html}{LABELEXP}
     structure Region : \href{regions.html}{REGION}
     structure Stream : \href{streams.html}{INSTRUCTION_STREAM}
     structure Extension : \mlrischref{mltree/mltree-extension.sig}{MLTREE_EXTENSION}
    ) : MLTREE
\end{SML}

\subsubsection{Basic Types}

  The basic types in MLTree are statements (\newtype{stm})
integer expressions (\newtype{rexp}), 
floating point expression (\newtype{fexp}), 
and conditional expressions (\newtype{ccexp}). 
Statements are evaluated for their effects,
while expressions are evaluated for their value. (Some expressions
could also have trapping effects.  The semantics of traps are unspecified.)
These types are parameterized by an extension
type, which we can use to extend the set of MLTree 
operators.  How this is used is described in Section~\ref{sec:mltree-extension}.

References to registers are represented internally as integers, and are denoted
as the type \sml{reg}. In addition, we use the types \sml{src} and \sml{dst}
as abbreviations for source and destination registers.
\begin{SML}
   type reg = int
   type src = reg
   type dst = reg
\end{SML}

All operators on MLTree are \emph{typed}
by the number of bits that 
they work on.  For example, 32-bit addition between \sml{a} and \sml{b}
is written as \sml{ADD(32,a,b)}, while 64-bit addition between the same
is written as \sml{ADD(64,a,b)}.  Floating point operations are
denoted in the same manner.  For example, IEEE single-precision floating
point add is written as \sml{FADD(32,a,b)}, while the same in
double-precision is written as \sml{FADD(64,a,b)} 

Note that these types are low level.  Higher level distinctions such
as signed and unsigned integer value, are not distinguished by the type.  
Instead, operators are usually partitioned into signed and unsigned versions, 
and it is legal (and often useful!) to mix signed and unsigned operators in
an expression.

Currently, we don't provide a direct way to specify non-IEEE floating point 
together with
IEEE floating point arithmetic.  If this distinction is needed then
it can be encoded using the extension mechanism described
in Section~\ref{sec:mltree-extension}.

We use the types \sml{ty} and \sml{fty} to stand for the number of
bits in integer and floating point operations.  
\begin{SML}
  type ty  = int
  type fty = int
\end{SML}

\subsubsection{The Basis}
The signature \mlrischref{mltree/mltree-basis.sig}{MLTREE\_BASIS}
defines the basic helper types used in the MLTREE signature.  
\begin{SML}
signature MLTREE_BASIS =
sig
 
  datatype cond = LT | LTU | LE | LEU | EQ | NE | GE | GEU | GT | GTU 

  datatype fcond = 
     ? | !<=> | == | ?= | !<> | !?>= | < | ?< | !>= | !?> |
     <= | ?<= | !> | !?<= | > | ?> | !<= | !?< | >= | ?>= |
     !< | !?= | <> | != | !? | <=> | ?<>

  datatype ext = SIGN_EXTEND | ZERO_EXTEND

  datatype rounding_mode = TO_NEAREST | TO_NEGINF | TO_POSINF | TO_ZERO

  type ty = int
  type fty = int

end
\end{SML}

The most important of these are the 
types \newtype{cond} and \newtype{fcond}, which represent the set of integer
and floating point comparisions.  These types can be combined with
the comparison constructors \verb|CMP| and \verb|FCMP| to form
integer and floating point comparisions.
\begin{Table}{|c|c|}{align=left} \hline
   Operator & Comparison \\ \hline
    \sml{LT}     & Signed less than \\
    \sml{LTU}    & Unsigned less than \\
    \sml{LE}     & Signed less than or equal \\
    \sml{LEU}    & Unsigned less than or equal \\
    \sml{EQ}     & Equal \\
    \sml{NE}     & Not equal \\
    \sml{GE}     & Signed greater than or equal \\
    \sml{GEU}    & Unsigned greater than or equal \\
    \sml{GT}     & Signed greater than \\
    \sml{GTU}    & Unsigned greater than \\
\hline
\end{Table}

Floating point comparisons can be ``decoded'' as follows.
In IEEE floating point, there are four different basic comparisons 
tests that we can performed given two numbers $a$ and $y$:
\begin{description}
   \item[$a < b$] Is $a$ less than $b$?
   \item[$a = b$] Is $a$ equal to $b$?
   \item[$a > b$] Is $a$ greater than to $b$?
   \item[$a ? b$] Are $a$ and $b$ unordered (incomparable)?
\end{description}
Comparisons can be joined together.  For example, 
given two double-precision floating point expressions $a$ and $b$,
the expression \verb|FCMP(64,<=>,a,b)| 
asks whether $a$ is less than, equal to or greater than $b$, i.e.~whether
$a$ and $b$ are comparable.  
The special symbol \verb|!| negates
the meaning the of comparison.    For example, \verb|FCMP(64,!>=,a,b)| 
means testing whether $a$ is less than or incomparable with $b$. 

\subsection{Integer Expressions}

A reference to the $i$th 
integer register with an $n$-bit value is written 
as \sml{REG(}$n$,$i$\sml{)}.  The operators \sml{LI}, \sml{LI32},
and \sml{LABEL}, \sml{CONST} are used to represent constant expressions 
of various forms.  The sizes of these constants are inferred from context.
\begin{SML}  
  REG   : ty * reg -> rexp
  LI    : int -> rexp
  LI32  : Word32.word -> rexp
  LABEL : LabelExp.labexp -> rexp
  CONST : Constant.const -> rexp
\end{SML}

The following figure lists all the basic integer operators and their
intuitive meanings.  All operators except \sml{NOTB, NEG, NEGT} are binary 
and have the type
\begin{SML}
  ty * rexp * rexp -> rexp
\end{SML}
The operators \sml{NOTB, NEG, NEGT} have the type
\begin{SML}
  ty * rexp -> rexp
\end{SML}

\begin{tabular}{|l|l|} \hline
   \sml{ADD} & Twos complement addition \\
  \sml{NEG}      & negation \\
  \sml{SUB}      & Twos complement subtraction \\
  \sml{MULS}     & Signed multiplication \\
  \sml{DIVS}     & Signed division, round to zero (nontrapping) \\
  \sml{QUOTS}    & Signed division, round to negative infinity (nontrapping) \\
  \sml{REMS}     & Signed remainder (???) \\
  \sml{MULU}     & Unsigned multiplication \\
  \sml{DIVU}     & Unsigned division \\
  \sml{REMU}     & Unsigned remainder \\
  \sml{NEGT}      & signed negation, trap on overflow \\
  \sml{ADDT}     & Signed addition, trap on overflow \\
  \sml{SUBT}     & Signed subtraction, trap on overflow \\
  \sml{MULT}     & Signed multiplication, trap on overflow \\
  \sml{DIVT}     & Signed division, round to zero,
   trap on overflow or division by zero \\
  \sml{QUOTT}    & Signed division, round to negative infinity, trap on overflow or division by zero \\
  \sml{REMT}     & Signed remainder, trap on division by zero \\
  \sml{ANDB}     & bitwise and \\
  \sml{ORB}      & bitwise or \\
  \sml{XORB}     & bitwise exclusive or \\
  \sml{NOTB}     & ones complement \\
  \sml{SRA}      & arithmetic right shift \\
  \sml{SRL}      & logical right shift \\
  \sml{SLL}      & logical left shift \\
\hline\end{tabular}

\subsubsection{Sign and Zero Extension}
Sign extension and zero extension are written using the operator
\sml{CVTI2I}. \sml{CVTI2I(}$m$,\sml{SIGN_EXTEND},$n$,$e$\sml{)} 
sign extends the $n$-bit value $e$ to an $m$-bit value, i.e. the
$n-1$th bit is of $e$ is treated as the sign bit.  Similarly,
\sml{CVTI2I(}$m$,\sml{ZERO_EXTEND},$n$,$e$\sml{)} 
zero extends an $n$-bit value to an $m$-bit
value.  If $m \le n$, then 
\sml{CVTI2I(}$m$,\sml{SIGN_EXTEND},$n$,$e$\sml{)} = 
\sml{CVTI2I}($m$,\sml{ZERO_EXTEND},$n$,$e$\sml{)}.

\begin{SML}
    datatype ext = SIGN_EXTEND | ZERO_EXTEND
    CVTI2I : ty * ext * ty * rexp -> rexp 
\end{SML}

\subsubsection{Conditional Move} \label{sec:cond-move}
Most new superscalar architectures incorporate conditional move 
instructions in their ISAs.  
Modern VLIW architectures also directly support full predication.  
Since branching (especially with data dependent branches) can
introduce extra latencies in highly pipelined architectures,
condtional moves should be used in place of short branch sequences. 
MLTree provide a conditional move instruction \sml{COND},
to make it possible to directly express conditional moves without using
branches. 
\begin{SML}
   COND : ty * ccexp * rexp * rexp -> rexp
\end{SML}

Semantically, \sml{COND(}\emph{ty},\emph{cc},$a$,$b$\sml{)} means to evaluate
\emph{cc}, and if \emph{cc} evaluates to true then the value of the entire expression is
$a$; otherwise the value is $b$.  Note that $a$ and $b$ are allowed to be
\emph{eagerly}
evaluated.  In fact, we are allowed to evaluate to \emph{both}
branches, one branch, or neither~\footnote{When possible.}. 

Various idioms of the \sml{COND} form are useful for expressing common
constructs in many programming languages.  For example, MLTree does not
provide a primitive construct for converting an integer value \sml{x} to a
boolean value (0 or 1).  But using \sml{COND}, this is expressible as
\sml{COND(32,CMP(32,NE,x,LI 0),LI 1,LI 0)}.  SML/NJ represents
the boolean values true and false as machine integers 3 and 1 respectively.
To convert a boolean condition $e$ into an ML boolean value, we can use
\begin{SML}
   COND(32,e,LI 3,LI 1)
\end{SML}

Common C idioms can be easily mapped into the \sml{COND} form. For example,
\begin{itemize}
  \item \verb|if (e1) x = y| translates into
  \sml{MV(32,x,COND(32,e1,REG(32,y),REG(32,x)))}
  \item
   \begin{verbatim}
     x = e1; 
     if (e2) x = y
   \end{verbatim}
    translates into 
  \sml{MV(32,x,COND(32,e2,REG(32,y),e1))}
  \item \verb|x = e1 == e2| translates into
  \sml{MV(32,x,COND(32,CMP(32,EQ,e1,e2),LI 1,LI 0)}
  \item \verb|x = ! e| translates into
   \sml{MV(32,x,COND(32,CMP(32,NE,e,LI 0),LI 1,LI 0)}
  \item \verb|x = e ? y : z| translates into
   \sml{MV(32,x,COND(32,e,REG(32,y),REG(32,z)))}, and
  \item \verb|x = y < z ? y : z| translates into
   \begin{alltt}
     MV(32,x,
         COND(32,
            CMP(32,LT,REG(32,y),REG(32,z)),
               REG(32,y),REG(32,z)))
   \end{alltt} 
\end{itemize}

In general, the \sml{COND} form should be used in place of MLTree's branching
constructs whenever possible, since the former is usually highly 
optimized in various MLRISC backends. 

\subsubsection{Integer Loads}

Integer loads are written using the constructor \verb|LOAD|.
\begin{SML}
   LOAD  : ty * rexp * Region.region -> rexp
\end{SML}
The client is required to specify a \href{regions.html}{region} that
serves as aliasing information for the load.  

\subsubsection{Miscellaneous Integer Operators}

An expression of the \sml{LET}($s$,$e$) evaluates the statement $s$ for
its effect, and then return the value of expression $e$.
\begin{SML}
  LET  : stm * rexp -> rexp
\end{SML}
Since the order of evaluation is MLTree operators are 
\emph{unspecified}
the use of this operator should be severely restricted to only 
\emph{side-effect}-free forms.

\subsection{Floating Point Expressions}

 Floating registers are referenced using the term \sml{FREG}.  The
$i$th floating point register with type $n$ is written 
as \sml{FREG(}$n$,$i$\sml{)}.
\begin{SML}
   FREG   : fty * src -> fexp
\end{SML}

Built-in floating point operations include addition (\sml{FADD}), 
subtraction (\sml{FSUB}), multiplication (\sml{FMUL}), division 
(\sml{FDIV}), absolute value (\sml{FABS}), negation (\sml{FNEG})
and square root (\sml{FSQRT}).
\begin{SML}
   FADD  : fty * fexp * fexp -> fexp
   FSUB  : fty * fexp * fexp  -> fexp
   FMUL  : fty * fexp * fexp -> fexp
   FDIV  : fty * fexp * fexp -> fexp
   FABS  : fty * fexp -> fexp
   FNEG  : fty * fexp -> fexp
   FSQRT : fty * fexp -> fexp
\end{SML}

A special operator is provided for manipulating signs.
To combine the sign of $a$ with the magnitude of $b$, we can
write \sml{FCOPYSIGN(}$a$,$b$\sml{)}\footnote{What should 
happen if $a$ or $b$ is nan?}.
\begin{SML}
   FCOPYSIGN : fty * fexp * fexp -> fexp
\end{SML}

To convert an $n$-bit signed integer $e$ into an $m$-bit floating point value,
we can write \sml{CVTI2F(}$m$,$n$,$e$\sml{)}\footnote{What happen to unsigned integers?}.
\begin{SML}
   CVTI2F : fty * ty * rexp -> fexp
\end{SML}

Similarly, to convert an $n$-bit floating point value $e$ to an $m$-bit
floating point value, we can write \sml{CVTF2F(}$m$,$n$,$e$\sml{)}\footnote{
What is the rounding semantics?}.
\begin{SML}
   CVTF2F : fty * fty * -> fexp
\end{SML}

\begin{SML}
  datatype rounding_mode = TO_NEAREST | TO_NEGINF | TO_POSINF | TO_ZERO
  CVTF2I : ty * rounding_mode * fty * fexp -> rexp
\end{SML}

\begin{SML}
   FLOAD : fty * rexp * Region.region -> fexp
\end{SML}

\subsection{Condition Expressions}
Unlike languages like C, MLTree makes the distinction between condition 
expressions and integer expressions.  This distinction is necessary for
two purposes:
\begin{itemize}
  \item It clarifies the proper meaning intended in a program, and
  \item It makes to possible for a MLRISC backend to map condition
expressions efficiently onto various machine architectures with different
condition code models.  For example, architectures like the Intel x86, 
Sparc V8, and PowerPC contains dedicated condition code registers, which
are read from and written to by branching and comparison instructions.
On the other hand, architectures such as the Texas Instrument C6, PA RISC,
Sparc V9, and Alpha does not include dedicated condition code registers.
Conditional code registers in these architectures
can be simulated by integer registers.
\end{itemize}


A conditional code register bit can be referenced using the constructors
\sml{CC} and \sml{FCC}.  Note that the \emph{condition} must be specified
together with the condition code register.
\begin{SML}
   CC   : Basis.cond * src -> ccexp 
   FCC  : Basis.fcond * src -> ccexp    
\end{SML}
For example, to test the \verb|Z| bit of the \verb|%psr| register on the
Sparc architecture, we can used \sml{CC(EQ,SparcCells.psr)}.  

The comparison operators \sml{CMP} and \sml{FCMP} performs integer and
floating point tests.  Both of these are \emph{typed} by the precision
in which the test must be performed under.
\begin{SML}
   CMP  : ty * Basis.cond * rexp * rexp -> ccexp  
   FCMP : fty * Basis.fcond * fexp * fexp -> ccexp
\end{SML}

Condition code expressions may be combined with the following
logical connectives, which have the obvious meanings.
\begin{SML}
   TRUE  : ccexp 
   FALSE : ccexp 
   NOT   : ccexp -> ccexp 
   AND   : ccexp * ccexp -> ccexp 
   OR    : ccexp * ccexp -> ccexp 
   XOR   : ccexp * ccexp -> ccexp 
\end{SML}

\subsection{Statements}

Statement forms in MLTree includes assignments, parallel copies,
jumps and condition branches, calls and returns, stores, sequencing,
and annotation.

\subsubsection{Assignments}

Assignments are segregated among the integer, floating point and
conditional code types.  In addition, all assignments are \emph{typed}
by the precision of destination register.

\begin{SML}
   MV   : ty * dst * rexp -> stm
   FMV  : fty * dst * fexp -> stm
   CCMV : dst * ccexp -> stm
\end{SML}  

\subsubsection{Parallel Copies}

Special forms are provided for parallel copies for integer and
floating point registers.  It is important to emphasize that
the semantics is that all assignments are performed in parallel.

\begin{SML}
   COPY  : ty * dst list * src list -> stm
   FCOPY : fty * dst list * src list -> stm
\end{SML}

\subsubsection{Jumps and Conditional Branches}  

Jumps and conditional branches in MLTree take two additional set of
annotations.  The first represents the \newdef{control flow} and is denoted
by the type \sml{controlflow}.  The second represent 
\newdef{control-dependence} and \newdef{anti-control-dependence} 
and is denoted by the type \sml{ctrl}.

\begin{SML}
   type controlflow = Label.label list
   type ctrl = reg list
\end{SML}
Control flow annotation is simply a list of labels, which represents
the set of possible targets of the associated jump.  Control dependence
annotations attached to a branch or jump instruction represents the
new definition of \newdef{pseudo control dependence predicates}.  These
predicates have no associated dynamic semantics; rather they are used
to constraint the set of potential code motion in an optimizer
(more on this later).

The primitive jumps and conditional branch forms are represented
by the constructors \sml{JMP}, \sml{BCC}.
\begin{SML}
   JMP : ctrl * rexp * controlflow  -> stm
   BCC : ctrl * ccexp * Label.label -> stm
\end{SML}

In addition to \sml{JMP} and \sml{BCC}, 
there is a \emph{structured} if/then/else statement.
\begin{SML}
   IF  : ctrl * ccexp * stm * stm -> stm
\end{SML}

Semantically, \sml{IF}($c,x,y,z$) is identical to
\begin{SML}
   BCC(\(c\), \(x\), L1)
   \(z\)
   JMP([], L2)
   DEFINE L1
   \(y\)
   DEFINE L2
\end{SML}
where \verb|L1| and \verb|L2| are new labels, as expected.

Here's an example of how control dependence predicates are used.
Consider the following MLTree statement:
\begin{SML}
   IF([p], CMP(32, NE, REG(32, a), LI 0),
        MV(32, b, PRED(LOAD(32, m, ...)), p),
        MV(32, b, LOAD(32, n, ...)))
\end{SML}
In the first alternative of the \verb|IF|, the \verb|LOAD|
expression is constrainted by the control dependence 
predicate \verb|p| defined in the \verb|IF|,
using the predicate constructor \verb|PRED|.  These states that
the load is \emph{control dependent} on the test of the branch,
and thus it may not be legally hoisted above the branch without
potentially violating the semantics of the program. 
For example,
semantics violation may happen  if the value of \verb|m| and \verb|a|
is corrolated, and whenever \verb|a| = 0, the address in \verb|m| is
not a legal address. 

Note that on architectures with speculative loads, 
the control dependence information can be used to 
guide the transformation of control dependent loads into speculative loads.

Now in constrast, the \verb|LOAD| in the second alternative is not
control dependent on the control dependent predicate \verb|p|, and
thus it is safe and legal to hoist the load above the test, as in
\begin{SML}
   MV(32, b, LOAD(32, n, ...));
   IF([p], CMP(32, NE, REG(32, a), LI 0),
        MV(32, b, PRED(LOAD(32, m, ...)), p),
        SEQ []
     )
\end{SML}
Of course, such transformation is only performed if the optimizer
phases think that it can benefit performance.  Thus the control dependence
information does \emph{not} directly specify any transformations, but it
is rather used to indicate when aggressive code motions are legal and safe.

\subsubsection{Calls and Returns}

Calls and returns in MLTree are specified using the constructors
\verb|CALL| and \verb|RET|, which have the following types.
\begin{SML}
   CALL : rexp * controlflow * mlrisc * mlrisc * 
          ctrl * ctrl * Region.region -> stm
   RET  : ctrl * controlflow -> stm
\end{SML}

The \verb|CALL| form is particularly complex, and require some explanation.
Basically the seven parameters are, in order:
\begin{description}
   \item[address] of the called routine.
   \item[control flow] annotation for this call.  This information 
specifies the potential targets of this call instruction.  Currently
this information is ignored but will be useful for interprocedural   
optimizations in the future.
   \item[definition and use]  These lists specify the list of
potential definition and uses during the execution of the call.
Definitions and uses are represented as the type \newtype{mlrisc} list.
The contructors for this type is:
\begin{SML}
  CCR : ccexp -> mlrisc
  GPR : rexp -> mlrisc
  FPR : fexp -> mlrisc
\end{SML}
   \item[definition of control and anti-control dependence] 
These two lists specifies definitions of control and anti-control dependence.
   \item[region] annotation for the call, which summarizes
the set of potential memory references during execution of the call.
\end{description}

The matching return statement constructor \verb|RET| has two
arguments.  These are:
\begin{description}
  \item[anti-control dependence]  This parameter represents
the set of anti-control dependence predicates defined by the return
statement.
  \item[control flow]  This parameter specifies the set of matching
procedure entry points of this return.  For example, suppose we have
a procedure with entry points \verb|f| and \verb|f'|.  
Then the MLTree statements 
\begin{verbatim}
  f:   ...
       JMP L1
  f':  ...
  L1:  ...
       RET ([], [f, f'])
\end{verbatim}
\noindent can be used to specify that the return is either from
the entries \verb|f| or \verb|f'|.  
\end{description}

\subsubsection{Stores}
Stores to integer and floating points are specified using the
constructors \verb|STORE| and \verb|FSTORE|.   
\begin{SML}
   STORE  : ty * rexp * rexp * Region.region -> stm
   FSTORE : fty * rexp * fexp * Region.region -> stm
\end{SML}

The general form is
\begin{SML}
   STORE(\(width\), \(address\), \(data\), \(region\))
\end{SML}

Stores for condition codes are not provided.
\subsubsection{Miscelleneous Statements}

Other useful statement forms of MLTree are for sequencing (\verb|SEQ|),
defining a local label (\verb|DEFINE|).
\begin{SML}
   SEQ    : stm list -> stm
   DEFINE : Label.label -> stm
\end{SML}
The constructor \sml{DEFINE L} has the same meaning as 
executing the method \sml{defineLabel L} in the 
\href{stream.html}{stream interface}.

\subsection{Annotations}
\href{annotations.html}{Annotations} are used as the generic mechanism for
exchanging information between different phases of the MLRISC system, and
between a compiler front end and the MLRISC back end.
The following constructors can be used to annotate a MLTree term with
an annotation:
\begin{SML}
   MARK : rexp * Annotations.annotation -> rexp
   FMARK : fexp * Annotations.annotation -> fexp
   CCMARK : ccexp * Annotations.annotation -> ccexp 
   ANNOTATION : stm * Annotations.annotation -> stm
\end{SML}

\subsection{Extending the MLTree Language} \label{sec:mltree-extension}

The following constructors are available for extending the MLTree language
with client defined types: 
\begin{SML}
  EXT : sext -> stm
  REXT : ty * rext -> rexp
  FEXT : fty * fext -> fexp
  CCEXT : ty * ccext -> ccexp
\end{SML}

The extension types \sml{sext, rext, fext, ccext} are defined in terms
of the client supplied extension types: 
\begin{SML}
   type sext = (stm, rexp, fexp, ccexp) Extension.sx
   type rext = (stm, rexp, fexp, ccexp) Extension.rx
   type fext = (stm, rexp, fexp, ccexp) Extension.fx
   type ccext = (stm, rexp, fexp, ccexp) Extension.ccx
\end{SML}

Here is an example of how these extension constructors may be used.
Suppose you are in the process of writing a compiler for a 
digital signal processing(\newdef{DSP}) programming
language using the MLRISC framework.  This wonderful language that you
are developing allows the programmer to specify high level looping and
iteration, and aggregation constructs that are common in DSP applications.
Furthermore, since saturated and fixed point arithmetic are common constructs
in DSP applications, the language and consequently the compiler should directly 
support these operators.   For simplicity, we would
like to have a unified intermediate representation that can be used to directly 
represent high level constructs in our language, and low level constructs
that are already present in MLTree.  
Since, MLTree does not directly support these
constructs, it seems that it is not possible to use MLRISC for such a
compiler infrastructure without substantial rewrite of the core components.

This is where the extension mechanism comes in.  
Let us suppose that for illustration that we would like to
implement high level looping and aggregation constructs such as
\begin{verbatim}
   for i := lower bound ... upper bound
       body
   x := sum{i := lower bound ... upper bound} expression
\end{verbatim}
together with saturated arithmetic mentioned above.

Here is a first attempt:
\begin{SML}
structure DSPMLTreeExtension
struct
   structure Basis = MLTreeBasis
   datatype ('s,'r,'f,'c) sx = 
      FOR of Basis.var * 'r * 'r * 's
   and ('s,'r,'f,'c) rx = 
      SUM of Basis.var * 'r * 'r * 'r
    | SADD of 'r * 'r
    | SSUB of 'r * 'r
    | SMUL of 'r * 'r
    | SDIV of 'r * 'r
   type ('s,'r,'f,'c) fx = unit
   type ('s,'r,'f,'c) ccx = unit
end
structure DSPMLTree : MLTreeF
    (structure Extension = DSPMLTreeExtension
     ...
    )
\end{SML}
In the above signature, we have defined two new datatypes \newtype{sx}
and \newtype{rx} that are used for representing the DSP statement
and integer expression extensions.  Integer expression extensions
include the high level sum construct, and the low levels saturated
arithmetic operators.  The recursive type definition is
necessary to ``inject'' these new constructors into the basic MLTree 
definition.

The following is an example of how these new constructors that we have defined can be used.  Suppose the source program in our DSP language is:
\begin{verbatim}
   for i := a ... b
   {  s := sadd(s, table[i]);
   }
\end{verbatim}
\noindent where \verb|sadd| is the saturated add operator.
For simplication, let us also assume that all operations and addresses
are in 32-bits.
Then the translation of the above into our extended DSP-MLTree could be:
\begin{SML}
   EXT(FOR(\(i\), REG(32, \(a\)), REG(32, \(b\)),
           MV(32, \(s\), REXT(32, SADD(REG(32, \(s\)), 
                LOAD(32, 
                    ADD(32, REG(32, \(table\)), 
                        SLL(32, REG(32, \(i\)), LI 2)),
                         \(region\)))))
          ))
\end{SML}

One potential short coming of our DSP extension to MLTree is that
the extension does not allow any further extensions.  This restriction
may be entirely satisfactory if DSP-MLTree is only used in your compiler
applications and no where else.  However, if DSP-MLTree is intended
to be an extension library for MLRISC, then  we must build in the flexibility
for extension.  This can be done in the same way as in the base MLTree
definition, like this: 
\begin{SML}
functor ExtensibleDSPMLTreeExtension
  (Extension : \mlrischref{mltree/mltree-extension.sig}{MLTREE_EXTENSION}) =
struct
   structure Basis = MLTreeBasis
   structure Extension = Extension
   datatype ('s,'r,'f,'c) sx = 
      FOR of Basis.var * 'r * 'r * 's
    | EXT of ('s,'r,'f,'c) Extension.sx 
   and ('s,'r,'f,'c) rx = 
      SUM of Basis.var * 'r * 'r * 'r
    | SADD of 'r * 'r
    | SSUB of 'r * 'r
    | SMUL of 'r * 'r
    | SDIV of 'r * 'r
    | REXT of ('s,'r,'f,'c) Extension.rx
   withtype
        ('s,'r,'f,'c) fx   = ('s,'r,'f,'c) Extension.fx
   and  ('s,'r,'f,'c) ccx  = ('s,'r,'f,'c) Extension.ccx
end
\end{SML}

As in MLTREE, we provide two new extension 
constructors \verb|EXT| and \verb|REXT| in
the definition of \sml{DSP_MLTREE}, which  can 
be used to further enhance the extended MLTREE language.

\subsubsection{Compilation for MLTree Extensions}
Of course, after extending the MLTree language, the client must provide
an \emph{extension} module describing how to compile the extensions.
How this is done is described in Section~\ref{sec:instrsel}.

\subsection{MLTree Utilities}

The \MLRISC{} system contains numerous utilities for working with
MLTree datatypes.  Some of the following utilizes are also useful for clients
use:
\begin{description}
  \item[MLTreeUtils] implements basic hashing, equality and pretty
printing functions,
  \item[MLTreeFold] implements a fold function over the MLTree datatypes,  
  \item[MLTreeRewrite] implements a generic rewriting engine,
  \item[MLTreeSimplify] implements a simplifier that performs algebraic
simplification and constant folding.
\end{description}
\subsubsection{Hashing, Equality, Pretty Printing}

The functor \mlrischref{mltree/mltree-utils.sml}{MLTreeUtils} provides
the basic utilities for hashing an MLTree term, comparing two
MLTree terms for equality and pretty printing.  The hashing and comparision
functions are useful for building hash tables using MLTree datatype as keys.
The signature of the functor is:
\begin{SML}
signature \mlrischref{mltree/mltree-utils.sig}{MLTREE_UTILS} =
sig
   structure T : MLTREE 

   (*
    * Hashing
    *)
   val hashStm   : T.stm -> word
   val hashRexp  : T.rexp -> word
   val hashFexp  : T.fexp -> word
   val hashCCexp : T.ccexp -> word

   (*
    * Equality
    *)
   val eqStm     : T.stm * T.stm -> bool
   val eqRexp    : T.rexp * T.rexp -> bool
   val eqFexp    : T.fexp * T.fexp -> bool
   val eqCCexp   : T.ccexp * T.ccexp -> bool
   val eqMlriscs : T.mlrisc list * T.mlrisc list -> bool

   (*
    * Pretty printing 
    *)
   val show : (string list * string list) -> T.printer

   val stmToString   : T.stm -> string
   val rexpToString  : T.rexp -> string
   val fexpToString  : T.fexp -> string
   val ccexpToString : T.ccexp -> string

end
functor \mlrischref{mltree/mltree-utils.sml}{MLTreeUtils} 
  (structure T : MLTREE
   (* Hashing extensions *)
   val hashSext  : T.hasher -> T.sext -> word
   val hashRext  : T.hasher -> T.rext -> word
   val hashFext  : T.hasher -> T.fext -> word
   val hashCCext : T.hasher -> T.ccext -> word

   (* Equality extensions *)
   val eqSext  : T.equality -> T.sext * T.sext -> bool
   val eqRext  : T.equality -> T.rext * T.rext -> bool
   val eqFext  : T.equality -> T.fext * T.fext -> bool
   val eqCCext : T.equality -> T.ccext * T.ccext -> bool

   (* Pretty printing extensions *)
   val showSext  : T.printer -> T.sext -> string
   val showRext  : T.printer -> T.ty * T.rext -> string
   val showFext  : T.printer -> T.fty * T.fext -> string
   val showCCext : T.printer -> T.ty * T.ccext -> string
  ) : MLTREE_UTILS =
\end{SML} 

The types \sml{hasher}, \sml{equality},
and \sml{printer} represent functions for hashing,
equality and pretty printing.   These are defined as:
\begin{SML} 
   type hasher =
      \{stm    : T.stm -> word,
       rexp   : T.rexp -> word,
       fexp   : T.fexp -> word,
       ccexp  : T.ccexp -> word
      \}    

   type equality =
      \{ stm    : T.stm * T.stm -> bool,
        rexp   : T.rexp * T.rexp -> bool,
        fexp   : T.fexp * T.fexp -> bool,
        ccexp  : T.ccexp * T.ccexp -> bool
      \} 
   type printer =
      \{ stm    : T.stm -> string,
        rexp   : T.rexp -> string,
        fexp   : T.fexp -> string,
        ccexp  : T.ccexp -> string,
        dstReg : T.ty * T.var -> string,
        srcReg : T.ty * T.var -> string
      \}
\end{SML}

For example, to instantiate a \sml{Utils} module for our \sml{DSPMLTree},
we can write:
\begin{SML}
   structure U = MLTreeUtils
     (structure T = DSPMLTree
      fun hashSext \{stm, rexp, fexp, ccexp\} (FOR(i, a, b, s)) =
           Word.fromIntX i + rexp a + rexp b + stm s
      and hashRext \{stm, rexp, fexp, ccexp\} e =
          (case e of
             SUM(i,a,b,c) => Word.fromIntX i + rexp a + rexp b + rexp c
           | SADD(a,b) => rexp a + rexp b
           | SSUB(a,b) => 0w12 + rexp a + rexp b
           | SMUL(a,b) => 0w123 + rexp a + rexp b
           | SDIV(a,b) => 0w1245 + rexp a + rexp b
          )
      fun hashFext _ _ = 0w0
      fun hashCCext _ _ = 0w0
      fun eqSext \{stm, rexp, fexp, ccexp\} 
        (FOR(i, a, b, s), FOR(i', a', b', s')) =
           i=i' andalso rexp(a,a') andalso rexp(b,b') andalso stm(s,s')
      fun eqRext \{stm, rexp, fexp, ccexp\} (e,e') =
       (case (e,e') of
          (SUM(i,a,b,c),SUM(i',a',b',c')) => 
            i=i' andalso rexp(a,a') andalso rexp(b,b') andalso stm(c,c')
        | (SADD(a,b),SADD(a',b')) => rexp(a,a') andalso rexp(b,b')
        | (SSUB(a,b),SSUB(a',b')) => rexp(a,a') andalso rexp(b,b')
        | (SMUL(a,b),SMUL(a',b')) => rexp(a,a') andalso rexp(b,b')
        | (SDIV(a,b),SDIV(a',b')) => rexp(a,a') andalso rexp(b,b')
        | _ => false
       )
      fun eqFext _ _ = true
      fun eqCCext _ _ = true

      fun showSext \{stm, rexp, fexp, ccexp, dstReg, srcReg\}  
            (FOR(i, a, b, s)) =
          "for("^dstReg i^":="^rexp a^".."^rexp b^")"^stm s
      fun ty t = "."^Int.toString t
      fun showRext \{stm, rexp, fexp, ccexp, dstReg, srcReg\} e = 
           (case (t,e) of
             SUM(i,a,b,c) => 
              "sum"^ty t^"("^dstReg i^":="^rexp a^".."^rexp b^")"^rexp c
           | SADD(a,b) => "sadd"^ty t^"("rexp a^","^rexp b^")"
           | SSUB(a,b) => "ssub"^ty t^"("rexp a^","^rexp b^")"
           | SMUL(a,b) => "smul"^ty t^"("rexp a^","^rexp b^")"
           | SDIV(a,b) => "sdiv"^ty t^"("rexp a^","^rexp b^")"
           )
      fun showFext _ _ = ""
      fun showCCext _ _ = ""
     )
\end{SML}

\subsubsection{MLTree Fold}
The functor \mlrischref{mltree/mltree-fold.sml}{MLTreeFold}
provides the basic functionality for implementing various forms of
aggregation function over the MLTree datatypes.  Its signature is
\begin{SML}
signature \mlrischref{mltree/mltree-fold.sig}{MLTREE_FOLD} =
sig
   structure T : MLTREE

   val fold : 'b folder -> 'b folder
end
functor \mlrischref{mltree/mltree-fold.sml}{MLTreeFold}
  (structure T : MLTREE
   (* Extension mechnism *)
   val sext  : 'b T.folder -> T.sext * 'b -> 'b
   val rext  : 'b T.folder -> T.ty * T.rext * 'b -> 'b
   val fext  : 'b T.folder -> T.fty * T.fext * 'b -> 'b
   val ccext : 'b T.folder -> T.ty * T.ccext * 'b -> 'b
  ) : MLTREE_FOLD =
\end{SML}
The type \newtype{folder} is defined as:
\begin{SML}
   type 'b folder =
       \{ stm   : T.stm * 'b -> 'b,
         rexp  : T.rexp * 'b -> 'b,
         fexp  : T.fexp * 'b -> 'b, 
         ccexp : T.ccexp * 'b -> 'b
       \}
\end{SML}


\subsubsection{MLTree Rewriting}

The functor \mlrischref{mltree/mltree-rewrite.sml}{MLTreeRewrite}
implements a generic term rewriting engine which is useful for performing
various transformations on MLTree terms. Its signature is
\begin{SML}
signature \mlrischref{mltree/mltree-rewrite.sig}{MLTREE_REWRITE} =
sig
   structure T : MLTREE

  val rewrite : 
       (* User supplied transformations *)
       \{ rexp  : (T.rexp -> T.rexp) -> (T.rexp -> T.rexp), 
         fexp  : (T.fexp -> T.fexp) -> (T.fexp -> T.fexp),
         ccexp : (T.ccexp -> T.ccexp) -> (T.ccexp -> T.ccexp),
         stm   : (T.stm -> T.stm) -> (T.stm -> T.stm)
       \} -> T.rewriters
end
functor \mlrischref{mltre/mltree-rewrite.sml}{MLTreeRewrite}
  (structure T : MLTREE
   (* Extension *)
   val sext : T.rewriter -> T.sext -> T.sext
   val rext : T.rewriter -> T.rext -> T.rext
   val fext : T.rewriter -> T.fext -> T.fext
   val ccext : T.rewriter -> T.ccext -> T.ccext
  ) : MLTREE_REWRITE =
\end{SML}

The type \newtype{rewriter} is defined in signature
\mlrischref{mltree/mltree.sig}{MLTREE} as:
\begin{SML}
   type rewriter = 
       \{ stm   : T.stm -> T.stm,
         rexp  : T.rexp -> T.rexp,
         fexp  : T.fexp -> T.fexp,
         ccexp : T.ccexp -> T.ccexp
       \} 
\end{SML}
 
\subsubsection{MLTree Simplifier}

The functor \mlrischref{mltree/mltree-simplify.sml}{MLTreeSimplify}
implements algebraic simplification and constant folding for MLTree.
Its signature is:
\begin{SML}
signature \mlrischref{mltree/mltree-simplify.sig}{MLTREE_SIMPLIFIER} =
sig

   structure T : MLTREE

   val simplify  :
       { addressWidth : int } -> T.simplifier
   
end
functor \mlrischref{mltree/mltree-simplify.sml}{MLTreeSimplifier}
  (structure T : MLTREE
   (* Extension *)
   val sext : T.rewriter -> T.sext -> T.sext
   val rext : T.rewriter -> T.rext -> T.rext
   val fext : T.rewriter -> T.fext -> T.fext
   val ccext : T.rewriter -> T.ccext -> T.ccext
  ) : MLTREE_SIMPLIFIER =
\end{SML}

Where type \newdef{simplifier} is defined in signature 
\mlrischref{mltree/mltree.sig}{MLTREE} as:
\begin{SML}
   type simplifier =
       \{ stm   : T.stm -> T.stm,
         rexp  : T.rexp -> T.rexp,
         fexp  : T.fexp -> T.fexp,
         ccexp : T.ccexp -> T.ccexp
       \}
\end{SML}



\section{MLTree Extensions} \label{sec:mltree-extension}
	Pattern matching over the MLTREE intermediate representation
may not be sufficient to provide access to all the registers or
operations provided on a specific architecture. MLTREE extensions is a 
method of extending the MLTREE intermediate language so that it is a
better match for the target architecture.


\subsection{Why Extensions}

	Pattern matching over the MLTREE intermediate representation
may not be sufficient to provide access to all the registers or
operations provided on a specific architecture. MLTREE extensions is a 
method of extending the MLTREE intermediate language so that it is a
better match for the target architecture.

For example there may be special registers to support the
increment-and-test operation on loop indices, or 
support for complex mathematical functions such as
square root, or access to hardware specific registers such as the
current register window pointer on the SPARC architecture. It is not
usually possible to write expression trees that would directly
generate these instructions.
Some complex operations can be generated by performing a peephole
optimization over simpler instructions, however this is not always the 
most convenient or simple thing to do.

\subsection{Cyclic Dependency}

The easiest way to provide extensions is to parameterize the MLTREE
interface with types that extend the various kinds of trees. Thus if
the type \sml{sext} represented statement extensions, we might define
MLTREE statement trees as :
\begin{SML}
  datatype stm
    = ...
    | SEXT of sext * mlrisc list * stm list

  and mlrisc = GPR of rexp | FPR of fexp | CCR of ccexp
\end{SML}
where the constructor \sml{SEXT} applies the extension to a list of
arguments. This approach is unsatisfactory in several ways, for
example, if one wanted to extend MLTREEs with for-loops, then the
following could be generated:
\begin{SML}
  SEXT(FORLOOP, [GPR from, GPR to, GPR step], body)
\end{SML}	
First, the loop arguments have to be wrapped up in \sml{GPR} and there
is little self documentation on the order of elements that are
arguments to the for-loop. It would be better to be able to write
something like:
\begin{SML}
  SEXT(FORLOOP\{from=f, to=t, step=s, body=b\}) 
\end{SML}

Where \sml{f}, \sml{t}, and \sml{s} are \sml{rexp} trees representing
the loop index start, end, and step size;  \sml{b} is a stm list
representing the body of the loop. Unfortunately, there is a cyclic
dependency as MLTREEs are defined in terms of \sml{sext}, and {\tt
sext} is defined in terms of MLTREEs. The usual way to deal with 
cyclic dependencies is to use polymorphic type variables. 

\subsection{MLTREE EXTENSION}

The statement extension type \sml{sext}, is now a type constructor
with arity four, i.e. 
\sml{('s, 'r, 'f, 'c) sx} where \sml{sx} is used instead of {\tt
sext}, and \sml{'s}, \sml{'r}, \sml{'f}, and \sml{'c} represents
MLTREE statement expressions, register expressions, floating point
expressions, and condition code expressions. Thus the for-loop
extension could be declared using something like:
\begin{SML}
  datatype ('s,'r,'f,'c) sx 
    = FORLOOP of \{from: 'r, to: 'r, step: 'r, body: 's\}
\end{SML}
and the MLTREE interface is defined as:
\begin{SML}
  signature MLTREE = sig
    type ('s, 'r, 'f, 'c) sx

    datatype stm =
      = ...
      | SEXT of sext

   withtype sext = (stm, rexp, fexp, cexp) sx
  end
\end{SML}

where \sml{sext} is the user defined statement extension but the
type variables have been instantiated to the final form the the MLTREE 
\sml{stm}, \sml{rexp}, \sml{fexp}, and \sml{cexp} components. 

\subsection{Compilation}

There are dedicated modules that perform pattern matching over MLTREEs 
and emit native instructions, and similar modules must be written for
extensions.  However, the same kinds of choices used in regular MLTREE 
patterns must be repeated for extensions. For example, one may define
an extension for the Intel IA32 of the form:

\begin{SML}
  datatype ('s,'r,'f,'c) sx = PUSHL of 'r | POPL of 'r | ...
\end{SML}

that translate directly to the Intel push and pop instructions; the
operands in each case are either memory locations or registers, but
immediates are allowed in the case of \sml{PUSHL}. Considerable effort 
has been invested into pattern matching the extensive set of
addressing modes for the Intel architecture, and
one would like to reuse this when compiling extensions. The pattern
matching functions are exposed by a set of functions exported from the 
instruction selection module, and provided in the MLTREE
interface. They are: 

\begin{SML}
  struture I : INSTRUCTIONS
  datatype reducer = 
    REDUCER of \{
      reduceRexp    : rexp -> reg,
      reduceFexp    : fexp -> reg,
      reduceCCexp   : ccexp -> reg,
      reduceStm     : stm * an list -> unit,
      operand       : rexp -> I.operand,
      reduceOperand : I.operand -> reg,
      addressOf     : rexp -> I.addressing_mode,
      emit          : I.instr * an list -> unit,
      instrStream   : (I.instr, I.regmap, I.cellset) stream,
      mltreeStream  : (stm, I.regmap, mlrisc list) stream
    \}
\end{SML}

where \sml{I} is the native instruction set. 
\begin{description}
\item[\tt reduceRexp]: reduces an MLTREE \sml{rexp} to a register, and
	similarly for \sml{reduceFexp} and \sml{reduceCCexp}.
\item[\tt reduceStm]: reduces an MLTREE \sml{stm} to a set of instructions
	that implement the set of statements.
\item[\tt operand]: reduced an MLTREE \sml{rexp} into an instruction
operand --- usually an immediate or memory address.
\item[\tt operand]: moves a native operand into a register.
\item[\tt addressOf]: reduces an MLTREE \sml{rexp} into a memory address.
\item[\tt emit]: emits an instruction together with an annotation.
\item[\tt instrStream]: is the native instruction output stream, and
\item[\tt mltreeStream]: is the MLTREE output stream.
\end{description}

Each extension must provide a function \sml{compileSext} that compiles
a statement extension into native instructions. In the
\sml{MLTREE_EXTENSION_COMP} interface we have:
\begin{SML}
  val compileSext: reducer -> {stm: MLTREE.sexp, an:MLTREE.an list} -> unit
\end{SML}

The use of extensions must follow a special structure. 
\begin{enumerate}
 \item A module defining the extension type using a type constructor
of arity four. Let us call this structure \sml{ExtTy} and must match
the \sml{MLTREE_EXTENSION} interface.
 \item The extension module must be used to specialize MLTREEs. 
 \item A module that describes how to compile the extension must be
created, and must match the \sml{MLTREE_EXTENSION_COMP} interace.
This module will typically be functorized over the MLTREE interface.
Let us call the result of applying the functor, \sml{ExtComp}.
 \item The extension compiler must be passed as a parameter to the
instruction selection module that will invoke it whenever an extension 
is seen.
\end{enumerate}


\subsection{Multiple Extensions}

Multiple extensions are handled in a similar fashion, except that the
extension type used to specialize MLTREEs is a tagged union of the
individual extensions. The functor to compile the extension dispatches 
to the compilation modules for the individual extensions.

\subsection{Example}
Suppose you are in the process of writing a compiler for a digital
signal processing(\newdef{DSP}) programming language using the MLRISC
framework.  This wonderful language that you are developing allows the
programmer to specify high level looping and iteration, and
aggregation constructs that are common in DSP applications.
Furthermore, since saturated and fixed point arithmetic are common
constructs in DSP applications, the language and consequently the
compiler should directly support these operators.  For simplicity, we
would like to have a unified intermediate representation that can be
used to directly represent high level constructs in our language, and
low level constructs that are already present in MLTree.  Since,
MLTree does not directly support these constructs, it seems that it is
not possible to use MLRISC for such a compiler infrastructure without
substantial rewrite of the core components.

Let us suppose that for illustration that we would like to
implement high level looping and aggregation constructs such as
\begin{verbatim}
   for i := lower bound ... upper bound
       body
   x := sum{i := lower bound ... upper bound} expression
\end{verbatim}
together with saturated arithmetic mentioned above.

Here is a first attempt:
\begin{SML}
structure DSPMLTreeExtension
struct
   structure Basis = MLTreeBasis
   datatype ('s,'r,'f,'c) sx = 
      FOR of Basis.var * 'r * 'r * 's
   and ('s,'r,'f,'c) rx = 
      SUM of Basis.var * 'r * 'r * 'r
    | SADD of 'r * 'r
    | SSUB of 'r * 'r
    | SMUL of 'r * 'r
    | SDIV of 'r * 'r
   type ('s,'r,'f,'c) fx = unit
   type ('s,'r,'f,'c) ccx = unit
end
structure DSPMLTree : MLTreeF
    (structure Extension = DSPMLTreeExtension
     ...
    )
\end{SML}
In the above signature, we have defined two new datatypes \newtype{sx}
and \newtype{rx} that are used for representing the DSP statement
and integer expression extensions.  Integer expression extensions
include the high level sum construct, and the low levels saturated
arithmetic operators.  The recursive type definition is
necessary to ``inject'' these new constructors into the basic MLTree 
definition.

The following is an example of how these new constructors that we have defined can be used.  Suppose the source program in our DSP language is:
\begin{verbatim}
   for i := a ... b
   {  s := sadd(s, table[i]);
   }
\end{verbatim}
\noindent where \verb|sadd| is the saturated add operator.
For simplicity, let us also assume that all operations and addresses
are in 32-bits.
Then the translation of the above into our extended DSP-MLTree could be:
\begin{SML}
   EXT(FOR(\(i\), REG(32, \(a\)), REG(32, \(b\)),
           MV(32, \(s\), REXT(32, SADD(REG(32, \(s\)), 
                LOAD(32, 
                    ADD(32, REG(32, \(table\)), 
                        SLL(32, REG(32, \(i\)), LI 2)),
                         \(region\)))))
          ))
\end{SML}

One potential short coming of our DSP extension to MLTree is that
the extension does not allow any further extensions.  This restriction
may be entirely satisfactory if DSP-MLTree is only used in your compiler
applications and no where else.  However, if DSP-MLTree is intended
to be an extension library for MLRISC, then  we must build in the flexibility
for extension.  This can be done in the same way as in the base MLTree
definition, like this: 
\begin{SML}
functor ExtensibleDSPMLTreeExtension
  (Extension : \mlrischref{mltree/mltree-extension.sig}{MLTREE_EXTENSION}) =
struct
   structure Basis = MLTreeBasis
   structure Extension = Extension
   datatype ('s,'r,'f,'c) sx = 
      FOR of Basis.var * 'r * 'r * 's
    | EXT of ('s,'r,'f,'c) Extension.sx 
   and ('s,'r,'f,'c) rx = 
      SUM of Basis.var * 'r * 'r * 'r
    | SADD of 'r * 'r
    | SSUB of 'r * 'r
    | SMUL of 'r * 'r
    | SDIV of 'r * 'r
    | REXT of ('s,'r,'f,'c) Extension.rx
   withtype
        ('s,'r,'f,'c) fx   = ('s,'r,'f,'c) Extension.fx
   and  ('s,'r,'f,'c) ccx  = ('s,'r,'f,'c) Extension.ccx
end
\end{SML}

As in MLTREE, we provide two new extension 
constructors \verb|EXT| and \verb|REXT| in
the definition of \sml{DSP_MLTREE}, which  can 
be used to further enhance the extended MLTREE language.

\section{MLTree Utilities} 

The \MLRISC{} system contains numerous utilities for working with
MLTree datatypes.  Some of the following utilizes are also useful for clients
use:
\begin{description}
  \item[MLTreeUtils] implements basic hashing, equality and pretty
printing functions,
  \item[MLTreeFold] implements a fold function over the MLTree datatypes,  
  \item[MLTreeRewrite] implements a generic rewriting engine,
  \item[MLTreeSimplify] implements a simplifier that performs algebraic
simplification and constant folding.
\end{description}
\subsubsection{Hashing, Equality, Pretty Printing}

The functor \mlrischref{mltree/mltree-utils.sml}{MLTreeUtils} provides
the basic utilities for hashing an MLTree term, comparing two
MLTree terms for equality and pretty printing.  The hashing and comparision
functions are useful for building hash tables using MLTree datatype as keys.
The signature of the functor is:
\begin{SML}
signature \mlrischref{mltree/mltree-utils.sig}{MLTREE_UTILS} =
sig
   structure T : MLTREE 

   (*
    * Hashing
    *)
   val hashStm   : T.stm -> word
   val hashRexp  : T.rexp -> word
   val hashFexp  : T.fexp -> word
   val hashCCexp : T.ccexp -> word

   (*
    * Equality
    *)
   val eqStm     : T.stm * T.stm -> bool
   val eqRexp    : T.rexp * T.rexp -> bool
   val eqFexp    : T.fexp * T.fexp -> bool
   val eqCCexp   : T.ccexp * T.ccexp -> bool
   val eqMlriscs : T.mlrisc list * T.mlrisc list -> bool

   (*
    * Pretty printing 
    *)
   val show : (string list * string list) -> T.printer

   val stmToString   : T.stm -> string
   val rexpToString  : T.rexp -> string
   val fexpToString  : T.fexp -> string
   val ccexpToString : T.ccexp -> string

end
functor \mlrischref{mltree/mltree-utils.sml}{MLTreeUtils} 
  (structure T : MLTREE
   (* Hashing extensions *)
   val hashSext  : T.hasher -> T.sext -> word
   val hashRext  : T.hasher -> T.rext -> word
   val hashFext  : T.hasher -> T.fext -> word
   val hashCCext : T.hasher -> T.ccext -> word

   (* Equality extensions *)
   val eqSext  : T.equality -> T.sext * T.sext -> bool
   val eqRext  : T.equality -> T.rext * T.rext -> bool
   val eqFext  : T.equality -> T.fext * T.fext -> bool
   val eqCCext : T.equality -> T.ccext * T.ccext -> bool

   (* Pretty printing extensions *)
   val showSext  : T.printer -> T.sext -> string
   val showRext  : T.printer -> T.ty * T.rext -> string
   val showFext  : T.printer -> T.fty * T.fext -> string
   val showCCext : T.printer -> T.ty * T.ccext -> string
  ) : MLTREE_UTILS =
\end{SML} 

The types \sml{hasher}, \sml{equality},
and \sml{printer} represent functions for hashing,
equality and pretty printing.   These are defined as:
\begin{SML} 
   type hasher =
      \{stm    : T.stm -> word,
       rexp   : T.rexp -> word,
       fexp   : T.fexp -> word,
       ccexp  : T.ccexp -> word
      \}    

   type equality =
      \{ stm    : T.stm * T.stm -> bool,
        rexp   : T.rexp * T.rexp -> bool,
        fexp   : T.fexp * T.fexp -> bool,
        ccexp  : T.ccexp * T.ccexp -> bool
      \} 
   type printer =
      \{ stm    : T.stm -> string,
        rexp   : T.rexp -> string,
        fexp   : T.fexp -> string,
        ccexp  : T.ccexp -> string,
        dstReg : T.ty * T.var -> string,
        srcReg : T.ty * T.var -> string
      \}
\end{SML}

For example, to instantiate a \sml{Utils} module for our \sml{DSPMLTree},
we can write:
\begin{SML}
   structure U = MLTreeUtils
     (structure T = DSPMLTree
      fun hashSext \{stm, rexp, fexp, ccexp\} (FOR(i, a, b, s)) =
           Word.fromIntX i + rexp a + rexp b + stm s
      and hashRext \{stm, rexp, fexp, ccexp\} e =
          (case e of
             SUM(i,a,b,c) => Word.fromIntX i + rexp a + rexp b + rexp c
           | SADD(a,b) => rexp a + rexp b
           | SSUB(a,b) => 0w12 + rexp a + rexp b
           | SMUL(a,b) => 0w123 + rexp a + rexp b
           | SDIV(a,b) => 0w1245 + rexp a + rexp b
          )
      fun hashFext _ _ = 0w0
      fun hashCCext _ _ = 0w0
      fun eqSext \{stm, rexp, fexp, ccexp\} 
        (FOR(i, a, b, s), FOR(i', a', b', s')) =
           i=i' andalso rexp(a,a') andalso rexp(b,b') andalso stm(s,s')
      fun eqRext \{stm, rexp, fexp, ccexp\} (e,e') =
       (case (e,e') of
          (SUM(i,a,b,c),SUM(i',a',b',c')) => 
            i=i' andalso rexp(a,a') andalso rexp(b,b') andalso stm(c,c')
        | (SADD(a,b),SADD(a',b')) => rexp(a,a') andalso rexp(b,b')
        | (SSUB(a,b),SSUB(a',b')) => rexp(a,a') andalso rexp(b,b')
        | (SMUL(a,b),SMUL(a',b')) => rexp(a,a') andalso rexp(b,b')
        | (SDIV(a,b),SDIV(a',b')) => rexp(a,a') andalso rexp(b,b')
        | _ => false
       )
      fun eqFext _ _ = true
      fun eqCCext _ _ = true

      fun showSext \{stm, rexp, fexp, ccexp, dstReg, srcReg\}  
            (FOR(i, a, b, s)) =
          "for("^dstReg i^":="^rexp a^".."^rexp b^")"^stm s
      fun ty t = "."^Int.toString t
      fun showRext \{stm, rexp, fexp, ccexp, dstReg, srcReg\} e = 
           (case (t,e) of
             SUM(i,a,b,c) => 
              "sum"^ty t^"("^dstReg i^":="^rexp a^".."^rexp b^")"^rexp c
           | SADD(a,b) => "sadd"^ty t^"("rexp a^","^rexp b^")"
           | SSUB(a,b) => "ssub"^ty t^"("rexp a^","^rexp b^")"
           | SMUL(a,b) => "smul"^ty t^"("rexp a^","^rexp b^")"
           | SDIV(a,b) => "sdiv"^ty t^"("rexp a^","^rexp b^")"
           )
      fun showFext _ _ = ""
      fun showCCext _ _ = ""
     )
\end{SML}

\subsubsection{MLTree Fold}
The functor \mlrischref{mltree/mltree-fold.sml}{MLTreeFold}
provides the basic functionality for implementing various forms of
aggregation function over the MLTree datatypes.  Its signature is
\begin{SML}
signature \mlrischref{mltree/mltree-fold.sig}{MLTREE_FOLD} =
sig
   structure T : MLTREE

   val fold : 'b folder -> 'b folder
end
functor \mlrischref{mltree/mltree-fold.sml}{MLTreeFold}
  (structure T : MLTREE
   (* Extension mechnism *)
   val sext  : 'b T.folder -> T.sext * 'b -> 'b
   val rext  : 'b T.folder -> T.ty * T.rext * 'b -> 'b
   val fext  : 'b T.folder -> T.fty * T.fext * 'b -> 'b
   val ccext : 'b T.folder -> T.ty * T.ccext * 'b -> 'b
  ) : MLTREE_FOLD =
\end{SML}
The type \newtype{folder} is defined as:
\begin{SML}
   type 'b folder =
       \{ stm   : T.stm * 'b -> 'b,
         rexp  : T.rexp * 'b -> 'b,
         fexp  : T.fexp * 'b -> 'b, 
         ccexp : T.ccexp * 'b -> 'b
       \}
\end{SML}


\subsubsection{MLTree Rewriting}

The functor \mlrischref{mltree/mltree-rewrite.sml}{MLTreeRewrite}
implements a generic term rewriting engine which is useful for performing
various transformations on MLTree terms. Its signature is
\begin{SML}
signature \mlrischref{mltree/mltree-rewrite.sig}{MLTREE_REWRITE} =
sig
   structure T : MLTREE

  val rewrite : 
       (* User supplied transformations *)
       \{ rexp  : (T.rexp -> T.rexp) -> (T.rexp -> T.rexp), 
         fexp  : (T.fexp -> T.fexp) -> (T.fexp -> T.fexp),
         ccexp : (T.ccexp -> T.ccexp) -> (T.ccexp -> T.ccexp),
         stm   : (T.stm -> T.stm) -> (T.stm -> T.stm)
       \} -> T.rewriters
end
functor \mlrischref{mltre/mltree-rewrite.sml}{MLTreeRewrite}
  (structure T : MLTREE
   (* Extension *)
   val sext : T.rewriter -> T.sext -> T.sext
   val rext : T.rewriter -> T.rext -> T.rext
   val fext : T.rewriter -> T.fext -> T.fext
   val ccext : T.rewriter -> T.ccext -> T.ccext
  ) : MLTREE_REWRITE =
\end{SML}

The type \newtype{rewriter} is defined in signature
\mlrischref{mltree/mltree.sig}{MLTREE} as:
\begin{SML}
   type rewriter = 
       \{ stm   : T.stm -> T.stm,
         rexp  : T.rexp -> T.rexp,
         fexp  : T.fexp -> T.fexp,
         ccexp : T.ccexp -> T.ccexp
       \} 
\end{SML}
 
\subsubsection{MLTree Simplifier}

The functor \mlrischref{mltree/mltree-simplify.sml}{MLTreeSimplify}
implements algebraic simplification and constant folding for MLTree.
Its signature is:
\begin{SML}
signature \mlrischref{mltree/mltree-simplify.sig}{MLTREE_SIMPLIFIER} =
sig

   structure T : MLTREE

   val simplify  :
       { addressWidth : int } -> T.simplifier
   
end
functor \mlrischref{mltree/mltree-simplify.sml}{MLTreeSimplifier}
  (structure T : MLTREE
   (* Extension *)
   val sext : T.rewriter -> T.sext -> T.sext
   val rext : T.rewriter -> T.rext -> T.rext
   val fext : T.rewriter -> T.fext -> T.fext
   val ccext : T.rewriter -> T.ccext -> T.ccext
  ) : MLTREE_SIMPLIFIER =
\end{SML}

Where type \newdef{simplifier} is defined in signature 
\mlrischref{mltree/mltree.sig}{MLTREE} as:
\begin{SML}
   type simplifier =
       \{ stm   : T.stm -> T.stm,
         rexp  : T.rexp -> T.rexp,
         fexp  : T.fexp -> T.fexp,
         ccexp : T.ccexp -> T.ccexp
       \}
\end{SML}



\section{Instruction Selection} \label{sec:instrsel}
Instruction selection modules are reponsible for translating 
\href{mltree.html}{MLTree} statements into a flowgraph consisting
of target machine instructions.  MLRISC decomposes this complex task 
into \emph{three} components:
\begin{description}
   \item[Instruction selection modules] which are responsible for
mapping a sequence of MLTree statements into a sequence target machine code,
   \item[Flowgraph builders]  which are responsible for constructing
the graph representation of the program from a sequence of target machine
instructions, and
   \item[Client Extender] which are responsible for compiling 
MLTree extensions (see also Section~\ref{sec:mltree-extension}).
\end{description}
By detaching these components, extra flexiblity is obtained.  For example,
the MLRISC system uses two different internal representations.  The
first, \href{cluster.html}{cluster}, is a \emph{light-weight} representation
which is suitable for simple compilers without extensive 
optimizations; the second, \href{mlrisc-ir.html}{MLRISC IR}, is a 
\emph{heavy duty} representation which allows very complex transformations
to be performed.  Since the flowgraph builders are detached from the
instruction selection modules, the same instruction selection modules
can be used for both representations.  

For consistency, the three components communicate to each other 
via the same \href{stream.html}{stream} interface.

\subsection{Interface Definition}
All instruction selection modules satisfy the following signature:

\begin{SML}
signature \mlrischref{mltree/mltreecomp.sig}{MLTREECOMP} = 
sig
   structure T : \href{mltree.html}{MLTREE}
   structure I : \href{instructions.html}{INSTRUCTIONS}
   structure C : \href{cells.html}{CELLS}
      sharing T.LabelExp = I.\href{labelexp.html}{LabelExp}
      sharing I.C = C

   type instrStream = (I.instruction,C.regmap,C.cellset) T.stream
   type mltreeStream = (T.stm,C.regmap,T.mlrisc list) T.stream

   val selectInstructions : instrStream -> mltreeStream
end
\end{SML}
Intuitively, this signature states that
the instruction selection module 
returns a function that can transform a stream of MLTree statements 
(\newtype{mltreeStream}) into a stream of instructions of the target 
machine (\newtype{instrStream}).  

\subsubsection{Compiling Client Extensions}

Compilation of client extensions to MLTREE is controlled by the
following signature: 
\begin{SML}
signature \mlrischref{mltree/mltreecomp.sig}{MLTREE_EXTENSION_COMP} =
sig
   structure T : \href{mltree.html}{MLTREE}
   structure I : \href{instructions.html}{INSTRUCTIONS}
   structure C : \href{cells.html}{CELLS}
      sharing T.LabelExp = I.\href{labelexp.html}{LabelExp}
      sharing I.C = C

   type reducer = 
     (I.instruction,C.regmap,C.cellset,I.operand,I.addressing_mode) T.reducer

   val compileSext : reducer -> \{stm:T.sext, an:T.an list\} -> unit
   val compileRext : reducer -> \{e:T.ty * T.rext, rd:C.cell, an:T.an list\} -> unit
   val compileFext : reducer -> \{e:T.ty * T.fext, fd:C.cell, an:T.an list\} -> unit
   val compileCCext : reducer -> \{e:T.ty * T.ccext, ccd:C.cell, an:T.an list\} -> unit
end
\end{SML}

Methods \verb|compileSext|, \verb|compileRext|, etc.~are callbacks that
are responsible for compiling MLTREE extensions.  The arguments
to these callbacks have the following meaning:
\begin{description}
  \item[reducer] The first argument is always the \newtype{reducer}, 
which contains internal methods for translating MLTree constructs
into machine code.  These methods are supplied by the instruction
selection modules.
  \item[an] This is a list of annotations that should be attached to the
generated code.
  \item[ty, fty] These are the types of the extension construct.
  \item[stm, e] This are the extension statement and expression.
  \item[rd, fd, cd] These are the target registers of the 
expression extension, i.e.~the callback should generate the appropriate
code for the expression and writes the result to this target.
\end{description}

For example, when an instruction selection encounters a
\verb|FOR(|$i,a,b,s$\verb|)| statement extension
defined in Section~\ref{sec:mltree-extension}, the callback
\begin{SML}   
  compileStm reducer \{ stm=FOR(\(i,a,b,s\)), an=an \}
\end{SML}
\noindent is be involved. 

The \newtype{reducer} type is defined
in the signature \mlrischref{mltree/mltree.sig}{MLTREE} and has the
following type:
\begin{SML}
  datatype reducer =
    REDUCER of
    \{ reduceRexp    : rexp -> reg,
      reduceFexp    : fexp -> reg,
      reduceCCexp   : ccexp -> reg,
      reduceStm     : stm * an list -> unit,
      operand       : rexp -> I.operand,
      reduceOperand : I.operand -> reg,
      addressOf     : rexp -> I.addressing_mode,
      emit          : I.instruction * an list -> unit,
      instrStream   : (I.instruction,C.regmap,C.cellset) stream,
      mltreeStream  : (stm,C.regmap,mlrisc list) stream
    \}
\end{SML}

The components of the reducer are
\begin{description}
  \item[reduceRexp, reduceFexp, reduceCCexp] These functions 
take an expression of type integer, floating point and condition code, 
translate them into machine code and return the 
register that holds the result. 
  \item[reduceStm] This function takes an MLTree statement and translates
it into machine code.  it also takes a second argument, which is the
list of annotations that should be attached to the statement.
  \item[operand] This function translates an \sml{rexp} into an
 \sml{operand} of the machine architecture.
  \item[reduceOperand] This function takes an operand of the machine
architecture and reduces it into an integer register.
  \item[addressOf] This function takes an \sml{rexp}, treats
it as an address expression and translates it into the appropriate
\sml{addresssing_mode} of the target architecture.
  \item[emit]  This function emits an instruction with attached annotations
to the instruction stream
  \item[instrStream, mltreeStream]  These are the instruction stream
and mltree streams that are currently bound to the extender.
\end{description}

\subsubsection{Extension Example}
Here is an example of how the extender mechanism can be used,
using the \sml{DSP_MLTREE} extensions defined in
Section~\ref{sec:mltree-extension}.   We need supply two
new functions, \verb|compileDSPStm| for compiling the \verb|FOR|
construct, and \verb|compileDSPRexp| for compiling the \verb|SUM|,
and saturated arithmetic instructions.

The first function, \sml{compileDSPStm}, is generic and simply
translates the \verb|FOR| loop into the appropriate branches.
Basically, we will translate \verb|FOR(|$i,start,stop,body$\verb|)| into
the following loop in pseudo code:
\begin{SML}
        limit = \(stop\)
        \(i\)  = \(start\)
        goto test
  loop: \(body\)
        \(i\) = \(i\) + 1
  test: if \(i\) <= limit goto loop
\end{SML}
This transformation can be implemented as follows:
\begin{SML}
 functor DSPMLTreeExtensionComp
    (structure I : DSP_INSTRUCTION_SET
     structure T : DSP_MLTREE
       sharing I.LabelExp = T.LabelExp
    ) =
 struct
   structure I = I
   structure T = T
   structure C = I.C

   type reducer = 
     (I.instruction,C.regmap,C.cellset,I.operand,I.addressing_mode) T.reducer
   
   fun mark(s, []) = s
     | mark(s, a::an) = mark(ANNOTATION(s, a), an)
   fun compileSext (REDUCER\{reduceStm, ...\}) 
      \{stm=FOR(i, start, stop, body), an\} =
   let val limit = C.newReg()
       val loop  = Label.newLabel ""
       val test  = Label.newLabel ""
   in  reduceStm(
         SEQ[MV(32, i, start),
             MV(32, limit, stop),
             JMP([], [LABEL(LabelExp.LABEL test)], []),
             LABEL loop,
             body,
             MV(32, i, ADD(32, REG(32, i), LI 1),
             LABEL test,
             mark(BCC([], 
                    CMP(32, LE, REG(32, i), REG(32, limit)), 
                      loop),
                  an),
            ]
      )
   end

   ...
\end{SML}
In this transformation, we have chosen to proprogate the annotation
\verb|an| into the branch constructor.

Assuming the target architecture that we are translated into contains
saturated arithmetic instructions \verb|SADD|, \verb|SSUB|, \verb|SMUL|
and \verb|SDIV|, the DSP extensions
\verb|SUM| and saturated arithmetic expressions can be handled as follows.
\begin{SML}
   fun compileRext (REDUCER\{reduceStm, reduceRexp, emit, ...\}) 
       \{ty, e, rd, an\} =
   (case (ty,e) of
      (_,T.SUM(i, a, b, exp)) =>
        reduceStm(SEQ[MV(ty, rd, LI 0),
                      FOR(i, a, b, 
                         mark(MV(ty, rd, ADD(ty, REG(ty, rd), exp)), an))
                     ]
                 )
   | (32,T.SADD(x,y)) => emit(I.SADD\{r1=reduceRexp x,r2=reduceRexp y,rd=rd\},an)
   | (32,T.SSUB(x,y)) => emit(I.SSUB\{r1=reduceRexp x,r2=reduceRexp y,rd=rd\},an)
   | (32,T.SMUL(x,y)) => emit(I.SMUL\{r1=reduceRexp x,r2=reduceRexp y,rd=rd\},an)
   | (32,T.SDIV(x,y)) => emit(I.SDIV\{r1=reduceRexp x,r2=reduceRexp y,rd=rd\},an)
   | ...
   )

   fun compileFext _ _ = ()
   fun compileCCext _ _ = ()

  end
\end{SML}

Note that in this example, we have simply chosen to reduce
a \verb|SUM| expression into the high level \verb|FOR| construct.
Clearly, other translation schemes are possible.

\subsection{Instruction Selection Modules}
Here are the actual code for the various back ends:
\begin{enumerate}
  \item \mlrischref{sparc/mltree/sparc.sml}{Sparc}
  \item \mlrischref{hppa/mltree/hppa.sml}{PA-RISC}
  \item \mlrischref{alpha/mltree/alpha.sml}{Alpha}
  \item \mlrischref{ppc/mltree/ppc.sml}{Power PC}
  \item \mlrischref{x86/mltree/x86.sml}{X86}
  \item C6xx 
\end{enumerate}

\section{Assemblers}

\subsubsection{Overview}
Assemblers in MLRISC satisfy the signature 
\mlrischref{emit/instruction-emitter.sig}{INSTRUCTION\_EMITTER},
which is defined as:
\begin{SML}
signature INSTRUCTION_EMITTER =
sig
   structure I : \href{instructions.html}{INSTRUCTIONS}
   structure C : \href{cells.html}{CELLS}
   structure S : \href{streams.html}{INSTRUCTION_STREAM}
   structure P : \href{pseudo-ops.html}{PSEUDO_OPS}
      sharing I.C = C
      sharing S.P = P

   val makeStream : Annotations.annotations ->
                     ((int -> int) -> I.instruction -> unit,
                      unit,'b,'c,'d,'e) S.stream
end
\end{SML}

The function \sml{makeStream} returns an instruction stream.
By default the output is bound to the stream \sml{AsmStream.asmOutStream} 
defined in the structure 
\mlrischref{emit/asmStream.sml}{AsmStream} at creation time.

The structure \sml{AsmStream} satisfy the following signature.
\begin{SML}
signature ASM_STREAM = sig
  val asmOutStream : TextIO.outstream ref
  val withStream : TextIO.outstream -> ('a -> 'b) -> 'a -> 'b
end
\end{SML}
\subsubsection{Redirecting the Output}
It is possible to redirect the output of an instruction stream.
For example, the following statement
\begin{SML}
   val asm = makeStream []
\end{SML}
binds the output of \sml{asm} to \sml{AsmStream.asmOutStream}, which
by default is just \sml{TextIO.stdOut}.  On the other hand, the
statement
\begin{SML}
   val asm = AsmStream.withStream mystream makeStream []
\end{SML}
binds the output of asm to \sml{mystream}.

\subsubsection{More Details}

Assemblers are automatically generated by the 
\href{mlrisc-md.html}{MDGen} tool.  Some specific generated
assemblers are listed below:
\begin{enumerate}
 \item \mlrischref{sparc/emit/sparcAsm.sml}{Sparc}
 \item \mlrischref{hppa/emit/hppaAsm.sml}{Hppa}
 \item \mlrischref{alpha/emit/alphaAsm.sml}{Alpha}
 \item \mlrischref{ppc/emit/ppcAsm.sml}{Power PC}
 \item \mlrischref{x86/emit/x86Asm.sml}{X86}
\end{enumerate}

\section{Machine Code Emitters}

\subsubsection{Overview}
MLRISC lets the client to directly emit machine code and bypass the traditional
assembly mechanism. 

Machine code emitters in MLRISC satisfy the signature 
\mlrischref{emit/instruction-emitter.sig}{INSTRUCTION\_EMITTER},
which is defined as:
\begin{SML}
signature INSTRUCTION_EMITTER =
sig

   structure I : \href{instructions.html}{INSTRUCTIONS}
   structure C : \href{cells.html}{CELLS}
   structure S : \href{streams.html}{INSTRUCTION_STREAM}
   structure P : \href{pseudo-ops.html}{PSEUDO_OPS}
      sharing I.C = C  
      sharing S.P = P

   val makeStream : Annotations.annotations ->
                     ((int -> int) -> I.instruction -> unit,
                      unit,'b,'c,'d,'e) S.stream

end
\end{SML}

The function \sml{makeStream} returns an instruction stream.
The output, a stream of bytes, is direct to the client supplied
structure which satisfy the 
\mlrischref{emit/code-string.sig}{CODE\_STRING} interface.
This signature is defined as follows:
\begin{SML}
signature CODE_STRING = sig
  type code_string
  val init          : int -> unit
  val update        : int * Word8.word -> unit
  val getCodeString : unit -> code_string
end
\end{SML}

\subsubsection{More Details}

Machine code emitters are automatically generated by the 
\href{mlrisc-md.html}{MDGen} tool.  Some specific generated
emitters are listed below:
\begin{enumerate}
 \item \mlrischref{sparc/emit/sparcMC.sml}{Sparc}
 \item \mlrischref{hppa/emit/hppaMC.sml}{Hppa}
 \item \mlrischref{alpha/emit/alphaMC.sml}{Alpha}
 \item \mlrischref{ppc/emit/ppcMC.sml}{Power PC}
 \item \mlrischref{x86/emit/x86MC.sml}{X86}
\end{enumerate}

\section{Delay Slot Filling}
\subsection{ Overview }

    Superscalar architectures such as the Sparc, MIPS, and PA-RISC
contain delayed branch and/or load instructions.  
Delay slot filling is necessary 
task of the back end to keep the instruction pipelines busy.  To accomodate
the intricate semantics of branch delay slot in various architectures, 
MLRISC uses the following very general framework for dealing with 
delayed instructions. 
   
\begin{description}
  \item[Instruction representation]
      To make it easy to deal with instruction with delay slot, MLRISC allow
       the following extensions to instruction representations.
  \begin{itemize}
    \item Instructions with delay slot may have a
        \begin{color}{#aa0000}nop\end{color} flag.   When this flag is true
        the delay slot is assumed to be filled with a NOP instruction.
    \item Instructions with delay slots that can be nullified may have a
        \begin{color}{#aa0000}nullified\end{color} flag.   
       When this flag is true the branch delay slot is assumed to be
       nullified.  
    \end{itemize}
   \item[Nullification semantics]
     Unfortunately, nullification semantics
        in architectures vary. In general, MLRISC allows the following
        additional nullification characteristics to be specified. 
     \begin{itemize}
     \item Nullification can be specified as illegal; this is needed 
           because some instructions can not be nullified
     \item When nullification is enabled, the semantics of the delay slot
          instruction may depend on the direction of the branch, and whether
          a conditional test succeeds. 
     \item Certain class of instructions may be declared to be illegal
          to fit into certain class of delay slots.
     \end{itemize}
\end{description} 

For example, conditional branch instructions on the Sparc are defined 
as follows:
\begin{verbatim}
   Bicc of {b:branch, a:bool, label:Label.label, nop:bool}
     asm: ``b<b><a>\t<label><nop>''
     padding: nop = true
     nullified: a = true and (case b of I.BA => false | _ => true)
     delayslot candidate: false
\end{verbatim}
\noindent where \sml{a} is \emph{annul} flag and \sml{nop} is the nop 
flag (see \mlrischref{sparc/sparc.md}{the Sparc machine description}).
A constructor term
\begin{SML}
   Bicc\{b=BE, a=true, label=label, nop=true\}
\end{SML}
denotes the instruction sequence
\begin{verbatim}
   be,a label
   nop
\end{verbatim}
while
\begin{SML}
   Bicc\{b=BE, a=false, label=label, nop=false\}
\end{SML}
denotes 
\begin{verbatim}
   be label
\end{verbatim}


\subsection{The Interface}

Architecture information about how delay slot filling is to be performed
is described in the signature
\mlrischref{backpatch/delaySlotProps.sig}{DELAY\_SLOT\_PROPERTIES}.
\begin{SML}
signature DELAY_SLOT_PROPERTIES =
sig
   structure I : INSTRUCTIONS

   datatype delay_slot = 
     D_NONE   | D_ERROR   | D_ALWAYS  
   | D_TAKEN  | D_FALLTHRU 

   val delaySlotSize : int 
   val delaySlot : \{ instr : I.instruction, backward : bool \} -> 
		   \{ n    : bool,      
		     nOn  : delay_slot,
		     nOff : delay_slot,
		     nop  : bool      
		   \} 
   val enableDelaySlot : 
	 \{instr : I.instruction, n:bool, nop:bool\} -> I.instruction
   val conflict : 
         \{regmap:int->int,src:I.instruction,dst:I.instruction\} -> bool
   val delaySlotCandidate : 
         \{ jmp : I.instruction, delaySlot : I.instruction \} -> bool
   val setTarget : I.instruction * Label.label -> I.instruction
end
\end{SML}

The components of this signature are:
\begin{description}
  \item[delay\_slot] This datatype describes properties related to a 
      delay slot.
   \begin{description}
     \item[D\_NONE]   This indicates that no delay slot is possible.
     \item[D\_ERROR]  This indicates that it is an error 
     \item[D\_ALWAYS] This indicates that the delay slot is always active
     \item[D\_TAKEN]  This indicates that the 
              delay slot is only active when branch is taken
     \item[D\_FALLTHRU]  This indicates that the delay slot 
       is only active when branch is not taken 
   \end{description} 
  \item[delaySlotSize] This is size of delay slot in bytes.

  \item[delaySlot]  This method takes an instruction \sml{instr}
      and a flag indicating whether the branch is \sml{backward},
     and returns the delay slot properties of an instruction.   The
      properties is described by four fields.
      \begin{description}
        \item[n : bool]  This bit is if the nullified bit in the
   instruction is currently set.
        \item[nOn : delay\_slot] This field indicates the delay slot 
          type when the instruction is nullified.
        \item[nOff : delay\_slot] This field indiciates the delay slot
         type when the instruction is not nullified. 
         \item[nop  : bool] This bit indicates whether there is an 
implicit padded nop.
      \end{description}


   \item[enableDelaySlot]
       This method set the nullification and nop flags of an instruction.

   \item[conflict] This method checks whether there are any conflicts
      between instruction \sml{src} and \sml{dst}.
   \item[delaySlotCandidate] 
       This method checks whether instruction \sml{delaySlot} is within the
       class of instructions that can fit within the delay slot of 
       instruction \sml{jmp}.

   \item[setTarget]
       This method changes the branch target of an instruction.
\end{description}

\subsubsection{Examples}
  For example,
\begin{SML}
    delaySlot\{instr=instr, backward=true\} =
    \{n=true, nOn=D_ERROR, nOff=D_ALWAYS, nop=true\}
\end{SML}
\noindent means that the instruction nullification bit is on, the
the nullification cannot be turned off, delay slot is always active 
(when not nullified), and there is currently an implicit padded nop.

\begin{SML}
   delaySlot\{instr=instr, backward=false\} =
  \{n=false, nOn=D_NONE, nOff=D_TAKEN, nop=false\}
\end{SML}
\noindent means that the nullification bit is off, the delay slot
is inactive when the nullification bit is off,  the delay slot is only
active when the (forward) branch is taken when \sml{instr} is 
not-nullified, and there
is no implicitly padded nop.

\begin{SML}
   delaySlot\{instr=instr, backward=true\} =
  \{n=true, nOn=D_TAKEN, nOff=D_ALWAYS, nop=true\}
\end{SML}
\noindent means that the nullification bit is on, the delay slot
is active on a taken (backward) branch when the nullification bit is off, 
the delay slot is always active when \sml{instr} is not-nullified, 
and there is currently an implicitly padded nop.


\section{Span Dependency Resolution} \label{sec:span-dep}

The span dependency resolution phase is used to resolve the values of
client defined \href{constants.html}{constants} and \href{labels.html}{labels}
in a program.  An instruction whose immediate operand field contains a
constant or \href{labexp.html}{label expression} which
is too large is rewritten into a sequence of instructions to compute
the same result.  Similarly, short branches referencing labels that are 
too far are rewritten into the long form.   For architectures
that require the filling of delay slots, this is performed at the same
time as span depedency resolution, to ensure maximum benefit results.

\subsubsection{The Interface}

The signature \sml{SDI_JUMPS} describes
architectural information about span dependence resolution.

\begin{SML}
signature \mlrischref{backpatch/sdi-jumps.sig}{SDI_JUMPS} = sig
  structure I : \href{instructions.html}{INSTRUCTIONS}
  structure C : \href{cells.html}{CELLS}
    sharing I.C = C

  val branchDelayedArch : bool
  val isSdi : I.instruction -> bool
  val minSize : I.instruction -> int
  val maxSize : I.instruction -> int
  val sdiSize : I.instruction * (C.cell -> C.cell)
                              * (Label.label -> int) * int -> int
  val expand : I.instruction * int * int -> I.instruction list
end
\end{SML}

The components in this interface are:
\begin{description}
  \item[branchDelayedArch] A flag indicating whether the architecture
contains delay slots.  For example, this would be true on the MIPS,
Sparc, PA RISC; but would be false on the x86 and on the Alpha.
   \item[isSdi] This function returns true if the instruction is 
\newdef{span dependent}, i.e.~its size depends either on some unresolved
constants, or on its position in the code stream.
   \item[sdiSize]  This function takes a span dependent instruction, 
a \href{regmap.html}{regmap},
a mapping from labels to code stream position, and 
its current code stream position and returns the size of its
expansion in bytes.
   \item[expand] This function takes a span dependent instruction,
its size, and its location and return its expansion.
\end{description}

The signature \sml{BBSCHED} is the signature of the phase that performs
span depedennce resolution and code generation.
\begin{SML}
signature \mlrischref{backpatch/bbsched.sig}{BBSCHED} = sig
  structure F : \href{cluster.html}{FLOWGRAPH}

  val bbsched : F.cluster -> unit
  val finish : unit -> unit
  val cleanUp : unit -> unit
end
\end{SML}

\subsubsection{The Modules}

Three different functors are present in the \MLRISC{} system for
performing span dependence resolution and code generator.
Functor \sml{BBSched2} is the simplest one, which does not perform
delay slot filling.
\begin{SML}
functor BBSched2
  (structure Flowgraph : \mlrischref{cluster/flowgraph.sig}{FLOWGRAPH}
   structure Jumps : \mlrischref{backpatch/sdi-jumps.sig}{SDI_JUMPS}
   structure Emitter : \href{mc.html}{INSTRUCTION_EMITTER}
     sharing Emitter.P = Flowgraph.P
     sharing Flowgraph.I = Jumps.I = Emitter.I
  ): BBSCHED 
\end{SML}

Functor \sml{SpanDependencyResolution} performs both span dependence
resolution and delay slot filling at the same time.
\begin{SML}
functor SpanDependencyResolution
  (structure Flowgraph : \mlrischref{cluster/flowgraph.sig}{FLOWGRAPH}
   structure Emitter : \href{mc.html}{INSTRUCTION_EMITTER}
   structure Jumps : \mlrischref{backpatch/sdi-jumps.sig}{SDI_JUMPS}
   structure DelaySlot : \href{delayslots.html}{DELAY_SLOT_PROPERTIES}
   structure Props : \mlrischref{instructions/insnProps.sig}{INSN_PROPERTIES}
     sharing Flowgraph.P = Emitter.P
     sharing Flowgraph.I = Jumps.I = DelaySlot.I = Props.I = Emitter.I
  ) : BBSCHED 
\end{SML}

Finally, functor \sml{BackPatch} is a span dependency resolution
module specially written for the \href{x86.html}{x86} architecture.
\begin{SML}
functor BackPatch
  (structure CodeString : \mlrischref{emit/code-string.sig}{CODE_STRING}
   structure Jumps: \mlrischref{backpatch/sdi-jumps.sig}{SDI_JUMPS}
   structure Props : \mlrischref{instructions/insnProps.sig}{INSN_PROPERTIES}
   structure Emitter : \mlrischref{backpatch/vlBatchPatch.sig}{MC_EMIT}
   structure Flowgraph : \href{cluster.html}{FLOWGRAPH}
   structure Asm : \href{asm.html}{INSTRUCTION_EMITTER}
      sharing Emitter.I = Jumps.I = Flowgraph.I = Props.I = Asm.I) : BBSCHED 
\end{SML}



%\section{The MLRISC Machine Description Language}

\subsection{ Overview }

\newdef{MDGen} is a machine description language 
is designed to automate
various mundane and error prone tasks in developing a back-end for 
MLRISC.  Currently, to target a new
architecture the programmer must provide the following set of modules
written in Standard ML:

\begin{itemize}
  \item \codehref{instructions/cells.sig}{CELLS} -- 
   the properties of the register set and (some part of) memory hierarchy. 
  \item \codehref{instructions/instructions.sig}{INSTRUCTIONS} -- 
   the concrete instruction set representation.
  \item \codehref{instructions/insnProps.sig}{INSNS_PROPERTIES}  --
   properties of the instructions.
  \item \codehref{instructions/shuffle.sig}{SHUFFLE} --
   methods to emit linearized code from parallel copies.
  \item \codehref{emit/instruction-emitter.sig}{ASSEMBLER} --
   the assembler
  \item \codehref{emit/instruction-emitter.sig}{MC} --
   the machine code emitter
  \item \codehref{../backpatch/sdi-jumps.sig}{ SDI_JUMPS } --
   methods for resolving span-dependent instructions. 
  \item <a href="../backpatch/delaySlotProps.sig" target=code> DELAY_SLOTS_PROPERTIES 
        </a> -- machine properties for delay slot filling, if a machine 
    architecture contains branch delay slots or load delay slots.
  \item \codehref{../SSA/ssaProps.sig}{ SSA_PROPERTIES } --
    semantics properties for performing optimizations in Static Single
  Assignment form.
\end{itemize}

In general, writing a backend is tedious even with  
SML's abstraction capabilities. 
Furthermore, the machine description is procedural in natural 
and must be checked by hand.  

\subsection{ What is in MDGen? }
The MDGen tool simplifies the process of developing a new MLRISC backend.  
MDGen provides the following:
\begin{itemize}
   \item A representation description language for specifying the
     machine encoding of the instruction set,
     using an extension of ML's algebraic datatype facility.
   \item A semantics description language for specifying the abstract semantics
      of the instructions.
\end{itemize}

Both sub-languages are based on ML's syntax and semantics, so
they should be readily familiar to all MLRISC users.

A backend developer can specify a new machine architecture using the MDGen 
language, and in turn, the MDGen tool generates ML modules that are
required by the MLRISC system.

The basic concepts of MDGen are inspired largely from 
Norman Ramsey's <a href="www.cs.virginia.edu/~nr/toolkit">
New Jersey Machine Code Tool Kit </a> and 
Ramsey and Davidson's
<a href="http://www.cs.virginia.edu/zephyr/csdl/lrtlindex.html">
Lambda RTL </a>

\subsection{A Sample Description}

Here we present a sample MDGen description, using the Alpha as an example.
We highlight all keywords in the MDGen language 
in.  A typical machine description
is structured as follows:

\begin{SML}
architecture Alpha =
   struct

   name "Alpha"

   superscalar

   little endian

   <font color=#FF0000>lowercase assembly</font>

   \href{#cells}{Storage cells and locations}
   \href{#encoding}{Instruction encoding formats specification}
   \href{#instruction}{Instruction definition}
<font color=#FF0000>end</font>
\end{SML}

Here, we declare that the Alpha is a superscalar machine using
little endian encoding.  Furthermore, assembly output should be displayed
in lowercase-- this is for personal esthetic reasons only; most assemblers
are case insensitive.



\subsubsection{ <a name="cells">Specifying Storage Cells and Locations </a>}

A <font color="#ff0000">cell</font> is an abstract resource location 
for holding data values.  On typical machines, the types of
cells include general purpose registers, floating point registers,
and condition code registers.

The \sml{storage} declaration defines different 
<font color="#ff0000">cellkinds</font>.  MLRISC requires the
cellkinds \sml{GP}, \sml{FP}, \sml{CC} to be defined.
These are the cellkinds for general purpose registers, floating point
registers and condition code registers.

In the following sequence of declarations, a few things are defined:
\begin{itemize}
  \item The cellkinds \sml{GP, FP, CC} are defined.
        Furthermore, the cellkinds \sml{MEM, CTRL}, which stand
        for memory and control (dependence), are also implicitly defined.
  \item The \sml{assembly as} clauses specify how a specific cell type is
       to be displayed.    Here, we specify that register 30, the
       stack pointer, should be displayed specially as \sml{$sp}.
  \item The \sml{in cellset} clause, when attached, tells MDGen that
       the associated cellkind should be part of the 
       \href{cellset.html}{ cellset }.  The clause \sml{in cellset GP}
       tells MDGen that the a cell of type \sml{CC} should be treated
       the same as a \sml{GP}
  \item The \sml{locations} declarations define a few abbreviations:
        \sml{stackptrR} is the stack pointer, \sml{asmTmpR} is
       the assembly temporary, \sml{fasmTmp} is the floating point
       assembly temporary etc.
\end{itemize}

<tt>
\begin{SML}
   <font color=#FF0000>storage</font>
     GP = 32 <font color=#FF0000>cells <font color=#FF0000>of</font> 64 bits in cellset called</font> "register" 
       	<font color=#FF0000>assembly as</font> (fn 30 => "$sp"
                      | r => "$"^Int.toString r)
   | FP = 32 <font color=#FF0000>cells <font color=#FF0000>of</font> 64 bits in cellset called</font> "floating point register" 
       	<font color=#FF0000>assembly as</font> (fn f => "f"^Int.toString f)
   | CC = <font color=#FF0000>cells <font color=#FF0000>of</font> 64 bits in cellset GP called</font> "condition code register"
                <font color=#FF0000>assembly as</font> "cc"
   <font color=#FF0000>locations</font>
       stackptrR = <font color=#008800>$</font>GP[30]
   <font color=#FF0000>and</font> asmTmpR   = <font color=#008800>$</font>GP[28]
   <font color=#FF0000>and</font> fasmTmp   = <font color=#008800>$</font>FP[30]
   <font color=#FF0000>and</font> GPReg r   = <font color=#008800>$</font>GP[r]
   <font color=#FF0000>and</font> FPReg f   = <font color=#008800>$</font>GP[f]
\end{SML}

<h3> <a name="instruction">
     Specifying the Representation of Instructions</a></h3> 
\begin{SML}
   <font color=#FF0000>structure</font> Instruction = 
   <font color=#FF0000>struct</font>
   <font color=#FF0000>datatype</font> ea = 
       Direct <font color=#FF0000>of</font> <font color=#008800>$</font>GP 
     | FDirect <font color=#FF0000>of</font> <font color=#008800>$</font>FP        
     | Displace <font color=#FF0000>of</font> {base: <font color=#008800>$</font>GP, disp:int}
 
   <font color=#FF0000>datatype</font> operand = 
       REGop <font color=#FF0000>of</font> <font color=#008800>$</font>GP       		``<GP>'' (GP)
     | IMMop <font color=#FF0000>of</font> int       		``<int>''
     | HILABop <font color=#FF0000>of</font> LabelExp.labexp       ``hi(<emit_labexp labexp>)''
     | LOLABop <font color=#FF0000>of</font> LabelExp.labexp       ``lo(<emit_labexp labexp>)''
     | LABop <font color=#FF0000>of</font> LabelExp.labexp       	``<emit_labexp labexp>''
     | CONSTop <font color=#FF0000>of</font> Constant.const       ``<emit_const const>''

   (* 
    * When I say ! after the datatype</font> name XXX, it means generate a
    * function emit_XXX that converts the constructors into the corresponding
    * assembly text.  By default, it uses the same name as the constructor,
    * but may be modified by the lowercase/uppercase assembly directive.
    * 
    *)
   <font color=#FF0000>datatype</font> branch! = 
      BR  0x30  
                | BSR 0x34  
                           | BLBC 0x3
    | BEQ  0x39 | BLT 0x3a | BLE  0x3b
    | BLBS 0x3c | BNE 0x3d | BGE  0x3e 
    | BGT  0x3f

   <font color=#FF0000>datatype</font> fbranch! =
                  FBEQ 0x31 | FBLT 0x32
    | FBLE 0x33             | FBNE 0x35
    | FBGE 0x36 | FBGT 0x37 
 
   <font color=#FF0000>datatype</font> load! = LDL 0x28 | LDL_L 0x2A | LDQ 0x29 | LDQ_L 0x2B | LDQ_U 0x0B
   <font color=#FF0000>datatype</font> store! = STL 0x2C | STQ 0x2D | STQ_U 0x0F
   <font color=#FF0000>datatype</font> fload[0x20..0x23]! = LDF | LDG | LDS | LDT 
   <font color=#FF0000>datatype</font> fstore[0x24..0x27]! = STF | STG | STS | STT 

   (* non-trapping opcodes *) 
   <font color=#FF0000>datatype</font> operate! = (* table C-5 *)
       ADDL  (0wx10,0wx00)                       | ADDQ (0wx10,0wx20) 
                           | CMPBGE(0wx10,0wx0f) | CMPEQ (0wx10,0wx2d) 
     | CMPLE (0wx10,0wx6d) | CMPLT (0wx10,0wx4d) | CMPULE (0wx10,0wx3d) 
     | CMPULT(0wx10,0wx1d) | SUBL  (0wx10,0wx09) 
     | SUBQ  (0wx10,0wx29) 
     | S4ADDL(0wx10,0wx02) | S4ADDQ (0wx10,0wx22) | S4SUBL (0wx10,0wx0b)
     | S4SUBQ(0wx10,0wx2b) | S8ADDL (0wx10,0wx12) | S8ADDQ (0wx10,0wx32)
     | S8SUBL(0wx10,0wx1b) | S8SUBQ (0wx10,0wx3b) 

     | AND   (0wx11,0wx00) | BIC    (0wx11,0wx08) | BIS    (0wx11,0wx20)
     | CMOVEQ(0wx11,0wx24) | CMOVLBC(0wx11,0wx16) | CMOVLBS(0wx11,0wx14)
     | CMOVGE(0wx11,0wx46) | CMOVGT (0wx11,0wx66) | CMOVLE (0wx11,0wx64)
     | CMOVLT(0wx11,0wx44) | CMOVNE (0wx11,0wx26) | EQV (0wx11,0wx48)
     | ORNOT (0wx11,0wx28) | XOR    (0wx11,0wx40)

     | EXTBL (0wx12,0wx06) | EXTLH  (0wx12,0wx6a) | EXTLL(0wx12,0wx26)
     | EXTQH (0wx12,0wx7a) | EXTQL  (0wx12,0wx36) | EXTWH(0wx12,0wx5a)
     | EXTWL (0wx12,0wx16) | INSBL  (0wx12,0wx0b) | INSLH(0wx12,0wx67)
     | INSLL (0wx12,0wx2b) | INSQH  (0wx12,0wx77) | INSQL(0wx12,0wx3b)
     | INSWH (0wx12,0wx57) | INSWL  (0wx12,0wx1b) | MSKBL(0wx12,0wx02)
     | MSKLH (0wx12,0wx62) | MSKLL  (0wx12,0wx22) | MSKQH(0wx12,0wx72)
     | MSKQL (0wx12,0wx32) | MSKWH  (0wx12,0wx52) | MSKWL(0wx12,0wx12)
     | SLL   (0wx12,0wx39) | SRA    (0wx12,0wx3c) | SRL  (0wx12,0wx34)
     | ZAP   (0wx12,0wx30) | ZAPNOT (0wx12,0wx31)
     | MULL  (0wx13,0wx00)                        | MULQ (0wx13,0wx20)
                           | UMULH  (0wx13,0wx30) 
     | SGNXL "addl" (0wx10,0wx00) (* same as ADDL *)

   (* conditional moves *) 
 
   <font color=#FF0000>datatype</font> pseudo_op! = DIVL | DIVLU
 
   <font color=#FF0000>datatype</font> operateV! = (* table C-5 opc/func *)
        ADDLV (0wx10,0wx40) | ADDQV (0wx10,0wx60)
      | SUBLV (0wx10,0wx49) | SUBQV (0wx10,0wx69) 
      | MULLV (0wx13,0wx00) | MULQV (0wx13,0wx60)
 
   <font color=#FF0000>datatype</font> foperate! =   (* table C-6 *)
      CPYS    (0wx17,0wx20)  | CPYSE (0wx17,0wx022)    | CPYSN   (0wx17,0wx021)
    | CVTLQ   (0wx17,0wx010) | CVTQL (0wx17,0wx030)    | CVTQLSV (0wx17,0wx530)
    | CVTQLV  (0wx17,0wx130)
    | FCMOVEQ (0wx17,0wx02a) | FCMOVEGE (0wx17,0wx02d) | FCMOVEGT (0wx17,0wx02f)
    | FCMOVLE (0wx17,0wx02e) | FCMOVELT (0wx17,0wx02c) | FCMOVENE (0wx17,0wx02b)
    | MF_FPCR (0wx17,0wx025) | MT_FPCR  (0wx17,0wx024)

                         (* table C-7 *)
    | CMPTEQ  (0wx16,0wx0a5) | CMPTLT (0wx16,0wx0a6)   | CMPTLE  (0wx16,0wx0a7)
    | CMPTUN  (0wx16,0wx0a4)

   <font color=#FF0000>datatype</font> foperateV! = 
          ADDSSUD  0wx5c0
        | ADDTSUD  0wx5e0
        | CVTQSC   0wx3c
        | CVTQTC   0wx3e
        | CVTTSC   0wx2c
        | CVTTQC   0wx2f
        | DIVSSUD  0wx5ec
        | DIVTSUD  0wx5c3
        | MULSSUD  0wx5c2
        | MULTSUD  0wx5e2
        | SUBSSUD  0wx5c1
        | SUBTSUD  0wx5e1
 
   <font color=#FF0000>datatype</font> osf_user_palcode! = 
      BPT 0x80 | BUGCHK 0x81 | CALLSYS 0x83 
    | GENTRAP 0xaa | IMB 0x86 | RDUNIQUE 0x9e | WRUNIQUE 0x9f

   end (* Instruction *)
\end{SML}

<h3> <a name="encoding">
     Specifying the Instruction Encoding Formats </a></h3>

    The Alpha has very simple instruction encoding formats.

<tt>
\begin{SML}
   <font color=#FF0000>instruction formats 32 bits</font>
     Memory{opc:6, ra:5, rb:GP 5, disp: signed 16} (* p3-9 *)
      (* derived from Memory *) 
   | LoadStore{opc,ra,rb,disp} =
       <font color=#FF00000>let val</font> disp = 
           <font color=#FF00000>case</font> disp <font color=#FF0000>of</font>
             I.REGop rb => emit_GP rb
           | I.IMMop i  => itow i
           | I.HILABop le => itow(LabelExp.valueOf le)
           | I.LOLABop le => itow(LabelExp.valueOf le)
           | I.LABop le => itow(LabelExp.valueOf le)
           | I.CONSTop c => itow(Constant.valueOf c)
       in  Memory{opc,ra,rb,disp}
       end
   | ILoadStore{opc,r:GP,b,d} = LoadStore{opc,ra=r,rb=b,disp=d}
   | FLoadStore{opc,r:FP,b,d} = LoadStore{opc,ra=r,rb=b,disp=d}

   | Jump{opc:6,ra:GP 5,rb:GP 5,h:2,disp:int signed 14}   (* table C-3 *)
   | Memory_fun{opc:6, ra:GP 5, rb:GP 5, func:16}     (* p3-9 *)
   | Branch{opc:branch 6, ra:GP 5, disp:signed 21}           (* p3-10 *)
   | Fbranch{opc:fbranch 6, ra:FP 5, disp:signed 21}          (* p3-10 *)
        (* p3-11 *)
   | Operate0{opc:6,ra:GP 5,rb:GP 5,sbz:13..15,_:1=0,func:5..11,rc:GP 5} 
        (* p3-11 *)
   | Operate1{opc:6,ra:GP 5,lit:signed 13..20,_:1=1,func:5..11,rc:GP 5} 
   | Operate{opc,ra,rb,func,rc} =
        (<font color=#FF00000>case</font> rb <font color=#FF0000>of</font>
          I.REGop rb => Operate0{opc,ra,rb,func,rc,sbz=0w0}
        | I.IMMop i  => Operate1{opc,ra,lit=itow i,func,rc}
        | I.HILABop le => Operate1{opc,ra,lit=itow(LabelExp.valueOf le),func,rc}
        | I.LOLABop le => Operate1{opc,ra,lit=itow(LabelExp.valueOf le),func,rc}
        | I.LABop le => Operate1{opc,ra,lit=itow(LabelExp.valueOf le),func,rc}
        | I.CONSTop c => Operate1{opc,ra,lit=itow(Constant.valueOf c),func,rc}
        )
   | Foperate{opc:6,fa:FP 5,fb:FP 5,func:5..15,fc:FP 5}
   | Pal{opc:6=0,func:26}
\end{SML}
</tt>

\subsubsection{ Specifying the instruction set }
<tt>
\begin{SML}
   <font color=#FF0000>structure</font> MC =
   <font color=#FF0000>struct</font>
      (* compute displacement address *)
      <font color=#FF0000>fun</font> disp lab = itow(Label.addrOf lab - !loc - 4) ~>> 0w2
   <font color=#FF0000>end</font>

   (*
    * The main instruction set definition consists <font color=#FF0000>of</font> the following:
    *  1) constructor-like declaration defines the view <font color=#FF0000>of</font> the instruction,
    *  2) assembly directive in funny quotes `` '',
    *  3) machine encoding expression,
    *  4) semantics expression in [[ ]],
    *  5) delay slot directives etc (not necessary in this architecture!)
    *) 
   <font color=#FF0000>instruction</font>
     DEFFREG <font color=#FF0000>of</font> <font color=#008800>$</font>FP       (* define a floating point register *)
       ``deffreg <FP>''
        (* Pseudo instruction for the register allocator *)
 
   (* Load/Store *)
   | LDA <font color=#FF0000>of</font> {r: <font color=#008800>$</font>GP, b: <font color=#008800>$</font>GP, d:operand}       (* use of REGop is illegal *)
     ``lda\t<r>, <d>()''
     ILoadStore{opc=0w08,r,b,d}

   | LDAH <font color=#FF0000>of</font> {r: <font color=#008800>$</font>GP, b: <font color=#008800>$</font>GP, d:operand} (* use of REGop is illegal *)
     ``ldah\t<r>, <d>()''
     ILoadStore{opc=0w09,r,b,d}

   | LOAD <font color=#FF0000>of</font> {ldOp:load, r: <font color=#008800>$</font>GP, b: <font color=#008800>$</font>GP, d:operand, mem:Region.region}
     ``<ldOp>\t<r>, <d>()''
     ILoadStore{opc=emit_load ldOp,r,b,d}

   | STORE <font color=#FF0000>of</font> {stOp:store, r: <font color=#008800>$</font>GP, b: <font color=#008800>$</font>GP, d:operand, mem:Region.region}
     ``<stOp>\t<r>, <d>()''
     ILoadStore{opc=emit_store stOp,r,b,d}

   | FLOAD <font color=#FF0000>of</font> {ldOp:fload, r: <font color=#008800>$</font>FP, b: <font color=#008800>$</font>GP, d:operand, mem:Region.region}
     ``<ldOp>\t<r>, <d>()''
     FLoadStore{opc=emit_fload ldOp,r,b,d}

   | FSTORE <font color=#FF0000>of</font> {stOp:fstore, r: <font color=#008800>$</font>FP, b: <font color=#008800>$</font>GP, d:operand, mem:Region.region}
     ``<stOp>\t<r>, <d>()''
     FLoadStore{opc=emit_fstore stOp,r,b,d}
 
   (* Control Instructions *)
   | JMPL <font color=#FF0000>of</font> {r: <font color=#008800>$</font>GP, b: <font color=#008800>$</font>GP, d:int} * Label.label list
     ``jmpl\t<r>, <d>()''
     Jump{opc=0wx1a,h=0w0,ra=r,rb=b,disp=d}   (* table C-3 *)

   | JSR <font color=#FF0000>of</font> {r: <font color=#008800>$</font>GP, b: <font color=#008800>$</font>GP, d:int} * C.cellset * C.cellset
     ``jsr\t<r>, <d>()''
     Jump{opc=0wx1a,h=0w1,ra=r,rb=b,disp=d}

   | RET <font color=#FF0000>of</font> {r: <font color=#008800>$</font>GP, b: <font color=#008800>$</font>GP, d:int} 
     ``ret\t<r>, <d>()''
     Jump{opc=0wx1a,h=0w2,ra=r,rb=b,disp=d}

   | BRANCH <font color=#FF0000>of</font> branch * <font color=#008800>$</font>GP * Label.label   
     ``<branch> <GP>, <label>''
     Branch{opc=branch,ra=GP,disp=disp label}

   | FBRANCH <font color=#FF0000>of</font> fbranch * <font color=#008800>$</font>FP * Label.label  
     ``<fbranch> <FP>, <label>''
     Fbranch{opc=fbranch,ra=FP,disp=disp label}
 
   (* Integer Operate *)
   | OPERATE <font color=#FF0000>of</font> {oper:operate, ra: <font color=#008800>$</font>GP, rb:operand, rc: <font color=#008800>$</font>GP}
       ``<oper>\t<ra>, <rb>, <rc>''
        (let val (opc,func) = emit_operate oper
         in  Operate{opc,func,ra,rb,rc} 
         end)

   | OPERATEV <font color=#FF0000>of</font> {oper:operateV, ra: <font color=#008800>$</font>GP, rb:operand, rc: <font color=#008800>$</font>GP}
       ``<oper>\t<ra>, <rb>, <rc>''
        (let val (opc,func) = emit_operateV oper
         in  Operate{opc,func,ra,rb,rc} 
         end)

   | PSEUDOARITH <font color=#FF0000>of</font> {oper: pseudo_op, ra: <font color=#008800>$</font>GP, rb:operand, rc: <font color=#008800>$</font>GP, 
       	     tmps: C.cellset}
       ``<oper>\t<ra>, <rb>, <rc>''
 
   (* Copy instructions *)
   | COPY <font color=#FF0000>of</font> {dst: <font color=#008800>$</font>GP list, src: <font color=#008800>$</font>GP list, 
              impl:instruction list option ref, tmp: ea option}
       ``<app emitInstr (Shuffle.shuffle{regmap,tmp,dst,src})>''
   | FCOPY <font color=#FF0000>of</font> {dst: <font color=#008800>$</font>FP list, src: <font color=#008800>$</font>FP list, 
               impl:instruction list option ref, tmp: ea option}
       ``<app emitInstr (Shuffle.shufflefp{regmap,tmp,dst,src})>''
 
   (* Floating Point Operate *)
   | FOPERATE <font color=#FF0000>of</font> {oper:foperate, fa: <font color=#008800>$</font>FP, fb: <font color=#008800>$</font>FP, fc: <font color=#008800>$</font>FP}
       ``<oper>\t<fa>, <fb>, <fc>''
       (let val (opc,func) = emit_foperate oper
        in  Foperate{opc,func,fa,fb,fc}
        end)

   (* Trapping versions <font color=#FF0000>of</font> the above *)
   | FOPERATEV <font color=#FF0000>of</font> {oper:foperateV, fa: <font color=#008800>$</font>FP, fb: <font color=#008800>$</font>FP, fc: <font color=#008800>$</font>FP}
       ``<oper>\t<fa>, <fb>, <fc>''
        Foperate{opc=0wx16,func=emit_foperateV oper,fa,fb,fc}
 
   (* Misc *)
   | TRAPB       			(* Trap barrier *)
       ``trapb''
        Memory_fun{opc=0wx18,ra=31,rb=31,func=0wx0}
 
   | CALL_PAL <font color=#FF0000>of</font> {code:osf_user_palcode, def: <font color=#008800>$</font>GP list, use: <font color=#008800>$</font>GP list}
       ``call_pal <code>''
        Pal{func=emit_osf_user_palcode code}
 end
\end{SML}
</tt>


\subsection{ 4 Machine Descriptions }
Here are some machine descriptions in varing degree of completion.

\begin{itemize}
 \item \codehref{../sparc/sparc.mdl}{ Sparc } 
 \item \codehref{../hppa/hppa.mdl}{ Hppa } 
 \item \codehref{../alpha/alpha.mdl}{ Alpha }
 \item \codehref{../ppc/ppc.mdl}{ PowerPC } 
 \item \codehref{../x86/x86.mdl}{ X86 } 
\end{itemize}

\subsection{ Syntax Highlighting Macros }

\begin{itemize}
  \item \href{md.vim}{ For vim 5.3 }
\end{itemize}

</body>
</html>


\section{The Graph Library}

\subsection{Overview}

Graphs are the most fundamental data structure in the \MLRISC{} system,
and in fact in many optimizing compilers.
\MLRISC{} now contains an extensive library for working with graphs.

All graphs in \MLRISC{} 
are modeled as edge- and node-labeled directed multi-graphs.
Briefly, this means that nodes and edges can carry user supplied data, and  
multiple directed edges can be attached between any two nodes.
Self-loops are also allowed.

A node is uniquely identified by its \sml{node_id}, which is
simply an integer.  Node ids can be assigned externally 
by the user, or else generated automatically by a graph.  All graphs
keep track of all node ids that are currently used,
and the method \sml{new_id : unit -> node_id} generates a new unused id.

A node is modeled as a node id and node label pair, $(i,l)$.
An edge is modeled as a triple $i \edge{l} j$, which contains
the \newdef{source} and \newdef{target} node ids $i$ and $j$,
and the edge label $l$.  These types are defined as follows:
\begin{SML}
   type 'n node = node_id * 'n 
   type 'e edge = node_id * node_id * 'e
\end{SML}

\subsubsection{The graph signature}

All graphs are accessed through an abstract interface
of the polymorphic type \sml{('n,'e,'g) graph}.
Here, \sml{'n} is the type of the node labels, \sml{'e} is the type
of the edge labels, and \sml{'g} is the type of any extra information
embedded in a graph.  We call the latter \sml{graph info}.

Formally, a graph $G$ is a quadruple $(V,L,E,I)$
where $V$ is a set of node ids, $L : V -> 'a$ is a node labeling
function from vertices to node labels, $E$ is a multi-set
of labeled-edges of type $V * V * 'e$, and $I: 'g$
is the graph info.

The interface of a graph is packaged into a 
record of methods that manipulate the base representation:  
\begin{SML}
 signature \mlrischref{graphs/graph.sig}{GRAPH} = sig
   type node_id = int
   type 'n node = node_id * 'n 
   type 'e edge = node_id * node_id * 'e

   exception Graph of string
   exception Subgraph        
   exception NotFound        
   exception Unimplemented        
   exception Readonly        

   datatype ('n,'e,'g) graph = GRAPH of ('n,'e,'g) graph_methods
   withtype ('n,'e,'g) graph_methods = 
       \{  name            : string,
          graph_info      : 'g,
          (* selectors *)
          (* mutators *)
          (* iterators *)
       \}
 end
\end{SML}

A few exceptions are predefined in this signature, which have
the following informal interpretation.
Exception \sml{Graph} is raised when a bug is encountered.
The exception \sml{Subgraph} is raised if certain semantics constraints
imposed on a graph are violated.
The exception \sml{NotFound} is raised if lookup of a node id fails.
The exception \sml{Unimplemented} is raised if a certain feature
is accessed but is undefined on the graph.  The exception 
\sml{Readonly} is raised if the graph is readonly and an update operation
is attempted.

\subsubsection{Selectors}

Methods that access the structure of a graph are listed below:
\begin{methods}
   nodes : unit -> $'n$ node list &
       Return a list of all nodes in a graph em \\
    edges : unit -> $'e$ edge list &
       Return a list of all edges in a graph \\
    order : unit -> int &
       Return the number of nodes in a graph.  The graph is empty
       if its order is zero \\
    size : unit -> int &
       Return the number of edges in a graph \\
    capacity : unit -> int & 
       Return the maximum node id in the graph, plus 1. 
       This can be used as a new id  \\
    succ : node\_id -> node\_id list &
       Given a node id $i$, return the node ids of all its successors,
       i.e. $\{ j | i \edge{l} j \in E\}$. \\
    pred : node\_id -> node\_id list &
      Given a node id $j$, return the node ids of all its predecessors,
       i.e. $\{ i | i \edge{l} j \in E\}$. \\
    out\_edges : node\_id -> $'e$ edge list &
       Given a node id $i$, return all the out-going edges from node $i$, 
       i.e. all edges whose source is $i$. \\
    in\_edges : node\_id -> $'e$ edge list &
       Given a node id $j$, return all the in-coming edges from node $j$,
       i.e. all edges whose target is $j$. \\
    has\_edge : node\_id * node\_id -> bool &
       Given two node ids $i$ and $j$, find out if an edge 
       with source $i$ and target $j$ exists. \\
    has\_node : node\_id -> bool &
        Given a node id $i$, find out if a node of id $i$ exists. \\
    node\_info : node\_id -> $'n$ &
       Given a node id, return its node label.  If the node does not
       exist, raise exception \sml{NotFound}. \\
\end{methods}

\subsubsection{Graph hierarchy}

A graph $G$ may in fact be a subgraph of a \newdef{base graph} $G'$, or
obtained from $G'$ via some transformation $T$.
In such cases the following methods may be used to determine of the
relationship between $G$ and $G'$.  
An \newdef{entry edge} is an edge
in $G'$ that terminates at a node in $G$, but is not an edge in $G$.
Similarly, an \newdef{exit edge} is an edge in $G'$ that originates from
a node in $G$, but is not an edge in $G$.  An \newdef{entry node}
is a node in $G$ that has an incoming entry edge.  
An \newdef{exit node} is a node in $G$ that has an out-going exit edge.  
If $G$ is not
a subgraph, all these methods will return nil.
\begin{methods}
    entries : unit -> node\_id list &
        Return the node ids of all the entry nodes. \\
    exits : unit -> node\_id list &
        Return the node ids of all the exit nodes. \\
    entry\_edges : node\_id -> $'e$ edge list &
       Given a node id $i$, return all the entry edges whose sources are
       $i$. \\
    exit\_edges : node\_id -> $'e$ edge list &
       Given a node id $i$, return all the exit edges whose targets are $i$.
\end{methods}

\subsubsection{Mutators}

Methods to update a graph are listed below:  
\begin{methods}
   new\_id : unit -> node\_id &
     Return a unique node id guaranteed to be
     absent in the current graph. \\
   add\_node : 'n node -> unit &
     Insert node into the graph.  If a node of the same node id
     already exists, replace the old node with the new. \\
   add\_edge : 'e edge -> unit & 
     Insert an edge into the graph. \\
   remove\_node : node\_id -> unit &
     Given a node id $n$, remove the node with the node id from the graph.
     This also automatically removes all edges with source or target $n$. \\
   set\_out\_edges : node\_id * 'e edge list -> unit &
      Given a node id $n$, and a list of edges $e_1,\ldots,e_n$
      with sources $n$, replace all out-edges of $n$ by $e_1,\ldots,e_n$. \\
   set\_in\_edges : node\_id * 'e edge list -> unit &
      Given a node id $n$, and a list of edges $e_1,\ldots,e_n$ 
      with targets $n$, replace all in-edges of $n$ by $e_1,\ldots,e_n$. \\
   set\_entries : node\_id list -> unit &
      Set the entry nodes of a graph. \\
   set\_exits : node\_id list -> unit &
      Set the exit nodes of a graph. \\
   garbage\_collect : unit -> unit &
      Reclaim all node ids of nodes that have been removed by 
     \sml{remove_node}.  Subsequent \sml{new_id} will reuse these
      node ids.  \\
\end{methods}

\subsubsection{Iterators}

Two primitive iterators are supported in the graph interface. 
Method \sml{forall_nodes} iterates over all the nodes in a graph,
while method \sml{forall_edges} iterates over all the edges.
Other more complex iterators can be found in other modules. 
\begin{methods}
 forall\_nodes : ($'n$ node -> unit) -> unit &
    Given a function $f$ on nodes, apply $f$ on all the nodes in the graph. \\
 forall\_edges : ($'e$ edge -> unit) -> unit &
    Given a function $f$ on edges, apply $f$ on all the edges in the graph.
\end{methods}

\subsubsection{Manipulating a graph}
 
Since operations on the graph type are packaged into
a record, an ``object oriented'' style of graph manipulation should be used.
For example, if \sml{G} is a graph object, then we can obtain all the
nodes and edges of \sml{G} as follows.
\begin{SML}
 val GRAPH g = G
 val edges = #edges g ()
 val nodes = #nodes g ()
\end{SML}
We can view \sml{#edges g} as sending the message to \sml{G}.
While all this seems like mere syntactic deviation from the usual
 signature/structure approach, there are two crucial differences which
we will exploit:
\emph{(i)} records are first class objects 
while structures are not (consequently
late binding of ``methods'' and cannot be easily simulated on the
structure level); \emph{(ii)} recursion
is possible on the type level, while recursive structures are not available.
The extra flexibility of this choice becomes apparent with the
introduction of views later. 

\subsubsection{Creating a Graph}

A graph implementation has the following signature
\begin{SML}
 signature \mlrischref{graphs/graphimpl.sig}{GRAPH_IMPLEMENTATION} = sig
   val graph : string * 'g * int -> ('n,'e,'g) graph
 end
\end{SML}
The function \sml{graph} takes a string (the name of the graph),
graph info, and a default size as arguments and create an empty graph.

The functor \sml{DirectedGraph}:
\begin{SML}
 functor DirectedGraph(A : ARRAY_SIG) : GRAPH_IMPLEMENTATION
\end{SML}
implements a graph using adjacency lists as internal representation.
It takes an array type as a parameter.  For graphs with
node ids that are dense enumerations, the \sml{DynamicArray} structure
should be used as the parameter to this functor. 
The structure \sml{DirectedGraph} is predefined as follows:
\begin{SML}
 structure \mlrischref{graphs/digraph.sml}{DirectedGraph} = DirectedGraph(DynamicArray)
\end{SML}

For node ids that are sparse enumerations, the structure \sml{HashArray}, 
which implements integer-keyed hash tables
with the signature of arrays, can be used
as argument to \sml{DirectedGraph}.  
For graphs with fixed sizes determined at creation time,
the structure \sml{Array} can be used (see also 
functor \mlrischref{library/undoable-array.sml}{\sml{UndoableArray}},
which creates arrays with undoable updates, and transaction-like semantics.)

\subsubsection{Basic Graph Algorithms}

\subsubsection{Depth-/Breath-First Search}

\begin{SML}
   val dfs : ('n,'e,'g) graph  ->
             (node_id -> unit) ->
             ('e edge -> unit) ->
             node_id list -> unit
\end{SML}
   The function \sml{dfs} takes as arguments a graph,
a function \sml{f : node_id -> unit}, a function 
\sml{g : 'e edge -> unit}, and a
set of source vertices.  It performs depth first search on the
graph.  The function \sml{f} is invoked 
whenever a new node is being visited, while the function \sml{g}
is invoked whenever a new edge is being traversed.
This algorithm has running time $O(|V|+|E|)$.

\begin{SML}
   val dfsfold : ('n,'e,'g) graph  ->
                 (node_id * 'a -> 'a) ->
                 ('e edge * 'b -> 'a) ->
                 node_id list -> 'a * 'b -> 'a * 'b
   val dfsnum :  ('n,'e,'g) graph  ->
                 (node_id * 'a -> 'a) ->
                 { dfsnum : int array, compnum : int array }
\end{SML}

   The function \sml{bfs} is similar to \sml{dfs}
except that breath first search is performed.
\begin{SML} 
   val bfs : ('n,'e,'g) graph  ->
             (node_id -> unit) ->
             ('e edge -> unit) ->
             node_id list -> unit
   val bfsdist : ('n,'e,'g) graph -> node_id list -> int array
\end{SML} 

\subsubsection{Preorder/Postorder numbering}
\begin{SML}
   val preorder_numbering  : ('n,'e,'g) graph -> int -> int array
   val postorder_numbering : ('n,'e,'g) graph -> int -> int array
\end{SML}  
   Both these functions take a tree $T$ and a root $v$, and return
the preorder numbering and the postorder numbering of the tree respectively. 

\subsubsection{Topological Sort}
\begin{SML}
  val topsort : ('n,'e,'g) graph -> node_id list -> node_id list
\end{SML}
   The function 
\sml{topsort} takes a graph $G$ and a set of source vertices $S$
as arguments.  It returns a topological sort of all the nodes reachable from
the set $S$.  
This algorithm has running time $O(|S|+|V|+|E|)$.

\subsubsection{Strongly Connected Components}
\begin{SML}
 val strong_components : ('n,'e,'g) graph -> 
   (node_id list * 'a -> 'a) -> 'a -> 'a
\end{SML}
   The function \sml{strong_components} takes a graph $G$ and
an aggregate function $f$ with type 
\begin{SML}
  node_id list * 'a -> 'a
\end{SML}
\noindent and an identity element \sml{x : 'a} as arguments.  
Function $f$ is invoked with a strongly connected component 
(represented as a list of node ids) as each is discovered.   
That is, the function \sml{strong_components} computes 

\[ 
   f(SCC_n,f(SCC_{n-1},\ldots, f(SCC_1,x)))
\] 

where $SCC_1,\ldots,SCC_n$ are the strongly connected components
in topological order.  This algorithm has running time $O(|V|+|E|)$.

\subsubsection{Biconnected Components}
\begin{SML}
 val biconnected_components : ('n,'e,'g) graph -> 
        ('e edge list * 'a -> 'a) -> 'a -> 'a
\end{SML}
   The function \sml{biconnected_components} takes a graph $G$ and
an aggregate function $f$ with type 
\begin{SML}
  'e edge list * 'a -> 'a
\end{SML}
\noindent and an identity element \sml{x : 'a} as arguments.  
Function $f$ is invoked with a biconnected component 
(represented as a list of edges) as each is discovered.
That is, the function \sml{biconnected_components} computes 

\[
   f(BCC_n,f(BCC_{n-1},\ldots, f(BCC_1,x))) 
\]

where $BCC_1,\ldots,BCC_n$ are the biconnected components.
This algorithm has running time $O(|V|+|E|)$.

\subsubsection{Cyclic Test}
\begin{SML}
 val is_cyclic : ('n,'e,'g) graph -> bool
\end{SML}
Function \sml{is_cyclic} tests if a graph is cyclic.
This algorithm has running time $O(|V|+|E|)$.

\subsubsection{Enumerate Simple Cycles}
\begin{SML}
 val cycles : ('n,'e,'g) graph -> ('e edge list * 'a -> 'a) -> 'a ->'a
\end{SML}
  A simple cycle is a circuit that visits each vertex only once.
  The function \sml{cycles} enumerates all simple cycles in a graph $G$.
  It takes as argument an aggregate function $f$ of type 
  \begin{SML}
       'e edge list * 'a -> 'a
  \end{SML}
  and an identity element $e$, and computes as result the expression
  \[
     f(c_n,f(c_{n-1},f(c_{n-2},\ldots, f(c_1,e)))) 
  \]
  where $c_1,\ldots,c_n$ are all the simple cycles in the graph.   
  All cycles $c_1,\ldots,c_n$ are guaranteed to be distinct.  
  A cycle is represented as a sequence of
  adjacent edges, i.e. in the order of 
  \[ 
     v_1 -> v_2, v_2 -> v_3, v_3 -> v_4, \ldots, v_{n-1} -> v_n, v_n -> v_1 
  \]
  Our implementation works by first decomposing the graph into
  its strongly connected components, then uses backtracking to enumerate
  simple cycles in each component.
\subsubsection{Minimal Cost Spanning Tree}
\begin{SML}
 signature \mlrischref{graphs/spanning-tree.sig}{MIN_COST_SPANNING_TREE} = sig
   exception Unconnected

   val spanning_tree : \{ weight    : 'e edge -> 'a,
                         <         : 'a * 'a -> bool
                       \} -> ('n, 'e, 'g) graph
                         -> ('e edge * 'a -> 'a) -> 'a -> 'a
 end
 structure \mlrischref{graphs/kruskal.sml}{Kruskal} : MIN_COST_SPANNING_TREE
\end{SML}

Structure \sml{Kruskal} implements Kruskal's algorithm for
computing a minimal cost spanning tree of a graph.
The function \sml{spanning_tree} takes as arguments:
\begin{itemize}
\item a \sml{weight} function which when given an edge returns its weight
\item an ordering function \sml{<}, which is used to compare the weights
\item a graph $G$
\item an accumulator function $f$, and
\item an identity element $x$
\end{itemize}
The function \sml{spanning_tree} computes
\[
   f(e_{n},f(e_{n-1},\ldots, f(e_1,x))) 
\]

where $e_1,\ldots,e_n$ are the edges in a minimal cost spanning tree 
of the graph.
The exception \sml{Unconnected} is raised if the graph is unconnected.

\subsubsection{Abelian Groups}
  Graph algorithms that deal with numeric weights or distances
are parameterized with respect to the signatures
\sml{ABELIAN_GROUP} or \sml{ABELIAN_GROUP_WITH_INF}.
These are defined as follows:
\begin{SML}
 signature \mlrischref{graphs/groups.sig}{ABELIAN_GROUP} = sig 
   type elem 
   val +    : elem * elem -> elem
   val -    : elem * elem -> elem
   val      : elem -> elem
   val zero : elem
   val <    : elem * elem -> bool
   val ==   : elem * elem -> bool
 end
 signature \mlrischref{graphs/groups.sig}{ABELIAN_GROUP_WITH_INF} = sig
   include ABELIAN_GROUP
   val inf : elem
 end
\end{SML}
Signature \sml{ABELIAN_GROUP} specifies an ordered commutative group,
while signature \sml{ABELIAN_GROUP_WITH_INF} specifies an ordered commutative
group with an infinity element \sml{inf}. 

\subsubsection{Single Source Shortest Paths}
\begin{SML}
 signature \mlrischref{graphs/shortest-paths.sig}{SINGLE_SOURCE_SHORTEST_PATHS} = sig 
   structure Num : ABELIAN_GROUP_WITH_INF
   val single_source_shortest_paths :
                 \{ graph : ('n,'e,'g) graph,
                   weight : 'e edge -> Num.elem,
                   s : node_id
                 \} ->
                 \{ dist : Num.elem array,
                   pred :  node_id array
                 \}
 end
 functor \mlrischref{graphs/dijkstra.sml}{Dijkstra}(Num : ABELIAN_GROUP_WITH_INF) 
    : SINGLE_SOURCE_SHORTEST_PATHS
\end{SML}
The functor \sml{Dijkstra} implements Dijkstra's algorithm
for single source shortest paths.  The function \linebreak
\sml{single_source_shortest_paths} takes as arguments: 
\begin{itemize}
\item a graph $G$, 
\item a \sml{weight} function on edges, and
\item the source vertex $s$.
\end{itemize}
It returns two arrays \sml{dist} and \sml{pred}
indexed by vertices.  These arrays have the following
interpretation.  Given a vertex $v$,
\begin{itemize}
\item \sml{dist}[$v$] contains the distance of $v$ from the source $s$
\item \sml{pred}[$v$] contains the predecessor of $v$ in the shortest
path from $s$ to $v$, or -1 if $v=s$.
\end{itemize}

Dijkstra's algorithm fails to work on graphs that have
negative edge weights.  
To handle negative weights, Bellman-Ford's algorithm can be used. 
The exception \sml{NegativeCycle} is raised if a cycle of
negative total weight is detected.
\begin{SML}
 functor \mlrischref{graphs/bellman-ford.sml}{BellmanFord}(Num : ABELIAN_GROUP_WITH_INF) : sig
    include SINGLE_SOURCE_SHORTEST_PATHS
    exception NegativeCycle
 end
\end{SML}
\subsubsection{All Pairs Shortest Paths}
\begin{SML}
 signature \mlrischref{graphs/shortest-paths.sig}{ALL_PAIRS_SHORTEST_PATHS} = sig 
   structure Num : ABELIAN_GROUP_WITH_INF
   val all_pairs_shortest_paths :
                 \{ graph : ('n,'e,'g) graph,
                   weight : 'e edge -> Num.elem
                 \} ->
                 \{ dist : Num.elem Array2.array,
                   pred :  node_id Array2.array
                 \}
 end
 functor \mlrischref{graphs/floyd-warshall.sml}{FloydWarshall}(Num : ABELIAN_GROUP_WITH_INF) 
    : ALL_PAIRS_SHORTEST_PATHS
\end{SML}
The functor \sml{FloydWarshall} implements Floyd-Warshall's algorithm
for all pairs shortest paths.  The function 
\sml{all_pairs_shortest_paths} takes as arguments: 
\begin{itemize}
\item a graph $G$, and
\item a \sml{weight} function on edges
\end{itemize}
It returns two 2-dimensional arrays \sml{dist} and \sml{pred}
indexed by vertices $(u,v)$.  These arrays have the following
interpretation.  Given a pair $(u,v)$,
\begin{itemize}
\item \sml{dist}[$u,v$] contains the distance from $u$ to $v$.
\item \sml{pred}[$u,v$] contains the predecessor of $v$ in the shortest
path from $u$ to $v$, or $-1$ if $u=v$.
\end{itemize}
This algorithm runs in time $O(|V|^3+|E|)$.

An alternative implementation is available that uses Johnson's algorithm, 
which works better for sparse graphs:
\begin{SML}
 functor \mlrischref{graphs/johnson.sml}{Johnson}(Num : ABELIAN_GROUP_WITH_INF) 
    : sig include ALL_PAIRS_SHORTEST_PATHS
          exception Negative Cycle
      end
\end{SML}

\subsubsection{Transitive Closure}
\begin{SML}
 signature \mlrischref{graphs/trans-closure.sml}{TRANSITIVE_CLOSURE} = sig
    val acyclic_transitive_closure : {  + : ('e * 'e -> 'e), simple : bool }
        -> ('n,'e,'g) graph -> unit
    val acyclic_transitive_closure2 : 
       \{  + : 'e * 'e -> 'e,
          max : 'e * 'e -> 'e
       \}  -> ('n,'e,'g) graph -> unit
    val transitive_closure : ('e * 'e -> 'e) -> ('n,'e,'g) graph -> unit
 structure \mlrischref{graphs/trans-closure.sml}{TransitiveClosure} : TRANSITIVE_CLOSURE
\end{SML}
Structure \sml{TransitiveClosure} implements
in-place transitive closures on graphs.   Three functions are implemented.
Functions \sml{acyclic_transitive_closure} and 
\sml{acyclic_transitive_closure2} can be used
to compute the transitive closure of an acyclic graph, whereas the
function \sml{transitive_closure} computes the transitive closure of
a cyclic graph.  All take a binary function 
\begin{SML}
  + : 'e * 'e -> 'e
\end{SML}
defined on edge labels.  
Transitive edges are inserted in the following manner:

\begin{itemize}
 \item \sml{acyclic_transitive_closure}:
   given $u \edge{l} v$ and $v \edge{l'} w$, 
if the flag \sml{simple} is false or if 
the transitive edge $u \rightarrow w$ does not exists,
then $u \edge{l + l'} w$ is added to the graph.
 \item \sml{acyclic_transitive_closure2}:
   given $u \edge{l} v$ and $v \edge{l'} w$, 
the transitive $u \edge{l + l'} w$ is added to the graph.
  Furthermore, all parallel edges 
\[ 
   u \edge{l_1} w, \ldots, u \edge{l_n} w 
\]
are coalesced into a single edge $u \edge{l} w$, where 
$l = {\tt max}_{i = 1 \ldots n} l_i$ 
\end{itemize}

\subsubsection{Max Flow}

   The function \sml{max_flow} computes the
maximum flow between the source vertex \sml{s} and the sink vertex
\sml{t} in the \sml{graph} when given a \sml{capacity} function. 
\begin{SML}
 signature \mlrischref{graphs/max-flow.sig}{MAX_FLOW} = sig
   structure Num : ABELIAN_GROUP
   val max_flow : \{ graph    : ('n,'e,'g) graph,
                    s        : node_id, 
                    t        : node_id, 
                    capacity : 'e edge -> Num.elem, 
                    flows    : 'e edge * Num.elem -> unit
                  \} -> Num.elem
 end
 functor \mlrischref{graphs/max-flow.sml}{MaxFlow}(Num : ABELIAN_GROUP) : MAX_FLOW
\end{SML}
The function \sml{max_flow} returns its result in the follow manner:
The function returns the total flow as its result value.
Furthermore, the function \sml{flows} is called once for each edge $e$ in the
graph with its associated flow $f_e$.  

This algorithm uses Goldberg's preflow-push approach, and runs
in $O(|V|^2|E|)$ time.
\subsubsection{Min Cut}
   The function \sml{min_cut} computes the
minimum (undirected) cut in a \sml{graph} 
when given a \sml{weight} function on
its edges.  
\begin{SML}
 signature \mlrischref{graphs/min-cut.sig}{MIN_CUT} = sig
   structure Num : ABELIAN_GROUP
   val min_cut : \{ graph    : ('n,'e,'g) graph,
                   weight : 'e edge -> Num.elem
                 \} -> node_id list * Num.elem
 end
 functor \mlrischref{graphs/min-cut.sml}{MinCut}(Num : ABELIAN_GROUP) : MIN_CUT
\end{SML}
The function \sml{min_cut} returns a list of node ids denoting
one side of the cut $C$ (the other side of the cut is $(V - C)$ and
the weight cut.

\subsubsection{Max Cardinality Matching}

\begin{SML}
   val matching : ('n,'e,'g) graph -> ('e edge * 'a -> 'a) -> 'a -> 'a * int
\end{SML}

The function \sml{BipartiteMatching.matching} computes the
maximal cardinality matching of a bipartite graph.  As result, 
the function iterates over all the matched edges and returns the
number of matched edges.  The algorithm runs in time $O(|V||E|)$.

\subsubsection{Node Partition}
\begin{SML}
 signature NODE_PARTITION = sig 
   type 'n node_partition

   val node_partition : ('n,'e,'g) graph -> 'n node_partition
   val !!    : 'n node_partition -> node_id -> 'n node
   val ==    : 'n node_partition -> node_id * node_id -> bool
   val union : 'n node_partition -> ('n node * 'n node -> 'n node) ->
                                        node_id * node_id -> bool
   val union': 'n node_partition -> node_id * node_id -> bool

 end
\end{SML}

\subsubsection{Node Priority Queue}
\begin{SML}
 signature NODE_PRIORITY_QUEUE = sig 
   type node_priority_queue

   exception EmptyPriorityQueue

   val create         : (node_id * node_id -> bool) -> node_priority_queue
   val fromGraph      : (node_id * node_id -> bool) -> 
      ('n,'e,'g) graph -> node_priority_queue
   val isEmpty        : node_priority_queue -> bool
   val clear          : node_priority_queue -> unit
   val min            : node_priority_queue -> node_id
   val deleteMin      : node_priority_queue -> node_id
   val decreaseWeight : node_priority_queue * node_id -> unit
   val insert         : node_priority_queue * node_id -> unit
   val toList         : node_priority_queue -> node_id list
 end
\end{SML}

\subsection{Views}\label{sec:views}
Simply put, a view is an alternative presentation
of a data structure to a client.  A graph, such as the control flow
graph, frequently has to be presented in different ways in a compiler.  
For example, when global scheduling is applied on a region 
(a subgraph of the CFG),
we want to be able to concentrate on just the region and ignore all
nodes and edges that are not part of the current focus.  
All transformations that are applied on the current region view should be
automatically reflected back to the entire CFG as a whole.
Furthermore, we want to be able to freely intermix
graphs and subgraphs of the same type in our program, without having
to introducing sums in our type representations.

The \sml{subgraph_view} view combinator accomplishes this.  \sml{Subgraph}
takes a list of nodes and produces a graph object which is a view of the
node induced subgraph of the original graph.
All modification to the subgraph are automatically
reflected back to the original graph.  From the client point of view,
a graph and a subgraph are entirely indistinguishable, and furthermore,
graphs and subgraphs can be freely mixed together (they are the same
type from ML's point of view.)

This transparency is obtained by selective method overriding, composition,
and delegation.  For example, a generic graph object provides the
following methods for setting and looking up the entries and exits
from a graph.
\begin{SML}
   set_entries  : node_id list -> unit
   set_exits    : node_id list -> unit
   entries      : unit -> node_id list
   exits        : unit -> node_id list
\end{SML}

For example, a CFG usually has a single entry and a single exit.
These methods allow the client to destinate one node as the
entry and another as
the exit.  In the case of subgraph view, these methods are overridden so
that the proper conventions are preserved:
a node in a subgraph is an entry (exit) iff there is an in-edge (out-edge)
from (to) outside the (sub-)graph.
Similarly, the methods \sml{entry_edges} and \sml{exit_edges} can be used
return the entry and exit edges associated with a node in a subgraph.
\begin{SML}
   entry_edges  : node_id -> 'e edge list
   exit_edges   : node_id -> 'e edge list
\end{SML}
These methods are initially defined to return \sml{[]} in a graph and
subsequently overridden in a subgraph.

\subsubsection{ Update Transparency }

Suppose a view $G'$ is created from some base graphs or views.
\newdef{Update transparency} refers to the fact that 
$G'$ behaves
consistently according to its conventions and semantics when updates
are performed. There are 4 different type of update transparencies:
\begin{itemize}
\item\newdef{update opaque}  A update opaque view disallows updates to both
itself and its base graphs.
\item\newdef{globally update transparent} A globally update transparent
view allows updates to its base graphs but not to itself.  Changes
will then be automatically reflected in the view.
\item\newdef{locally update transparent}  A locally update transparent
view allows updates to itself but not to its base graphs.
Changes will be automatically reflected to the base graphs.
\item\newdef{fully update transparent}  A fully update transparent
view allows updates through its methods or through its base
graphs'.  
\end{itemize}

\subsubsection{Structural Views}\label{sec:structural-views}

\subsubsection{Reversal}
\begin{SML}
   val \mlrischref{graphs/revgraph.sml}{ReversedGraphView.rev_view} : ('n,'e,'g) graph -> ('n,'e,'g) graph
\end{SML}
   This combinator takes a graph $G$ and produces a view $G^R$
which reverses the direction
of all its edges, including entry and exit edges.  Thus 
the edge $i \edge{l} j$ in $G$ becomes the edge
$j \edge{l} i$ in $G^R$.  This view is fully update transparent.

\subsubsection{Readonly}
\begin{SML}
   val \mlrischref{graphs/readonly.sml}{ReadOnlyGraphView.readonly_view} : ('n,'e,'g) graph -> ('n,'e,'g) graph
\end{SML} 
  This function takes a graph $G$ and produces a view $G'$
in which no mutator methods can be used.  Invoking a mutator
method raises the exception \sml{Readonly}.
This view is globally update transparent.

\subsubsection{Snapshot}
\begin{SML}
   functor \mlrischref{graphs/snap-shot.sml}{GraphSnapShot}(GI : GRAPH_IMPLEMENTATION) : GRAPH_SNAPSHOT 
   signature GRAPH_SNAPSHOT = sig
      val snapshot : ('n,'e,'g) graph -> 
        \{ picture : ('n,'e,'g) graph, button : unit -> unit \}
   end
\end{SML}

The function \sml{snapshot} can be used to keep a cached copy
of a view a.k.a the \sml{picture}.    This cached copy
can be updated locally but the modification will not be reflected back
to the base graph.  The function \sml{button} can be used to
keep the view and the base graph up-to-date.

\subsubsection{Map}
\begin{SML}
   val \mlrischref{graphs/isograph.sml}{IsomorphicGraphView.map} :
     ('n node -> 'n') -> ('e edge -> 'e') -> ('g -> 'g') -> 
     ('n,'e,'g) graph -> ('n','e','g') graph
\end{SML}
The function \sml{map} is a generalization of the \sml{map}
function on lists.  It takes three functions 
\begin{SML}
f : 'n node -> 'n
g : 'e edge -> 'e
h : 'g -> g'
\end{SML}
and a graph $G=(V,L,E,I)$ as arguments.  
It computes the view $G'=(V,L',E',I')$ where
\begin{eqnarray*}
  L'(v) & = & f(v,L(v)) \mbox{\ for all $v \in V$} \\
  E'    & = & { i \edge{g(i,j,l)} j | i \edge{l} j \in E } \\
  I'    & = & h(I) 
\end{eqnarray*}

\subsubsection{Singleton}
\begin{SML}
   val \mlrischref{graphs/singleton.sml}{SingletonGraphView.singleton_view} : ('n,'e,'g) graph -> node_id -> ('n,'e,'g) graph
\end{SML}
Function \sml{singleton_view} 
takes a graph $G$ and a node id $v$ (which must exists in $G$)
and return an edge-free graph with only one node ($v$).
This view is opaque.

\subsubsection{Node id renaming}
\begin{SML}
   val \mlrischref{graphs/renamegraph.sml}{RenamedGraphView.rename_view} : int -> ('n,'e,'g) graph -> ('n','e','g') graph
\end{SML}
The function \sml{rename_view} takes an integer $n$ and
a graph $G$ and create a fully update transparent
view where all node ids are incremented by $n$.  Formally,
given graph $G=(V,E,L,I)$ it computes the view $G'=(V',E',L',I)$
where
\begin{eqnarray*}
   V' & = & { v + n | v \in V } \\
   E' & = & { i+n \edge{l} j+n | i \edge{l} j \in E } \\
   L' & = & \lambda v. L(v-n) 
\end{eqnarray*}

\subsubsection{Union and Sum}
\begin{SML}
   val \mlrischref{graphs/uniongraph.sml}{UnionGraphView.union_view} : ('g * 'g') -> 'g'') ->
      ('n,'e,'g) graph * ('n,'e,'g') graph -> ('n','e','g'') graph
   GraphCombinations.unions : ('n,'e,'g) graph list -> ('n,'e,'g) graph
   GraphCombinations.sum : ('n,'e,'g) graph * ('n,'e,'g) graph -> ('n,'e,'g) graph
   GraphCombinations.sums : ('n,'e,'g) graph list -> ('n,'e,'g) graph
\end{SML}

Function \sml{union_view} takes as arguments
a function $f$, and two graphs
$G=(V,L,E,I)$ and $G'=(V',L',E',I')$, it computes the union $G+G'$ of
these graphs.  Formally, $G \union G'=(V'',L'',E'',I'')$ where
\begin{eqnarray*}
   V'' & = & V \union V' \\
   L'' & = & L \overrides L' \\
   E'' & = & E \union E' \\
   I'' & = & f(I,I')
\end{eqnarray*}

The function \sml{sum} constructs a \newdef{disjoint sum} of two
graphs.
\subsubsection{Simple Graph View}
\begin{SML}
  val \mlrischref{graphs/simple-graph.sml}{SimpleGraph.simple_graph} : (node_id * node_id * 'e list -> 'e) ->
   ('n,'e,'g) graph -> ('n,'e,'g) graph
\end{SML}
  Function \sml{simple_graph} takes a merge function $f$ 
  and a multi-graph $G$ as arguments and return a view in which
  all parallel multi-edges (edges with the same source and target) are combined
  into a single edge: i.e. any collection of multi-edges between
  the same source $s$ and target $t$ and with labels $l_1,\ldots,l_n$, 
  are replaced by the edge $s \edge{l_{st}} t$ in the view, where
  $l_{st} = f(s,t,[l_1,\ldots,l_n])$.  The function $f$ is assumed
  to satisfy the equality $l = f(s,t,[l])$ for all $l$, $s$ and $t$.

\subsubsection{No Entry or No Exit} 
\begin{SML}
  val \mlrischref{graphs/no-exit.sml}{NoEntryView.no_entry_view} : ('n,'e,'g) graph -> ('n,'e,'g) graph
  NoEntryView.no_exit_view : ('n,'e,'g) graph -> ('n,'e,'g) graph
\end{SML}

The function \sml{no_entry_view} creates a view in which
all entry edges (and thus entry nodes) are removed.   The function
\sml{no_exit_view} is the dual of this and creates a view in which
all exit edges are removed.  This view is fully update transparent.
It is possible to remove all entry and exit edges by composing these
two functions.

\subsubsection{Subgraphs} 
\begin{SML}
   val \mlrischref{graphs/subgraph.sml}{SubgraphView.subgraph_view} : node_id list -> ('e edge -> bool) -> 
     ('n,'e,'g) graph -> (n','e','g') graph
\end{SML}

   The function \sml{subgraph_view} takes as arguments a set of node ids
$S$, an edge predicate $p$ and a graph $G=(V,L,E,I)$.  It
returns a view in which only the visible nodes are $S$ and
the only visible edges $e$ are those that satisfy $p(e)$ and
with sources and targets in $S$.  $S$ must be a subset of $V$.

\begin{SML}
   val \mlrischref{graphs/subgraph-p.sml}{Subgraph_P_View.subgraph_p_view} : node_id list -> 
     (node_id -> bool) -> (node_id * node_id -> bool) ->
     ('n,'e,'g) graph -> ('n','e','g') graph
\end{SML}

   The function \sml{subgraph_view} takes as arguments a set of node ids
$S$, a node predicate $p$, an edge predicate $q$ and a graph $G=(V,L,E,I)$.  It
returns a view in which only the visible nodes $v$ are those 
in $S$ satisfying $p(v)$, and
the only visible edges $e$ are those that satisfy $q(e)$ and
with sources and targets in $S$.  $S$ must be a subset of $V$.

\subsubsection{Trace}
\begin{SML}
   val \mlrischref{graphs/trace-graph.sml}{TraceView.trace_view} : node_id list -> ('n,'e,'g) graph -> ('n','e','g') graph
\end{SML}

\begin{wrapfigure}{r}{3in}
  \begin{Boxit}
%  \psfig{figure=../pictures/eps/trace.eps,width=2.8in}
  \includegraphics[width=2.8in]{../pictures/pdf/trace}
  \end{Boxit}
  \label{fig:trace-view}
  \caption{A trace view}
\end{wrapfigure}
A \newdef{trace} is an acyclic path in a graph.
The function \sml{trace_view} takes a trace of node ids
$v_1,\ldots,v_n$ and a graph $G$ and 
returns a view in which only the nodes are visible.
Only the edges that connected two adjacent nodes on the trace, i.e. 
$v_i -> v_{i+1}$ for some $i = 1 \ldots n-1$ are considered be within
the view.  Thus if there is an edge $v_i -> v_j$ in $G$ where
$j \ne i+1$ this edge is not considered to be within the view --- it
is considered to be an exit edge from $v_i$ and an entry edge
from $v_j$ however.  Trace views can be used to construct a CFG region
suitable for trace scheduling \cite{trace-scheduling,bulldog}.   

Figure \ref{fig:trace-view} illustrates this concept graphically.
Here, the trace view is formed from the
nodes \sml{A, C, D, F} and \sml{G}.  The
solid edges linking the trace is visible within the view.  All other
dotted edges are considered to be either entry of exit edges into
the trace.  The edge from node \sml{G} to \sml{A} is considered to
be both since it exits from \sml{G} and enters into \sml{A}.

\subsubsection{Acyclic Subgraph}
\begin{SML}
   val \mlrischref{graphs/acyclic-graph.sml}{AcyclicSubgraphView.acyclic_view} : 
     node_id list -> 
     ('n,'e,'g) graph -> ('n,'e,'g) graph
\end{SML}
\begin{wrapfigure}{r}{3in}
  \begin{Boxit}
  \includegraphics[width=2.8in]{../pictures/pdf/subgraph}
%  \psfig{figure=../pictures/eps/subgraph.eps,width=2.8in}
  \end{Boxit}
  \label{fig:acyclic-subgraph-view}
  \caption{An acyclic subgraph}
\end{wrapfigure}
The function \sml{acyclic_view} takes an ordered
list of node ids $v_1,\ldots,v_n$ and a graph $G$ as arguments
and return a view $G'$ such that only the nodes $v_1,\ldots,v_n$
are visible.  In addition, only the edges with directions consistent
with the order list are considered to be within the view.
Thus an edge $v_i -> v_j$ from $G$ is in $G'$ iff $1 \le i < j \le n$.
Acyclic views can be used to construct a CFG region suitable
for DAG scheduling.
Figure \ref{fig:acyclic-subgraph-view} illustrates this concept graphically.

\subsubsection{Start and Stop}
\begin{SML}
   val \mlrischref{graphs/start-stop.sml}{StartStopView.start_stop_view} :
     \{ start : 'n node,
        stop  : 'n node,
        edges : 'e edge list
     \} -> ('n,'e,'g) graph -> ('n','e','g') graph
\end{SML}

The function \sml{start_stop_view}

\subsubsection{Single-Entry/Multiple-Exits}
\begin{SML}
   \mlrischref{graphs/SEME.sml}{SingleEntryMultipleExit.SEME}
     exit : 'n node -> ('n,'e,'g) graph -> ('n,'e,'g) graph
\end{SML}

The function \sml{SEME} converts a single-entry/multiple-exits 
graph $G$ into a single entry/single exit graph.
It takes an exit node $e$ and a graph $G$ and returns
a view $G'$.  Suppose $i \edge{l} j$ is an exit edge in $G$.
In view $G$ this edge is replaced by a new normal edge $i \edge{l} e$
and a new exit edge $e \edge{l} j$.  Thus $e$ becomes the sole exit
node in the new view.  

\subsubsection{Behavioral Views}

\subsubsection{Behavioral Primitives}

Figure \ref{fig:behavioral-view-primitives} lists
the set of behavioral primitives defined
in structure \mlrischref{graphs/wrappers.sml}{\sml{GraphWrappers}}.  
These functions allow the user
to attach an action $a$ to a mutator method $m$ such that whenever $m$
is invoked so does $a$.  Given a graph $G$, the combinator 
\begin{SML}
   do_before_\(xxx\) : f -> ('n,'e,'g) graph -> ('n,'e,'g) graph
\end{SML}
\noindent returns a view $G'$ such that whenever method $xxx$ is invoked
in $G'$, the function $f$ is called. 
Similarly, the combinator 
\begin{SML}
   do_after_\(xxx\) : f -> ('n,'e,'g) graph -> ('n,'e,'g) graph
\end{SML}
\noindent creates a new view $G''$ such that the function $f$
is called after the method is invoked.
\begin{Figure}
\begin{boxit}
\begin{SML}
 do_before_new_id : (unit -> unit) -> ('n,'e,'g) graph -> ('n,'e,'g) graph
 do_after_new_id : (node_id -> unit) -> ('n,'e,'g) graph -> ('n,'e,'g) graph
 do_before_add_node : ('n node -> unit) -> ('n,'e,'g) graph -> ('n,'e,'g) graph
 do_after_add_node : ('n node -> unit) -> ('n,'e,'g) graph -> ('n,'e,'g) graph
 do_before_add_edge : ('e edge -> unit) -> ('n,'e,'g) graph -> ('n,'e,'g) graph
 do_after_add_edge : ('e edge -> unit) -> ('n,'e,'g) graph -> ('n,'e,'g) graph
 do_before_remove_node : (node_id -> unit) -> ('n,'e,'g) graph -> ('n,'e,'g) graph
 do_after_remove_node : (node_id -> unit) -> ('n,'e,'g) graph -> ('n,'e,'g) graph 
 do_before_set_in_edges : (node_id * 'e edge list -> unit) -> 
    ('n,'e,'g) graph -> ('n,'e,'g) graph
 do_after_set_in_edges : (node_id * 'e edge list -> unit) -> 
    ('n,'e,'g) graph -> ('n,'e,'g) graph
 do_before_set_out_edges : (node_id * 'e edge list -> unit) -> 
    ('n,'e,'g) graph -> ('n,'e,'g) graph
 do_after_set_out_edges : (node_id * 'e edge list -> unit) -> 
    ('n,'e,'g) graph -> ('n,'e,'g) graph
 do_before_set_entries : (node_id list -> unit) -> ('n,'e,'g) graph -> ('n,'e,'g) graph
 do_after_set_entries : (node_id list -> unit) -> ('n,'e,'g) graph -> ('n,'e,'g) graph
 do_before_set_exits : (node_id list -> unit) -> ('n,'e,'g) graph -> ('n,'e,'g) graph
 do_after_set_exits : (node_id list -> unit) -> ('n,'e,'g) graph -> ('n,'e,'g) graph
\end{SML}
\end{boxit}
\label{fig:behavioral-view-primitives} 
\caption{Behavioral view primitives}
\end{Figure}

Frequently it is not necessary to know precisely by which method a graph's
structure has been modified, only that it is.  The following two methods
take a notification function $f$ and returns a new view.  $f$ is
invoked before a modification is attempted in a view created
by \sml{do_before_changed}.  It is invoked after the modification in
a view created by \sml{do_after_changed}.
\begin{SML}
   do_before_changed : (('n,'e,'g) graph -> unit) -> ('n,'e,'g) graph -> ('n,'e,'g) graph
   do_after_changed : (('n,'e,'g) graph -> unit) -> ('n,'e,'g) graph -> ('n,'e,'g) graph
\end{SML}

Behavioral views created by the above functions are all fully update
transparent.

\section{The Graph Visualization Library}
\subsection{Overview}
Visualization is an important aid for debugging graph algorithms.
MLRISC provides a simple facility for displaying graphs that
adheres to the graph interface.  Two graph viewer 
back-ends are currently supported.  (An interface to the \emph{dot}
tool is still available but is unsupported.)
\begin{itemize}
 \item \externhref{http://www.cs.uni-sb.de/RW/users/sander/html/gsvcg1.html}{vcg} -- 
     this tool supports the browsing
 of hierarchical graphs, zoom in/zoom out functions.  It can
 handle up to around 5000 nodes in a graph.
 \item \externhref{http://www.Informatik.Uni-Bremen.DE/~davinci/}{daVinci} -- 
   this tool supports a separate
 ``survey'' view from the main view and text searching.  This tool is
slower than vcg but it has a nicer interface, and
 can handle up to around 500 nodes in a graph.
\end{itemize}
All graph viewing back-ends work in the same manner.  
They take a graph whose nodes and edges are annotated with \newdef{layout}
instructions and translate these layout instructions
into the target description language.  For vcg, the target description
language is GDL.  For daVinci, it is a language based on s-expressions.

\subsection{Graph Layout}
Some basic layout formats are defined structure \sml{GraphLayout} are:
\begin{SML}
 structure \mlrischref{visualization/graphLayout.sml}{GraphLayout} = struct
   datatype format =
     LABEL of string
   | COLOR of string
   | NODE_COLOR of string
   | EDGE_COLOR of string
   | TEXT_COLOR of string
   | ARROW_COLOR of string
   | BACKARROW_COLOR of string
   | BORDER_COLOR of string
   | BORDERLESS 
   | SHAPE of string 
   | ALGORITHM of string
   | EDGEPATTERN of string

   type ('n,'e,'g) style = 
      \{ edge  : 'e edge -> format list,
        node  : 'n node -> format list,
        graph : 'g -> format list
      \}
   type layout = (format list, format list, format list) graph
 end
\end{SML}

The interpretation of the layout formats are as follows:
\begin{center}
\begin{tabular}{|l|l|} \hline
   \sml{LABEL} $l$ &  Label a node or an edge with the string $l$ \\
   \sml{COLOR} $c$ &  Use color $c$ for a node or an edge \\
   \sml{NODE_COLOR} $c$ & Use color $c$ for a node  \\
   \sml{EDGE_COLOR} $c$ & Use color $c$ for an edge \\
   \sml{TEXT_COLOR} $c$ & Use color $c$ for the text within a node \\
   \sml{ARROW_COLOR} $c$ & Use color $c$ for the arrow of an edge \\
   \sml{BACKARROW_COLOR} $c$ & Use color $c$ for the arrow of an edge \\
   \sml{BORDER_COLOR} $c$ & Use color $c$ for the border in a node \\
   \sml{BORDERLESS} & Disable border for a node \\
   \sml{SHAPE} $s$ &  Use shape $s$ for a node \\
   \sml{ALGORITHM} $a$ & Use algorithm $a$ to layout the graph \\
   \sml{EDGEPATTERN} $p$ & Use pattern $p$ to layout an edge \\
\hline
\end{tabular}
\end{center}

Exactly how these formats are interpreted is determined by
the visualization tool that is used.    If a feature is unsupported
then the corresponding format will be ignored.
Please see the appropriate reference manuals of vcg and daVinci for details.

\subsection{Layout style}
How a graph is layout is determined by its \newdef{layout style}:
\begin{SML}
   type ('n,'e,'g) style = 
      \{ edge  : 'e edge -> format list,
        node  : 'n node -> format list,
        graph : 'g -> format list
      \}
\end{SML}
which is simply three functions that convert nodes, edges and graph
info into layout formats.
The function \sml{makeLayout} can be used to convert a 
layout style into a layout, which can then be passed to a graph
viewer to be displayed.
\begin{SML}
   GraphLayout.makeLayout : ('n,'e,'g) style -> ('n,'e,'g) graph -> layout
\end{SML}

\subsection{Graph Displays}

A \newdef{graph display} is an abstraction for the
interface that converts a layout graph into an external graph 
description language.  This abstraction is defined in the
signature below.
\begin{SML}
 signature \mlrischref{visualization/graphDisplay.sig}{GRAPH_DISPLAY} = sig
   val suffix    : unit -> string
   val program   : unit -> string
   val visualize : (string -> unit) -> GraphLayout.layout -> unit
 end
\end{SML}
\begin{itemize}
\item \sml{suffix} is the common file suffix used for the graph description
language 
\item \sml{program} is the common name of the graph visualization tool
\item \sml{visualize} is a function that takes a 
string output function and a layout graph $G$ as arguments
and generates a graph description based on $G$
\end{itemize}

\subsection{Graph Viewers}

The graph viewer functor 
\mlrischref{visualization/graphViewer.sml}{GraphViewer}
takes a graph display back-end and creates a graph viewer
that can be used to display any layout graph.

\begin{SML}
 signature \mlrischref{visualization/graphViewer.sig}{GRAPH_VIEWER} = sig
    val view : GraphLayout.layout -> unit
 end
 functor GraphViewer(D : GRAPH_DISPLAY) : GRAPH_VIEWER
\end{SML}

\section{Basic Compiler Graphs}

\subsection{Introduction}
In this section we describe the set of core compiler specific graphs and
algorithms implemented in MLRISC.
Mostly of these algorithms are parameterized with respect
to the actual intermediate representation, and as such they
do not provide many facilities that are provided by higher abstraction
layers, such as in \href{mlrisc-ir.html}{MLRISC IR}, 
or in \href{SSA.html}{SSA}.

\subsubsection{Dominator/Post-dominator Trees}
\newdef{Dominance}
is a fundamental concept in compiler optimizations.
Node $A$ $dominates$ $B$ 
iff all paths from the start node
to $B$ intersects A.  A dual notion is the concept of 
$post-dominance$:
$A$ \newdef{post-dominates} $B$ iff all paths from $B$ to the stop node
intersects $A$.  A (post-)dominator tree can be used
to summarize the dominance/post-dominance relationship.

\begin{SML}
 functor \mlrischref{ir/dominator.sml}{DominatorTree}
    (GraphImpl : GRAPH_IMPLEMENTATION) : DOMINATOR_TREE
\end{SML}
   The functor implements dominator analysis and 
creates a dominator/post-dominator tree from a graph $G$.  A dominator tree is implemented as a graph
with the following definition:
\begin{SML}
 signature \mlrischref{ir/dominator.sig}{DOMINATOR_TREE} = sig
    exception Dominator
    datatype 'n dom_node =
       DOM of \{ node : 'n, level : int, preorder : int, postorder : int \}
    type ('n,'e,'g) dom_info
    type ('n,'e,'g) dominator_tree = ('n dom_node,unit,('n,'e,'g) dom_info) graph
    type ('n,'e,'g) postdominator_tree = ('n dom_node,unit,('n,'e,'g) dom_info) graph
\end{SML}

We annotated each node in
a dominator tree with three extra fields of information, which
is useful for other algorithms:
\begin{itemize}
  \item\sml{level} is the nesting level of the tree.  The root
  node has level 0, children of the root has level 1 and so on.
  \item\sml{preorder} is the preorder numbering of a node
  \item\sml{preorder} is the postorder numbering of a node.
\end{itemize}

To create a dominator tree and a postdominator tree
from a graph, the following function should be called.
\begin{SML}
 val dominator_trees : ('n,'e,'g) graph ->
         ('n,'e,'g) dominator_tree * ('n,'e,'g) postdominator_tree
\end{SML}
We use the algorithm of Tarjan and Lengauer, which
runs in time $O(|V+E|\alpha(|V+E|))$ where $\alpha$ is the functional
inverse of the Ackermann function.

To perform many common queries on a dominator tree, we first
call the function \sml{methods} to obtain a method object.
\begin{SML} 
  val methods : ('n,'e,'g) dominator_tree -> dominator_methods
\end{SML}

The methods are packed into the following type:
\begin{SML}
   type dominator_methods =
         \{ dominates              : node_id * node_id -> bool,
           immediately_dominates  : node_id * node_id -> bool,
           strictly_dominates     : node_id * node_id -> bool,
           postdominates          : node_id * node_id -> bool,
           immediately_postdominates : node_id * node_id -> bool,
           strictly_postdominates : node_id * node_id -> bool,
           control_equivalent     : node_id * node_id -> bool,
           idom         : node_id -> node_id, $(* ~1 if none *)$
           idoms        : node_id -> node_id list,
           doms         : node_id -> node_id list,
           ipdom        : node_id -> node_id, $(* ~1 if none *)$
           ipdoms       : node_id -> node_id list,
           pdoms        : node_id -> node_id list,
           dom_lca      : node_id * node_id -> node_id,
           pdom_lca     : node_id * node_id -> node_id,
           dom_level    : node_id -> int,
           pdom_level   : node_id -> int,
           control_equivalent_partitions : unit -> node_id list list
         \}
\end{SML}

The query methods are as follows:
\begin{methods}
  dominates($a,b$)             & returns true iff $a$ dominates $b$ \\
  immediately\_dominates($a,b$) & returns true iff $a$ immediately dominates $b$ \\
  strictly\_dominates($a,b$)    & returns true iff $a$ strictly dominates $b$ \\
  postdominates($a,b$)            & returns true iff $a$ post-dominates $b$ \\
  immediately\_postdominates($a,b$) & returns true iff $a$ immediately post-dominates $b$ \\
  strictly\_postdominates($a,b$) & returns true iff $a$ strictly post-dominates $b$ \\
  control\_equivalent($a,b$) & 
  returns true iff $a$ dominates $b$ and vice versa \\ 
  idom($a$) & returns the immediate dominator of $a$, or $-1$ if none exists \\
  idoms($a$) & returns all nodes that $a$ immediately dominates \\
  doms($a$) & returns all nodes that $a$ dominates (including $a$ itself) \\
  ipdom($a$) & returns the immediate post-dominator of $a$, or $-1$ if none exists \\
  ipdoms($a$) & returns all nodes that $a$ immediately post-dominates \\
  pdoms($a$) & returns all nodes that $a$ post-dominates (including $a$ itself) \\
  dom\_lca($a,b$) & returns the least common ancestor of $a$ and $b$ in
  the dominator tree \\
  pdom\_lca($a,b$) & returns the least common ancestor of $a$ and $b$
  in the post-dominator tree \\
  dom\_level($a$) & returns the nesting level of $a$ in the dominator tree \\
  pdom\_level($b$) & returns the nesting level of $a$ in the post-dominator 
  tree \\
  control\_equivalent\_partitions & partitions the graph into
  a set of control equivalent nodes.
\end{methods}

The methods \sml{dom_lca}, \sml{pdom_lca} and 
\sml{control_equivalent_partitions} executes in $O(n)$ time, where
$n$ is the size of the dominator tree.  The other methods run in $O(1)$ time.

\subsubsection{Control Dependence Graph}
Given two nodes $A$ and $B$ in a control flow graph $G$, 
we say that $B$ is \newdef{control dependent} on $A$ iff
\begin{itemize}
  \item $B$ post-dominates a successor of $A$
  \item $B$ does not strictly post-dominates $A$
\end{itemize}
Intuitively, $B$ is control dependent on $A$ means that
some path in the program that goes through $A$ can by-passed $B$,
and furthermore, $A$ is the point in which this divergence can occur.
Control dependence is used to various kinds of analysis and optimizations in
a compiler, such as code motion and global scheduling~\cite{bernstein-rodeh}.

To build a control dependence graph, the functor
\sml{ControlDependenceGraph} can be used:
\begin{SML}
 signature \mlrischref{ir/cdg.sig}{CONTROL_DEPENDENCE_GRAPH} = sig
    type ('n,'e,'g) cdg = ('n,'e,'g) graph

    val control_dependence_graph :
          ('e -> bool) ->
          ('n,'e,'g) dominator_tree *
          ('n,'e,'g) postdominator_tree ->
          ('n,'e,'g) cdg
 end
 functor \mlrischref{ir/cdg.sml}{ControlDependenceGraph}
    (structure Dom : DOMINATOR_TREE
     structure GraphImpl : GRAPH_IMPLEMENTATION
    ) : CONTROL_DEPENDENCE_GRAPH
\end{SML}
The control depedence graph is a subcomponent of the
program dependence graph commonly used in
modern compiler optimizations.

\subsubsection{Dominance Frontiers}

Many algorithms involving the notion of control dependence or dominance
can be rephrased in terms of \newdef{dominance frontiers}.
A node $A$ is in the dominance frontiers of $B$ iff
$B$ dominates a predecessor of $A$ but $B$ does not strictly-dominate $A$.
We denote this as $A \in DF(B)$. 
The dual notion of \newdef{post-dominance frontiers} can be defined
analogously using the post-dominator tree\footnote{Control dependence
can be defined in terms of post-dominance frontiers.}.  

\begin{SML}
  functor \mlrischref{ir/dominance-frontier.sml}{DominanceFrontiers}(Dom : DOMINATOR_TREE) : DOMINANCE_FRONTIERS
\end{SML}
The functor \sml{DominanceFrontiers} can be used to
compute all the dominance frontiers of all the nodes in a graph.
It has the following signature. 

\begin{SML}
 signature \mlrischref{ir/dominance-frontier.sig}{DOMINANCE_FRONTIERS} = sig
   structure Dom : DOMINATOR_TREE
   type dominance_frontiers = node_id list array
   val DFs : ('n,'e,'g) Dom.dominator_tree -> dominance_frontiers
 end
\end{SML}

\subsubsection{Iterated Dominance Frontiers}

\newdef{Iterated dominance frontiers} (denoted as $DF^+$) are defined
as the least fixed point of iterating the operation $DF$. Formally,
define the dominance frontiers on a set $S$ as follows:
\[ 
   DF(S) \defas \Union_{A \in S} DF(A) 
\]
Define iteration of $DF$, denoted as $DF^n$, as follows:
\begin{eqnarray*}
  DF^1(S)     & \defas & DF(S) \\
  DF^{n+1}(S) & \defas & DF(S \union DF^n(S)) \\
\end{eqnarray*}
The iterated dominance frontiers $DF^+(S)$ on a set $S$ are defined as
the limit:
\[  
   DF^+(S) \defas \lim_{n \to \infty} DF^n(S) 
\]

Iterated dominance frontiers of a set $S$ can be computed in
time $O(|S|+|V|+|E|)$ using the 
algorithm by Sreedhar and Gao~\cite{linear-time-IDF}\footnote{
In practice it is often sub-linear in $|V|+|E|$.}.

\begin{SML}
  functor \mlrischref{ir/djgraph.sml}{DJGraph}(Dom : DOMINATOR_TREE) : DJ_GRAPH
\end{SML}
The functor \sml{DJGraph} implements this algorithm.
It satisfies the signature below:
\begin{SML}
 signature \mlrischref{ir/djgraph.sig}{DJ_GRAPH} = sig
    structure Dom : DOMINATOR_TREE
    type ('n,'e,'g) dj_graph = ('n,'e,'g) Dom.dominator_tree
    val dj_graph : ('n,'e,'g) dj_graph ->
        \{  DF   : node_id -> node_id list,
           IDF  : node_id -> node_id list,
           IDFs : node_id list -> node_id list
        \}
 end
\end{SML}
The function \sml{dj_graph} takes a dominator tree and returns
three query methods for computing dominance and iterated dominance frontiers.
Method \sml{DF} computes $DF(v)$ for a single node $v$.
Method \sml{IDF} computes the $DF^+(v)$, and method
\sml{IDFs} computes $DF^+(S)$ when given a set of node ids.
The dominator tree must not be updated while these operations
are being performed. 

Sreedhar's original algorithm is phrased in terms of the
DJ-graph, which is a fusion of the dominator tree
with its underlying flowgraph.  Our variant operates on the
dominator tree and the flowgraph at the same time, without
building an intermediate data structure.  

Iterated dominance frontiers are used
in many algorithms that deal with the notion of dominance.
For example, our SSA construction algorithm uses iterated
dominance frontiers to identify confluent points in the program
where $phi$-functions are to be placed.

\subsubsection{Loop Nesting Tree}

A \newdef{natural loop} $L$ in a graph is a maximal 
strongly connected component 
such that all nodes in $L$ are dominated by a single node $h$, called
the \newdef{loop header}.  Loops tend to form good optimization candidates
and consequently \newdef{loop detection} is an essential task in a compiler.
The functor 
\begin{SML}
 functor \mlrischref{ir/loop-structure.sml}{LoopStructure} 
  (structure GraphImpl : GRAPH_IMPLEMENTATION
   structure Dom       : DOMINATOR_TREE
  ) : LOOP_STRUCTURE 
\end{SML}
recognizes all natural loops in a graph and built a 
\newdef{loop nesting tree}
that describes the loop nesting relationship between graphs.

\begin{SML}
 signature \mlrischref{ir/loop-structure.sig}{LOOP_STRUCTURE} = sig
   structure Dom : DOMINATOR_TREE
   datatype ('n,'e,'g) loop =
      LOOP of \{ nesting    : int,
                header     : node_id,
                loop_nodes : node_id list,
                backedges  : 'e edge list,
                exits      : 'e edge list
              \}

   type ('n,'e,'g) loop_info
   type ('n,'e,'g) loop_structure = (('n,'e,'g) loop,unit, ('n,'e,'g) loop_info) graph

   val loop_structure : ('n,'e,'g) Dom.dominator_tree -> ('n,'e,'g) loop_structure
   val nesting_level : ('n,'e,'g) loop_structure -> node_id array
   val header        : ('n,'e,'g) loop_structure -> node_id array
 end
\end{SML}

Our algorithm computes the loop nesting tree in time 
$O((|V|+|E|)\alpha(|V|+|E|))$.
Each node in this tree represents a loop in the flowgraph, except the
root of the tree, which represents the entire graph.    
Given a flowgraph $G$, the root
of the loop nesting tree is defined to be the sole vertex in 
\sml{#entry} $G$.  Other nodes in the tree
are indexed by the loop header node ids.

Loop detection classifies each loop and for 
each loop $L$, the following information is obtained:
\begin{itemize}
 \item An integer \sml{nesting}.   The root of the tree has nesting
 depth 0.  The top level loops have nesting depth 1, etc.
 \item The node id of the loop \sml{header} $h$.
 \item A set of \sml{loop_nodes}.  Loop nodes are
  nodes that are in the strongly connected
  component $L$, but excluding the header $h$ 
  and all nodes that are part of any nested loops.
   Thus all nodes are uniquely partitioned in header nodes and
   loop nodes, and loop nodes are further partitioned into different
   sets according to which headers they are immediately nested under.
 \item A set of \sml{backedges}.  A back-edge is an
    edge that targets the header $h$ and originates from a loop node
    in $L$.
 \item A set of loop \sml{exits}. An exit-edge is an edge
   that originates from a loop node within $L$
   targets a node outside of $L$.  Note that this set does not include
   any exit-edges contained in loops nested in $L$ but 
   target a node out of $L$.
\end{itemize}

\subsubsection{Static Single Assignment}

An SSA construction algorithm based on~\cite{SSA,Briggs-SSA,linear-time-IDF}
is implemented in the following functor:
\begin{SML}
  functor \mlrischref{ir/ssa.sml}{StaticSingleAssignmentForm}
     (Dom : DOMINATOR_TREE) : STATIC_SINGLE_ASSIGNMENT_FORM
\end{SML}

SSA-based optimizations in MLRISC
are actually implemented on top of a
high-level SSA layer described in Section~\ref{sec:ssa}. 
So it is not necessary to use this module directly.  Nevertheless,
there can be situations in which this module can be specialized in other
ways; for example, in the construction of sparse evaluation graphs.

\begin{SML}
 signature \mlrischref{ir/ssa.sig}{STATIC_SINGLE_ASSIGNMENT_FORM} = sig
   structure Dom : DOMINATOR_TREE
   type var     = int 
   type phi  = var * var * var list $(* orig def/def/uses *)$
   type renamer = \{defs : var list, uses: var list\} ->
                  \{defs : var list, uses: var list\}
   type copy    = \{dst : var list, src: var list\} -> unit

   val compute_ssa : 
       ('n,'e,'g) Dom.dominator_tree ->
       \{ max_var      : var,  
         defs         : 'n node -> var list,
         is_live      : var * int -> bool,
         rename_var   : var -> var,
         rename_stmt  : \{rename:renamer,copy:copy\} -> 'n node -> unit,
         insert_phi   : \{block    : 'n node,
                         in_edges : 'e edge list,
                         phis     : phi list 
                        \} -> unit
       \} -> unit
 end
\end{SML}

This module defines the function \sml{compute_ssa}, which
constructs an SSA graph.  It requires 
the following information from the client:
\begin{itemize}
\item A dominator tree of the flowgraph.
\item \sml{max_var} -- the maximum variable id (integer) that exists
in the flowgraph.  All variables are assumed to be indexed by non-negative
 integers.
\item \sml{defs}($X$) -- a function that returns $defs(X)$, 
i.e.~the set of variable names defined in block $X$.
If a minimal SSA form is desired, this set should include all the definitions
in $X$.  If a pruned SSA form is required, this set should
include only the set of names that are live-out in $X$.
\item \sml{is_live}($v,X$) -- a function that determines if
variable $v$ is live-in into block $X$.  If not, a $\phi$-function will
not be placed in $X$.  For example, to compute
the minimal-SSA form, this function should always return true. 
\item \sml{rename_var}($v$) -- a function that returns a new 
unique name for variable $v$.   
\item \sml{rename_stmt} -- a function of type
       \sml{{rename:renamer,copy:copy} -> 'n node -> unit} where
\begin{SML}
   type renamer = \{defs : var list, uses: var list\} ->
                  \{defs : var list, uses: var list\}
   type copy    = \{dst : var list, src: var list\} -> unit
\end{SML}
Function \sml{rename_stmt} is called for each block
in the flowgraph in the order of the dominator tree, and
is responsible for renaming all the variables in $X$ by
calling the functions \sml{renamer} or \sml{copy}.
Function \sml{renamer} renames all definitions and uses of
a statement, while function \sml{copy} renames
of a set of parallel assignments
\item \sml{insert_phi}($X$,$es$,$phis$) --
   a function that inserts a set of 
   $\phi$-definitions $phis$ in block $X$, where $es$
   is the list of control flow edges that merge into block $X$.
\end{itemize}      

\subsubsection{IDEFS/IUSE sets}
Reif and Tarjan define the following useful notions for
computing approximate birth-points for expressions,  which in turn
can be used to drive other optimizations.
Given a node $X$, let $idom(X)$ denote the immediate dominator of $X$.
Let $def(X)$ ($use(X)$) denote all the definitions (uses) in $X$. 
Given a path $p \equiv v_1\ldots v_n$, define $def(p)$ ($use(p)$) as
\begin{eqnarray*}
   def(v_1\ldots v_n) & \equiv &\union_{i \in 1 \ldots n} def(v_i) \\
   use(v_1\ldots v_n) & \equiv &\union_{i \in 1 \ldots n} use(v_i)
\end{eqnarray*}

Let $P(X)$ denotes all the paths from $idom(X)$ to $X$
that does not cross $idom(X)$ internally.    Then define
$idef(X)$ ($iuse(X)$) as:
\begin{eqnarray*}
  idef(X) & \equiv & \Union_{idom(X) v_1 \ldots v_n X \in P(X)} 
     def(v_1\ldots v_n) \\
  iuse(X) & \equiv & \Union_{idom(X) v_1 \ldots v_n X \in P(X)} 
     use(v_1\ldots v_n) 
\end{eqnarray*}
The sets $ipostdef(X)$ and $ipostuse(X)$ are defined analogously
using the postdominator tree.

\begin{SML}
 signature \mlrischref{ir/idefs2.sig}{IDEFS} = sig
   type var = int
   val compute_idefs : 
       \{def_use : 'n Graph.node -> var list * var list,
        cfg     : ('n,'e,'g) Graph.graph
       \} ->
       \{ idefuse      : unit -> (RegSet.regset * RegSet.regset) Array.array,
         ipostdefuse  : unit -> (RegSet.regset * RegSet.regset) Array.array
       \}
 end
 structure \mlrischref{ir/idefs2.sml}{IDefs} : IDEFS
\end{SML}
Structure \sml{IDefs} implements the function 
\sml{comput_idefs} for computing
the $idef$, $iuse$, $ipostdef$ and $ipostuse$ sets of a control flow
graph.  It takes as arguments a flowgraph and a function \sml{def_use}, which
takes a graph node and returns the def/use sets of the node.
It returns two functions \sml{idefuse} and \sml{ipostdefuse} which
compute the $idef/iuse$ and $ipostdef/ipostuse$ sets.  These sets
are returned as arrays indexed by node ids.

\section{The MLRISC IR}
\subsection{Introduction}

In this section we will describe the MLRISC intermediate representation.

\subsubsection{Control Flow Graph}
The control flow graph is the main view of the IR.  
A control flow graph satisfies the following signature:
\begin{SML}
 signature \mlrischref{IR/mlrisc-cfg.sig}{CONTROL_FLOW_GRAPH} = sig
   structure I : INSTRUCTIONS
   structure P : PSEUDO_OPS
   structure C : CELLS
   structure W : FIXED_POINT 
      sharing I.C = C
   
   \italics{definitions}
 end
\end{SML}

The following structures nested within a CFG:
\begin{itemize}
   \item \sml{I : INSTRUCTIONS} is the instruction structure.
   \item \sml{P : PSEUDO_OPS} is the structure with the definition
of pseudo ops.
   \item \sml{C : CELLS} is the cells structure describing the
register conventions of the architecture.
   \item \sml{W : FIXED_POINT} is a structure that contains
a fixed point type used in execution frequency annotations.
\end{itemize}

The type \sml{weight} below is used in execution frequency annotations:
\begin{SML}
   type weight = W.fixed_point
\end{SML}

There are a few different kinds of basic blocks, described
by the type \sml{block_kind} below:
\begin{SML}
   datatype block_kind = 
       START          
     | STOP          
     | FUNCTION_ENTRY
     | NORMAL        
     | HYPERBLOCK   
\end{SML}

A basic block is defined as the datatype \sml{block}, defined below:
\begin{SML}
   and data = LABEL  of Label.label
            | PSEUDO of P.pseudo_op

   and block = 
      BLOCK of
      \{  id          : int,                      
         kind        : block_kind,                 
         name        : B.name,                    
         freq        : weight ref,                
         data        : data list ref,             
         labels      : Label.label list ref,     
         insns       : I.instruction list ref,     
         annotations : Annotations.annotations ref 
      \}
\end{SML}

Edges in a CFG are annotated with the type \sml{edge_info},
defined below:
\begin{SML}
   and edge_kind = ENTRY           
                 | EXIT           
                 | JUMP          
                 | FALLSTHRU     
                 | BRANCH of bool
                 | SWITCH of int 
                 | SIDEEXIT of int 

   and edge_info = 
       EDGE of \{ k : edge_kind,                 
                 w : weight ref,               
                 a : Annotations.annotations ref
               \}
\end{SML}

Type \sml{cfg} below defines a control flow graph:
\begin{SML}
   type edge = edge_info edge
   type node = block node

   datatype info = 
       INFO of \{ regmap      : C.regmap,
                 annotations : Annotations.annotations ref,
                 firstBlock  : int ref,
                 reorder     : bool ref
               \}
   type cfg = (block,edge_info,info) graph
\end{SML}

\subsubsection{Low-level Interface}
   The following subsection describes the low-level interface to a CFG.
These functions should be used with care since they do not
always maintain high-level structural invariants imposed on
the representation.  In general, higher level interfaces exist
so knowledge of this interface is usually not necessary for customizing
MLRISC. 
   
   Various kinds of annotations on basic blocks are defined below:
\begin{SML}
   exception LIVEOUT of C.cellset   
   exception CHANGED of unit -> unit
   exception CHANGEDONCE of unit -> unit
\end{SML}
The annotation \sml{LIVEOUT} is used record live-out information
on an escaping block.
The annotations \sml{CHANGED} and \sml{CHANGEDONCE} are used
internally for maintaining views on a CFG.  These should not be used
directly. 

    The following are low-level functions for building new basic blocks.
The functions \sml{new}\emph{XXX} build empty basic blocks of a specific
type.  The function \sml{defineLabel} returns a label to a basic block;
and if one does not exist then a new label will be generated automatically.
The functions \sml{emit} and \sml{show_block} are low-level
routines for displaying a basic block.
\begin{SML}
   val newBlock          : int * B.name -> block      
   val newStart          : int -> block              
   val newStop           : int -> block             
   val newFunctionEntry  : int -> block            
   val copyBlock         : int * block -> block   
   val defineLabel       : block -> Label.label  
   val emit              : C.regmap -> block -> unit
   val show_block        : C.regmap -> block -> string 
\end{SML}

   Methods for building a CFG are listed as follows:
\begin{SML}
   val cfg      : info -> cfg    
   val new      : C.regmap -> cfg
   val subgraph : cfg -> cfg     
   val init     : cfg -> unit    
   val changed  : cfg -> unit   
   val removeEdge : cfg -> edge -> unit
\end{SML}
 Again, these methods should be used only with care.

  The following functions allow the user to extract low-level information
from a flowgraph.  Function \sml{regmap} returns the current register map.
Function \sml{regmap} returns a function that lookups the current register
map.  Function \sml{liveOut} returns liveOut information from a block;
it returns the empty cellset if the block is not an escaping block.
Function \sml{fallsThruFrom} takes a node id $v$ and locates the
block $u$ (if any) that flows into $v$ without going through a branch
instruction.  Similarly, the function \sml{fallsThruTo}  takes
a node id $u$ and locates the block (if any) that $u$ flows into
with going through a branch instruction.  If $u$ falls through to
$v$ in any feasible code layout $u$ must preceed $v$.
\begin{SML}
   val regmap    : cfg -> C.regmap
   val reglookup : cfg -> C.register -> C.register
   val liveOut   : block -> C.cellset
   val fallsThruFrom : cfg * node_id -> node_id option
   val fallsThruTo   : cfg * node_id -> node_id option
\end{SML}

   To support graph viewing of a CFG, the following low-level
primitives are provided: 
\begin{SML}
   val viewStyle      : cfg -> (block,edge_info,info) GraphLayout.style
   val viewLayout     : cfg -> GraphLayout.layout
   val headerText     : block -> string
   val footerText     : block -> string
   val subgraphLayout : { cfg : cfg, subgraph : cfg } -> GraphLayout.layout
\end{SML}

   Finally, a miscellany function for control dependence graph building.
\begin{SML} 
   val cdgEdge : edge_info -> bool
\end{SML}

\subsubsection{IR}
The MLRISC intermediate representation is a composite
view of various compiler data structures, including the control
flow graph, (post-)dominator trees, control dependence graph, and
loop nesting tree.   Basic compiler optimizations in MLRISC
operate on this data structure; advance optimizations
operate on more complex representations which use this
representation as the base layer.  
\begin{wrapfigure}{r}{4.5in}
   \begin{Boxit}
   \psfig{figure=../pictures/eps/mlrisc-IR.eps,width=4.5in} 
   \end{Boxit}
   \caption{The MLRISC IR}
\end{wrapfigure}

This IR provides a few additional functionalities:
\begin{itemize}
  \item Edge frequencies -- execution frequencies
are maintained on all control flow edges.
  \item Extensible annotations -- semantics information can be 
       represented as annotations on the graph. 
  \item Multiple facets -- 
   Facets are high-level views that automatically keep themselves
up-to-date.  Computed facets are cached and out-of-date facets 
are recomputed by demand.
The IR defines a mechanism to attach multiple facets to the IR.
\end{itemize}

The signature of the IR is listed below
\begin{SML}
 signature \mlrischref{IR/mlrisc-ir.sig}{MLRISC_IR} = sig
   structure I    : INSTRUCTIONS
   structure CFG  : CONTROL_FLOW_GRAPH
   structure Dom  : DOMINATOR_TREE
   structure CDG  : CONTROL_DEPENDENCE_GRAPH
   structure Loop : LOOP_STRUCTURE
   structure Util : CFG_UTIL
      sharing Util.CFG = CFG
      sharing CFG.I = I 
      sharing Loop.Dom = CDG.Dom = Dom
  
   type cfg  = CFG.cfg  
   type IR   = CFG.cfg  
   type dom  = (CFG.block,CFG.edge_info,CFG.info) Dom.dominator_tree
   type pdom = (CFG.block,CFG.edge_info,CFG.info) Dom.postdominator_tree
   type cdg  = (CFG.block,CFG.edge_info,CFG.info) CDG.cdg
   type loop = (CFG.block,CFG.edge_info,CFG.info) Loop.loop_structure
 
   val dom   : IR -> dom
   val pdom  : IR -> pdom
   val cdg   : IR -> cdg
   val loop  : IR -> loop

   val changed : IR -> unit  
   val memo : (IR -> 'facet) -> IR -> 'facet
   val addLayout : string -> (IR -> GraphLayout.layout) -> unit
   val view : string -> IR -> unit      
   val views : string list -> IR -> unit      
   val viewSubgraph : IR -> cfg -> unit 
 end
\end{SML}

The following facets are predefined: dominator, post-dominator tree,
control dependence graph and loop nesting structure.
The functions \sml{dom}, \sml{pdom}, \sml{cdg}, \sml{loop}
are \newdef{facet extraction} methods that
compute up-to-date views of these facets.

The following protocol is used for facets:
\begin{itemize}
\item When the IR is changed, 
the function \sml{changed} should be called to 
signal that all facets attached to the IR should be updated.
\item To add a new facet of type \sml{F} that is computed by demand,
the programmer has to provide a facet construction 
function \sml{f : IR -> F}.  Call the function \sml{mem}
to register the new facet.  For example, let \sml{val g = memo f}. 
Then the function \sml{g} can be used to as a new facet extraction
function for facet \sml{F}.
\item To register a graph viewing function, call
the function \sml{addLayout} and provide an appropriate 
graph layout function.  For example, we can say
\sml{addLayout "F" layoutF} to register a graph layout function
for a facet called ``F''.
\end{itemize}

To view an IR, the functions \sml{view}, \sml{views} or
\sml{viewSubgraph} can be used.  They have the following interpretation:
\begin{itemize}
\item \sml{view} computes a layout for one facet of the IR and displays
it.  The predefined facets are called
``dom'', ``pdom'', ``cdg'', ``loop.''  The IR can be
viewed as the facet ``cfg.'' In addition, there is a layout
named ``doms'' which displays the dominator tree and the post-dominator
tree together, with the post-dominator inverted.
\item \sml{views} computes a set of facets and displays it together
in one single picture.
\item \sml{viewSubgraph} layouts a subgraph of the IR.
This creates a picture with the subgraph highlighted and embedded
in the whole IR.
\end{itemize}

\subsubsection{Building a CFG}

There are two basic methods of building a CFG:
\begin{itemize}
\item convert a sequence of machine instructions
into a CFG through the emitter interface, described below, and 
\item convert it from a \newdef{cluster}, which is
the basic linearized representation used in the MLRISC system.
\end{itemize}
The first method requires you to perform instruction selection
from a compiler front-end, but allows you to bypass all other
MLRISC phases if desired.  The second method allows you
to take advantage of various MLRISC's instruction selection modules
currently available.  We describe these methods in this section.

\paragraph{Directly from Instructions}
 Signature \sml{CODE_EMITTER} below describes an abstract emitter interface
for accepting a linear stream of instructions from a source 
and perform a sequence of actions based on this
stream\footnote{Unlike the signature {\tt EMITTER\_NEW} or 
{\tt FLOWGRAPH\_GEN}, it has the advantage that it is not 
tied into any form of specific flowgraph representation.}.  

\begin{SML}
 signature \mlrischref{extensions/code-emitter.sig}{CODE_EMITTER} = sig 
   structure I : INSTRUCTIONS
   structure C : CELLS
   structure P : PSEUDO_OPS
      sharing I.C = C

   type emitter =
   \{  defineLabel : Label.label -> unit,   
      entryLabel  : Label.label -> unit,   
      exitBlock   : C.cellset -> unit,    
      pseudoOp    : P.pseudo_op -> unit,  
      emitInstr   : I.instruction -> unit, 
      comment     : string -> unit,        
      init        : int -> unit,           
      finish      : unit -> unit   
   \} 
 end
\end{SML}

The code emitter interface has the following informal protocol. 
\begin{methods}
 init($n$)   & Initializes the emitter and signals that
               the back-end should 
               allocate space for $n$ bytes of machine code.
               The number is ignored for non-machine code back-ends. \\
 defineLabel($l$) & Defines a new label $l$ at the current position.\\
 entryLabel($l$)  & Defines a new entry label $l$ at the current position.  
 An entry label defines an entry point into the current flow graph.
 Note that multiple entry points are allowed\\
 exitBlock($c$) & Defines an exit at the current position. 
 The cellset $c$ represents the live-out information \\
 pseudOp($p$)  & Emits an pseudo op $p$ at the current position \\
 emitInstr($i$)  & Emits an instruction $i$ at the current position \\
 blockName($b$) & Changes the block name to $b$ \\
 comment($msg$) & Emits a comment $msg$ at the current position \\
 finish      & Signals that the use of the emitter is finished.
 The emitter is free to perform its post-processing functions.
 When this is finished the CFG is built. 
\end{methods}

The functor \sml{ControlFlowGraphGen} below can be
used to create a CFG builder that uses the \sml{CODE_EMITTER} interface.
\begin{SML}
 signature \mlrischref{IR/mlrisc-cfg-gen.sig}{CONTROL_FLOW_GRAPH_GEN} = sig
   structure CFG     : CONTROL_FLOW_GRAPH
   structure Emitter : CODE_EMITTER
       sharing Emitter.I = CFG.I
       sharing Emitter.P = CFG.P
   val emitter : CFG.cfg -> Emitter.emitter
 end
 functor \mlrischref{IR/mlrisc-cfg-gen.sml}{ControlFlowGraphGen}
    (structure CFG     : CONTROL_FLOW_GRAPH
     structure Emitter : CODE_EMITTER
     structure P       : INSN_PROPERTIES
         sharing CFG.I = Emitter.I = P.I
         sharing CFG.P = Emitter.P
         sharing CFG.B = Emitter.B
    ) : CONTROL_FLOW_GRAPH_GEN
\end{SML}

\paragraph{Cluster to CFG}

The core \MLRISC{} system implements many instruction selection
front-ends.  The result of an instruction selection module is a linear 
code layout block called a cluster.  The functor \sml{Cluster2CFG} below 
generates a translator that translates a cluster into a CFG:
\begin{SML}
 signature \mlrischref{IR/mlrisc-cluster2cfg.sig}{CLUSTER2CFG} = sig
   structure CFG : CONTROL_FLOW_GRAPH
   structure F   : FLOWGRAPH
      sharing CFG.I = F.I
      sharing CFG.P = F.P
      sharing CFG.B = F.B
   val cluster2cfg : F.cluster -> CFG.cfg
 end 
 functor \mlrischref{IR/mlrisc-cluster2cfg.sml}{Cluster2CFG}
   (structure CFG : CONTROL_FLOW_GRAPH 
    structure F   : FLOWGRAPH
    structure P   : INSN_PROPERTIES
       sharing CFG.I = F.I = P.I 
       sharing CFG.P = F.P
       sharing CFG.B = F.B
   ) : CLUSTER2CFG 
\end{SML}

\paragraph{CFG to Cluster}

The basic \MLRISC{} system also implements many back-end functions
such as register allocation, assembly output and machine code output.
These modules all utilize the cluster representation.  The 
functor \mlrischref{IR/mlrisc-cfg2cluster.sml}{CFG2Cluster} 
below generates a translator
that converts a CFG into a cluster.  With the previous functor,
the CFG and the cluster presentation can be freely inter-converted.
\begin{SML}
 signature \mlrischref{IR/mlrisc-cfg2cluster.sig}{CFG2CLUSTER} = sig
   structure CFG : CONTROL_FLOW_GRAPH
   structure F   : FLOWGRAPH
      sharing CFG.I = F.I
      sharing CFG.P = F.P
      sharing CFG.B = F.B
   val cfg2cluster : { cfg : CFG.cfg, relayout : bool } -> F.cluster
 end 
 functor \mlrischref{IR/mlrisc-cfg2cluster.sml}{CFG2Cluster}
   (structure CFG  : CONTROL_FLOW_GRAPH
    structure F    : FLOWGRAPH
       sharing CFG.I = F.I
       sharing CFG.P = F.P
       sharing CFG.B = F.B
    val patchBranch : {instr:CFG.I.instruction, backwards:bool} -> 
                         CFG.I.instruction list
   ) : CFG2CLUSTER
\end{SML}

When a CFG originates from a cluster, we try to preserve
the same code layout through out all optimizations when possible.
The function \sml{cfg2cluster} takes an optional flag 
that specifies we should force the recomputation of
the code layout of a control flow graph when translating a CFG
back into a cluster.

\subsubsection{Basic CFG Transformations}

Basic CFG transformations are implemented in the functor 
\sml{CFGUtil}.  These transformations include splitting edges, merging
edges, removing unreachable code and tail duplication.
\begin{SML}
   functor \mlrischref{IR/mlrisc-cfg-util.sml}{CFGUtil}
      (structure CFG : CONTROL_FLOW_GRAPH
       structure P   : INSN_PROPERTIES
          sharing P.I = CFG.I
      ) : CFG_UTIL
\end{SML}

The signature of \sml{CFGUtil} is defined below:
\begin{SML}
 signature \mlrischref{IR/mlrisc-cfg-util.sig}{CFG_UTIL} = sig
    structure CFG : CONTROL_FLOW_GRAPH
    val updateJumpLabel : CFG.cfg -> node_id -> unit
    val mergeEdge       : CFG.cfg -> CFG.edge -> bool
    val eliminateJump   : CFG.cfg -> node_id -> bool
    val insertJump      : CFG.cfg -> node_id -> bool
    val splitEdge  : CFG.cfg -> { edge : CFG.edge, jump : bool }
                      -> { edge : CFG.edge, node : CFG.node }
    val isMerge        : CFG.cfg -> node_id -> bool
    val isSplit        : CFG.cfg -> node_id -> bool
    val hasSideExits   : CFG.cfg -> node_id -> bool
    val isCriticalEdge : CFG.cfg -> CFG.edge -> bool
    val splitAllCriticalEdges : CFG.cfg -> unit
    val ceed : CFG.cfg -> node_id * node_id -> bool
    val tailDuplicate : CFG.cfg -> \{ subgraph : CFG.cfg, root : node_id \} 
                                -> \{ nodes : CFG.node list, 
                                     edges : CFG.edge list \} 
    val removeUnreachableCode : CFG.cfg -> unit
    val mergeAllEdges : CFG.cfg -> unit
 end
\end{SML}

These functions have the following meanings:
\begin{itemize}
  \item  \sml{updateJumpLabel} $G u$.  This function
     updates the label of the branch instruction in a block $u$
     to be consistent with the control flow edges with source $u$.  
     This is an nop if the CFG is already consistent. 
  \item \sml{mergeEdge} $G e$. This function merges edge 
        $e \equiv u \edge{} v$
        in the graph $G$ if possible.   This is successful only if
        there are no other edges flowing into $v$ and no other edges
        flowing out from $u$.  It returns true if the merge
        operation is successful.  If successful, the nodes $u$ and $v$
        will be coalesced into the block $u$.   The jump instruction (if any)
        in the node $u$ will also be elided.
  \item \sml{eliminateJump} $G u$.  This function eliminate the
      jump instruction at the end of block $u$ if it is feasible.
  \item \sml{insertJump} $G u$.  This function inserts a jump
       instruction in block $u$ if it is feasible.
  \item \sml{splitEdge} $G e$.  This function 
     split the control flow edge $e$, and return a new edge $e'$ and the 
     new block $u$ as return values.  It addition, it takes as
     argument a flag \sml{jump}.  If this flag is true, 
     then a jump instruction is always placed in the 
     split; otherwise, we try to eliminate the jump when feasible.
  \item \sml{isMerge} $G u$.  This function tests whether block $u$
          is a \newdef{merge} node.  A merge node is a node that
          has two or more incoming flow edges.
  \item \sml{isSplit} $G u$.  This function tests whether block $u$
           is a \newdef{split} node.  A split node is a node that
            has two or more outgoing flow edges.
  \item \sml{hasSideExits} $G u$.  This function tests whether
           a block has side exits $G$.  This assumes that $u$
           is a hyperblock.
  \item \sml{isCriticalEdge} $G e$.  This function tests whether
      the edge $e$ is a \newdef{critical} edge.  The 
       edge $e \equiv u \edge{} v$ is critical iff 
      there are $u$ is merge node and $v$ is a split node.
  \item  \sml{splitAllCriticalEdges} $G$.  This function goes
        through the CFG $G$ and splits
      all critical edges in the CFG.
      This can introduce extra jumps and basic blocks in the program.
  \item  \sml{mustPreceed} $G (u,v)$.   This function
      checks whether two blocks $u$ and $v$ are necessarily adjacent.
      Blocks $u$ and $v$ must be adjacent iff $u$ must preceed $v$
      in any feasible code layout.
  \item  \sml{tailDuplicate}.  
   \begin{SML}
    val tailDuplicate : CFG.cfg -> \{ subgraph : CFG.cfg, root : node_id \} 
                                -> \{ nodes : CFG.node list, 
                                     edges : CFG.edge list \} 
   \end{SML}
\begin{Figure}
\begin{boxit}
\cpsfig{figure=../pictures/eps/tail-duplication.eps,width=3in}
\end{boxit}
\label{fig:tail-duplication} 
\caption{Tail-duplication}
\end{Figure}

      This function tail-duplicates the region \sml{subgraph}
      until it only has a single entry \sml{root}.
      Return the set of new nodes and new edges.  
      The region is represented as a subgraph view of the CFG.
      Figure~\ref{fig:tail-duplication} illustrates 
      this transformation.

  \item  \sml{removeUnreachableCode} $G$. This function
          removes all unreachable code  from the graph.
  \item  \sml{mergeAllEdges} $G$.  This function tries to merge all
         the edges in the flowgraph $G$.  Merging is performed in the
         non-increasing order of edge frequencies. 
\end{itemize}

\subsubsection{Dataflow Analysis}
MLRISC provides a simple customizable module for performing
iterative dataflow analysis.   A dataflow analyzer
has the following signature:

\begin{SML}
 signature \mlrischref{IR/dataflow.sig}{DATAFLOW_ANALYZER} = sig
   structure CFG : CONTROL_FLOW_GRAPH
   type dataflow_info
   val analyze : CFG.cfg * dataflow_info -> dataflow_info
 end
\end{SML}

A dataflow problem is described by the signature \sml{DATAFLOW_PROBLEM}, 
described below:
\begin{SML}
 signature \mlrischref{IR/dataflow.sig}{DATAFLOW_PROBLEM} = sig
   structure CFG : CONTROL_FLOW_GRAPH
   type domain
   type dataflow_info
   val forward   : bool
   val bot       : domain
   val ==        : domain * domain -> bool
   val join      : domain list -> domain
   val prologue  : CFG.cfg * dataflow_info ->
                       CFG.block node ->
                           \{ input    : domain,
                             output   : domain,
                             transfer : domain -> domain
                           \}
   val epilogue  : CFG.cfg * dataflow_info ->
                       \{ node   : CFG.block node,
                         input  : domain,
                         output : domain
                       \} -> unit
 end
\end{SML}
This description contains the following items
\begin{itemize}
\item \sml{type domain} is the abstract lattice domain $D$.
\item \sml{type dataflow_info} is where the dataflow information
is stored.
\item \sml{forward} is true iff the dataflow problem is in the
forward direction
\item \sml{bot} is the bottom element of $D$.
\item \sml{==} is the equality function on $D$.
\item \sml{join} is the least-upper-bound function on $D$.
\item \sml{prologue} is a user-supplied function that performs
pre-processing and setup.  For each CFG node $X$, this function
computes
\begin{itemize}
 \item  \sml{input} -- which is the initial input value of $X$
 \item \sml{output} -- which is the initial output value of $X$
 \item \sml{transfer} -- which is the transfer function on $X$.
\end{itemize}
\item \sml{epilogue} is a function that performs post-processing.
It visits each node $X$ in the flowgraph and return the resulting
\sml{input} and \sml{output} value for $X$. 
\end{itemize}

To generate a new dataflow analyzer from a dataflow problem, 
the functor \sml{Dataflow} can be used:
\begin{SML}
 functor \mlrischref{IR/dataflow.sml}{Dataflow}(P : DATAFLOW_PROBLEM) : DATAFLOW_ANALYZER =
\end{SML}

\subsubsection{Static Branch Prediction} 

\subsubsection{Branch Optimizations}

\section{SSA Optimizations}\label{sec:ssa}

All SSA optimization modules satisfy the signature
\mlrischref{SSA/ssa-optimization.sig}{SSA\_OPTIMIZATION},
which is defined as:
\begin{SML}
signature SSA_OPTIMIZATION = sig
   structure SSA : SSA 

   val optimize : SSA.ssa -> SSA.ssa
end
\end{SML}

The following SSA based scalar optimizations have been implemented in MLRISC.
\begin{itemize}
\item \mlrischref{SSA/ssa-dead-code-elim.sml}{Dead code elimination}
\item \mlrischref{SSA/ssa-gvn.sml}{Global value numbering, constant folding, algebraic simplication}
\item \mlrischref{SSA/ssa-gcm.sml}{Global code motion} 
\item \mlrischref{SSA/ssa-cond-const-prop.sml}{Conditional constant propagation}
\item \mlrischref{SSA/ssa-op-str-red.sml}{Strength reduction}
\end{itemize}

\section{ILP Optimizations}
\subsection{Introduction}
    This section is under construction.  A new scheduler framework
for superscalars that ties into the machine description language
is currently being developed.
\subsection{The ILP ToolBox}
\subsubsection{List Scheduler}
\subsubsection{Ranking Algorithms}
   Some more complex ranking algorithms (than say critical path) have been
implemented.  These are:
\begin{itemize}
 \item The algorithm of
 \mlrischref{scheduling/PalemSimons.sig}{Palem and Simons} 
  which appeared in TOPLAS '93.  This algorithm
      computes the modified deadlines of a set instructions, with
      precedence, latency, and deadlines constraints.
      
 \item The algorithm of 
      \mlrischref{scheduling/LeungPalemPnueli.sig}{Leung, Palem, and Pnueli} 
       which appeared in PACT '98.
      This algorithm computes the modified deadlines of a set of instructions,
      with precedence, latency, release-times and deadline constraints.
\end{itemize}

\section{Optimizations for VLIW/EPIC Architectures}

\subsection{Overview}
Many newer architectures such as the upcoming IA-64 and the
DSPs such as the C6 are VLIW or so called EPIC machines.  
These architectures depends on the compiler to 
extract instruction level parallelism (\newdef{ILP})
and data level parallelism (\newdef{DLP}).

Optimizations for these architectures include:
\begin{itemize}
  \item Hyperblock construction
  \item Predication and predicate analysis
  \item Hyperblock scheduling
  \item Modulo scheduling
\end{itemize}

\subsection{Hyperblocks}
\subsection{Predicate Analysis}
\subsection{Hyperblock Scheduling}
\subsection{Modulo Scheduling}

\section{Register Allocator}

The MLRISC register allocator implements the iterated-coalescing algorithm
described in POPL '96 [George, Appel].  The details are described in these
papers
\begin{enumerate}
\item \externhref{http://cm.bell-labs.com/cm/cs/what/smlnj/compiler-notes/new-ra.ps}{A New MLRISC Register Allocator}
\end{enumerate}


\majorsection{Back Ends}
\section{The Alpha Back End}

\subsection{Trap Shadows, Floating Exceptions, and Denormalized Numbers on the DEC Alpha}

  \emph{By Andrew W. Appel and Lal George, Nov 28, 1995}

  See section 4.7.5.1 of the \emph{Alpha Architecture Reference Manual}.

  The Alpha has imprecise exceptions, meaning that if a floating
  point instruction raises an IEEE exception, the exception may
  not interrupt the processor until several successive instructions have
  completed.  ML, on the other hand, may want a "precise" model
  of floating point exceptions.

  Furthermore, the Alpha hardware does not support denormalized numbers
  (for ``gradual underflow'').  Instead, underflow always rounds to zero.
  However, each floating operation (add, mult, etc.) has a trapping
  variant that will raise an exception (imprecisely, of course) on
  underflow; in that case, the instruction will produce a zero result
  AND an exception will occur.  In fact, there are several variants
  of each instruction; three variants of MULT are:
\begin{description}
 \item[MULT  s1,s2,d]  truncate denormalized result to zero; no exception
 \item[MULT/U  s1,s2,d] truncate denormalized result to zero; raise UNDERFLOW
 \item[MULT/SU  s1,s2,d]  software completion, producing denormalized result
\end{description}

  The hardware treats the \verb|MULT/U| and \verb|MULT/SU| 
  instructions identically,
  truncating a denormalized result to zero and raising the UNDERFLOW
  exception.  But the operating system, on an UNDERFLOW exception,
  examines the faulting instruction to see if it's an \verb|/SU| 
  form, and if so,
  recalculates \verb|s1*s2|, puts the right answer in \verb|d|, and continues,
  all without invoking the user's signal handler.

  Because most machines compute with denormalized numbers in hardware,
  to maximize portability of SML programs, we use the \verb|MULT/SU| form.
  (and \verb|ADD/SU|, \verb|SUB/SU|, etc.)  But to use this form successfully,
  certain rules have to be followed.  Basically, d cannot be the same
  register as s1 or s2, because the opsys needs to be able to 
  recalculate the operation using the original contents of s1 and s2,
  and the MULT/SU instruction will overwrite d even if it traps.

  More generally, we may want to have a sequence of floating-point
  instructions.  The rules for such a sequence are:

  1. The sequence should end with a \verb|TRAPB| (trap barrier) instruction.
     (This could be relaxed somewhat, but certainly a \verb|TRAPB| would
      be a good idea sometime before the next branch instruction or
      update of an ML reference variable, or any other ML side effect.)
  2. No instruction in the sequence should destroy any operand of itself
     or of any previous instruction in the sequence.
  3. No two instructions in the sequence should write the same destination
     register.

  We can achieve these conditions by the following trick in the
  Alpha code generator.  Each instruction in the sequence will write
  to a different temporary; this is guaranteed by the translation from
  ML-RISC.  At the beginning of the sequence, we will put a special
  pseudo-instruction (we call it \verb|DEFFREG|) that ``defines'' 
   the destination
  register of the arithmetic instruction.  If there are $K$ arithmetic
  instructions in the sequence, then we'll insert $K$ 
   \verb|DEFFREG| instructions
  all at the beginning of the sequence.
  Then, each arithop will not only ``define'' its destination temporary
  but will ``use'' it as well.  When all these instructions are fed to
  the liveness analyzer, the resulting interference graph will then
  have inteference edges satisfying conditions 2 and 3 above.

  Of course, \verb|DEFFREG| doesn't actually generate any code.  In our model
  of the Alpha, every instruction generates exactly 4 bytes of code
  except the ``span-dependent'' ones.  Therefore, we'll specify \verb|DEFFREG|
  as a span-dependent instruction whose minimum and maximum sizes are zero.

  At the moment, we do not group arithmetic operations into sequences;
  that is, each arithop will be preceded by a single \verb|DEFFREG| and
  followed by a \verb|TRAPB|.  To avoid the cost of all those \verb|TRAPB|'s, 
  we should improve this when we have time.  Warning:  Don't put more 
  than 31 instructions in the sequence, because they're all required
  to write to different destination registers!  

  What about multiple traps?  For example, suppose a sequence of
  instructions produces an Overflow and  a Divide-by-Zero exception?
  ML would like to know only about the earliest trap, but the hardware
  will report \emph{BOTH} traps to the operating system.  However, as long
  as the rules above are followed (and the software-completion versions
  of the arithmetic instructions are used), the operating system will
  have enough information to know which instruction produced the
  trap.  It is very probable that the operating system will report \emph{ONLY}
  the earlier trap to the user process, but I'm not sure.

  For a hint about what the operating system is doing in its own
  trap-handler (with software completion), see section 6.3.2 of
  ``\emph{OpenVMS Alpha Software}'' (Part II of the Alpha Architecture
  Manual).  This stuff should apply to Unix (OSF1) as well as VMS.

\section{The PA RISC Back End}

No documentation yet.

\section{The Sparc Back End}

The Sparc back end can function in two different modes:
\begin{description}
  \item[Sparc V8]  This is V8 instruction set is used.  In this mode the processor
behaves like a 32-bit processor.   In this mode we assume we
have 16 floating point registers numbered \verb|%f0, %f2, %f4, ..., %f30|.
These are all in IEEE double precision.
  \item[Sparc V9]  This generates code assuming the V9 instruction set is used.
In this mode the processor functions at 64-bit.  In this mode the 
floating point processors can number from \verb|%f0, %f2, %f4, ..., %f62|.
These are all in IEEE double precision.

  New V9 instructions include the 64-bit extended version of multiplications,
divisions, shifts, and load and store.
\begin{verbatim}
    MULX SMULX DIVX SLLX SRLX SRAX LDX STX
\end{verbatim}

  Also, V9 includes conditional moves and more general form of branches.
\begin{description}
  \item[MOVcc]  conditional moves on condition code 
  \item[FMOVcc] conditional moves on condition code 
  \item[MOVR]   conditional moves on integer condition 
  \item[BR]     branch on integer register with prediction 
  \item[BP]     branch on integer condition with prediction 
\end{description}
\end{description}

\subsection{General Setup for V8}

 The SPARC architecture has 32 general purpose registers 
 (\verb|%g0| is always 0)
 and 32 single precision floating point registers.

 Some Ugliness: double precision floating point registers are
 register pairs.  There are no double precision moves, negation and absolute
 values.  These require two single precision operations.  I've created
 composite instructions \verb|FMOVd|, 
  \verb|FNEGd| and 
  \verb|FABSd| to stand for these.
 
 All integer arithmetic instructions can optionally set the condition
 code register.  We use this to simplify certain comparisons with zero
 in the instruction selection process.

 Integer multiplication, division and conversion from integer to floating
 go thru the pseudo instruction interface, since older sparcs do not
 implement these instructions in hardware.

 In addition, the trap instruction for detecting overflow is a parameter.
 This allows different trap vectors to be used.

\subsection{General Setup for V9}

\subsection{Specializing the Sparc Back End}

\section{The Intel x86 Back End}

No documentation yet.

\section{The PowerPC Back End}

No documentation yet.

\input nwmac
\filename{mipsglue.nw}
\begindocs{0}
\enddocs
\begindocs{1}
For the Mips, all comments are no-ops, so to get diagnostics in
assembly mode we have to slip in and define a new comment function.
Our life is further complicated by the fact that the stream on which
comments are to be written is bound late, so we have to save up comments
and then write them when asked to \code{}generate\edoc{}.
\enddocs
\begincode{2}
\moddef{*}\endmoddef
structure MipsAC : ASSEMBLER = struct
    val diag_out = ref std_out
    structure MCN = struct
        open MipsCoder
        structure M = struct
            open M
            fun comment s = output (!diag_out) s
        end
    end
        
    structure CM = MipsCM(MCN)
        
    structure Gen = CPScomp(CM)
    fun generate (lexp, stream) = (
        diag_out := stream;
        Gen.compile lexp;
        MipsCoder.codestats stream;
        Emitters.address := 0;
        MipsCoder.codegen (Emitters.MipsAsm stream);
        ())
end

\endcode
\begincode{3}
\moddef{*}\endmoddef
structure MipsCodeStats : ASSEMBLER = struct
    val diag_out = ref std_out
    structure MCN = MipsCoder
        
    structure CM = MipsCM(MCN)
        
    structure Gen = CPScomp(CM)
    fun generate (lexp, stream) = (
        Gen.compile lexp;
        MipsCoder.codestats stream;
        ())
end

\endcode
\begindocs{4}
Mips machines come in two byte-orders, so we need two of
every machinelike thing.
\enddocs
\begincode{5}
\moddef{*}\endmoddef
structure MipsMCBig : CODEGENERATOR = struct
    structure CM = MipsCM(MipsCoder)
    structure Gen = CPScomp(CM)
 
    fun generate lexp = (
        Gen.compile lexp;
        MipsCoder.codegen (Emitters.BigEndian);
        Emitters.emitted_string ()
        )
end

structure MipsMCLittle : CODEGENERATOR = struct
    structure CM = MipsCM(MipsCoder)
    structure Gen = CPScomp(CM)
fun diag (s : string) f x =
        f x handle e =>
                (print "?exception "; print (System.exn_name e);
                 print " in mipsglue."; print s; print "\\n";
                 raise e)
 
    fun generate lexp = (
        diag "Gen.compile" Gen.compile lexp;
        diag "MipsCoder.codegen" MipsCoder.codegen (Emitters.LittleEndian);
        diag "Emitters.emitted_string" Emitters.emitted_string ()
        )
end


structure CompMipsLittle = Batch(structure M=MipsMCLittle and A=MipsAC)
structure IntMipsLittle = IntShare(MipsMCLittle)

structure CompMipsBig = Batch(structure M=MipsMCBig and A=MipsAC)
structure IntMipsBig = IntShare(MipsMCBig)

structure CompMipsStats = Batch(structure M=MipsMCLittle and A=MipsCodeStats)
\endcode
\filename{emitters.nw}
\begindocs{0}
\section{Emitters}
We have an odd problem---we need to be able to emit either a 32-bit
integer or a string.
The order in which the bytes of the integer are emitted depends on
whether the target machine is BigEndian or LittleEndian, but the
bytes of the string should be emitted in the same order on both machines.
This means that the two emission functions depend on each other, but
in a machine-dependent way, so we bundle them up.
We also have to be able to emit two words for floating point constants.
The way to do this can be derived from \code{}emit_word\edoc{}, and this
code seems to be the sensible place to do that.
So we define a type \code{}emitter_triple\edoc{} 
and pass them around that way.

\enddocs
\begindocs{1}
Eventually we want to take all the words and strings that have been
emitted and bundle them up into a single string using \code{}implode\edoc{}.
We'll take the following tack with that:
each emitter will squirrel some info away in a reference variable.
A function \code{}emitted_string: unit -> string\edoc{} will take the
squirreled information and return a string that represents
everything emitted so far.
As a side effect, it will reset the emitter system to its initial 
state, where ``everything emitted so far'' is the empty string.

The actual implementation will be a list of strings which is reversed 
and imploded.
\enddocs
\begincode{2}
\moddef{signature}\endmoddef
signature EMITTERS = sig
    type emitter_triple
    val LittleEndian : emitter_triple
    val BigEndian : emitter_triple
    val emitted_string : unit -> string
    val MipsAsm : outstream -> emitter_triple
    val address : int ref
end

\endcode
\begindocs{3}
First something that's capable of emitting real code, then
something that can print assembly code.
We emit the assembly code to output right away, without any fooling.
\enddocs
\begincode{4}
\moddef{*}\endmoddef
structure Emitters : EMITTERS = struct
    type emitter_triple = (int * int -> unit) * (int -> string -> unit)
                                        * (string -> unit)
    local 
        \LA{}memory and basic services\RA{}
    in
        \LA{}string emitter\RA{}
        \LA{}little-endian emitter\RA{}
        \LA{}big-endian emitter\RA{}
        fun emitted_string () =
            let val s = implode (rev (!so_far))
            in  so_far := nil; s
            end
    end
    structure BigReal = MipsReal(struct val emit_word = emit_pair_big end)
    structure LittleReal =MipsReal(struct val emit_word = emit_pair_little end)
    val LittleEndian = (emit_pair_little,emit_string,LittleReal.realconst)
    val BigEndian = (emit_pair_big,emit_string,BigReal.realconst)
    \LA{}assembly-code emitters\RA{}
end


\endcode
\begindocs{5}
Here is a variable to remember what's been emitted so far
\enddocs
\begincode{6}
\moddef{memory and basic services}\endmoddef
val so_far = ref nil : string list ref
fun squirrel s = so_far := s :: !so_far
fun emit_byte n = squirrel (chr n)
\endcode
\begincode{7}
\moddef{string emitter}\endmoddef
fun emit_string n s = squirrel (substring(s,0,n))
                handle e =>
                        (print "?exception "; print (System.exn_name e);
                         print (" in emitters.emit_string "^
                                (Integer.makestring n) ^ " \\""^s^"\\"\\n");
                         raise e)

\endcode
\begindocs{8}
Little-endian means the little (least significant) end first,
like the VAX.
 We parameterize the real emitters by a function that emits a byte.
\enddocs
\begincode{9}
\moddef{little-endian emitter}\endmoddef
fun emit_pair_little(hi,lo) =
    let open Bits
        fun emit_word(n) =
          (emit_byte(andb(n,255));emit_byte(andb(rshift(n,8),255)))
    in  (emit_word(lo);emit_word(hi))
    end

\endcode
\begincode{10}
\moddef{big-endian emitter}\endmoddef
fun emit_pair_big(hi,lo) =
     let open Bits
        fun emit_word(n) =
          (emit_byte(andb(rshift(n,8),255));emit_byte(andb(n,255)))
    in  (emit_word(hi);emit_word(lo))
    end


\endcode
\begindocs{11}
Now the assembly code.
to make it easier to debug, we print out addresses in the same format
as {\tt dbx}: we use the byte address and we print it in hex.

We have to bend over backwards to handle real numbers

\enddocs
\begincode{12}
\moddef{assembly-code emitters}\endmoddef
val address = ref 0             (* address of next instruction in words *)
\LA{}real number state info and \code{}decode_real_ptr\edoc{}\RA{}
fun MipsAsm stream =
    let fun say s = (output stream s; flush_out stream)
        fun printaddr addrref = 
           let val n = !addrref
           in  (if n<10 then "  " else if n < 100 then " " else "") 
                ^ (Integer.makestring n) 
           end
        local 
            open Bits
            fun hexdigit n = 
                let val d = andb(n,15)
                in  if d <= 9 then chr(d+ord("0"))
                              else chr(d-10+ord("a"))
                end
            fun hex1 n = hexdigit(rshift(n,4))^hexdigit(n)
            fun hex2 n = hex1(rshift(n,8))^hex1(n)
            fun hex4 n = hex2(rshift(n,16))^hex2(n)
        in
            fun hex(hi,lo) = hex2(hi) ^ hex2(lo)
            fun printaddr addrref = 
                let val n = 4 * (!addrref)      (* address in bytes *)
                in "0x" ^ (hex4 n) 
           end
        end
        fun decode x = (
                say ((printaddr address) ^ ": (" ^ (hex x) ^") " 
                 ^ (MipsDecode.decode x));
           address := !address + 1; ()
           )
        fun do_decode_real(w,s) = (
                say ((printaddr address) ^ ": (" ^ (hex w) ^") " 
                 ^ s ^ "\\n");
           address := !address + 1; ()
           )
        fun decode_real s = (real_string := s; AsmReal.realconst s)
        fun decode_string n s =
            if n > 0 then
                (say ((printaddr address) 
                            ^ ": \\"" ^substring(s,0,4) ^"\\"\\n");
                   address := !address + 1;
                   decode_string (n-4) (substring(s,4,String.length(s)-4))
                   )
            else ()
    in
        decode_real_ptr := SOME do_decode_real;
        (decode,decode_string,decode_real) : emitter_triple
    end
\endcode
\begincode{13}
\moddef{real number state info and \code{}decode_real_ptr\edoc{}}\endmoddef
val decode_real_ptr = ref NONE : ((int * int) * string -> unit) option ref
                                   (* used to emit asm code for a real word *)
    
val real_least = ref NONE : (int * int) option ref
                                   (* least significant word of real *)
val real_string = ref ""
fun emit_real_word w =
    let val decode_real = case !decode_real_ptr of 
                  SOME f => f 
                | NONE => ErrorMsg.impossible "missed real decoder in mips asm"
    in
        case !real_least of 
            NONE => real_least := SOME w
          | SOME least => 
                (decode_real(least,"[low  word of "^(!real_string)^"]");
                 decode_real(w,"[high word of "^(!real_string)^"]"))
    end

structure AsmReal = MipsReal(struct val emit_word = emit_real_word end)

\endcode
\filename{mipsreal.nw}
\begindocs{0}
\subsection{Handling IEEE floating point constants}
Here we take care of converting floating point constants from
string representation to 64-bit IEEE representation.
We use the machinery developed for the Sparc by John Reppy.

Reppy's functor accepts a simple structure with a single value,
\code{}emitWord : int -> unit\edoc{}, which emits a 16-bit word.
It produces a \code{}PRIMREAL\edoc{}.
When \code{}RealConst\edoc{} is applied to the result, it produces a 
structure containing a single function, \code{}val realconst : string -> unit\edoc{}.
This function, when applied to a string, emits the four sixteen-bit words
of the IEEE representation, most significant first.

Our job will be to convert this to something that emits the two 32-bit
words of the constant, least significant word first.
First, let's consider the state information that has to be retained
while the halfwords are being emitted, and functions that change that state.
\enddocs
\begincode{1}
\moddef{state info}\endmoddef
val halfwords = ref nil : int list ref          (* halfwords already out *)
val count = ref 0                               (* length of halfwords *)
fun reset_state () = (halfwords := nil; count := 0)
fun add_half h = (count := !count + 1; halfwords := h :: (!halfwords))

\endcode
\begincode{2}
\moddef{emitting a halfword}\endmoddef
fun emit_half h = 
    if !count = 3 then (emit_four (h::(!halfwords)); reset_state())
    else add_half h

\endcode
\begindocs{3}
To emit the whole list, we have to emit the words, one at a time.
We use descriptive names to remind ourselves what is signficant
(highest is most significant).
\enddocs
\begincode{4}
\moddef{emitting four halfwords}\endmoddef
fun emit_four [lowest,low,high,highest] = 
                        (emit_word(low,lowest);emit_word(highest,high))
  | emit_four _ = ErrorMsg.impossible "bad floating pt constant in mips"

\endcode
\begindocs{5}
Now, we bundle up the whole thing in a functor that
gets passed a structure holding \code{}emit_word\edoc{} and returns 
one containing \code{}realconst\edoc{}.
\enddocs
\begincode{6}
\moddef{*}\endmoddef
functor MipsReal(E: sig val emit_word : int * int -> unit end) : REALCONST = 
struct
    open E
    \LA{}state info\RA{}
    \LA{}emitting four halfwords\RA{}
    \LA{}emitting a halfword\RA{}
    structure IEEERealConst =
        RealConst(IEEEReal(struct val emitWord = emit_half end))
    val realconst = IEEERealConst.realconst
end




\endcode
\filename{mips.nw}
\begindocs{0}
\section{Using \code{}MIPSCODER\edoc{} to implement a \code{}CMACHINE\edoc{}}

\enddocs
\begincode{1}
\moddef{*}\endmoddef
functor MipsCM(MipsC : MIPSCODER) : CMACHINE = struct

    open MipsC System.Tags

    \LA{}utility functions\RA{}

    \LA{}immediate and register functions\RA{}

    \LA{}register definitions\RA{}

    \LA{}move\RA{}
    \LA{}alignment, marks, and constants\RA{}
    \LA{}labels\RA{}
    \LA{}record manipulation\RA{}
    \LA{}indexed fetch and store (byte)\RA{}
    \LA{}indexed fetch and store (word)\RA{}
    \LA{}arithmetic\RA{}
    \LA{}shifts\RA{}
    \LA{}arithmetic and shifts with overflow detection\RA{}
    \LA{}bitwise operations\RA{}
    \LA{}branches\RA{}

    \LA{}floating point\RA{}

    \LA{}memory check\RA{}

    \LA{}omitted functions\RA{}

    val comment = M.comment

(* +DEBUG *)
    \LA{}DEBUG code\RA{}
(* -DEBUG *)

end (* MipsCM *)

\endcode
\begindocs{2}
The debugging code replaces possibly offensive functions with functions
that diagnose their own exceptions.
\enddocs
\begincode{3}
\moddef{DEBUG code}\endmoddef
fun diag (s : string) f x =
        f x handle e =>
                (print "?exception "; print (System.exn_name e);
                 print " in mips."; print s; print "\\n";
                 raise e)

\endcode
\begincode{4}
\moddef{immediate and register functions}\endmoddef
val immed = Immed
fun isimmed(Immed i) = SOME i
  | isimmed _ = NONE

fun isreg(Direct(Reg i)) = SOME i | isreg _ = NONE
fun eqreg (a: EA) b = a=b


\endcode
\begindocs{5}
Here's what our register conventions are:
\input regs
\enddocs
\begincode{6}
\moddef{register definitions}\endmoddef
val standardarg = Direct(Reg 2)
val standardcont = Direct(Reg 3)
val standardclosure = Direct(Reg 4)
val miscregs = map (Direct o Reg) [5,6,7,8,9,10,11,12,13,14,
                                   15,16,17,18,19]
val storeptr as Direct storeptr' = Direct(Reg 22)
val dataptr  as Direct dataptr'  = Direct(Reg 23)
val exnptr = Direct(Reg 30)

  (* internal use only *)
val my_arithtemp as Direct my_arithtemp'= Direct(Reg 20) 
val my_ptrtemp as Direct my_ptrtemp' = Direct(Reg 21)

  (* exported for external use *)
val arithtemp as Direct arithtemp' = Direct(Reg 24) 
val arithtemp2 as Direct arithtemp2'= Direct(Reg 25)

\endcode
\begincode{7}
\moddef{move}\endmoddef
fun move (src,Direct dest) = M.move(src, dest)
  | move _ = ErrorMsg.impossible "destination of move not register in mips"
\endcode
\begincode{8}
\moddef{alignment, marks, and constants}\endmoddef
val align = M.align
val mark = M.mark

val emitlong = M.emitlong
val realconst = M.realconst
val emitstring = M.emitstring

\endcode
\begincode{9}
\moddef{labels}\endmoddef
fun emitlab(i,Immedlab lab) = M.emitlab(i,lab)
  | emitlab _ = ErrorMsg.impossible "bad emitlab arg in mips"
fun newlabel() = Immedlab(M.newlabel())
fun define (Immedlab lab) = M.define lab
  | define _ = ErrorMsg.impossible "bad define arg in mips"
\endcode
\begincode{10}
\moddef{DEBUG code}\endmoddef
val emitlab = diag "emitlab" emitlab
val define = diag "define" define


\endcode
\begindocs{11}
We only ever put the address of a newly created record into a register.
If I make this out correctly, the first word on the list of
values \code{}vl\edoc{} is actually a descriptor.
BUGS: The original routine put the address of the descriptor
into \code{}z\edoc{}.  
What needs to go into \code{}z\edoc{} is the address of the first word in the record.
We can get this by adding 4 to the \code{}dataptr'\edoc{}.
\enddocs
\begincode{12}
\moddef{record manipulation}\endmoddef
fun record(vl, Direct z) =
    let open CPS
        val len = List.length vl
        fun f(i,nil) = ()
          | f(i,(r, SELp(j,p))::rest) = (* follow ptrs to get the item *)
                (M.lw(my_ptrtemp', r, j*4); f(i,(my_ptrtemp,p)::rest))
          | f(i,(Direct r,OFFp 0)::rest) =  (* simple store, last first *) 
                (M.sw(r, dataptr, i*4); f(i-1,rest))
          | f(i,(Direct r, OFFp j)::rest) = 
                (M.add(r, Immed(4*j), my_ptrtemp'); 
                                f(i,(my_ptrtemp,OFFp 0)::rest))
          | f(i,(ea,p)::rest) = (* convert to register-based *)
                (M.move(ea, my_ptrtemp'); f(i,(my_ptrtemp,p)::rest))
      in f(len - 1, rev vl); (* store first word in \code{}0(dataptr')\edoc{} *)
         M.add(dataptr', Immed 4, z);
         M.add(dataptr', Immed(4*len), dataptr')
     end
   | record _ = ErrorMsg.impossible "result of record not register in mips"

fun select(i, r, Direct s) = M.lw(s, r, i*4)
  | select _ = ErrorMsg.impossible "result of select not register in mips"

fun offset(i, Direct r, Direct s) = M.add(r,Immed(i*4), s)
  | offset _ = ErrorMsg.impossible "nonregister arg to offset in mips"
\endcode
\begincode{13}
\moddef{DEBUG code}\endmoddef
val record = diag "record" record
val select = diag "select" select
val offset = diag "offset" offset

\endcode
\begindocs{14}
For the indexed fetch and store, arithtemp is {\em not} tagged---the
tags are removed at a higher level (in {\tt generic.sml}).
These could be made faster for the case when they're called with immediate
constants as \code{}x\edoc{}.
\enddocs
\begincode{15}
\moddef{indexed fetch and store (byte)}\endmoddef
(* fetchindexb(x,y) fetches a byte: y <- mem[x+arithtemp]
        y cannot be arithtemp *)
fun fetchindexb(x,Direct y) =
    (M.add(arithtemp',x,my_arithtemp');    
     M.lbu(y,my_arithtemp,0))
  | fetchindexb _ = ErrorMsg.impossible "fetchb result not register in mips"

(* storeindexb(x,y) stores a byte: mem[y+arithtemp] <- x; *)
fun storeindexb(Direct x,y) =
    (M.add(arithtemp',y,my_arithtemp');
     M.sb(x,my_arithtemp,0))
  | storeindexb _ = ErrorMsg.impossible "storeb arg not register in mips"

(* jmpindexb(x)    pc <- (x+arithtemp) *)
fun jmpindexb x = (M.add(arithtemp',x,my_arithtemp');
                     M.jump(my_arithtemp'))

\endcode
\begincode{16}
\moddef{DEBUG code}\endmoddef
val fetchindexb = diag "fetchindexb" fetchindexb
val storeindexb = diag "storeindexb" storeindexb
val jmpindexb = diag "jmpindexb" jmpindexb


\endcode
\begindocs{17}
Here it looks like \code{}z\edoc{} is a tagged integer number of words,
so that \code{}2*(z-1)\edoc{} converts to the appropriate byte offset.
But I'm just guessing.
In any case, it saves an instruction to compute \code{}2*z\edoc{} (actually \code{}z+z\edoc{})
and 
load (or store) with offset \code{}~2\edoc{}.

Anything stored with \code{}storeindexl\edoc{} is being put into an array, so it
is safe to treat it as a pointer. 
\enddocs
\begincode{18}
\moddef{indexed fetch and store (word)}\endmoddef
   (* fetchindexl(x,y,z) fetches a word:   y <- mem[x+2*(z-1)] *)
   (* storeindexl(x,y,z) stores a word:    mem[y+2*(z-1)] <- x *)

fun fetchindexl(x,Direct y, Direct z) = 
      (M.sll(Immed 1,z,my_arithtemp');
       M.add(my_arithtemp',x,my_arithtemp');
       M.lw(y, my_arithtemp,~2))
  | fetchindexl(x,Direct y, Immed z) = M.lw(y, x, z+z-2)
  | fetchindexl _ = ErrorMsg.impossible "fetchl result not register in mips"

fun storeindexl(Direct x,y, Immed 1) = M.sw(x,y,0)
  | storeindexl(Direct x,y,Direct z) = 
    (M.sll(Immed 1,z,my_arithtemp');
     M.add(my_arithtemp',y,my_arithtemp');
     M.sw(x, my_arithtemp,~2))
  | storeindexl(Direct x,y,Immed z) = M.sw(x,y,z+z-2)

  | storeindexl(Direct _,_,Immedlab _) =
        ErrorMsg.impossible "storeindexl(Direct _,_,Immedlab _) in mips"

  | storeindexl(Immedlab label,y,z) =
    (M.move(Immedlab label,my_ptrtemp');
     storeindexl(my_ptrtemp,y,z))

  | storeindexl(Immed constant,y,offset) =
        (M.move(Immed constant,my_ptrtemp');
         storeindexl(my_ptrtemp,y,offset))

\endcode
\begincode{19}
\moddef{DEBUG code}\endmoddef
val fetchindexl = diag "fetchindexl" fetchindexl
val storeindexl = diag "storeindexl" storeindexl


\endcode
\begindocs{20}
The function \code{}three\edoc{} makes commutative three-operand
instructions easier to call.
All three operands become \code{}EA\edoc{}s, and it is enough if either of the
first two operands is a register.
\enddocs
\begincode{21}
\moddef{utility functions}\endmoddef
fun three f (Direct x, ea, Direct y) = f(x,ea,y)
  | three f (ea, Direct x, Direct y) = f(x,ea,y)
  | three f _ =ErrorMsg.impossible "neither arg to three f is register in mips"

\endcode
\begindocs{22}
I assume that shifts are only ever done on arithmetic quantities,
not pointers, so that I am justified in using \code{}my_arithtemp'\edoc{} to
store intermediate values.  This is consistent with being unwilling
to shift things matching \code{}Immedlab _\edoc{}.
Appel agrees that pointers aren't shifted, as far as he can remember.
\enddocs
\begincode{23}
\moddef{shifts}\endmoddef
fun ashr(shamt, Direct op1, Direct result) = M.sra(shamt,op1,result)
  | ashr(shamt, Immed op1, Direct result) = 
        (M.move(Immed op1,my_arithtemp'); M.sra(shamt,my_arithtemp',result))
  | ashr _ = ErrorMsg.impossible "ashr args don't match in mips"
fun ashl(shamt, Direct op1, Direct result) = M.sll(shamt,op1,result)
  | ashl(shamt, Immed op1, Direct result) = 
        (M.move(Immed op1,my_arithtemp'); M.sll(shamt,my_arithtemp',result))
  | ashl _ = ErrorMsg.impossible "ashl args don't match in mips"
\endcode
\begincode{24}
\moddef{DEBUG code}\endmoddef
val ashr = diag "ashr" ashr
val ashl = diag "ashl" ashl

\endcode
\begincode{25}
\moddef{bitwise operations}\endmoddef
val orb = three M.or
val andb = three M.and'
fun notb (a,b) = subl3(a, Immed ~1, b) (* ~1 - a == one's complement *)
val xorb = three M.xor
\endcode
\begincode{26}
\moddef{DEBUG code}\endmoddef
val orb = diag "orb" orb
val andb = diag "andb" andb
val notb = diag "notb" notb
val xorb = diag "xorb" xorb


\endcode
\begindocs{27}
Subtraction may appear a bit odd.
The MIPS machine instruction and  \code{}MIPSCODER.sub\edoc{} both subtract
their second operand from their first operand.
The VAX machine instruction and \code{}CMACHINE.subl3\edoc{} both subtract
their first operand from their second operand.
This will certainly lead to endless confusion.
\enddocs
\begincode{28}
\moddef{arithmetic}\endmoddef
val addl3 = three M.add

fun subl3(Immed k, x, y) = addl3(x, Immed(~k), y)
  | subl3(Direct x, Direct y, Direct z) = M.sub(y,x,z)
  | subl3(x, Immed k, dest) = 
            (M.move(Immed k, my_arithtemp');
             subl3(x, my_arithtemp, dest))
  | subl3 _ = ErrorMsg.impossible "subl3 args don't match in mips"

\endcode
\begindocs{29}
We assume that any quantities being multiplied are arithmetic
quantities, not pointers.
\enddocs
\begincode{30}
\moddef{arithmetic}\endmoddef
fun mull2(Direct x, Direct y) = M.mult(y,x,y)
  | mull2(Immed x, Direct y) = (M.move(Immed x,my_arithtemp');
                                M.mult(y,my_arithtemp',y))
  | mull2 _ = ErrorMsg.impossible "mull2 args don't match in mips"
fun divl2(Direct x, Direct y) = M.div(y,x,y)
  | divl2(Immed x, Direct y) = (M.move(Immed x,my_arithtemp');
                                M.div(y,my_arithtemp',y))
  | divl2 _ = ErrorMsg.impossible "divl2 args don't match in mips"

\endcode
\begincode{31}
\moddef{DEBUG code}\endmoddef
val addl3 = diag "addl3" addl3
val subl3 = diag "subl3" subl3
val mull2 = diag "mull2" mull2
val divl2 = diag "divl2" divl2


\endcode
\begindocs{32}
The Mips hardware detects two's complement integer overflow on 
add and subtract instructions only.  
The exception is not maskable (see the Mips book, page 5-18).
At the moment we don't implement any overflow detection for multiplications
or for left shifts.
This has consequences only for coping with real constants and for
compiling user programs.  
\enddocs
\begincode{33}
\moddef{arithmetic and shifts with overflow detection}\endmoddef
val addl3t = addl3
val subl3t = subl3
\endcode
\begindocs{34}
The Mips multiplies two 32-bit quantities to get a 64-bit result.
That result fits in 32 bits if and only if the high-order word is zero or
negative one, and it has the same sign as the low order word.
Thus, we can add the sign bit of the low order word to the high order
word, and we have overflow if and only if the result is nonzero.
\enddocs
\begincode{35}
\moddef{arithmetic and shifts with overflow detection}\endmoddef
fun mull2t(x,y as Direct y') = 
    let val ok = M.newlabel()
    in  mull2(x,y);
        M.mfhi(my_arithtemp');
        M.slt(y',Direct (Reg 0),my_ptrtemp'); (* 0 or 1 OK in pointer *)
        M.add(my_arithtemp',my_ptrtemp,my_arithtemp');
        M.beq(true,my_arithtemp',Reg 0,ok);    (* OK if not overflow *)
        M.lui(my_arithtemp',32767);
        M.add(my_arithtemp',my_arithtemp,my_arithtemp');  (* overflows *)
        M.define(ok)
    end
  | mull2t _ = ErrorMsg.impossible "result of mull2t not register in mips"

\endcode
\begincode{36}
\moddef{DEBUG code}\endmoddef
val addl3t = diag "addl3t" addl3t
val subl3t = diag "subl3t" subl3t
val mull2t = diag "mull2t" mull2t
val ashlt = diag "ashlt" ashlt


\endcode
\begindocs{37}
We hack \code{}ibranch\edoc{} to make sure it will only reverse once.
It's easier than thinking.
\enddocs
\begincode{38}
\moddef{branches}\endmoddef
datatype condition = NEQ | EQL | LEQ | GEQ | LSS | GTR
local 
fun makeibranch reverse = 
let
fun ibranch (cond, Immed a, Immed b, Immedlab label) =
            if (case cond of EQL => a=b | NEQ => a<>b | LSS => a<b |
                             LEQ => a<=b | GTR => a>b | GEQ => a>=b)
                then M.beq(true,Reg 0, Reg 0, label) else ()
  | ibranch (NEQ, Direct r, Direct s, Immedlab label) =
                    M.beq(false, r, s, label)
  | ibranch (NEQ, Direct r, x, Immedlab label) =
                    (M.move(x, my_arithtemp');
                     M.beq(false, r, my_arithtemp', label))
  | ibranch (EQL, Direct r, Direct s, Immedlab label) =
                    M.beq(true, r, s, label)
  | ibranch (EQL, Direct r, x, Immedlab label) =
                    (M.move(x, my_arithtemp');
                     M.beq(true, r, my_arithtemp', label))
  | ibranch (LSS, Direct r, x, Immedlab lab) =
                (M.slt(r,x,my_arithtemp');
                 M.beq(false,Reg 0, my_arithtemp',lab))
  | ibranch (GEQ, Direct r, x, Immedlab lab) =
                (M.slt(r,x,my_arithtemp'); 
                 M.beq(true,Reg 0, my_arithtemp',lab))
  | ibranch (GTR, x, Direct r, Immedlab lab) =
                (M.slt(r,x,my_arithtemp'); 
                 M.beq(false,Reg 0, my_arithtemp',lab))
  | ibranch (LEQ, x, Direct r, Immedlab lab) =
                (M.slt(r,x,my_arithtemp'); 
                 M.beq(true,Reg 0, my_arithtemp',lab))
(* These two cases added to prevent infinite reversal *)
  | ibranch (GTR, Direct r, x, Immedlab lab) =
                (M.move(x, my_arithtemp');
                 M.slt(my_arithtemp',Direct r,my_arithtemp');
                 M.beq(false,Reg 0,my_arithtemp',lab))
  | ibranch (LEQ, Direct r, x, Immedlab lab) =
                (M.move(x, my_arithtemp');
                 M.slt(my_arithtemp',Direct r,my_arithtemp');
                 M.beq(true,Reg 0,my_arithtemp',lab))
  | ibranch (_, Immedlab _, Immedlab _, _) = 
                ErrorMsg.impossible "bad ibranch args 1 in mips"
  | ibranch (_, Immedlab _, _, _) = 
                ErrorMsg.impossible "bad ibranch args 1a in mips"
  | ibranch (_, _, Immedlab _, _) = 
                ErrorMsg.impossible "bad ibranch args 1b in mips"
  | ibranch (_, _, _, Direct _) = 
                ErrorMsg.impossible "bad ibranch args 2 in mips"
  | ibranch (_, _, _, Immed _) = 
                ErrorMsg.impossible "bad ibranch args 3 in mips"
  | ibranch (cond, x, y, l) = 
        let fun rev LEQ = GEQ
              | rev GEQ = LEQ
              | rev LSS = GTR
              | rev GTR = LSS
              | rev NEQ = NEQ
              | rev EQL = EQL
        in  if reverse then (makeibranch false) (rev cond, y,x,l) 
            else ErrorMsg.impossible "infinite ibranch reversal in mips"
        
        end
in ibranch
end
in
val ibranch = makeibranch true
end
    
\endcode
\begincode{39}
\moddef{branches}\endmoddef
fun jmp (Direct r) = M.jump(r)
  | jmp (Immedlab lab) = M.beq(true,Reg 0,Reg 0,lab)
  | jmp (Immed i) = ErrorMsg.impossible "jmp (Immed i) in mips"


        (* branch on bit set *)
fun bbs (Immed k, Direct y, Immedlab label) =
        (M.and'(y,Immed (Bits.lshift(1,k)),my_arithtemp');
         M.beq(false,my_arithtemp',Reg 0,label))
  | bbs _ = ErrorMsg.impossible "bbs args don't match in mips"

\endcode
\begincode{40}
\moddef{DEBUG code}\endmoddef
val ibranch = diag "ibranch" ibranch
val jmp = diag "jmp" jmp
val bbs = diag "bbs" bbs

\endcode
\begindocs{41}
We decided not to include floating point registers in our galaxy of
effective addresses.
This means that floating point registers are used only at this level, and
only to contain intermediate results.
All operands and final results will be stored in memory, in the usual
ML format (i.e. as 8-byte strings).

In fact, we can be much more strict than that, and claim that
all floating point operands will live in FPR0 and FPR2, and that all 
results will appear in FPR0.

We don't make a distinction between general-purpose and floating point
registers; it's up to the instructions to know the difference.
\enddocs
\begincode{42}
\moddef{floating point}\endmoddef
val floatop1 = Reg 0
val floatop2 = Reg 2
val floatresult = Reg 0

\endcode
\begindocs{43}
One very common operation is to take the result of a floating point
operation and put it into a fresh record, newly allocated on the heap.
This operation is traditionally called \code{}finish_real\edoc{}, and it takes one
argument, the destination register for the new value.
All real values on the heap are labelled as 8-byte strings.
To store a floating point, we store the least significant
word in the lower address, but we store the most significant word
first, in case that triggers a garbage collection.
\enddocs
\begincode{44}
\moddef{floating point}\endmoddef
val real_tag = Immed(8*System.Tags.power_tags + System.Tags.tag_string)

fun store_float(Reg n,ea,offset) = 
    if n mod 2 <> 0 then ErrorMsg.impossible "bad float reg in mips"
    else (M.swc1(Reg (n+1),ea,offset+4);M.swc1(Reg n,ea,offset))

fun finish_real (Direct result) = (
    store_float(floatresult,dataptr,4);
    M.move(real_tag,my_arithtemp');
    M.sw(my_arithtemp',dataptr,0);
    M.add(dataptr',Immed 4,result);
    M.add(dataptr',Immed 12,dataptr'))
  | finish_real _ = 
     ErrorMsg.impossible "ptr to result of real operation not register in mips"

\endcode
\begindocs{45}
Loading a floating point quantity is analogous.
\enddocs
\begincode{46}
\moddef{floating point}\endmoddef
fun load_float(Reg dest,src,offset) =
    if dest mod 2 <> 0 then ErrorMsg.impossible "bad float reg in mips"
    else (M.lwc1(Reg dest,src,offset); M.lwc1(Reg (dest+1),src,offset+4))

\endcode
\begindocs{47}
Now we can do a general two- and three-operand floating point operationa.
The only parameter is the function in \code{}MipsCoder\edoc{} that
emits the floating point register operation.
\enddocs
\begincode{48}
\moddef{floating point}\endmoddef
fun two_float instruction (op1,result) = (
    load_float(floatop1,op1,0);
    instruction(floatop1,floatresult);
    finish_real(result))

fun three_float instruction (op1,op2,result) = (
    load_float(floatop1,op1,0);
    load_float(floatop2,op2,0);
    instruction(floatop1,floatop2,floatresult);
    finish_real(result))

\endcode
\begindocs{49}
That takes care of everything except branch
\enddocs
\begincode{50}
\moddef{floating point}\endmoddef
val mnegg = two_float M.neg_double
val mulg3 = three_float M.mul_double
val divg3 = three_float M.div_double
val addg3 = three_float M.add_double
val subg3 = three_float M.sub_double


\endcode
\begindocs{51}
The Mips doesn't provide all six comparisons in hardware, so the
next function does the comparison using only less than and equal.
The result tells \code{}bcop1\edoc{} whether to branch on condition true
or condition false.
\enddocs
\begincode{52}
\moddef{floating point compare}\endmoddef
fun compare(LSS,op1,op2) = (M.slt_double(op1,op2); true)
  | compare(GEQ,op1,op2) = (M.slt_double(op1,op2); false)
  | compare(EQL,op1,op2) = (M.seq_double(op1,op2); true)
  | compare(NEQ,op1,op2) = (M.seq_double(op1,op2); false)
  | compare(LEQ,op1,op2) = compare(GEQ,op2,op1)
  | compare(GTR,op1,op2) = compare(LSS,op2,op1)
\endcode
\begincode{53}
\moddef{floating point}\endmoddef
local
    \LA{}floating point compare\RA{}
in
    fun gbranch (cond, op1, op2, Immedlab label) = (
            load_float(floatop1,op1,0);
            load_float(floatop2,op2,0);
            M.bcop1(compare(cond,floatop1,floatop2),label))
      | gbranch _ = ErrorMsg.impossible "insane gbranch target in mips.nw"
end
        

\endcode
\begindocs{54}
When a function begins execution, it checks to make sure there is sufficient
memory available that it can do all its allocation.
generic does this by calling \code{}checkLimit : int -> unit\edoc{}.
At the moment, we implement this check by doing a store,
taking advantage of the virtual memory hardware, which will cause an exception
if there's not enough memory.
Later we will replace this store with a check against a limit register,
which will avoid virtual memory hacking and which will have advantages
for concurrency.
\enddocs
\begincode{55}
\moddef{memory check}\endmoddef
fun checkLimit max_allocation = M.sw(Reg 0, dataptr, max_allocation-4)
                              (* store zero in last location to be used *)

\endcode
\begindocs{56}
These two functions have null implementations.
\code{}beginStdFn\edoc{} is necessary only on the SPARC, since that machine needs to get 
its program counter, and it is awkward to do so in the middle of a function.

\code{}profile\edoc{} is a mysterious relic.
\enddocs
\begincode{57}
\moddef{omitted functions}\endmoddef
fun beginStdFn _ = ()           (* do nothing, just like the Vax *)

fun profile(i,incr) = ()

\endcode
\filename{mipscoder.nw}
\begindocs{0}
\input verbatim
\input itemize
\chapter{A small assembler for the MIPS}
This is part of the code generator for Standard ML of New Jersey.
We generate code in several stages.
This is nearly the lowest stage; it is like an assembler.
The user can call any function in the MIPSCODER signature.
Each one corresponds to an assembler pseudo-instruction.
Most correspond to single MIPS instructions.
The assembler remembers all the instructions that have been 
requested, and when \code{}codegen\edoc{} is called it generates MIPS
code for them.

Some other structure will be able to use the MIPS structure to implement
a \code{}CMACHINE\edoc{}, which is the abstract machine that ML thinks it is running
on.
(What really happens is a functor maps some structure 
implementing \code{}MIPSCODER\edoc{} to a different structure implementing 
\code{}CMACHINE\edoc{}.)

{\em Any function using a structure of this signature must avoid
touching registers 1~and~31.
Those registers are reserved for use by the assembler.}

\enddocs
\begindocs{1}
Here is the signature of the assembler, \code{}MIPSCODER\edoc{}.
It can be extracted from this file by
$$\hbox{\tt notangle mipsinstr.nw -Rsignature}.$$
\enddocs
\begincode{2}
\moddef{signature}\endmoddef
signature MIPSCODER = sig

(* Assembler for the MIPS chip *)

eqtype Label
datatype Register = Reg of int
    (* Registers 1 and 31 are reserved for use by this assembler *)
datatype EA = Direct of Register | Immed of int | Immedlab of Label
                                (* effective address *)

structure M : sig

    (* Emit various constants into the code *)

    val emitstring : string -> unit     (* put a literal string into the
                                           code (null-terminated?) and
                                           extend with nulls to 4-byte 
                                           boundary. Just chars, no 
                                           descriptor or length *)
    val realconst : string -> unit      (* emit a floating pt literal *)
                                                (* NOT RIGHT YET *)
    val emitlong : int -> unit          (* emit a 4-byte integer literal *)


    (* Label bindings and emissions *)

    val newlabel : unit -> Label        (* new, unbound label *)
    val define : Label -> unit          (* cause the label to be bound to
                                           the code about to be generated *)
    val emitlab : int * Label -> unit   (* L3: emitlab(k,L2) is equivalent to
                                           L3: emitlong(k+L2-L3) *)

    (* Control flow instructions *)

    val slt : Register * EA * Register -> unit
                (* (operand1, operand2, result) *)
                                        (* set less than family *)
    val beq : bool * Register * Register * Label -> unit
                (* (beq or bne, operand1, operand2, branch address) *)
                                        (* branch equal/not equal family *)
    
    val jump : Register -> unit         (* jump register instruction *)

    val slt_double : Register * Register -> unit
                                        (* floating pt set less than *)
    val seq_double : Register * Register -> unit
                                        (* floating pt set equal *)
    val bcop1 : bool * Label -> unit    (* floating pt conditional branch *)


    (* Arithmetic instructions *)
            (* arguments are (operand1, operand2, result) *)

    val add : Register * EA * Register -> unit
    val and' : Register * EA * Register -> unit
    val or : Register * EA * Register -> unit
    val xor : Register * EA * Register -> unit
    val sub : Register * Register * Register -> unit
    val div : Register * Register * Register -> unit
    val mult : Register * Register * Register -> unit
    val mfhi : Register -> unit         (* high word of 64-bit multiply *)
    
    (* Floating point arithmetic *)

    val neg_double : Register * Register -> unit
    val mul_double : Register * Register * Register -> unit
    val div_double : Register * Register * Register -> unit
    val add_double : Register * Register * Register -> unit
    val sub_double : Register * Register * Register -> unit

    (* Move pseudo-instruction :  move(src,dest) *)

    val move : EA * Register -> unit

    (* Load and store instructions *)
            (* arguments are (destination, source address, offset) *)
 
    val lbu  : Register * EA * int -> unit (* bytes *)
    val sb  : Register * EA * int -> unit
    val lw  : Register * EA * int -> unit  (* words *)
    val sw  : Register * EA * int -> unit
    val lwc1: Register * EA * int -> unit  (* floating point coprocessor *)
    val swc1: Register * EA * int -> unit
    val lui : Register * int -> unit

    (* Shift instructions *)
            (* arguments are (shamt, operand, result) *)
            (* shamt as Immedlab _ is senseless *)

    val sll : EA * Register * Register -> unit
    val sra : EA * Register * Register -> unit
    

    (* Miscellany *)

    val align : unit -> unit            (* cause next data to be emitted on
                                           a 4-byte boundary *)
    val mark : unit -> unit             (* emit a back pointer, 
                                           also called mark *)

    val comment : string -> unit

  end (* signature of structure M *)

  val codegen : (int * int -> unit) * (int -> string -> unit) 
                        * (string -> unit) -> unit

  val codestats : outstream -> unit     (* write statistics on stream *)

end (* signature MIPSCODER *)
\endcode
\begindocs{3}
The basic strategy of the implementation is to hold on, via the \code{}kept\edoc{}
pointer, to the list of instructions generated so far.
We use \code{}instr\edoc{} for the type of an instruction, so
\code{}kept\edoc{} has type \code{}instr list ref\edoc{}.

The instructions will be executed in the following order: the 
instruction at the head of the \code{}!kept\edoc{} is executed last.
This enables us to accept calls in the order of execution but
add the new instruction(s) to the list in constant time.


\enddocs
\begindocs{4}

We structure the instruction stream a little bit by factoring
out the difference between multiplication and division; these
operations are treated identically in that the result has to be
fetched out of the MIPS' LO register.

We also factor the different of load and store instructions that can
occur: we have load byte, load word, and load to coprocessor (floating point).
\enddocs
\begincode{5}
\moddef{types auxiliary to \code{}instr\edoc{}}\endmoddef
datatype size = Byte | Word | Floating
datatype muldiv = MULT | DIV
\endcode
\begindocs{6}

Here are the instructions that exist.
We list them in more or less the order of the MIPSCODER signature.
\enddocs
\begincode{7}
\moddef{definition of \code{}instr\edoc{}}\endmoddef
\LA{}types auxiliary to \code{}instr\edoc{}\RA{}

datatype instr = 
    STRINGCONST of string               (* constants *)
  | REALCONST of string
  | EMITLONG of int

  | DEFINE of Label                     (* labels *)
  | EMITLAB of int * Label

  | SLT of Register * EA * Register     (* control flow *)
  | BEQ of bool * Register * Register * Label
  | JUMP of Register 
  | SLT_D of Register * Register
  | SEQ_D of Register * Register
  | BCOP1 of bool * Label

  | NOP (* no-op for delay slot *)

  | ADD of Register * EA * Register     (* arithmetic *)
  | AND of Register * EA * Register
  | OR  of Register * EA * Register
  | XOR of Register * EA * Register
  | SUB of Register * Register * Register
  | MULDIV of muldiv * Register * Register
  | MFLO of Register    (* mflo instruction used with
                           64-bit multiply and divide *)
  | MFHI of Register

  | NEG_D of Register * Register
  | MUL_D of Register * Register * Register
  | DIV_D of Register * Register * Register
  | ADD_D of Register * Register * Register
  | SUB_D of Register * Register * Register

  | MOVE of EA * Register    (* put something into a register *)
  | LDI_32 of int * Register (* load in a big immediate constant (>16 bits) *)
  | LUI of Register * int    (* Mips lui instruction *)

  | LOAD of size * Register * EA * int  (* load and store *)
  | STORE  of size * Register * EA * int

  | SLL of EA * Register * Register     (* shift *)
  | SRA of EA * Register * Register

  | COMMENT of string                   (* generates nothing *)
  | MARK                                (* a backpointer *)
\endcode
\begindocs{8}

Here is the code that handles the generated stream, \code{}kept\edoc{}.
It begins life as \code{}nil\edoc{} and returns to \code{}nil\edoc{} every time code is
generated.
The function \code{}keep\edoc{} is a convenient way of adding a single \code{}instr\edoc{} to
the list; it's very terse.
Sometimes we have to add multiple \code{}instr\edoc{}s; then we use \code{}keeplist\edoc{}.
We also define a function \code{}delay\edoc{} that is just like a \code{}keep\edoc{} but
it adds a NOP in the delay slot.
\enddocs
\begincode{9}
\moddef{instruction stream and its functions}\endmoddef
  val kept = ref nil : instr list ref
  fun keep f a = kept := f a :: !kept
  fun delay f a = kept := NOP :: f a :: !kept
  fun keeplist l = kept := l @ !kept
\endcode
\begincode{10}
\moddef{reinitialize \code{}kept\edoc{}}\endmoddef
  kept := nil
\endcode
\begindocs{11}

\subsection{Exporting functions for \code{}MIPSCODER\edoc{}}
We now know enough to implement most of the functions called for in
\code{}MIPSCODER\edoc{}.
We still haven't decided on an implementation of labels,
and there is one subtlety in multiplication and division,
but the rest is set.
\enddocs
\begincode{12}
\moddef{\code{}MIPSCODER\edoc{} functions}\endmoddef
  val emitstring = keep STRINGCONST     (* literals *)
  val realconst = keep REALCONST
  val emitlong = keep EMITLONG

  \LA{}label functions\RA{}                   (* labels *)

  val slt = keep SLT                    (* control flow *)
  val beq = delay BEQ
  val jump = delay JUMP
  val slt_double = delay SLT_D
  val seq_double = delay SEQ_D
  val bcop1 = delay BCOP1

  val add = keep ADD                    (* arithmetic *)
  val and' = keep AND
  val or = keep OR
  val xor = keep XOR
  val op sub = keep SUB
  \LA{}multiplication and division functions\RA{}

  val neg_double = keep NEG_D
  val mul_double = keep MUL_D
  val div_double = keep DIV_D
  val add_double = keep ADD_D
  val sub_double = keep SUB_D

  val move = keep MOVE

  fun lbu (a,b,c) = delay LOAD (Byte,a,b,c) (* load and store *)
  fun lw (a,b,c)  = delay LOAD (Word,a,b,c)
  fun lwc1 (a,b,c)  = delay LOAD (Floating,a,b,c)
  fun sb (a,b,c)  = keep STORE (Byte,a,b,c)
  fun sw (a,b,c)  = keep STORE (Word,a,b,c)
  fun swc1 (a,b,c)  = delay STORE (Floating,a,b,c)
  val lui = keep LUI

  val sll = keep SLL                    (* shift *)
  val sra = keep SRA

  fun align() = ()                      (* never need to align on MIPS *)
  val mark = keep (fn () => MARK)
  val comment = keep COMMENT
\endcode
\begindocs{13}

Multiplication and division have a minor complication; the
result has to be fetched from the LO register.
\enddocs
\begincode{14}
\moddef{multiplication and division functions}\endmoddef
  fun muldiv f (a,b,c) = keeplist [MFLO c, MULDIV (f,a,b)]
  val op div = muldiv DIV
  val mult = muldiv MULT
  val mfhi = keep MFHI
\endcode
\begindocs{15}

For now, labels are just pointers to integers.
During code generation, those integers will be set to positions
in the instruction stream, and then they'll be useful as addresses
relative to the program counter pointer (to be held in \code{}Reg pcreg\edoc{}).
\enddocs
\begincode{16}
\moddef{definition of \code{}Label\edoc{}}\endmoddef
  type Label = int ref
\endcode
\begincode{17}
\moddef{label functions}\endmoddef
  fun newlabel () = ref 0
  val define = keep DEFINE
  val emitlab = keep EMITLAB
\endcode
\begindocs{18}

Here's the overall plan of this structure:
\enddocs
\begincode{19}
\moddef{*}\endmoddef
structure MipsCoder : MIPSCODER = struct

  \LA{}definition of \code{}Label\edoc{}\RA{}

  datatype Register = Reg of int

  datatype EA = Direct of Register
              | Immed of int
              | Immedlab of Label

  \LA{}definition of \code{}instr\edoc{}\RA{}

  \LA{}instruction stream and its functions\RA{}
  
  structure M = struct
    \LA{}\code{}MIPSCODER\edoc{} functions\RA{}
  end

  open M

  \LA{}functions that assemble \code{}instr\edoc{}s into code\RA{}

  \LA{}statistics\RA{}

end (* MipsInstr *)
\endcode
\begindocs{20}
\subsection{Sizes of \code{}instr\edoc{}s}
Now let's consider the correspondence between our \code{}instr\edoc{} type and the
actual MIPS instructions we intend to emit.
One important problem to solve is figuring out how big things are, 
so that we know what addresses to generate for the various labels.
We will also want to know what address is currently stored in the program
counter regsiter (\code{}pcreg\edoc{}),
because we'll need to know when something is close 
enough that we can use a sixteen-bit address relative to that register.
The kind of address we can use will determine how big things are.

We'll rearrange the code so that we have a list of \code{}ref int * instr\edoc{} pairs,
where the \code{}ref int\edoc{} stores the position in the list. 
(Positions start at zero.)
Since in the MIPS all instructions are the same size, we measure
position as number of instructions.
While we're at it, we reverse the list so that the head will execute first,
then the rest of the list.

We begin with each position set to zero, and make a pass over the list 
trying to set the value of each position. 
We do this by estimating the size of (number of MIPS instructions 
generated for) each \code{}instr\edoc{}.
Since there are forward references, we may not have all the distances right
the first time, so we have to make a second pass.
But during this second pass we could find that something is further
away than we thought, and we have to switch from using a pc-relative mode to
something else (or maybe grab the new pc?), which changes the size again,
and moves things even further away.
Because we can't control this process, we just keep making passes over the
list until the process quiesces (we get the same size twice).

In order to guarantee termination, we have to make sure later passes only 
increase the sizes of things.
This is sufficient since there is a maximum number of MIPS instructions
we can generate for each \code{}instr\edoc{}.


While we're at it, we might want to complicate things by making the function
that does the passes also emit code.
For a single pass we hand an optional triple of emitters, the initial position,
an \code{}int option\edoc{} for the program counter pointer (if known), and the
instructions.



I'm not sure what explains the use of the \code{}ref int\edoc{} to track the position,
instead of just an \code{}int\edoc{}---it might be a desire to avoid the
overhead of creating a bunch of new objects, or it might be really hard
to do the passes cheaply.
It should think a variation on \code{}map\edoc{} would do the job, but maybe I'm
missing something.

\enddocs
\begindocs{21}
\code{}emitters\edoc{} is actually a triple \code{}emit,emit_string,emit_real\edoc{}.
\code{}emit : int * int -> unit\edoc{} emits one instruction, 
and \code{}emit_string : int -> string -> unit\edoc{} emits a string constant.
\code{}emit_string\edoc{} could be specified as a function of \code{}emit\edoc{},
but the nature of the function would depend on whether the target
machine was little-endian or big-endian, and we don't want to have
that dependency built in.
\code{}emit_real : string -> unit\edoc{} emits two words corresponding to an IEEE
floating point constant in string form (e.g. \code{}"3.14159"\edoc{}).
In principle, it can always be derived from \code{}emit_word\edoc{}, but it's
easier to pass it in because then we can use John Reppy's code;
see the \code{}MipsReal\edoc{} functor for more details, as well as
\code{}mipsglue.sml\edoc{}.


 \code{}instrs\edoc{} is the
list of instructions (in execute-head-last order).

The second argument to \code{}pass\edoc{} indicates for what instructions code
is to be generated.
It is a record (position of next instruction, program counter pointer if any,
remaining instructions to generate [with positions]).

\indent \code{}prepare\edoc{} produces two results: the instruction stream with
size pointers added, and the total size of code to be generated.
We add the total size because that is the only way to find the number
of \code{}bltzal\edoc{}s, which are implicit in the instruction stream.

\enddocs
\begincode{22}
\moddef{assembler}\endmoddef
fun prepare instrs =
 let fun add_positions(done, inst::rest) =  
                 add_positions( (ref 0, inst) :: done, rest)
       | add_positions(done, nil) = done

     val instrs' = add_positions(nil, instrs) (* reverse and add \code{}ref int\edoc{}s*)

     fun passes(oldsize) = 
                (* make passes with no emission until size is stable*)
        let val size = pass NONE (0,NONE,instrs')
        in  if size=oldsize then size
            else passes size
        end
  in \{size = passes 0, stream = instrs'\}
  end

fun assemble emitters instrs =
        pass (SOME emitters) (0,NONE,#stream (prepare instrs))

\endcode
\begincode{23}
\moddef{functions that assemble \code{}instr\edoc{}s into code}\endmoddef
fun get (SOME x) = x 
  | get NONE = ErrorMsg.impossible "missing pcptr in mipscoder"

\LA{}\code{}pcptr\edoc{} functions\RA{}
\LA{}single pass\RA{}
\LA{}assembler\RA{}

fun codegen emitters = (
    assemble emitters (!kept);
    \LA{}reinitialize \code{}kept\edoc{}\RA{}
    )
\endcode
\begindocs{24}

The program counter pointer is a device that enables us to to addressing
relative to the pcp register, register 31.
The need for it arises when we want to access a data element which we know
only by its label.
The labels give us addresses relative to the beginning of the function,
but we can only use addresses relative to some register.
The answer is to set register~31 with a \code{}bltzal\edoc{} instruction,
then use that for addressing.

The function \code{}needs_a_pcptr\edoc{} determines when it is necessary
to have a known value in register~31.
That is, we need the program counter pointer
\itemize
\item 
at \code{}NOP\edoc{} for a reason to be named later?
\item
at any operation that uses an effective address that refers to a label
(since all labels have to be relative to the program counter).
\item
BEQ's and BCOP1's to very far away, 
since we have to compute the address for a JUMP 
knowing the value of the program counter pointer.
\enditemize
\enddocs
\begincode{25}
\moddef{\code{}pcptr\edoc{} functions}\endmoddef
fun needs_a_pcptr(_,SLT(_,Immedlab _,_)) = true
  | needs_a_pcptr(_,ADD(_,Immedlab _,_)) = true
  | needs_a_pcptr(_,AND(_,Immedlab _,_)) = true
  | needs_a_pcptr(_,OR(_,Immedlab _,_)) = true
  | needs_a_pcptr(_,XOR(_,Immedlab _,_)) = true
  | needs_a_pcptr(_,MOVE(Immedlab _,_)) = true
  | needs_a_pcptr(_,LOAD(_,_,Immedlab _,_)) = true
  | needs_a_pcptr(_,STORE(_,_,Immedlab _,_)) = true
  | needs_a_pcptr(_,SLL(Immedlab _,_,_)) = true
  | needs_a_pcptr(_,SRA(Immedlab _,_,_)) = true
  | needs_a_pcptr(1, BEQ _) = false  (* small BEQ's dont need pcptr *)
  | needs_a_pcptr(_, BEQ _) = true   (* but large ones do *)
  | needs_a_pcptr(1, BCOP1 _) = false  (* small BCOP1's dont need pcptr *)
  | needs_a_pcptr(_, BCOP1 _) = true   (* but large ones do *)
  | needs_a_pcptr _ = false
\endcode
\begindocs{26}

Creating the program counter pointer once, with a \code{}bltzal\edoc{}, is not
enough; we have to invalidate the program counter pointer at every
label, since control could arrive at the label from God knows where, and
therefore we don't know what the program counter pointer is.

We use the function \code{}makepcptr\edoc{} to create a new program counter pointer
``on the fly'' while generating code for other \code{}instrs\edoc{}.
(I chose not to create a special \code{}instr\edoc{} for \code{}bltzal\edoc{}, which I
could have inserted at appropriate points in the instruction stream.)
To try and find an odd bug, I'm adding no-ops after each \code{}bltzal\edoc{}.
I don't really believe they're necessary.

The function \code{}gen\edoc{}, which generates the instructions (or computes
their size), takes three arguments.
Third: the list of instructions to be generated (paired with pointers
to their sizes); first: the position (in words) at which to generate 
those instructions;  second: the current value of the program counter
pointer (register~31), if known.

The mutual recursion between \code{}gen\edoc{} and \code{}makepcptr\edoc{} maintains
the program counter pointer.
\code{}gen\edoc{} invalidates it at labels, and calls \code{}makepcptr\edoc{} to create a valid
one when necessary (as determined by \code{}needs_a_pcptr\edoc{}).
\enddocs
\begincode{27}
\moddef{single pass}\endmoddef
fun pass emitters =
let fun makepcptr(i,x) = 
         (* may need to emit NOP for delay slot if next instr is branch *)
  let val size = case x of ((_,BEQ _)::rest) => 2 
                         | ((_,BCOP1 _)::rest) => 2 
                         | _ => 1
  in  case emitters of NONE                     => () 
                     | SOME (emit, _, _ ) => (
                              emit(Opcodes.bltzal(0,0));
                              if size=2 then emit(Opcodes.add(0,0,0)) else ());
      gen(i+size, SOME (i+2), x)
  end
and gen(i,_,nil) = i
  | gen(i, _, (_,DEFINE lab) :: rest) = (lab := i; gen(i,NONE, rest))
                        (* invalidate the pc pointer at labels *)
  (* may want to do special fiddling with NOPs *)
  | gen(pos, pcptr, x as ((sizeref as ref size, inst) :: rest)) =
       if (pcptr=NONE andalso needs_a_pcptr(size, inst)) then makepcptr(pos,x)
       else case emitters of
          SOME (emit : int*int -> unit, emit_string : int -> string -> unit,
                emit_real : string -> unit) =>
             \LA{}emit MIPS instructions\RA{}
        | NONE =>
             \LA{}compute positions\RA{}
in  gen
end

\endcode
\begindocs{28}
\subsection{Generating the instructions}
Now we need to consider the nitty-gritty details of just what instructions
are generated for each \code{}instr\edoc{}.
In  early passes, we'll just need to know how many instructions are
required (and that number may change from pass to pass, so it must be
recomputed).
In the last pass, the sizes are stable (by definition), so we can look
at the sizes to see what instructions to generate.

We'll consider the \code{}instrs\edoc{} in groups, but first, here's the
way we will structure things:
\enddocs
\begincode{29}
\moddef{compute positions}\endmoddef
let \LA{}functions for computing sizes\RA{}
    val newsize = case inst of
        \LA{}cases for sizes to be computed\RA{}
in  if newsize > size then sizeref := newsize else ();
    gen(pos+(!sizeref) (* BUGS -- was pos+size*),pcptr,rest)
end
\endcode
\begincode{30}
\moddef{emit MIPS instructions}\endmoddef
let fun gen1() = gen(pos+size,pcptr,rest) 
                                (* generate the rest of the \code{}instr\edoc{}s *)
    open Bits
    open Opcodes
    \LA{}declare reserved registers \code{}tempreg\edoc{} and \code{}pcreg\edoc{}\RA{}
    \LA{}functions for emitting instructions\RA{}
in  case inst of
    \LA{}cases of instructions to be emitted\RA{}
end
\endcode
\begindocs{31}
When we get around to generating code, we may need to use a temporary
register.
For example, if we want to load into a register
an immediate constant that won't fit
into 16~bits, we will have to load the high-order part of the constant
with \code{}lui\edoc{}, then use \code{}addi\edoc{} to add then the low-order part.
The MIPS assembler has a similar problem, and on page D-2 of
the MIPS book we notice that register~1 is reserved for the use of the
assembler.
So we do the same.

We need to reserve a second register for use in pointing to the program 
counter.
We will use register 31 because the \code{}bltzal\edoc{} instruction automatically
sets register 31 to the PC.
\enddocs
\begincode{32}
\moddef{declare reserved registers \code{}tempreg\edoc{} and \code{}pcreg\edoc{}}\endmoddef
val tempreg = 1
val pcreg = 31
\endcode
\begindocs{33}

Before showing the code for the actual instructions, we should
point out that
we have two different ways of emitting a long word.
\code{}emitlong\edoc{} just splits the bits into two pieces for those cases
when it's desirable to put a word into the memory image.
\code{}split\edoc{} gives something that will load correctly
when the high-order piece is loaded into a high-order halfword
(using \code{}lui\edoc{}),
and the low-order piece is sign-extended and then added to the 
high-order piece.
This is the way we load immediate constants of more than sixteen bits.
It is also useful for generating load or store instructions with
offsets of more than sixteen bits: we \code{}lui\edoc{} the \code{}hi\edoc{} part and
add it to the base regsiter, then use the \code{}lo\edoc{} part as an offset.
\enddocs
\begincode{34}
\moddef{functions for emitting instructions}\endmoddef
fun emitlong i = emit(rshift(i,16), andb(i,65535))
                                (* emit one long word (no sign fiddling) *)
fun split i = let val hi = rshift(i,16) and lo = andb(i,65535)
                     in if lo<32768 then (hi,lo) else (hi+1, lo-65536)
                    end

\endcode
\begindocs{35}
We begin implementing \code{}instrs\edoc{} by considering those that emit constants.
String constants are padded with nulls out to a word boundary, and 
real constants take up two words.
{\bf At the moment real constants seem to be zeros.
One day this will have to be fixed.}
Integer constants are just emitted with \code{}emitlong\edoc{}.
\enddocs
\begincode{36}
\moddef{cases for sizes to be computed}\endmoddef
  STRINGCONST s => Integer.div(String.length(s)+3,4)
| REALCONST _ => 2
| EMITLONG _ => 1
\endcode
\begincode{37}
\moddef{cases of instructions to be emitted}\endmoddef
  STRINGCONST s => 
        let val s' = s ^ "\\000\\000\\000\\000"
        in  gen1(emit_string (4*size) s')
                                      (* doesn't know Big vs Little-Endian *)
        end
| REALCONST s => gen1(emit_real s)  (* floating pt constant *)
| EMITLONG i => gen1(emitlong i)
\endcode
\begindocs{38}

Next consider the labels.
A \code{}DEFINE\edoc{} should never reach this far, and \code{}EMITLAB\edoc{} is almost like
an \code{}EMITLONG\edoc{}.
\enddocs
\begincode{39}
\moddef{cases for sizes to be computed}\endmoddef
| DEFINE _ => ErrorMsg.impossible "generate code for DEFINE in mipscoder"
| EMITLAB _ => 1
\endcode
\begincode{40}
\moddef{cases of instructions to be emitted}\endmoddef
| DEFINE _ => gen1(ErrorMsg.impossible "generate code for DEFINE in mipscoder")
| EMITLAB(i, ref d) => gen1(emitlong((d-pos)*4+i))
\endcode
\begindocs{41}

Now we have to start worrying about instructions with \code{}EA\edoc{} in them.
The real difficulty these things present is that they may have an
immediate operand that won't fit in 16~bits.
So we'll need to get this large immediate operand into a register,
sixteen bits at a time, and then do the operation on the register.

Since all of the arithmetic instructions have this difficulty, and since
we can use them to implement the others, we'll start with those and
catch up with the control-flow instructions later.
\enddocs
\begindocs{42}
\code{}SUB\edoc{}, \code{}MULDIV\edoc{}, and \code{}MFLO\edoc{} all use registers only, so they are easy.
The other arithmetic operations get treated exactly the same, so we'll
use a function to compute the size.
{\bf move this to follow the definition of \code{}arith\edoc{}?}
\enddocs
\begincode{43}
\moddef{cases for sizes to be computed}\endmoddef
| ADD(_, ea, _) => easize ea
| AND(_, ea, _) => easize ea
| OR (_, ea, _) => easize ea
| XOR(_, ea, _) => easize ea
| SUB _ => 1
| MULDIV _ => 1 
| MFLO _ => 1
| MFHI _ => 1
\endcode
\begindocs{44}
Register operations take one instruction.
Immediate operations take one instruction for 16~bit constants,
and 3 for larger constants (since it costs two instructions to load
a big immediate constant into a register).
An immediate instruction with \code{}Immedlab l\edoc{} means that the operand
is intended to be the machine address associated with that label.
To compute that address, we need to add 
\code{}4*(l-pcptr)\edoc{} to the contents of 
register~\code{}pcreg\edoc{} (which holds \code{}4*pcptr\edoc{}), 
put the results in a register, and operate on that register.

This tells us enough to compute the sizes.
\enddocs
\begincode{45}
\moddef{functions for computing sizes}\endmoddef
fun easize (Direct _) = 1
  | easize (Immed i) = if abs(i)<32768 then 1 else 3
  | easize (Immedlab(ref lab)) = 1 + easize(Immed (4*(lab-(get pcptr))))
\endcode
\begindocs{46}

As we have seen, 
to implement any arithmetic operation, we need to know the register 
form and the sixteen-bit immediate form.
We will also want the operator from \code{}instr\edoc{}, since we do the
large immediate via a recursive call.
We'll set up a function, \code{}arith\edoc{}, that does the job.
\enddocs
\begincode{47}
\moddef{functions for emitting instructions}\endmoddef
fun arith (opr, rform, iform) =
   let fun ar (Reg op1, Direct (Reg op2), Reg result) = 
                gen1(emit(rform(result,op1,op2)))
         | ar (Reg op1, Immed op2, Reg result) =
                (case size of
                    1 (* 16 bits *) => gen1(emit(iform(result,op1,op2)))
                  | 3 (* 32 bits *) => 
                     gen(pos,pcptr,
                          (ref 2, LDI_32(op2, Reg tempreg))::
                          (ref 1, opr(Reg op1, Direct(Reg tempreg), Reg result))::
                          rest)
                  | _ => gen(ErrorMsg.impossible 
                                "bad size in arith Immed in mipscoder")
                )
         | ar (Reg op1, Immedlab (ref op2), Reg result) =
                gen(pos, pcptr, 
                      (ref (size-1), 
                            ADD(Reg pcreg,Immed(4*(op2-(get pcptr))), Reg tempreg))::
                      (ref 1, opr(Reg op1, Direct(Reg tempreg), Reg result))::
                      rest)
   in  ar
   end
\endcode
\begindocs{48}

The generation itself may be a bit anticlimactic.
The MIPS has no ``subtract immediate'' instruction, and \code{}SUB\edoc{} has
a different type than the others, so we emit it directly.
\enddocs
\begincode{49}
\moddef{cases of instructions to be emitted}\endmoddef
| ADD stuff => arith (ADD,add,addi) stuff
| AND stuff => arith (AND,and',andi) stuff
| OR  stuff => arith (OR,or,ori) stuff
| XOR stuff => arith (XOR,xor,xori) stuff
| SUB (Reg op1, Reg op2, Reg result) => gen1(emit(sub(result,op1,op2)))
| MULDIV(DIV, Reg op1, Reg op2) => gen1(emit(div(op1,op2)))
| MULDIV(MULT,Reg op1, Reg op2) => gen1(emit(mult(op1,op2)))
| MFLO(Reg result) => gen1(emit(mflo(result)))
| MFHI(Reg result) => gen1(emit(mfhi(result)))
\endcode
\begindocs{50}
Floating point arithmetic is pretty easy because we always do it in
registers.  
We also support only one format, double precision.
\enddocs
\begincode{51}
\moddef{cases for sizes to be computed}\endmoddef
| NEG_D _ => 1
| MUL_D _ => 1
| DIV_D _ => 1
| ADD_D _ => 1
| SUB_D _ => 1
\endcode
\begindocs{52}
When emitting instructions we have to remember the Mips instructions
use result on the left, but the \code{}MIPSCODER\edoc{} signature requires result
on the right.
\enddocs
\begincode{53}
\moddef{cases of instructions to be emitted}\endmoddef
| NEG_D (Reg op1,Reg result) => gen1(emit(neg_fmt(D_fmt,result,op1)))
\endcode
\begincode{54}
\moddef{functions for emitting instructions}\endmoddef
fun float3double instruction (Reg op1,Reg op2,Reg result) =
   gen1(emit(instruction(D_fmt,result,op1,op2)))
\endcode
\begincode{55}
\moddef{cases of instructions to be emitted}\endmoddef
| MUL_D x => float3double mul_fmt x
| DIV_D x => float3double div_fmt x
| ADD_D x => float3double add_fmt x
| SUB_D x => float3double sub_fmt x


\endcode
\begindocs{56}
We offer a separate \code{}MOVE\edoc{} instruction because of large immediate
constants.  
It is always possible to do \code{}move(src,dest)\edoc{} by doing 
\code{}add(Reg 0,src,dest)\edoc{}, but the general form \code{}add(Reg i, Immed c, dest)\edoc{}
takes three instructions when \code{}c\edoc{} is a large constant (more than 16 bits).
Rather than clutter up the code for \code{}add\edoc{} (and \code{}or\edoc{} and \code{}xor\edoc{}) by
trying to recognize register~0, we provide \code{}move\edoc{} explicitly.

\indent \code{}LDI_32\edoc{} takes care of the particular case in which we are 
loading a 32-bit immediate constant into a register.  
It dates from the bad old days before \code{}MOVE\edoc{}, and it might be a good idea
to remove it sometime.
\enddocs
\begincode{57}
\moddef{functions for emitting instructions}\endmoddef
fun domove (Direct (Reg src), Reg dest) = gen1(emit(add(dest,src,0)))
  | domove (Immed src, Reg dest) =
        (case size of
            1 (* 16 bits *) => gen1(emit(addi(dest,0,src)))
          | 2 (* 32 bits *) => 
                        gen(pos,pcptr,(ref 2, LDI_32(src, Reg dest))::rest)
          | _ => gen(ErrorMsg.impossible "bad size in domove Immed in mipscoder")
        )
  | domove (Immedlab (ref src), Reg dest) =
        gen(pos, pcptr, 
              (ref size, 
                    ADD(Reg pcreg,Immed(4*(src-(get pcptr))), Reg dest))::rest)
\endcode
\begindocs{58}
Notice we use \code{}easize\edoc{} and not \code{}movesize\edoc{} in the third clause
because when we reach this point the treatment of a \code{}MOVE\edoc{} is the same
as that of an \code{}ADD\edoc{}.
\enddocs
\begincode{59}
\moddef{functions for computing sizes}\endmoddef
fun movesize (Direct _) = 1
  | movesize (Immed i) = if abs(i)<32768 then 1 else 2
  | movesize (Immedlab(ref lab)) = easize(Immed (4*(lab-(get pcptr))))

\endcode
\begincode{60}
\moddef{cases for sizes to be computed}\endmoddef
| MOVE (src,_) => movesize src
| LDI_32 _ => 2
| LUI _ => 1
\endcode
\begincode{61}
\moddef{cases of instructions to be emitted}\endmoddef
| MOVE stuff => domove stuff
| LDI_32 (immedconst, Reg dest) =>
         let val (hi,lo) = split immedconst
         in  gen1(emit(lui(dest,hi));emit(addi(dest,dest,lo)))
         end 
| LUI (Reg dest,immed16) => gen1(emit(lui(dest,immed16)))

\endcode
\begindocs{62}

Now that we've done arithmetic, we can see how to do control flow without
too much trouble.
\code{}SLT\edoc{} can be treated just like an arithmetic operator.
\code{}BEQ\edoc{} is simple if the address to which we branch is close enough.
Otherwise we use the following sequence for \code{}BEQ(Reg op1, Reg op2, ref dest)\edoc{}:
\verbatim
        bne op1,op2,L
        ADD (Reg pcreg, Immed (4*(dest-pcptr)), Reg tempreg)
        jr tempreg
     L: ...
\endverbatim
Notice we don't have to put a \code{}NOP\edoc{} in the delay slot of the \code{}bne\edoc{}.
We don't need one after the jump unless we needed one after the 
original \code{}BEQ\edoc{}, in which case one will be there.
If the branch is taken, we're doing as well as we can.
If the branch is not taken, we will have executed an \code{}add\edoc{} or \code{}lui\edoc{} in the 
delay slot of the \code{}bne\edoc{}, but the results just get thrown away.
\enddocs
\begincode{63}
\moddef{cases for sizes to be computed}\endmoddef
| SLT(_, ea, _) => easize ea
| BEQ(_,_,_,ref dest) => 
        if abs((pos+1)-dest) < 32768 then 1 (* single instruction *)
        else 2+easize (Immed (4*(dest-(get pcptr))))
| JUMP _ => 1
| SLT_D _ => 1
| SEQ_D _ => 1
| BCOP1(_,ref dest) => 
        if abs((pos+1)-dest) < 32768 then 1 (* single instruction *)
        else 2+easize (Immed (4*(dest-(get pcptr))))
| NOP => 1
\endcode
\begindocs{64}
The implementation is as described, except we use a 
non-standard \code{}nop\edoc{}.
There are many Mips instructions that have no effect, and the standard
one is the word with all zeroes (\code{}sll 0,0,0\edoc{}).
We use \code{}add\edoc{},  adding 0 to 0 and store the result in 0, because it
will be easy to distinguish from a data word that happens to be zero.
\enddocs
\begincode{65}
\moddef{cases of instructions to be emitted}\endmoddef
| SLT stuff => arith (SLT,slt,slti) stuff
| BEQ(b, Reg op1, Reg op2, ref dest) =>
    if size = 1 then 
         gen1(emit((if b then beq else bne)(op1,op2,dest-(pos+1))))
    else gen(pos,pcptr,
                  (ref 1, BEQ(not b, Reg op1, Reg op2, ref(pos+size)))
                  ::(ref (size-2), 
                      ADD(Reg pcreg, Immed(4*(dest-(get pcptr))), Reg tempreg))
                  ::(ref 1, JUMP(Reg tempreg))
                  ::rest)
| JUMP(Reg dest) => gen1(emit(jr(dest)))
| SLT_D (Reg op1, Reg op2) => 
     gen1(emit(c_lt(D_fmt,op1,op2)))
| SEQ_D (Reg op1, Reg op2) => 
     gen1(emit(c_seq(D_fmt,op1,op2)))
| BCOP1(b, ref dest) =>
    let fun bc1f offset = cop1(8,0,offset)
        fun bc1t offset = cop1(8,1,offset)
    in  if size = 1 then 
             gen1(emit((if b then bc1t else bc1f)(dest-(pos+1))))
        else gen(pos,pcptr,
                  (ref 1, BCOP1(not b, ref(pos+size)))
                  ::(ref (size-2), 
                      ADD(Reg pcreg, Immed(4*(dest-(get pcptr))), Reg tempreg))
                  ::(ref 1, JUMP(Reg tempreg))
                  ::rest)
    end
| NOP => gen1(emit(add(0,0,0)))         (* one of the many MIPS no-ops *)
\endcode
\begindocs{66}

Our next problem is to tackle load and store.
The major difficulty is if the offset is too large to fit in
sixteen bits; if so, we have to create a new base register.
If we have \code{}Immedlab\edoc{}, we do it as an offset from \code{}pcreg\edoc{}.
\enddocs
\begincode{67}
\moddef{functions for emitting instructions}\endmoddef
fun memop(rform,Reg dest, Direct (Reg base), offset) =
      (case size
       of 1 => gen1(emit(rform(dest,offset,base)))
        | 3 => let val (hi,lo) = split offset
               in  gen1(emit(lui(tempreg,hi));       (* tempreg = hi \LL{} 16 *)
                        emit(add(tempreg,base,tempreg));(* tempreg += base *)
                        emit(rform(dest,lo,tempreg)) (* load dest,lo(tempreg) *)
                       )
               end
        | _ => gen1(ErrorMsg.impossible "bad size in memop Direct in mipscoder")
       )
  | memop(rform,Reg dest, Immed address, offset) =
      (case size
       of 1 => gen1(emit(rform(dest,offset+address,0)))
        | 2 => let val (hi,lo) = split (offset+address)
               in  gen1(emit(lui(tempreg,hi)); 
                        emit(rform(dest,lo,tempreg))
                       )
               end
        | _ => gen1(ErrorMsg.impossible "bad size in memop Immed in mipscoder")
       )
  | memop(rform,Reg dest, Immedlab (ref lab), offset) =
      memop(rform, Reg dest, Direct (Reg pcreg), offset+4*(lab - get pcptr))
\endcode
\begindocs{68}
The actual registers don't matter for computing sizes, and in fact
the value of \code{}pcreg\edoc{} is not visible here, so we use an arbitrary
register (\code{}Reg 0\edoc{}) to compute the size.
\enddocs
\begincode{69}
\moddef{functions for computing sizes}\endmoddef
fun adrsize(_, Reg _, Direct _, offset) = 
            if abs(offset)<32768 then 1 else 3
  | adrsize(_, Reg _, Immed address, offset) = 
            if abs(address+offset) < 32768 then 1 else 2
  | adrsize(x, Reg dest, Immedlab (ref lab), offset) =
            adrsize(x, Reg dest, Direct (Reg 0  (* pcreg in code *) ), 
                    offset+4*(lab-(get pcptr)))
\endcode
\begincode{70}
\moddef{cases for sizes to be computed}\endmoddef
| LOAD  x => adrsize x
| STORE x => adrsize x
\endcode
\begincode{71}
\moddef{cases of instructions to be emitted}\endmoddef
| LOAD  (Byte,dest,address,offset) => memop(lbu,dest,address,offset)
| LOAD  (Word,dest,address,offset) => memop(lw,dest,address,offset)
| LOAD  (Floating,dest,address,offset) => memop(lwc1,dest,address,offset)
| STORE (Byte,dest,address,offset) => memop(sb,dest,address,offset)
| STORE (Word,dest,address,offset) => memop(sw,dest,address,offset)
| STORE (Floating,dest,address,offset) => memop(swc1,dest,address,offset)
\endcode
\begindocs{72}
 
For the shift instructions, only register and immediate operands
make sense.
Immediate operands make sense if and only if they are representable
in five bits.
If everything is right, these are single instructions.
\enddocs
\begincode{73}
\moddef{cases for sizes to be computed}\endmoddef
| SLL _ => 1  
| SRA _ => 1
\endcode
\begincode{74}
\moddef{cases of instructions to be emitted}\endmoddef
| SLL (Immed shamt, Reg op1, Reg result) => gen1(
        if (shamt >= 0 andalso shamt < 32) then emit(sll(result,op1,shamt))
        else ErrorMsg.impossible ("bad sll shamt "
                ^ (Integer.makestring shamt) ^ " in mipscoder"))
| SLL (Direct(Reg shamt), Reg op1, Reg result) => 
        gen1(emit(sllv(result,op1,shamt)))
| SLL (Immedlab _,_,_) => ErrorMsg.impossible "sll shamt is Immedlab in mipscoder"
| SRA (Immed shamt, Reg op1, Reg result) => gen1(
        if (shamt >= 0 andalso shamt < 32) then emit(sra(result,op1,shamt))
        else ErrorMsg.impossible ("bad sra shamt "
                ^ (Integer.makestring shamt) ^ " in mipscoder"))
| SRA (Direct(Reg shamt), Reg op1, Reg result) =>
        gen1(emit(srav(result,op1,shamt)))
| SRA (Immedlab _,_,_) => ErrorMsg.impossible "sra shamt is Immedlab in mipscoder"
\endcode
\begindocs{75}

Finally, comments are ignored, and marks (backpointers) are written into the
instruction stream.

Comments are used by the front end to give diagnostics.
In the bad old days we would have had two different \code{}MIPSCODER\edoc{}s, one
which generated machine code (and ignored comments), and one which
wrote out assembly code (and copied comments).
Today we have just one, which means the rerouting of comments takes place
at a much higher level.  Look in \code{}cps/mipsglue.nw\edoc{}.
\enddocs
\begincode{76}
\moddef{cases for sizes to be computed}\endmoddef
| COMMENT _ => 0
| MARK => 1                     (* backpointer takes one word *)
\endcode
\begindocs{77}
Just for the record, here's the description of what a mark (backpointer)
is.
``Take the byte address at which the mark resides and add 4, giving
the byte address of the object following the mark.
(That object is the marked object.)
Subtract the byte address of the initial word that marks the
start of this instruction stream.
Now divide by 4, giving the distance in words between the
beginning of the block and the marked object.
Take that quantity and shift it left by multiplying by \code{}power_tags\edoc{},
and indicate the result is a mark by adding the tag bits \code{}tag_backptr\edoc{}
into the low order part.''
 \code{}pos+1\edoc{} is exactly the required distance in words.
\enddocs
\begincode{78}
\moddef{cases of instructions to be emitted}\endmoddef
| COMMENT _ => gen1()
| MARK => gen1(
    let open System.Tags
    in  emitlong((pos+1) * power_tags + tag_backptr)
    end)
\endcode
\begindocs{79}

\enddocs
\begindocs{80}
\subsection{Optimization}
The first step towards optimization is to take statistics.
We will count: \code{}instrs\edoc{}, Mips words, \code{}NOP\edoc{}s in load and branch delays,
and \code{}bltzal\edoc{}s.
In the current implementation the \code{}bltzal\edoc{}s are implicit, so there
is no way to count them or optimize them.
\enddocs
\begincode{81}
\moddef{statistics}\endmoddef
fun printstats stream
 \{inst : int, code : int, data : int, 
  load : int, branch : int, compare : int, size : int\} =
    let val print = output stream
        val nop = load+branch+compare
        val bltzal = size - (code + data)
        val code = code + bltzal
        \LA{}definition of \code{}sprintf\edoc{}\RA{}
        fun P x = substring(makestring(100.0 * x),0,4)  (* percent *)
        fun printf f d = print (sprintf f d)
    in  printf ["Counted "," instrs in "," words (",
                                " code, "," data)\\n" ^
                "Used "," NOPs ("," load, "," branch,"," compare) and "," bltzals\\n" ^
                "","% of code words were NOPs; ","% were bltzals\\n" ^
                "","% of all words were code; ","% of all words were NOPs\\n"]
               [I inst, I size, I code, I data, 
                I nop, I load, I branch,  I compare, I bltzal,
                P (real nop / real code), P (real bltzal / real code),
                P (real code / real size), P (real nop / real size)]
        handle Overflow => print "[Overflow in computing Mips stats]\\n"
             | Real s => print ("[FPE ("^s^") in computing Mips stats]\\n")
    end
                
\endcode
\begincode{82}
\moddef{statistics}\endmoddef
\LA{}definition of \code{}iscode\edoc{}\RA{}
fun addstats (counts as \{inst,code,data,load,branch,compare\}) =
  fn nil => counts
   | (sizeref,first)::(_,NOP)::rest => addstats
          \{inst=inst+2, code=code+(!sizeref)+1, data=data,
           load=load+ (case first of LOAD _ => 1 | _ => 0),
           branch=branch +(case first of BEQ _ => 1 | JUMP _ => 1 
                                       | BCOP1 _ => 1 | _ => 0),
           compare=compare+(case first of SLT_D _ => 1 | SEQ_D _ => 1 
                                        | _ => 0)
          \} rest
   | (sizeref,first)::rest => addstats
          \{inst=inst+1, 
           code = code + if iscode(first) then !sizeref else 0,
           data = data + if not (iscode first) then !sizeref else 0,
           load=load,
           branch=branch,
           compare=compare
          \} rest


fun codestats outfile =
    let val \{size,stream=instrs\} = prepare (!kept)
        val zero = \{inst=0, code=0, data=0, load=0, branch=0, compare=0\}
        val counts as \{inst,code,data,load,branch,compare\} = 
                                                addstats zero instrs
    in  printstats outfile 
            \{inst=inst,code=code,data=data,
             load=load,branch=branch,compare=compare,size=size\}
    end
        
\endcode
\begincode{83}
\moddef{definition of \code{}iscode\edoc{}}\endmoddef
val iscode = fn
    STRINGCONST _ => false
  | REALCONST _ => false
  | EMITLONG _ => false
  | DEFINE _ => false
  | EMITLAB _ => false

  | SLT _ => true
  | BEQ _ => true
  | JUMP _ => true
  | NOP => true
  | SLT_D _ => true
  | SEQ_D _ => true
  | BCOP1 _ => true

  | ADD _ => true
  | AND _ => true
  | OR  _ => true
  | XOR _ => true
  | SUB _ => true
  | MULDIV _ => true
  | MFLO _ => true
  | MFHI _ => true

  | NEG_D _ => true
  | MUL_D _ => true
  | DIV_D _ => true
  | ADD_D _ => true
  | SUB_D _ => true

  | MOVE _ => true
  | LDI_32 _ => true
  | LUI _ => true

  | LOAD _ => true
  | STORE  _ => true

  | SLL _ => true
  | SRA _ => true

  | COMMENT _ => false
  | MARK => false

\endcode
\begincode{84}
\moddef{definition of \code{}sprintf\edoc{}}\endmoddef
val I = Integer.makestring
val R = Real.makestring
exception Printf
fun sprintf format values =
    let fun merge([x],nil) = [x]
          | merge(nil,nil) = nil
          | merge(x::y,z::w) = x::z:: merge(y,w)
          | merge _ = raise Printf
    in  implode(merge(format,values))
    end

\endcode
\begindocs{85}

At the moment these functions are meaningless junk.
\enddocs
\begincode{86}
\moddef{functions that remove pipeline bubbles}\endmoddef
val rec squeeze =

 fn (x as LOAD(_,Reg d, m, i))::NOP::instr::rest =>
        if use(instr,d) then ??
        else squeeze(x::instr::rest)
  | (x as STORE _)::(y as LOAD _)::rest => 
        x :: squeeze(y::rest)
  | instr::(x as LOAD(_,Reg d, Direct(Reg s), i))::NOP::rest =>
        if use(instr, d) orelse gen(instr, s) then ??
        else squeeze(x::instr::rest)
  | instr::(x as LOAD(_,Reg d, _, i))::NOP::rest =>
        if use(instr,d) then ??
        else squeeze(x::instr::rest)
  | (x as MFLO _):: (y as MULDIV _) :: rest =>
        x :: squeeze (y::rest)
  | (x as MFLO(Reg d))::instr::rest =>
        if (use(instr,d) orelse gen(instr,d) then ??
        else squeeze(instr::x::rest)
  | instr :: (x as MULDIV(Reg a, Reg b)) :: rest =>
        if gen(instr,a) orelse gen(instr,b) then ??
        else squeeze(x::instr::rest)

val rec final =
 fn
  | instr::(x as LOAD(_,Reg d, Direct(Reg s), i))::NOP::rest =>
        if gen(instr, s) then instr::final(x::NOP::rest)
        else x::instr::(final rest)
  | instr :: (x as JUMP _) :: NOP :: rest =>
        x :: instr :: final rest
  | instr :: (x as BEQ(_,Reg a, Reg b, _)) :: NOP :: rest =>
        if gen(instr,a) orelse gen(instr,b) then instr::x::NOP::(final rest)
        else x::instr::(final rest)


\endcode
\filename{opcodes.nw}
\begindocs{0}
\chapter{Handling the MIPS opcodes}
\section{Introduction}

This file generates the code necessary to handle MIPS instructions
in a natural, mnemonic way from within ML.
All MIPS instructions occupy 32 bits, and since ML has no simple
32~bit data type, we use pairs of integerss to represent MIPS instructions.
A pair \code{}(hi,lo)\edoc{} of 16-bit integers holds the most and least significant
halfwords of the MIPS word.
ML integers are 31 bits, so this is more than adequate.

The biggest hassle in converting between these integer pairs and more
mnemonic representations is that it is too easy to make mistakes
(especially typographical errors) in writing the code.
For that reason, I have added an extra level of indirection to the
whole business by putting all of the instruction descriptions in
tables.
These tables are read by an awk script, which writes two ML files:
{\tt opcodes.sml} and {\tt mipsdecode.sml}.
The {\tt opcodes.sml} file contains the code needed to convert from
a mnemonic like \code{}add(3,4,9)\edoc{} (add the contents of register~3 to
the contents of register~4, placing the result in register~9) to 
the integer pair representation of the actual bits in that add instruction
(in this case \code{}(137,6176)\edoc{}).
The {\tt mipsdecode.sml} file contains a \code{}decode\edoc{} function that converts
from the integer pair representation of instructions to a string
representation.
The string representation is a little hokey at the moment (that is,
it's different from the one used in the MIPS book), but it represents
a nice compromise between being readable and easy to generate.

I have contemplating generating a third file to test the whole
business.
The idea would be to have a function that would write out (to files)
two
parallel representations of the same instruction stream (presumably
one copy of each known instruction).
One representation would be the binary one understood by the MIPS.
The other representation would be a string representation.
We could then use a tool like {\tt gdb} or {\tt adb} to print out
the binary as an instruction sequence (i.e. convert back to
a second string representation) and compare the string representations
to see if they make sense.

\paragraph{Possible bugs}
This code should be gone over with care to make sure that negative
operands (e.g. in \code{}offset\edoc{}) won't break the code.


\enddocs
\begindocs{1}

We need a special line in the Makefile to handle this file, since
it writes both an awk program and that program's input.  The input
is in module {\tt \LL{}opcodes table\GG{}} so the line is
$$\hbox{\code{}      $(NOTANGLE) '-Ropcodes table' opcodes.ow > opcodes\edoc{}}$$
The input is nothing but a sequence of tables, each labelled, and
processed one after anothing according to the label.
The label is always a single word on a line by itself.
Tables end with blank lines.
\enddocs
\begindocs{2}
The opcode-to-pair code is written to the standard output, in 
\code{}structure Opcodes\edoc{}.
The pair-to-string code is written to \code{}"mipsdecode.sml"\edoc{}, in
\code{}structure MipsDecode\edoc{}.

We begin by defining and and shift functions.
We make pessimistic assumptions about shifting, trying always to
keep the arguments between 0 and 31 inclusive.
\enddocs
\begincode{3}
\moddef{BEGIN}\endmoddef
print "structure Opcodes = struct"
print "val andb = Bits.andb"
print "fun lshift(op1,amt) = "
print "    if amt<0 then Bits.rshift(op1,0-amt)"
print "    else Bits.lshift(op1,amt)"
print "nonfix sub"      # bug fixes; want \code{}sub\edoc{} to be a MIPS opcode
print "nonfix div"      # bug fixes; want \code{}div\edoc{} to be a MIPS opcode

decode = "mipsdecode.sml";
print "structure MipsDecode = struct" > decode
print "val andb = Bits.andb" > decode
print "fun rshift(op1,amt) = " > decode
print "    if amt<0 then Bits.lshift(op1,0-amt)" > decode
print "    else Bits.rshift(op1,amt)" > decode
\endcode
\begincode{4}
\moddef{END}\endmoddef
\LA{}write out the definitions of the decoding functions\RA{}
print "end (* Opcodes *)"
print "end (* Decode *)" > decode
\endcode
\begindocs{5}
The sections BEGIN and END are drawn from 
 our universal model of an awk program:
\enddocs
\begincode{6}
\moddef{*}\endmoddef
BEGIN \{
  \LA{}BEGIN\RA{}
\}
\LA{}functions\RA{}
\LA{}statements\RA{}
END \{
  \LA{}END\RA{}
\}
\endcode
\begindocs{7}
\section{The opcode tables}
The numeric codes for all the MIPS opcodes are described in three
tables in the MIPS book on page~A-87.
Normal opcodes are six bits, and appear in the \code{}opcode\edoc{} field of the
instruction.
Two opcodes \code{}special\edoc{} and \code{}bcond\edoc{} stand for several instructions.
These instructions are decoded by checking the bit-pattern in the
\code{}funct\edoc{} and \code{}cond\edoc{} fields of the instructions, respectively.

The tables show which opcodes correspond to which bit-patterns.
For example, the \code{}slti\edoc{} corresponds to an \code{}opcode\edoc{} value of octal~12.
The table headed \code{}opcode\edoc{} gives the mnemonics for all six-bit patterns
in the \code{}opcode\edoc{} field.
The \code{}special\edoc{} table shows patterns for the \code{}funct\edoc{} field, used with
the \code{}special\edoc{} opcode.
The \code{}bcond\edoc{} table shows five-bit patterns for the \code{}cond\edoc{} field,
used with the \code{}bcond\edoc{} opcode.
In all tables, stars (\code{}*\edoc{}) stand for unused fields.

Each table is terminated with a blank line.
\enddocs
\begincode{8}
\moddef{opcodes table}\endmoddef
                            opcode
special bcond   j       jal     beq     bne     blez    bgtz
addi    addiu   slti    sltiu   andi    ori     xori    lui
cop0    cop1    cop2    cop3    *       *       *       *
*       *       *       *       *       *       *       *
lb      lh      lwl     lw      lbu     lhu     lwr     *
sb      sh      swl     sw      *       *       swr     *
lwc0    lwc1    lwc2    lwc3    *       *       *       *
swc0    swc1    swc2    swc3    *       *       *       *

                            special
sll     *       srl     sra     sllv    *       srlv    srav
jr      jalr    *       *       syscall break   *       *
mfhi    mthi    mflo    mtlo    *       *       *       *
mult    multu   div     divu    *       *       *       *
add     addu    sub     subu    and'    or      xor     nor
*       *       slt     sltu    *       *       *       *
*       *       *       *       *       *       *       *
*       *       *       *       *       *       *       *

                            bcond
bltz    bgez    *       *       *       *       *       *
*       *       *       *       *       *       *       *
bltzal  bgezal  *       *       *       *       *       *
*       *       *       *       *       *       *       *


\endcode
\begindocs{9}
The instructions codes for Coprocessor 1 (floating point)
are takin from page B-28 of the Mips book.
\enddocs
\begincode{10}
\moddef{opcodes table}\endmoddef
                            cop1
add_fmt sub_fmt mul_fmt div_fmt *       abs_fmt mov_fmt neg_fmt
*       *       *       *       *       *       *       *
*       *       *       *       *       *       *       *
*       *       *       *       *       *       *       *
cvt_s   cvt_d   *       *       cvt_w   *       *       *
*       *       *       *       *       *       *       *
c_f     c_un    c_eq    c_ueq   c_olt   c_ult   c_ole   c_ule
c_sf    c_ngle  c_seq   c_ngl   c_lt    c_nge   c_le    c_ngt

\endcode
\begindocs{11}

Now we have to deal with reading these tables, and extracting the
information stored therein.
First of all, for each mnemonic \code{}$i\edoc{} we store the corresponding bit
pattern (as an integer, \code{}code\edoc{}) in the array \code{}numberof[$i] \edoc{}.
Then, we store the type of the mnemonic (ordinary \code{}OPCODE\edoc{}, 
\code{}SPECIAL\edoc{}, \code{}BCOND\edoc{}, of \code{}COP1\edoc{}) in the array \code{}typeof[$i] \edoc{}.
Finally, we store inverse (a map from type and bit pattern to mnemonic)
in the \code{}opcode\edoc{} array.
\enddocs
\begincode{12}
\moddef{store opcode information}\endmoddef
if ($i != "*") \{
        numberof[$i] = code
        typeof[$i] = type
        opcode[type,code] = $i
\} else \{
        opcode[type,code] = "reserved"
\}
\endcode
\begindocs{13}
The types are just constants set at the beginning.
\enddocs
\begincode{14}
\moddef{BEGIN}\endmoddef
OPCODE = 1 ; SPECIAL = 2 ; BCOND = 3 ; COP1 = 4
\endcode
\begindocs{15}
We determine the type by scanning the header word that precedes
each table.
Once we see the appropriate table header, we set one of \code{}opcodes\edoc{},
\code{}specials\edoc{}, and \code{}bconds\edoc{}, so that determining the type is easy:
\enddocs
\begincode{16}
\moddef{set \code{}type\edoc{}}\endmoddef
type = OPCODE * opcodes + SPECIAL * specials + BCOND * bconds + COP1 * cop1s
\endcode
\begindocs{17}
Seeing the right table header causes us to set the right variable.
We also remember the line number, because we use the positions of later
lines to help extract the bit patterns from the table.
\enddocs
\begincode{18}
\moddef{statements}\endmoddef
NF == 1 && $1 == "opcode" \{
        startline = NR
        opcodes = 1
        next
\}
NF == 1 && $1 == "special" \{
        startline = NR
        specials = 1
        next
\}
NF == 1 && $1 == "bcond" \{
        startline = NR
        bconds = 1
        next
\}
NF == 1 && $1 == "cop1" \{
        startline = NR
        cop1s = 1
        next
\}
\endcode
\begindocs{19}
Any time we see a blank line, that ends the appropriate table.
\enddocs
\begincode{20}
\moddef{statements}\endmoddef
NF == 0 \{opcodes = 0; specials = 0; bconds = 0; cop1s = 0
        \LA{}blank line resets\RA{}
\}
\endcode
\begindocs{21}
Here is the code that actually extracts the bit patterns from
the opcode tables.
The code is the same for each of the three tables.

The \code{}insist_fields(8)\edoc{} issues an error message and returns false (0)
unless there are exactly 8 fields on the input line.
\enddocs
\begincode{22}
\moddef{statements}\endmoddef
opcodes || specials || bconds || cop1s \{
        if (!insist_fields(8)) next
        \LA{}set \code{}type\edoc{}\RA{}
        major = NR - startline - 1              # major octal digit from row
        for (i=1; i<= NF; i++) \{
                minor = i-1                     # minor octal digit from column
                code = minor + 8 * major
                \LA{}store opcode information\RA{}
        \}
\}
\endcode
\begindocs{23}
\section{The instruction fields}
Now that we've dealt with the opcodes, we'll handle other fields of
the instruction.
This table tells us the position of each field within the word,
so that if we know a bit-pattern for each field, we can assemble
all the fields into an instruction.

Not all fields are used in all instructions.
Later we'll have a table that indicates exactly which fields are used in
which instructions.
For now, we just list the fields and their positions with the
understanding that some fields will overlap.

The table is taken from the MIPS book, page A-3.
The numbers are the numbers of the starting and ending bit positions,
where 0 is the least and 31 the most significant bit.
The names are exactly those used in the book except \code{}op'\edoc{} has been
substituted for \code{}op\edoc{} since \code{}op\edoc{} is a reserved word in ML.

If a field is signed, we put a \code{}+\edoc{}~sign as the first character
of its name.
The sign information is used only in decoding (I think).
\enddocs
\begincode{24}
\moddef{opcodes table}\endmoddef
                        fields
op' 26 31
rs 21 25
rt 16 20
+immed 0 15
+offset 0 15
base 21 25
target 0 25
rd 11 15
shamt 6 10
funct 0 5
cond 16 20
\LA{}floating point load/store fields\RA{}
\LA{}floating point computation fields\RA{}

\endcode
\begindocs{25}
From page B-5.  Most fields are the same as the CPU instruction formats.
\enddocs
\begincode{26}
\moddef{floating point load/store fields}\endmoddef
ft 16 20
\endcode
\begindocs{27}
From page B-6.  Many fields are reused from earlier specifications.
The computational instructions all have a one bit in position 25.
Instead of trying to insert special code to handle that, we cheat on
it by making that bit part of the format, and cheating on the format.
Thus:
\enddocs
\begincode{28}
\moddef{floating point computation fields}\endmoddef
fmt 21 25
fs 11 15
fd 6 10
\endcode
\begincode{29}
\moddef{write format info}\endmoddef
print "val S_fmt = 16+0"
print "val D_fmt = 16+1"
print "val W_fmt = 16+4"

\endcode
\begindocs{30}
The setup for the fields is similar to that used for the opcodes.
\enddocs
\begincode{31}
\moddef{statements}\endmoddef
NF == 1 && $1 == "fields" \{
        startline = NR
        fields = 1
        \LA{}write format info\RA{}
        next
\}
\endcode
\begincode{32}
\moddef{blank line resets}\endmoddef
fields = 0
\endcode
\begincode{33}
\moddef{statements}\endmoddef
fields \{
        if (!insist_fields(3)) next
        fieldname = $1;  low = $2; high = $3
        \LA{}look for sign in \code{}fieldname\edoc{} and set \code{}signed\edoc{}\RA{}
        fieldnames[fieldname]= 1        # rememeber all the field names

        \LA{}write to standard output a function to convert bit-pattern to pair\RA{}
        \LA{}write to \code{}decode\edoc{} a function to extract field from pair\RA{}

\}
\endcode
\begincode{34}
\moddef{look for sign in \code{}fieldname\edoc{} and set \code{}signed\edoc{}}\endmoddef
if (substr(fieldname,1,1)=="+") \{
        signed = 1
        fieldname = substr(fieldname,2)
\} else \{
        signed = 0
\}
\endcode
\begindocs{35}

The idea is that for each of these fields, we want to write a function
that will take an integer argument and shift it by the right amount.
Since we have to represent the 32-bit quantities as pairs of integers,
we actually use two functions, one for the high half and one for the low.
So, for example, for the \code{}rd\edoc{} field we will produce two function definitions,
\code{}rdHI\edoc{} and \code{}rdLO\edoc{}.

The awk function \code{}function_definition\edoc{} is used to compute ML function
definitions.
It takes as arguments the name of the function and the number of arguments
to that function.
The arguments are numbered \code{}A1\edoc{}, \code{}A2\edoc{}, et cetera.

The functions themselves are all tedious combinations of ands and shifts.
At one time I had convinced myself that this worked.
\enddocs
\begincode{36}
\moddef{write to standard output a function to convert bit-pattern to pair}\endmoddef
if (low >= 16) \{
        printf "%s", function_definition(fieldname "LO",1); print "0"
\} else \{
        printf "%s", function_definition(fieldname "LO",1)
        printf "andb(lshift(A1,%d),65535)\\n", low
\}
if (high < 16) \{
        printf "%s", function_definition(fieldname "HI",1); print "0"
\} else \{
        printf "%s", function_definition(fieldname "HI",1)
        printf "lshift(A1,%s)\\n", mlnumber(low - 16)
\}
\endcode
\begindocs{37}
The inverse operation is
to extract a bit pattern from a pair.
We'll want that if we ever care to decode instructions.
This time, the function to extract e.g.\ field \code{}rd\edoc{} from a pair
is the function \code{}THErd\edoc{} applied to that pair.

The functions work first by extracting from the low part, then
from the high part, and adding everything together.
If the field is signed, we make the value negative if it is too high.
\enddocs
\begincode{38}
\moddef{write to \code{}decode\edoc{} a function to extract field from pair}\endmoddef
printf "%s", function_definition("THE" fieldname,2) > decode
if (signed) printf "let val n = " > decode
\LA{}print expression for unsigned value\RA{}
if (signed) \{
        printf "in if n < %d then n else n - %d\\nend\\n",
                2**(high-low), 2**(high-low+1) > decode
\}

\endcode
\begincode{39}
\moddef{print expression for unsigned value}\endmoddef
if (low >= 16) \{
        printf "0" > decode
\} else \{
        printf "andb(rshift(A2,%d),%d)", low,
                        (2**(min(15,high)-low+1)-1) > decode
\}
printf " + " > decode
if (high < 16) \{
        printf "0\\n" > decode
\} else \{
        printf "rshift(andb(A1,%d),%s)\\n", (2**(high-16+1)-1),
                        mlnumber(low - 16) > decode
\}
\endcode
\begindocs{40}
ML uses a strange minus sign (\code{}~\edoc{} instead of \code{}-\edoc{}), 
so we print numbers that might be negative like this:
\enddocs
\begincode{41}
\moddef{functions}\endmoddef
function mlnumber(n, s) \{
        if (n<0) s = sprintf("~%d", -n)
        else s = sprintf("%d", n)
        return s
\}
\endcode
\begindocs{42}
For reasons best known to its designers, awk has no \code{}min\edoc{} function.
\enddocs
\begincode{43}
\moddef{functions}\endmoddef
function min(x,y)\{
        if (x<y) return x
        else return y
\}
\endcode
\begindocs{44}
\section{The list of instructions and their formats}
This is the section that tells which fields are used in what instructions,
and in what order the fields appear.
The information is from Appendix A
of the MIPS book and should be proofread.

To cut down on the number of ML functions generated, we can comment out
instructions with a \code{}#\edoc{} in the first column.
This means that no code will be generated for the instruction, and
it won't appear in the \code{}structure Opcodes\edoc{}.
\enddocs
\begincode{45}
\moddef{opcodes table}\endmoddef
                        instructions
add rd rs rt
addi rt rs immed
addiu rt rs immed
addu rd rs rt
and' rd rs rt
andi rt rs immed
beq rs rt offset
bgez rs offset
bgezal rs offset
bgtz rs offset
blez rs offset
bltz rs offset
bltzal rs offset
bne rs rt offset
break
div rs rt
divu rs rt
j target
jal target
jalr rs rd
jr rs
lb rt offset base
lbu rt offset base
lh rt offset base
lb rt offset base
lhu rt offset base
lui rt immed
lw rt offset base
lwl rt offset base
lwr rt offset base
mfhi rd
mflo rd
mthi rs
mtlo rs
mult rs rt
multu rs rt
nor rd rs rt
or rd rs rt
ori rt rs immed
sb rt offset base
sh rt offset base
sll rd rt shamt
sllv rd rt rs
slt rd rs rt
slti rt rs immed
sltiu rt rs immed
sltu rd rs rt
sra rd rt shamt
srav rd rt rs
srl rd rt shamt
srlv rd rt rs
sub rd rs rt
subu rd rs rt
sw rt offset base
swl rt offset base
swr rt offset base
syscall
xor rd rs rt
xori rt rs immed
\LA{}floating point instructions\RA{}


\endcode
\begindocs{46}
We define only those floating point instructions we seem likely to need.
To distinguish them as floating point we append an f to their names.
\enddocs
\begincode{47}
\moddef{floating point instructions}\endmoddef
add_fmt fmt fd fs ft
div_fmt fmt fd fs ft
lwc1 ft offset base
mul_fmt fmt fd fs ft
neg_fmt fmt fd fs
sub_fmt fmt fd fs ft
swc1 ft offset base
c_seq fmt fs ft
c_lt fmt fs ft
\endcode
\begindocs{48}

 Here is a terrible hack to enable us to construct branch on coprocessor~1
true or false.
We will use \code{}fun bc1f offset = cop1(0,offset)\edoc{} and
        \code{}fun bc1t offset = cop1(1,offset)\edoc{}.
\enddocs
\begincode{49}
\moddef{floating point instructions}\endmoddef
cop1 rs rt offset
\endcode
\begindocs{50}



\enddocs
\begindocs{51}
For each instruction, we define an ML function with the appropriate
number of arguments.
When that function is given an integer in each argument,
it converts the whole thing to one MIPS instruction, represented as an
integer pair.

The implementation is a bit of a grubby mess.
Doing the fields is straightforward enough, but
for each mnemonic we have to do something different based
on its type, because each type of opcode goes in a different
field.
Moreover, for mnemonics of type \code{}SPECIAL\edoc{}, \code{}BCOND\edoc{}, and \code{}COP1\edoc{} we
have to generate \code{}special\edoc{}, \code{}bcond\edoc{}, and \code{}cop1\edoc{} in the \code{}op'\edoc{} field.
Finally, we have to do it all twice; once for the high order
halfword and once for the low order halfword.
\enddocs
\begincode{52}
\moddef{compute function that generates this instruction}\endmoddef
        printf "%s", function_definition(opname, NF-1)
        printf "("      # open parenthesis for pair
        for (i=2; i<= NF; i++) \{
                if (!($i in fieldnames)) \LA{}bad field name\RA{}
                printf "%sHI(A%d)+", $i, i-1
        \}
        if (typeof[opname]==OPCODE) \{
                printf "op'HI(%d)", numberof[opname]
        \} else if (typeof[opname]==SPECIAL) \{
                printf "op'HI(%d)+", numberof["special"]
                printf "functHI(%d)", numberof[opname]
        \} else if (typeof[opname]==BCOND) \{
                printf "op'HI(%d)+", numberof["bcond"]
                printf "condHI(%d)", numberof[opname]
        \} else if (typeof[opname]==COP1) \{
                printf "op'HI(%d)+", numberof["cop1"]
                printf "functHI(%d)", numberof[opname]
        \} else \LA{}bad operator name\RA{}
        printf ", "
        for (i=2; i<= NF; i++) \{
                if (!($i in fieldnames)) \LA{}bad field name\RA{}
                printf "%sLO(A%d)+", $i, i-1
        \}
        if (typeof[opname]==OPCODE) \{
                printf "op'LO(%d)", numberof[opname]
        \} else if (typeof[opname]==SPECIAL) \{
                printf "op'LO(%d)+", numberof["special"]
                printf "functLO(%d)", numberof[opname]
        \} else if (typeof[opname]==BCOND) \{
                printf "op'LO(%d)+", numberof["bcond"]
                printf "condLO(%d)", numberof[opname]
        \} else if (typeof[opname]==COP1) \{
                printf "op'LO(%d)+", numberof["cop1"]
                printf "functLO(%d)", numberof[opname]
        \} else \LA{}bad operator name\RA{}
        printf ")\\n"
\endcode
\begindocs{53}

Setup is as before.
\enddocs
\begincode{54}
\moddef{statements}\endmoddef
NF == 1 && $1 == "instructions" \{
        startline = NR
        instructions = 1
        next
\}
\endcode
\begincode{55}
\moddef{blank line resets}\endmoddef
instructions= 0
\endcode
\begincode{56}
\moddef{statements}\endmoddef
instructions && $0 !~ /^#/ \{
        opname = $1

        \LA{}compute string displayed when this instruction is decoded\RA{}
########        gsub("[^a-z']+"," ")   ### ill-advised

        \LA{}compute function that generates this instruction\RA{}
\}

\endcode
\begindocs{57}
\paragraph{Decoding instructions}
When we've decoded an instruction, we have to display some sort of
string representation that tells us what the instruction is.
Ideally we should display either just what the assembler expects,
or perhaps just what dbx displays when asked about actual instructions
in memory images.

For now, we just give the mnemonic for the instruction, followed
by a description of each field (followed by a newline).
The fields are described as name-value pairs.

We rely on the fact that for a field e.g.\ \code{}rd\edoc{}, the string
representation of the value of that field is in \code{}Srd\edoc{}.
\enddocs
\begincode{58}
\moddef{compute string displayed when this instruction is decoded}\endmoddef
temp = "\\"" opname " \\""
for (i=2; i<=NF; i++) \{
        temp = sprintf( "%s ^ \\"%s = \\" ^ S%s", temp, $i, $i)
        if (i<NF) temp = sprintf("%s ^ \\",\\" ", temp)
\}
displayof[opname]=temp " ^ \\"\\\\n\\""

\endcode
\begindocs{59}
The implementation of the decoding function is split into several parts.
First, we have to be able to extract any field from an instruction.
Then, we have to be able to decode four kinds of opcodes:
\code{}OPCODE\edoc{}s, \code{}BCOND\edoc{}s,  \code{}SPECIAL\edoc{}s, and \code{}COP1\edoc{}s.
The main function is the one that does ordinary opcodes.
The others are auxiliary.
\enddocs
\begincode{60}
\moddef{write out the definitions of the decoding functions}\endmoddef
printf "%s", function_definition("decode",2) > decode
print "let" > decode
  \LA{}write definitions of integer and string representations of each field\RA{}
  \LA{}write expression that decodes the \code{}funct\edoc{} field for \code{}special\edoc{}s\RA{}
  \LA{}write expression that decodes the \code{}cond\edoc{} field for \code{}bcond\edoc{}s\RA{}
  \LA{}write expression that decodes the \code{}funct\edoc{} field for \code{}cop1\edoc{}s\RA{}
print "in" > decode
  \LA{}write \code{}case\edoc{} expression that decodes the \code{}op'\edoc{} field for each instruction\RA{}
print "end" > decode
\endcode
\begindocs{61}
We give each field its own name for an integer version, and its name
preceded by \code{}S\edoc{} for its string version.
These values are all computed just once, from the arguments to the
enclosing function (\code{}decode\edoc{}).
\enddocs
\begincode{62}
\moddef{write definitions of integer and string representations of each field}\endmoddef
for (f in fieldnames) \{
        printf "val %s = THE%s(A1,A2)\\n", f, f  > decode
        printf "val S%s = Integer.makestring %s\\n", f, f  > decode
\}
\endcode
\begindocs{63}
The next three functions are very much of a piece.
They are just enormous \code{}case\edoc{} expressions that match up integers
(bit patterns) to strings.
The fundamental operation is printing out a decimal value and a string
for each opcode:
\enddocs
\begincode{64}
\moddef{if \code{}name\edoc{} is known, display a case for it}\endmoddef
if (name != ""  && name != "reserved") \{
        \LA{}print space or bar (\code{}|\edoc{})\RA{}
        disp = displayof[name]
        if (disp=="") disp="\\"" name "(??? unknown format???)\\\\n\\""
        printf "%d => %s\\n", code, disp > decode
\}
\endcode
\begindocs{65}
Cases must be separated by vertical bars.
We do the separation by putting a vertical bar before each case except
the first.
We use a hack to discover the first; we assume that code~0 is always
defined, and so it will always be the first.
\enddocs
\begincode{66}
\moddef{print space or bar (\code{}|\edoc{})}\endmoddef
if (code!=0) printf " | "  > decode # hack but it works
else printf "   " > decode
\endcode
\begincode{67}
\moddef{write expression that decodes the \code{}funct\edoc{} field for \code{}special\edoc{}s}\endmoddef
print "val do_special ="  > decode
print "(case funct of" > decode
for (code=0; code<256; code++) \{
        name = opcode[SPECIAL,code]
        \LA{}if \code{}name\edoc{} is known, display a case for it\RA{}
\}
printf " | _ => \\"unknown special\\\\n\\"\\n" > decode
print "   ) " > decode
\endcode
\begincode{68}
\moddef{write expression that decodes the \code{}cond\edoc{} field for \code{}bcond\edoc{}s}\endmoddef
print "val do_bcond =" > decode
print "(case cond of" > decode
for (code=0; code<256; code++) \{
        name = opcode[BCOND,code]
        \LA{}if \code{}name\edoc{} is known, display a case for it\RA{}
\}
printf " | _ => \\"unknown bcond\\\\n\\"\\n" > decode
print "   ) " > decode
\endcode
\begincode{69}
\moddef{write expression that decodes the \code{}funct\edoc{} field for \code{}cop1\edoc{}s}\endmoddef
print "val do_cop1 =" > decode
print "(case funct of" > decode
for (code=0; code<256; code++) \{
        name = opcode[COP1,code]
        \LA{}if \code{}name\edoc{} is known, display a case for it\RA{}
\}
printf " | _ => \\"unknown cop1\\\\n\\"\\n" > decode
print "   ) " > decode
\endcode
\begindocs{70}
The major expression is a little more complicated, because it has to
check for \code{}special\edoc{}, \code{}bcond\edoc{}, and \code{}cop1\edoc{} and handle those separately.
\enddocs
\begincode{71}
\moddef{write \code{}case\edoc{} expression that decodes the \code{}op'\edoc{} field for each instruction}\endmoddef
print "(case op' of" > decode
for (code=0; code<256; code++) \{
        name = opcode[OPCODE,code]
        if (name=="special") \{
                \LA{}print space or bar (\code{}|\edoc{})\RA{}
                printf "%d => %s\\n", code, "do_special" > decode
        \} else if (name=="bcond") \{
                \LA{}print space or bar (\code{}|\edoc{})\RA{}
                printf "%d => %s\\n", code, "do_bcond" > decode
        \} else if (name=="cop1") \{
                \LA{}print space or bar (\code{}|\edoc{})\RA{}
                printf "%d => %s\\n", code, "do_cop1" > decode
        \} else \LA{}if \code{}name\edoc{} is known, display a case for it\RA{}
\}
printf " | _ => \\"unknown opcode\\\\n\\"\\n" > decode
print "   ) " > decode
\endcode
\begindocs{72}
\section{testing}
One day someone will have to modify the instruction handler so that
it generates a test invocation of each instruction.
Then the results can be handed to something like adb or dbx and we can
see whether the system agrees with us about what we're generating.

\enddocs
\begindocs{73}
\section{Defining ML functions}
The awk function \code{}function_definition\edoc{} is used to
come up with ML function definitions.
It takes as arguments the name of the function and the number of arguments
to that function, and returns a string containing the initial part of
the function definition.
Writing an expression following that string will result in a complete
ML function.

If we ever wanted to define these things as C preprocessor macros instead,
we could do it by substituting \code{}macro_definition\edoc{}.
I'm not sure it would ever make sense to do so, but I'm leaving the
code here anyway.
\enddocs
\begincode{74}
\moddef{functions}\endmoddef
function function_definition(name, argc,  i, temp) \{
        if (argc==0) \{
                temp = sprintf("val %s = ", name)
        \} else \{
                temp = sprintf( "fun %s(", name)
                for (i=1; i< argc; i++) temp = sprintf("%sA%d,", temp,i)
                temp = sprintf( "%sA%d) = ", temp, argc)
        \}
        return temp
\}
\endcode
\begincode{75}
\moddef{useless functions}\endmoddef
function macro_definition(name, argc,  i, temp) \{
        if (argc==0) \{
                temp = sprintf("#define %s ", name)
        \} else \{
                temp = sprintf( "#define %s(", name)
                for (i=1; i< argc; i++) temp = sprintf("%sA%d,", temp,i)
                temp = sprintf( "%sA%d) ", temp, argc)
        \}
        return temp
\}
\endcode
\begindocs{76}
\section{Handling error conditions}
Here are a bunch of uninteresting functions and modules
that handle error conditions.
\enddocs
\begincode{77}
\moddef{bad operator name}\endmoddef
\{
        print "unknown opcode", opname, "on line", NR > stderr
        next
\}
\endcode
\begincode{78}
\moddef{bad field name}\endmoddef
\{
        print "unknown field", $i, "on line", NR > stderr
        next
\}
\endcode
\begincode{79}
\moddef{BEGIN}\endmoddef
stderr="/dev/tty"
\endcode
\begincode{80}
\moddef{functions}\endmoddef
function insist_fields(n) \{
        if (NF != n) \{
                print "Must have", n, "fields on line",NR ":", $0 > stderr
                return 0
        \} else \{
                return 1
        \}
\}
\endcode
\begindocs{81}
\section{Leftover junk}
Like a pack rat, I never throw out anything that might be useful again later.
\enddocs
\begincode{82}
\moddef{junk}\endmoddef
function thetype(n) \{
        if (n==OPCODE) return "OPCODE"
        else if (n==SPECIAL) return "SPECIAL"
        else if (n==BCOND) return "BCOND"
        else if (n==COP1) return "COP1"
        else return "BADTYPE"
\}
\endcode
\begincode{83}
\moddef{decoding junk}\endmoddef
for (f in fieldnames) \{
        printf "^ \\"\\\\n%s = \\" ^ Integer.makestring %s\\n",f,f > decode
\}
printf "^\\"\\\\n\\"\\n" > decode
\endcode
\filename{/u/nr/sml/36/src/runtime/MIPS.prim.nw}
\begindocs{0}
\section{Assembly-language primitives for the run-time system}
This file is derived from the similar file for the VAX.
We include \code{}<regdef.h>\edoc{}, which defines names for the registers.
\enddocs
\begincode{1}
\moddef{*}\endmoddef
#include "tags.h"
#include "prof.h"
#include "ml.h"
#include "prim.h"
\LA{}register definitions\RA{}
\LA{}\code{}String\edoc{} and \code{}Closure\edoc{} definitions\RA{}
        .data
\LA{}data segment items\RA{}
        .text
\LA{}run vector\RA{}
\LA{}array\RA{}
\LA{}string and bytearray\RA{}
\LA{}C linkage\RA{}
\LA{}calling C routines\RA{}
\LA{}system calls\RA{}
\LA{}floating point\RA{}
/* this bogosity is for export.c */
        .globl  startptr
startptr: .word    __start      /* just a guess... */


\endcode
\begindocs{2}
We define a couple of macros for creating strings and closures.
When calling \code{}String\edoc{} we should use a literal string whose length 
is a multiple of~4.

The closure of a primitive function
 is a record of length~1, containing a pointer to the first 
instruction in the function.
All closures have length~1 because there aren't any free variables in any 
of the primitive functions.
\enddocs
\begincode{3}
\moddef{\code{}String\edoc{} and \code{}Closure\edoc{} definitions}\endmoddef
#define String(handle,len,str) .align 2;\\
                               .set noreorder;\\
                               .word len*power_tags+tag_string;\\
                               handle: .ascii str;\\
                               .set reorder
#define Closure(name) .align    2;\\
                      .set noreorder;\\
                      .word     mak_desc(1,tag_record);\\
                      name:     .word 9f; /* address of routine */ \\
                      .word     1; /* here for historical reasons */\\
                      .word     tag_backptr;\\
                      .set reorder;\\
                      9:
\endcode
\begindocs{4}
\subsection{Allocation and garbage collection}
Put a brief summary here: gc is caused by storing beyond the end of high 
memory.
For that reason we store the last word of objects first.

\enddocs
\begindocs{5}
\subsection{Register usage}
\input regs
\enddocs
\begincode{6}
\moddef{register definitions}\endmoddef
#define stdarg 2
#define stdcont 3
#define stdclos 4
#define storeptr 22
#define dataptr 23
#define exnptr 30
#define artemp1 24
#define artemp2 25
#define artemp3 20
#define ptrtemp 21
\endcode
\begindocs{7}
The MIPS version of Unix doesn't put underscores in front of global names.

First we define the global \code{}runvec\edoc{}.
This is an Ml object that represents the substructure \code{}A\edoc{} in 
{\tt boot/assembly.sig}.
All the ML functions will call these primitives by grabbing them
out of this record, which contains pointers to all the primitives.
\enddocs
\begincode{8}
\moddef{run vector}\endmoddef

        .globl  runvec
        .align  2
        .word   mak_desc(8,tag_record)
runvec:
        .word   array_v
        .word   callc_v
        .word   create_b_v
        .word   create_s_v
        .word   floor_v
        .word   logb_v
        .word   scalb_v
        .word   syscall_v

\endcode
\begindocs{9}
\subsection{Creating arrays, strings, and bytearrays}
\code{}array(m,x)\edoc{} creates an array of length $n$, each element initialized
to $x$.
(The corresponding record creation code is not implemented as a primitive; 
the ML compiler generates that code in line.)
$n$~is a tagged integer representing the length in words; 
$x$ can be any value.
This routine will loop forever (or until something strange happens
in memory) if $n<0$.

We need to be careful in the implementation to make sure all register values 
are sensible when garbage collection might occur.

If the order of instructions seems a little strange, it's because we try to
make sensible use of load delay slots.
\enddocs
\begincode{10}
\moddef{array}\endmoddef
Closure(array_v)
        lw $artemp1,0($stdarg)          /* tagged length in $artemp1 */
        lw $10,4($stdarg)               /* get initial value in $10 */
        sra $artemp1,1                  /* drop the tag bit */
        sll $artemp2,$artemp1,width_tags /* length for descr into $artemp2 */
        ori $artemp2,tag_array          /* complete descriptor into $artemp2 */
        sll $artemp1,2                  /* get length in bytes into $artemp1 */
.set noreorder  /* can't reorder because collection might occur */
        add $artemp3,$artemp1,$dataptr  /* $artemp3 points to last word 
                                                        of new array*/
badgc1: sw $0,($artemp3)                        /* clear; causes allocation */
.set reorder  /* can rearrange instructions again */
\endcode
\begincode{11}
\moddef{load bad pc into \code{}$artemp2\edoc{}; branch to \code{}badpc\edoc{} if == \code{}$artemp1\edoc{}}\endmoddef
        la $artemp2,badgc1
        beq $artemp1,$artemp2,badpc
\endcode
\begindocs{12}
At this point garbage collection may have occurred.
Ordinarily, we couldn't rely
on the value in \code{}$artemp3\edoc{}, because it's not forwarded.
However, this is one of the special garbage collection locations, 
so we do know that \code{}$artemp3\edoc{} is sensible.
But, since we really want something larger, we recompute it, 
using the (possibly changed) value of the data pointer.
Extra cleverness here might enable us to save one instruction.
\enddocs
\begincode{13}
\moddef{array}\endmoddef
        sw $artemp2,0($dataptr)         /* store the descriptor */
        add $dataptr,4                  /* points to new object */
        add $artemp3,$artemp1,$dataptr  /* beyond last word of new array*/
        add $stdarg,$dataptr,$0         /* put ptr in return register
                                        (return val = arg of continuation) */
        \LA{}store the initial value in every slot, leaving \code{}$dataptr\edoc{} pointing to the first free word\RA{}
        lw $10,0($stdcont)              /* grab continuation */
        j $10                           /* return */

\endcode
\begindocs{14}
With some clever thinking, the size of this loop could probably be cut 
from four instructions to three instructions.
\enddocs
\begincode{15}
\moddef{store the initial value in every slot, leaving \code{}$dataptr\edoc{} pointing to the first free word}\endmoddef
        b 2f
1:      sw $10,0($dataptr)              /* store the value */
        addi $dataptr,4                 /* on to the next word */
2:      bne $dataptr,$artemp3,1b        /* if not off the end, repeat */


\endcode
\begindocs{16}
\code{}create_b(n)\edoc{} creates a byte-array of length $n$, and
\code{}create_s(n)\edoc{} creates a string of length $n$.

We use the same code to create byte arrays and strings, since the only
difference is in the tags.
The odd arrangement of closures (odd because each one starts a new record)
causes no problems because this code isn't in the garbage-collectible region.
\enddocs
\begincode{17}
\moddef{string and bytearray}\endmoddef
Closure(create_b_v)
        addi $artemp3,$0,tag_bytearray  /* tag into $artemp3 */
        b       2f
Closure(create_s_v)
        addi $artemp3,$0,tag_string     /* tag into $artemp3 */
2:      
\endcode
\begindocs{18}
The length computation may be a bit confusing.
We are handed a tagged integer $2n+1$, and we need to compute the required
number of words, which is $\lfloor{n+3\over 4}\rfloor$.
This is just $\lfloor{(2n+1)+5\over 8}\rfloor$.
However, we'll save an instruction later if we happen to have one more than 
the number of words tucked away in a register, 
because $1+\lfloor{n+3\over 4}\rfloor$ is the number of words
we're taking from the data space (we include the descriptor).
So we compute $(2n+1)+13$ and continue accordingly
\enddocs
\begincode{19}
\moddef{string and bytearray}\endmoddef
        addi    $artemp1,$stdarg,13     /* $2n+14$ */
        sra     $artemp1,3              /* number of words in string+tag */
        sll     $artemp1,2              /* # of bytes allocated for str+tag */
.set noreorder /* don't cross gc boundary */
        add     $artemp2,$artemp1,$dataptr /* beyond last word of string */
badgc2: sw $0,-4($artemp2)              /* clear last; causes allocation */
.set reorder
        sra     $artemp2,$stdarg,1      /* untagged length in bytes */
        sll     $artemp2,width_tags     /* room for descriptor */
        or      $artemp2,$artemp3       /* descriptor */
        sw      $artemp2,0($dataptr)    /* store descriptor */
        addi    $stdarg,$dataptr,4      /* pointer to new string */
        add     $dataptr,$artemp1       /* advance; save 1 instruction */
        lw $10,0($stdcont)              /* grab continuation */
        j $10                           /* return */

\endcode
\begincode{20}
\moddef{load bad pc into \code{}$artemp2\edoc{}; branch to \code{}badpc\edoc{} if == \code{}$artemp1\edoc{}}\endmoddef
        la $artemp2,badgc2
        beq $artemp1,$artemp2,badpc

\endcode
\begindocs{21}
\subsection{Linkage with C code}
C always gains control first, and stuffs something appropriate into
the register save areas before starting ML by calling \code{}restoreregs\edoc{}.
It also puts something appropriate in the ML \code{}saved_pc\edoc{}.
\code{}restoreregs\edoc{} squirrels away the current state of the C runtime stack,
restores the ML registers, and finally jumps to the saved program counter.

When ML wants to call C, it calls \code{}saveregs\edoc{}, which saves the ML state 
in the appropriate save areas, then restores the C runtime stack and returns.
Before returning to C, it sets \code{}cause\edoc{} to something appropriate.

All programs must ensure that \code{}restoreregs\edoc{} {\em never} calls itself
recursively, because it is {\em not} reentrant.

The C end of this connection is on display in the \code{}runML()\edoc{} function of
\code{}~ml/src/runtime/callgc.c\edoc{}.
\enddocs
\begincode{22}
\moddef{data segment items}\endmoddef
bottom: .word 0                 /* C's saved stack pointer */
\endcode
\begincode{23}
\moddef{C linkage}\endmoddef
        .globl saveregs
        .globl handle_c
        .globl return_c
        .globl restoreregs
.ent restoreregs
restoreregs:
\LA{}save caller's stuff using MIPS calling conventions\RA{}
\LA{}enable floating point overflow and zerodivide exceptions\RA{}
        sw      $sp,bottom      /* save C's stack pointer */
\LA{}if \code{}saved_pc\edoc{} points to a bad spot, adjust it (destroys arithmetic temps)\RA{}
\LA{}restore the ML registers\RA{}
.set noat /* This trick will cause a warning, but the code is OK */
        lw      $at,saved_pc    /* grab the saved program counter */
        j       $at             /* and continue executing at that spot */
.set at

\endcode
\begindocs{24}
The next two functions are an exception handler and a continuation for
ML programs called from C.
Although neither appears to return any result (by manipulating \code{}$stdarg\edoc{},
they do return results.  
It's just that the C code on the other end gets the result out of 
\code{}saved_ptrs[0],\edoc{} where it expects to find \code{}$stdarg\edoc{}.
\enddocs
\begincode{25}
\moddef{C linkage}\endmoddef
Closure(handle_c) /* exception handler for ML functions called from C */
        li      $artemp1,CAUSE_EXN
        sw      $artemp1,cause
        b       saveregs
Closure(return_c) /* continuation for ML functions called from C */
        li      $artemp1,CAUSE_RET
        sw      $artemp1,cause
saveregs: 
\LA{}save the ML registers\RA{}
        lw      $sp,bottom      /* recover C's stack pointer */
\LA{}restore caller's stuff using MIPS calling conventions\RA{}
        j       $31     /* return to C program */
.end restoreregs

\endcode
\begincode{26}
\moddef{enable floating point overflow and zerodivide exceptions}\endmoddef
.set noat
        cfc1 $at,$31            /* grab fpa control register */
        ori  $at,$at,0x600      /* set O and Z bits */
        ctc1 $at,$31            /* return fpa control register */
.set at

\endcode
\begindocs{27}
The MIPS calling conventions are described in gory detail in Appendix~D
of the MIPS book; pages D-18 and following.

At the moment we don't save any floating point registers.
We save (on the stack) nine general-purpose registers, the global pointer,
 and the return address.

We always have to allocate at least 16 bytes for argument build,
because any C function might be varargs, and might begin by
spilling all of its registers into the argument build area (Hanson).
We allocate exactly sixteen bytes, planning to fiddle the stack if
(God forbid) we are ever asked to issue a system call with more than
16 bytes worth of arguments.

\enddocs
\begincode{28}
\moddef{save caller's stuff using MIPS calling conventions}\endmoddef
#define regspace 44
#define localspace 4
#define argbuild 16
#define framesize (regspace+localspace+argbuild) /* must be multiple of 8 */
#define frameoffset (0-localspace)
        subu $sp,framesize
\LA{}give .mask and save the C registers\RA{}

\endcode
\begincode{29}
\moddef{restore caller's stuff using MIPS calling conventions}\endmoddef
\LA{}restore the C registers\RA{}
        addu $sp,framesize
\endcode
\begindocs{30}
We don't save floating point regs yet.
\enddocs
\begincode{31}
\moddef{give .mask and save the C registers}\endmoddef
.mask 0xd0ff0000,0-localspace
        sw      $31,argbuild+40($sp)
        sw      $30,argbuild+36($sp)
        sw      $gp,argbuild+32($sp)
        sw      $23,argbuild+28($sp)
        sw      $22,argbuild+24($sp)
        sw      $21,argbuild+20($sp)
        sw      $20,argbuild+16($sp)
        sw      $19,argbuild+12($sp)
        sw      $18,argbuild+8($sp)
        sw      $17,argbuild+4($sp)
        sw      $16,argbuild($sp)
\endcode
\begincode{32}
\moddef{restore the C registers}\endmoddef
        lw      $31,argbuild+40($sp)
        lw      $30,argbuild+36($sp)
        lw      $gp,argbuild+32($sp)
        lw      $23,argbuild+28($sp)
        lw      $22,argbuild+24($sp)
        lw      $21,argbuild+20($sp)
        lw      $20,argbuild+16($sp)
        lw      $19,argbuild+12($sp)
        lw      $18,argbuild+8($sp)
        lw      $17,argbuild+4($sp)
        lw      $16,argbuild($sp)


\endcode
\begindocs{33}
There are two save areas; one for pointers and one for non-pointers.
(The pointer area may, of course, include tagged integers.)
The pointer area has special spots for standard argument, continuation,
and closure.
In addition there are special save areas for the special registers.
Register 31 is to be maintained constant relative to the program counter,
so we store the difference with \code{}saved_pc\edoc{}.
\enddocs
\begincode{34}
\moddef{save the ML registers}\endmoddef
                                        /* needn't save $1 */
        /* the big three: argument, continuation, closure */
        sw      $stdarg,saved_ptrs
        sw      $stdcont,saved_ptrs+4
        sw      $stdclos,saved_ptrs+8
        
        /* All the miscellaneous guys */
        sw      $5,saved_ptrs+12
        sw      $6,saved_ptrs+16
        sw      $7,saved_ptrs+20
        sw      $8,saved_ptrs+24
        sw      $9,saved_ptrs+28
        sw      $10,saved_ptrs+32
        sw      $11,saved_ptrs+36
        sw      $12,saved_ptrs+40
        sw      $13,saved_ptrs+44
        sw      $14,saved_ptrs+48
        sw      $15,saved_ptrs+52
        sw      $16,saved_ptrs+56
        sw      $17,saved_ptrs+60
        sw      $18,saved_ptrs+64
        sw      $19,saved_ptrs+68

        sw      $21, saved_ptrs+72

        sw      $artemp1,saved_nonptrs
        sw      $artemp2,saved_nonptrs+4
        sw      $artemp3,saved_nonptrs+8

        /* don't touch registers $26 and $27 */

        sw      $storeptr,saved_storeptr
        sw      $dataptr,saved_dataptr
        sw      $exnptr,saved_exnptr

\LA{}save $\code{}$31\edoc{}-\code{}saved_pc\edoc{}$ in \code{}saved_pc_diff\edoc{} (destroys \code{}$artemp1\edoc{})\RA{}


\endcode
\begincode{35}
\moddef{restore the ML registers}\endmoddef
        /* the big three: argument, continuation, closure */
        lw      $stdarg,saved_ptrs
        lw      $stdcont,saved_ptrs+4
        lw      $stdclos,saved_ptrs+8
        
        /* All the miscellaneous guys */
        lw      $5,saved_ptrs+12
        lw      $6,saved_ptrs+16
        lw      $7,saved_ptrs+20
        lw      $8,saved_ptrs+24
        lw      $9,saved_ptrs+28
        lw      $10,saved_ptrs+32
        lw      $11,saved_ptrs+36
        lw      $12,saved_ptrs+40
        lw      $13,saved_ptrs+44
        lw      $14,saved_ptrs+48
        lw      $15,saved_ptrs+52
        lw      $16,saved_ptrs+56
        lw      $17,saved_ptrs+60
        lw      $18,saved_ptrs+64
        lw      $19,saved_ptrs+68

        lw      $21, saved_ptrs+72

\LA{}restore \code{}$31\edoc{} from \code{}saved_pc\edoc{} \& \code{}saved_pc_diff\edoc{} (destroys \code{}$artemp1\edoc{})\RA{}
        lw      $artemp1,saved_nonptrs
        lw      $artemp2,saved_nonptrs+4
        lw      $artemp3,saved_nonptrs+8

        /* don't touch registers $26 and $27 */

        lw      $storeptr,saved_storeptr
        lw      $dataptr,saved_dataptr
        lw      $exnptr,saved_exnptr

\endcode
\begincode{36}
\moddef{save $\code{}$31\edoc{}-\code{}saved_pc\edoc{}$ in \code{}saved_pc_diff\edoc{} (destroys \code{}$artemp1\edoc{})}\endmoddef
        lw $artemp1,saved_pc
        subu $artemp1,$31,$artemp1      /* mustn't overflow */
        sw $artemp1,saved_pc_diff
\endcode
\begincode{37}
\moddef{restore \code{}$31\edoc{} from \code{}saved_pc\edoc{} \& \code{}saved_pc_diff\edoc{} (destroys \code{}$artemp1\edoc{})}\endmoddef
        lw $artemp1,saved_pc
        lw $31,saved_pc_diff
        addu $31,$artemp1               /* mustn't overflow */
\endcode
\begincode{38}
\moddef{data segment items}\endmoddef
saved_pc_diff:  .word 0


\endcode
\begindocs{39}
Because the Mips has no indexed addressing modes, there are special
circumstances under which we have to adjust the program counter before
a garbage collection.
The problem arises when we want to create an object whose size is not
known at compile time.
In order to do that, we have to add the size of the object to \code{}$dataptr\edoc{},
putting the result in a new register.
We then store at offset $-4$ from that register to allocate (and possibly
cause garbage collection).
That register can't be a pointer, because at the time of the gc it doesn't
point to anything sensible (in fact, by definition it points out of the
garbage-collectible region entirely).
If it is a nonpointer, though, it isn't changed by the garbage collection,
so when the collection is over, we attempt once again to store in exactly
the same place, causing another fault (unless the heap has been resized).

The solution is a hack.  Since there are only two places this problem
can occur, we check \code{}saved_pc\edoc{} against the offending program counters.
If we find one, we reduce \code{}saved_pc\edoc{} by 4 (the size of one instruction),
causing the addition to be repeated.

\enddocs
\begincode{40}
\moddef{if \code{}saved_pc\edoc{} points to a bad spot, adjust it (destroys arithmetic temps)}\endmoddef
    lw $artemp1,saved_pc
    \LA{}load bad pc into \code{}$artemp2\edoc{}; branch to \code{}badpc\edoc{} if == \code{}$artemp1\edoc{}\RA{}
    b 1f
badpc:
    subu $artemp1,4             /* adjust */
    sw $artemp1,saved_pc        /* save */
1:


\endcode
\begindocs{41}
\code{}callc(f,a)\edoc{} calls a C-language function \code{}f\edoc{} with argument \code{}a\edoc{}.
We don't have to save a register unless we'll need its value later.

The closure of this routine is irrelevant, since \code{}callc\edoc{} doesn't
have any free variables.
Therefore the only things that have to be restored after the call to~C
are the continuation, the store pointer, the data pointer, and
the exception handler.
If we wanted \code{}callc\edoc{} to be more efficient, we would
rearrange things so that all those registers fell into \code{}s0\edoc{}--\code{}s8\edoc{},
where they would automatically be preserved across procedure calls.
As it stands, everything except the continuation is preserved,
so we're not doing too badly.

Miraculously, C routines return integer results in \code{}$2\edoc{}, which is
exactly the register we need to pass to our continuation (in order to
return a value).
I decided not to rely on this, and to include a \code{}move\edoc{} instruction
anyway.  Maybe the assembler will park it in a delay slot since it
is a nop.
\enddocs
\begincode{42}
\moddef{calling C routines}\endmoddef
Closure(callc_v)
        sw $stdcont,argbuild+regspace($sp) /* save continuation on stack */
        lw $4,4($stdarg)        /* get value a into arg register */
        lw $10,0($stdarg)       /* get address of f into misc reg */
        jal $10                 /* call f ($31 can be trashed) */
        move $stdarg,$2         /* return val is argument to continuation */
        lw $stdcont,argbuild+regspace($sp) /* recover continuation */
\LA{}put zeroes in all forwardable regs that might hold garbage\RA{}
        lw $artemp3,cause       /* get cause */
        bne $artemp3,$0,saveregs /* if cause != 0, save ML & return to C */
        lw $10,0($stdcont)      /* grab continuation */
        j $10                   /* return */
\endcode
\begindocs{43}
A forwardable register can hold garbage unless it was saved
by C (is in \code{}s0\edoc{}--\code{}s8\edoc{}) or is \code{}$stdarg\edoc{} of \code{}$stdcont\edoc{}.
\enddocs
\begincode{44}
\moddef{put zeroes in all forwardable regs that might hold garbage}\endmoddef
        move $stdclos,$0
        move $5,$0
        move $6,$0
        move $7,$0
        move $8,$0
        move $9,$0
        move $10,$0
        move $11,$0
        move $12,$0
        move $13,$0
        move $14,$0
        move $15,$0
        /* $16--$23 and $30 are saved by the callee */

\endcode
\begindocs{45}
This interface is going to be agony, because the rules for passing
arguments are passing strange.
The interface is \code{}syscall_v(callnumber,argvector,argcount)\edoc{}.

The system call interface is the same as the procedure call interface,
but instead of a \code{}jal\edoc{} we use a \code{}syscall\edoc{} instruction, and
we put the system call number in register \code{}$2\edoc{}.
It appears that, after the execution of the \code{}syscall\edoc{} handler,
the result is in \code{}$2\edoc{}, and \code{}$7\edoc{} is zero unless an error occurred.
We will put all the arguments on the stack, then load the first four
into \code{}$4\edoc{}--\code{}$7\edoc{}.
\enddocs
\begincode{46}
\moddef{system calls}\endmoddef
Closure(syscall_v)
        sw $stdcont,argbuild+regspace($sp) /* save continuation on stack */
        lw $artemp1,8($stdarg)  /* 2*argc+1 in $artemp1 */
        sra $artemp1,1          /* argc in $artemp1 */
        move $16,$sp            /* save our $sp */
\LA{}extend argbuild area to be big enough for all arguments\RA{}
        lw $ptrtemp,4($stdarg)  /* argv in $ptrtemp */
\LA{}put all arguments onto the stack\RA{}
\LA{}load first four arguments into \code{}$4\edoc{}--\code{}$7\edoc{}\RA{}
9:      lw $2,0($stdarg)        /* get syscall # in $2; trash $stdarg */
        sra $2,1                /* throw out the tag bit */
        syscall
        move $sp,$16            /* recover the good stack pointer */
        lw $stdcont,argbuild+regspace($sp) /* recover continuation */
        bnez $7,1f              /* if error, return ~1 */
        move $stdarg,$2         /* return val is argument to continuation */
        add $stdarg,$stdarg     /* double return value */
        addi $stdarg,1          /* and add tag bit */
        b 2f
1:      li $stdarg,-1
2:
\LA{}put zeroes in all forwardable regs that might hold garbage\RA{}
        lw $10,0($stdcont)      /* grab continuation */
        j $10                   /* return */

\endcode
\begindocs{47}
At this point we know that the number of arguments is in \code{}$artemp1\edoc{}.
We have room for four arguments; if there are more 
we'll have to increase the stack size by the appropriate multiple of 8
so that it stays doubleword-aligned.
\enddocs
\begincode{48}
\moddef{extend argbuild area to be big enough for all arguments}\endmoddef
        ble $artemp1,4,1f               /* big enough */
        sub $artemp2,$artemp1,3         /* (temp2 = argc - 4 + 1) > 1 */
        sra $artemp2,1                  
        sll $artemp2,3                  /* temp2 = 4 * roundup (argc-4,2) */
        subu $sp,$artemp2               /* increase stack */
1:

\endcode
\begindocs{49}
Now we have a list of arguments pointed to by \code{}$ptrtemp\edoc{}.
We have the count of the arguments in \code{}$artemp1\edoc{}.
We have to put them on the stack.
We have to remove tag bits where appropriate.
\enddocs
\begincode{50}
\moddef{put all arguments onto the stack}\endmoddef
        move $artemp2,$sp               /* destination in $artemp2 */
        b 1f                            /* branch forward to test */
2:      /* argc > 0 */
        lw $artemp3,0($ptrtemp)         /* get list element */
        andi $10,$artemp3,1             /* tagged? */
        beqz $10,3f
        sra $artemp3,1                  /* drop tag bit */
3:      sw $artemp3,0($artemp2)         /* save the argument */
        lw $ptrtemp,4($ptrtemp)         /* next element */
        add $artemp2,4                  /* next arg build area */
        sub $artemp1,1                  /* --argc */
1:      bgtz $artemp1,2b                /* if argc>0, store another */

\endcode
\begindocs{51}
It doesn't matter if we load arguments that aren't there; the
system call will just ignore them.
\enddocs
\begincode{52}
\moddef{load first four arguments into \code{}$4\edoc{}--\code{}$7\edoc{}}\endmoddef
        lw $4,0($sp)
        lw $5,4($sp)
        lw $6,8($sp)
        lw $7,12($sp)

\endcode
\begindocs{53}
\subsection{Floating point}
We store floating point constants in two words, with the least significant
word first.  
We use the 64 bit IEEE format.

We begin with instructions to change the rounding modes.
See the MIPS book, pages 6-5--6-7.
\enddocs
\begincode{54}
\moddef{tell the floating point unit to round toward $-\infty$}\endmoddef
.set noat
        cfc1 $at,$31            /* grab fpa control register */
        ori  $at,0x03           /* set rounding bits to 11 */
        ctc1 $at,$31            /* return fpa control register */
.set at
\endcode
\begincode{55}
\moddef{tell the floating point unit to round to nearest}\endmoddef
.set noat
        cfc1 $at,$31            /* grab fpa control register */
        ori  $at,0x03           /* set rounding bits to 11 */
        xori $at,0x03           /* set rounding bits to 00
        ctc1 $at,$31            /* return fpa control register */
.set at
\endcode
\begindocs{56}
These floating point functions are used int floating to integer conversion.
\enddocs
\begincode{57}
\moddef{floating point}\endmoddef
/* Floating exceptions raised (assuming ROP's are never passed to functions):
 *      DIVIDE BY ZERO - (div)
 *      OVERFLOW/UNDERFLOW - (add,div,sub,mul) as appropriate
 *
 * floor raises integer overflow if the float is out of 32-bit range,
 * so the float is tested before conversion, to make sure it is in (31-bit)
 * range */

\endcode
\begindocs{58}
\code{}floor(x)\edoc{} returns the smallest integer less than or equal to \code{}x\edoc{}.
\enddocs
\begincode{59}
\moddef{floating point}\endmoddef
Closure(floor_v)
        lwc1 $f4,0($stdarg)             /* get least significant word */
        lwc1 $f5,4($stdarg)             /* get most significant word */
\LA{}tell the floating point unit to round toward $-\infty$\RA{}
        cvt.w.d $f6,$f4                 /* convert to integer */
\LA{}tell the floating point unit to round to nearest\RA{}
        mfc1 $stdarg,$f6                /* get in std argument register */
        sll $stdarg,1           /* make room for tag bit */
        add $stdarg,1           /* add the tag bit */
        lw $10,0($stdcont)      /* grab continuation */
        j $10                   /* return */


\endcode
\begindocs{60}
\code{}logb(x)\edoc{} returns the exponent part of the floating point \code{}x\edoc{}.
We grab the 11-bit exponent from the word, then unbias it (according
to the IEEE standard) by subtracting 1023.
\enddocs
\begincode{61}
\moddef{floating point}\endmoddef
Closure(logb_v)
        lw      $stdarg,4($stdarg)      /* most significant part */
        srl     $stdarg,20              /* throw out 20 low bits */
        andi    $stdarg,0x07ff          /* clear all but 11 low bits */
        sub     $stdarg,1023            /* subtract 1023 */
        sll     $stdarg,1               /* make room for tag bit */
        add     $stdarg,1               /* add the tag bit */
        lw      $10,0($stdcont)         /* grab continuation */
        j       $10                     /* return */

\endcode
\begindocs{62}
\code{}scalb(x,n)\edoc{} adds \code{}n\edoc{} to the exponent of floating
point \code{}x\edoc{}.
Since we don't want the resulting float to be anything
special, we insist that the unbiased exponent of the result
satisfy $-1022 \le E \le 1023$, i.e.\ that the biased exponent satisfy
 $1 \le e \le 2046$.
\enddocs
\begincode{63}
\moddef{floating point}\endmoddef
Closure(scalb_v)
        lw      $artemp1,4($stdarg)     /* get tagged n */
        sra     $artemp1,1              /* get real n */
        beqz    $artemp1,9f             /* if zero, return the old float */
        lw      $ptrtemp,0($stdarg)     /* get pointer to float */
        lw      $artemp2,4($ptrtemp)    /* most significant part */
        srl     $artemp2,20             /* throw out 20 low bits */
        andi    $artemp2,0x07ff         /* clear all but 11 low bits */
        add     $artemp3,$artemp2,$artemp1      /* new := old + n */
        blt     $artemp3,1,under        /* punt if underflow */
        bgt     $artemp3,2046,over      /* or overflow */
\LA{}allocate and store new floating point constant and set \code{}$stdarg\edoc{}\RA{}
        lw      $10,0($stdcont)         /* grab continuation */
        j       $10                     /* return */

9:      lw      $stdarg,0($stdarg)      /* get old float */
        lw      $10,0($stdcont)         /* grab continuation */
        j       $10                     /* return */

over:   la      $stdarg,1f              /* exception name in $stdarg */
        b       raise_real
String(1,8,"overflow")
under:  la      $stdarg,1f              /* exception name in $stdarg */
        b       raise_real
String(1,9,"underflow\\0\\0\\0")

raise_real:
 /* build new record to pass to exception handler */
 /*    [descriptor]
 /*    [exception (string)]
 /*    [real_e (more exception info)]
  */
        la      $10,real_e              /* get address of real_e */
.set noreorder
        sw      $10,8($dataptr)         /* allocate; may cause gc */
.set reorder
        sw      $stdarg,4($dataptr)
        li      $10,mak_desc(2,tag_record)
        sw      $10,0($dataptr)
        add     $stdarg,$dataptr,4      /* new record is argument */
        addi    $dataptr,12             /* $dataptr restored */
        move    $stdclos,$exnptr        /* make sure closure is right */
        lw      $10,0($exnptr)          /* grab handler */
        j       $10                     /* raise the exception */

\endcode
\begindocs{64}
Here we indulge in a little cleverness to save a couple of instructions.
Since the old value is in \code{}$artemp2\edoc{} and the new in \code{}$artemp3\edoc{},
we can \code{}xor\edoc{} them, then store the new one with a second \code{}xor\edoc{}.
\enddocs
\begincode{65}
\moddef{allocate and store new floating point constant and set \code{}$stdarg\edoc{}}\endmoddef
        xor     $artemp3,$artemp2       /* at3 = new xor old */
        sll     $artemp3,20             /* put exponent in right position */
        lw      $artemp2,4($ptrtemp)    /* most significant word */
        xor     $artemp2,$artemp3       /* change to new exponent */
.set noreorder
        sw      $artemp2,8($dataptr)    /* allocate; may cause gc */
.set reorder
        lw      $artemp2,0($ptrtemp)    /* get least significant word */
        li      $10,mak_desc(8,tag_string) /* make descriptor */
        sw      $artemp2,4($dataptr)    /* save lsw */
        sw      $10,0($dataptr)         /* save descriptor */
        add     $stdarg,$dataptr,4      /* get pointer to new float */
        add     $dataptr,12             /* point to new free word */
\endcode
\bye

\section{The TI C6x Back End}

No documentation yet.


\majorsection{Basic Types}
\section{Annotations}

\subsection{Overview}
A compiler front-end has to be propagate information to
the back-end.  An optimization phase may have to leave behind information
at various places of the IR so that other phases can reuse such information.
MLRISC uses the \newdef{annotations}
mechanism for these functions.  
Individual instructions, basic blocks, and flow graph edges, 
can be attached one or more annotations.  

The basic MLRISC system understands many annotations.  Some examples are:
\begin{description}
   \item[COMMENT] 
         these can be used to attach comments.  If attached to
         an instruction, the assemblers will output 
         them as part of their assembly output.
   \item[BRANCH\_PROB]
          these can be attached to a branch instruction to indicate
          the probability in which is it taken.
   \item[EXECUTION\_FREQ]
          these can be attached to a basic block to indicate 
          its expected execution frequency 
\end{description}

\subsection{Details}
The primitive annotations datatype is defined
to have this \mlrischref{library/annotations.sig}{signature}.
In addition, MLRISC predefined a few primitive annotations that are
recognized by the core system.  This signature is
\mlrischref{instructions/mlriscAnnotations.sig}{MLRISC\_ANNOTATIONS}.
More detailed documentation can be found in this 
\href{http://cm.bell-labs.com/cm/cs/what/smlnj/compiler-notes/annotations.ps}{paper}.

\section{Cells}

MLRISC uses
the \mlrischref{instructions/cells.sig}{CELLS} 
interface to define all readable/writable resources
in a machine architecture,  or \emph{cells} 
The types defined herein are:
\begin{itemize}
 \item \sml{cellkind} -- different classes of cells are assigned
   difference cellkinds.  The following cellkinds should be present
   \begin{itemize}
     \item \sml{GP} -- general purpose registers.
     \item \sml{FP} -- floating point registers.
     \item \sml{CC} -- condition code registers.
   \end{itemize}
   In addition, the cellkinds \sml{MEM} and \sml{CTRL}
   should also be defined.  These are used for representing
   memory based data dependence and control dependence.
   \begin{itemize}
     \item \sml{MEM} -- memory 
     \item \sml{CTRL} -- control dependence
   \end{itemize} 
 \item \sml{regmap} -- \href{regmap.html}{register map}
 \item \sml{cellset} -- a cellset represent a set of cells.  This
   type can be used to denote live-in/live-out information.  Cellsets are
   implemented as immutable abstract types.
\end{itemize}

These core definitions are defined in the following signature
\begin{SML}
signature \mlrischref{instructions/cells.sig}{CELLS\_BASIS} =
sig
   eqtype cellkind 
   type cell = int
   type regmap = cell Intmap.intmap
   exception Cells

   val cellkinds : cellkind list 
   val cellkindToString : cellkind -> string
   val firstPseudo : cell                    
   val Reg   : cellkind -> int -> cell
   val GPReg : int -> cell 
   val FPReg : int -> cell
   val cellRange : cellkind -> {low:int, high:int}
   val newCell   : cellkind -> 'a -> cell 
   val cellKind : cell -> cellkind         
   val updateCellKind : cell * cellkind -> unit        
   val numCell   : cellkind -> unit -> int              
   val maxCell   : unit -> cell
   val newReg    : 'a -> cell              
   val newFreg   : 'a -> cell              
   val newVar    : cell -> cell
   val regmap    : unit -> regmap
   val lookup    : regmap -> cell -> cell
   val reset     : unit -> unit
end
\end{SML}

\begin{itemize}
  \item\sml{cellkinds} -- this is a list of all the cellkinds defined in the
architecture
  \item\sml{cellkindToString} -- this function maps a cellkind into its name
  \item\sml{firstPseudo} -- MLRISC numbered physical resources
   in the architecture from 0 to firstPseudo-1.  
   This is the first usable virtual register number.
  \item\sml{Reg} -- This function maps the $i$th physical
   resource of a particular cellkind to its internal encoding used by MLRISC.
   Note that all resources in MLRISC are named uniquely.
  \item\sml{GPReg} -- abbreviation for \sml{Reg GP} 
  \item\sml{FPReg} -- abbreviation for \sml{Reg FP} 
  \item \sml{cellRange} -- this returns a range \sml{{low, high}}
   when given a cellkind, with denotes the range of physical resources
  \item \sml{newCell}  -- This function returns a new virtual register 
   of a particular cellkind.
  \item \sml{newReg} -- abbreviation as \sml{newCell GP}
  \item \sml{newFreg} -- abbreviation as \sml{newCell FP}
  \item \sml{cellKind}  -- When given a cell number, this returns its
    cellkind.  Note that this feature is not enabled by default.
  \item \sml{updateCellKind} -- updates the cellkind of a cell.
  \item \sml{numCell} -- returns the number of virtual cells allocated for one cellkind.
  \item \sml{maxCell} --  returns the next virtual cell id.
  \item \sml{newVar}  -- given a cell id, return a new cell id of
     the same cellkind.
  \item \sml{regmap} -- This function returns a new empty regmap
  \item \sml{lookup} -- This converts a regmap into a lookup function.
  \item \sml{reset} -- This function resets all counters associated
with all virtual cells.
\end{itemize}

\begin{SML}
signature CELLS = sig
   include CELLS_BASIS
   val GP   : cellkind 
   val FP   : cellkind
   val CC   : cellkind 
   val MEM  : cellkind 
   val CTRL : cellkind 
   val toString : cellkind -> cell -> string
   val stackptrR : cell 
   val asmTmpR : cell  
   val fasmTmp : cell 
   val zeroReg : cellkind -> cell option

   type cellset

   val empty      : cellset
   val addCell    : cellkind -> cell * cellset -> cellset
   val rmvCell    : cellkind -> cell * cellset -> cellset
   val addReg     : cell * cellset -> cellset
   val rmvReg     : cell * cellset -> cellset
   val addFreg    : cell * cellset -> cellset
   val rmvFreg    : cell * cellset -> cellset
   val getCell    : cellkind -> cellset -> cell list
   val updateCell : cellkind -> cellset * cell list -> cellset

   val cellsetToString : cellset -> string
   val cellsetToString' : (cell -> cell) -> cellset -> string

   val cellsetToCells : cellset -> cell list
end
\end{SML}

\begin{itemize} 
  \item \sml{toString} -- convert a cell id of a certain cellkind into
its assembly name.
  \item \sml{stackptrR} -- the cell id of the stack pointer register. 
  \item \sml{asmTmpR} -- the cell id of the assembly temporary 
  \item \sml{fasmTmp} -- the cell id of the floating point temporary
  \item \sml{zeroReg} -- given the cellkind, returns the cell id of the
   source that always hold the value of zero, if there is any. 
  \item \sml{empty} -- an empty cellset
  \item \sml{addCell} -- inserts a cell into a cellset
  \item \sml{rmvCell} -- remove a cell from a cellset
  \item \sml{addReg} -- abbreviation for \sml{addCell GP}
  \item \sml{rmvReg} -- abbreviation for \sml{rmvCell GP} 
  \item \sml{addFreg} -- abbreviation for \sml{addCell FP}
  \item \sml{rmvFreg} -- abbreviation for \sml{rmvCell FP} 
  \item \sml{getCell} -- lookup all cells of a particular cellkind from
the cellset
  \item \sml{updateCell} -- replace all cells of a particular cellkind
from the cellset. 
   \item \sml{cellsetToString} -- pretty print a cellset 
   \item \sml{cellsetToString'} -- pretty print a cellset, but first
apply a regmap function.
   \item \sml{cellsetToCells} -- convert a cellset into list form.
\end{itemize}

\section{Cluster}

A \newdef{cluster}
represents a compilation unit in linearized form,
and contains information about the control flow, global annotations, 
block and edge execution frequencies, and live-in/live-out information.

Its signature is:
\begin{SML}
signature FLOWGRAPH = sig
  structure C : \href{cells.html}{CELLS}
  structure I : \href{instructions.html}{INSTRUCTIONS}
  structure P : \href{pseudo-ops.html}{PSEUDO_OPS}
  structure W : \href{freq.html}{FREQ}
     sharing I.C = C

  datatype block =
      PSEUDO of P.pseudo_op
    | LABEL of Label.label
    | BBLOCK of
        \{ blknum      : int,
          freq        : W.freq ref,
          annotations : Annotations.annotations ref,
	  liveIn      : C.cellset ref,
	  liveOut     : C.cellset ref,
	  succ 	      : edge list ref,
	  pred 	      : edge list ref,
	  insns	      : I.instruction list ref
        \}
    | ENTRY of 
        \{blknum : int, freq : W.freq ref, succ : edge list ref\}
    | EXIT of 
        \{blknum : int, freq : W.freq ref, pred : edge list ref\}
  withtype edge = block * W.freq ref

  datatype cluster = 
      CLUSTER of \{
        blocks: block list, 
        entry : block,
        exit  : block,	  
        regmap: C.regmap,
        blkCounter : int ref,
        annotations : Annotations.annotations ref
      \}
end
\end{SML}

Clusters are used in
\href{span-dep.html}{span dependency resolution}, 
\href{delayslots.html}{delay slot filling},
\href{asm.html}{assembly}, 
and \href{mc.html}{machine code} 
output, since these phases require the code laid out in linearized form.

\section{Client Defined Constants}
\subsubsection{Introduction}
MLRISC allows the client to inject abstract 
\newdef{constants} that are resolved
only at the end of the compilation phase into the instruction stream.
These constants can be used whereever an integer literal is expected.
Typical usage are stack frame offsets for spill locations which are only
known after register allocation, 
and garbage collection and exception map which are resolved only
when all address calculation are performed.

\subsubsection{The Details}
Client defined constants should satsify the following signature:
\begin{SML}
signature \mlrischref{instructions/constant.sig}{CONSTANT} = sig
   type const

   val toString : const -> string
   val valueOf  : const -> int
   val hash     : const -> word
   val ==       : const * const -> bool
end
\end{SML}

The methods are:
\begin{methods}
 toString & a pretty printing function \\
 valueOf & returns the value of the constant \\
 hash & returns the hash value of the constant \\
 == & compare two constants for identity \\
\end{methods}

The method \sml{toString} should be implemented in all cases.
The method \sml{valueOf} is necessary only if machine code generation
is used.  The last two methods, \sml{hash} and \sml{==} are necessary
only if SSA optimizations are used.

\section{Client Defined Pseudo Ops}
\subsection{Introduction}
\newdef{Pseudo ops}
are client defined instruction stream markers.  They
can be used to represent assembly directives.
Pseudo ops should satisfy the following signature:
\begin{SML}
signature \mlrischref{instructions/pseudoOps.sig}{PSEUDO_OPS} = sig
  type pseudo_op
  val toString : pseudo_op -> string
  val emitValue : {pOp:pseudo_op, loc:int, emit:Word8.word -> unit} -> unit
  val sizeOf : pseudo_op * int -> int
  val adjustLabels : pseudo_op * int -> bool
end
\end{SML}

The method that is required is:
\begin{itemize}
 \item \sml{toString} -- pretty printing the pseudo in assembly format.
\end{itemize}

When machine code generation is used, we also have to implement
the following methods:
\begin{itemize}
 \item \sml{emitValue} --
    emit value of pseudo op give current location counter and output
    stream. The value emitted should respect the endianness of the
    target machine.
 \item \sml{sizeOf} --
    Size of the pseudo op in bytes given the current location counter
    The location counter is provided in case some pseudo ops are 
    dependent on alignment considerations.
 \item \sml{adjustLabels} --
    adjust the value of labels in the pseudo op given the current
    location counter.
\end{itemize}
These methods are involved during the 
\href{span-dep.html}{span dependence resolution} phase to determine
the size and layout of the pseudo ops.

\section{Instructions}

  Instructions in MLRISC are implemented as abstract datatypes and
must satisfy the signature 
\mlrischref{instructions/instructions.sig}{INSTRUCTIONS}, defined as follows:

\begin{SML}
signature INSTRUCTIONS =
sig
   structure C        : \href{cells.html}{CELLS}
   structure Constant : \href{constants.html}{CONSTANT}
   structure LabelExp : \href{labelexp.html}{LABELEXP}
      sharing LabelExp.Constant = Constant

   type operand   
   type ea         
   type addressing_mode
   type instruction 
end
\end{SML}

Type \sml{operand} is used to represent ioperands,
\sml{ea} is used to represent effective addresses, type 
\sml{addressing_mode} is used to represent the internal addressing mode
used by the architecture.  Note that these are all abstract according to 
the signature, so the client has complete freedom in choosing the most
convenient representation for these things.

\subsection{Predication}
   For architectures that have full \newdef{predication}
built-in, such as the C6xx or IA-64, the instruction set should be
extended to satisfy the signature: 
\begin{SML}
signature \mlrischref{instructions/pred-instructions.sig}{PREDICATED_INSTRUCTIONS} =
sig
   include INSTRUCTIONS
   
   type predicate  
end
\end{SML}
This basically says that the type that is used to represent a predicate
can be implemented however the client wants.  This flexibility
is quite important since the predication model may differ substantially
from architecture to architecture.

For example, in the TI C6, there are no seperate predicate register files
and integer registers double as predicate registers, and the predicate
true is any non-zero value.  Each instruction can be predicated under a
predicate register or its negation.  In contrasts, architectures such as
IA-64 and HP's Playdoh incorporate separate predicate registers into their 
architectures.  In Playdoh, \newdef{predicate defining} instructions 
actually set a pair of complementary predicate registers, 
and instructions can only
be predicated under the value of a predicate register, not its negation.

\subsection{VLIW}
   VLIW architectures differ from superscalars in that
resource assignments are statically determined at compile time.
We distinguish between two different types of resources, namely
\newdef{functional units} and \newdef{data paths}.  
The latter type is particularly
important for clustered architectures.
The following signature
is used to describe VLIW instructions:
\begin{SML}
signature \mlrischref{instructions/vliw-instructions.sig}{VLIW\_INSTRUCTIONS} =
sig

   include INSTRUCTIONS
   structure FU : \mlrischref{instructions/funits.sig}{FUNITS}
   structure DP : \mlrischref{instructions/datapaths.sig}{DATAPATHS}
end
\end{SML}
The signature \sml{FUNITS} is used to describe functional unit
resources, while the signature \sml{DATAPATHS} is used to describe
data paths.

\subsection{Predicated VLIW}

Finally, instructions sets for predicated VLIW/EPIC machines should match
the signature 
\begin{SML}
signature \mlrischref{instructions/pred-vliw-instructions.sig}{PREDICATED_VLIW_INSTRUCTIONS} =
sig
   include VLIW_INSTRUCTIONS
   type predicate
end
\end{SML}

\section{Instruction Streams}

\subsubsection{Overview}
An \newdef{instruction stream}
is an abstraction used by MLRISC to describe linearized instructions.
This abstraction turns out to fit the function of
many MLRISC modules.  For example,
a phase such as \href{instrsel.html}{Instruction Selection} 
can be viewed as taking an stream of 
\href{mltree.html}{MLTREE} statements and return a
stream of \href{instructions.html}{instructions}.  Similarly,
phases such as \href{asm.html}{assembly output} and
\href{mc.html}{machine code generation} can be seen 
as taking a stream of instructions and 
returning a stream of characters and a stream of bytes.

\subsubsection{The Details}
An instruction stream satisfy the following abstract signature:
\begin{SML}
signature \mlrischref{instructions/stream.sig}{INSTRUCTION_STREAM} =
sig
   structure P : \href{pseudo-ops.html}{PSEUDO_OPS}

   datatype ('a,'b,'c,'d,'e,'f) stream =
      STREAM of
      \{ beginCluster: int -> 'b,  
        endCluster  : 'c -> unit, 
        emit        : 'a,        
        pseudoOp    : P.pseudo_op -> unit,
        defineLabel : Label.label -> unit,
        entryLabel  : Label.label -> unit,
        comment     : string -> unit,    
        annotation  : Annotations.annotation -> unit,
        exitBlock   : 'd -> unit,
        alias       : 'e -> unit, 
        phi         : 'f -> unit  
      \}
end
\end{SML}
This type is specialized in other modules as such the
\href{asm.html}{assembler}, the \href{mc.html}{machine code emitter},
and the \href{instrsel.html}{instruction selection modules}.
\subsubsection{The protocol}
All instruction streams, irrespective of their actual types, 
follow the following protocol:
\begin{itemize}
  \item The method \sml{beginCluster} should be called at the beginning of
        the stream to mark the start of a new compilation unit.  
         The integer passed to this method is the number
        of bytes in the stream.  This integer is only used for 
        machine code emitter, which uses it to allocate space for the
        code string.  
  \item The method \sml{endCluster} should be called when the entire
       compilation unit has been sent.
  \item In between these calls, the following methods can be called in any
       order:
  \begin{itemize}
   \item \sml{emit} -- this method emits an instruction.  It takes
         a \href{regmap.html}{regmap} as argument.
   \item \sml{pseudoOp} -- this method emits a pseudo op.
   \item \sml{defineLabel} -- this method defines a \emph{local} label, i.e.
a label that is only referenced within the same compilation unit.
   \item \sml{entryLabel} -- this method defines an \emph{enternal} label that
          marks an procedure entry, and may be referenced from other 
compilation units.
   \item \sml{comment} -- this emits a comment string
   \item \sml{annotation} -- this function attaches an annotation to 
     the current basic block.
   \item \sml{exitBlock} -- 
          this marks the current block as an procedure exit.
  \end{itemize}
\end{itemize}  

\section{Label Expressions}

A \newdef{label expression} is a constant
expression defined in terms of labels, or user 
defined \href{constants.html}{constants}.  MLRISC uses the type
\sml{labexp} to represent label expressions.  Label expressions
are defined in the structure 
\mlrischref{instructions/labelExp.sml}{LabelExp}.

The datatype \sml{labexp} has the following definition:
\begin{SML}
  datatype labexp = 
      LABEL of Label.label
    | CONST of Constant.const
    | INT of int
    | PLUS of labexp * labexp
    | MINUS of labexp * labexp
    | MULT of labexp * labexp
    | DIV of labexp * labexp
    | LSHIFT of labexp * word
    | RSHIFT of labexp * word
    | AND of labexp * word
    | OR of labexp * word
\end{SML}

In addition, the following functions are defined in \sml{labexp}:
\begin{itemize}
  \item \sml{valueOf : labexp -> int}  -- Returns the value associated with
a label expression
  \item \sml{toString : labexp -> string} -- Return the pretty printed representation of an expression
  \item \sml{hash : labexp -> word} -- Returns the hash value of an expression
  \item \sml{== : labexp * labexp -> bool} -- Tests whether two label expression are lexically identical
\end{itemize}

The type \sml{labexp} is depends on client defined 
\href{constants.html}{constants} typed.  The functor \sml{LabelExp}
is parameterized as follows.
\begin{SML}
   functor \mlrischref{instructions/labelExp.sml}{LabelExp}(Constant : \mlrischref{instructions/constant.sig}{CONSTANT})
\end{SML}

\section{Labels}

\newdef{Labels} are used as symbolic names for address.
The structure \mlrischref{instructions/labels.sml}{Label}
defines the label datatype.  The following operations are defined
on labels:
\begin{itemize}
\item \sml{newLabel : string -> label} --  Generate a new label with
    a given name.  If the name is \sml{""}, a new name is generated.
\item \sml{nameOf : label -> string} -- Returns the name of
   a label
\item \sml{id : label -> int} -- Return the unique id of a label
\item \sml{reset : unit -> unit} -- Return the label id counter to 0.  
\end{itemize}

For machine code generation, the following two additional methods are
defined.
\begin{itemize}
\item  \sml{addrOf : label -> int} -- Return the address associated with
a label
\item  \sml{setAddr : label * int -> unit} --  Set the address associated
with a label
\end{itemize}

See also \href{labelexp.html}{Label Expressions}.

\section{Regions}
\subsubsection{Overview}

The MLRISC system uses user defined type called
\newdef{regions} to propagate
aliasing information to the backend. This type is
abstract and no constraint is imposed on how it is implemented.
The advantage of this is that the client can optimize the representation
of the region information according to the semantics of the source language.
The downside of this freedom is that the client has to implement
various modules to extract information from the regions datatype
required by various optimization phases.

For clients that do not want to implement their own regions datatype,
there is now a new generic mechanism, called 
\newdef{MLRiscRegions}, built on top of
the regions concept, for propagating both:
\begin{itemize}
  \item Aliasing information, and
  \item Control dependence/anti-control dependence information
\end{itemize}
Both kinds of information are crucial for extracting parallelism
from the target code, and are used in all optimizations that perform code
motion, such as SSA optimizations and all scheduling optimizations. 

\subsubsection{MLRisc Regions}

\section{Regmap}
A \newdef{regmap}
is a mapping from virtual register to virtual or physical
register, and is used by MLRISC register allocators to
represent the current binding of virtual registers.  Regmaps are implemented
as \mlrischref{library/intmap.sml}{Intmap} 
in MLRISC, and are defined in the
\href{cells.html}{CELLS} interface.

Regmaps are used in phases such as 
\href{asm.html}{assembly generation} and 
\href{mc.html}{machine code}.   MLRISC program representations such
\href{cluster.html}{clusters} and \href{mlrisc-ir.html}{IR}
each contains a global regmap per compilation unit.  Representations
such as \href{hyperblock.html}{hyperblock} may contain its own
regmap, which overrides the global regmap. 


\bibliography{mlrisc}

\end{document}
