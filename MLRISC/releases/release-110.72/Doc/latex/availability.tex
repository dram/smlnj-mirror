\section{How to Obtain MLRISC}

There are a few ways to obtain the MLRISC system.
\begin{enumerate}
\item
An old version of MLRISC is available from
\externhref{http://cm.bell-labs.com/cm/cs/what/smlnj/doc/MLRISC/quick-tour/index.html}{this link}.   
This version is stable but very out-dated, and does
not contain the most up-to-date features. 
\item
New experimental versions are available from the 
\externhref{http://cm.bell-labs.com/cm/cs/what/smlnj/software.html}{SML/NJ software page} as part of the SML/NJ compiler releases.  
These versions are relative stable, but
do not include the entire MLRISC source tree.
\item \href{mailto:leunga@cs.nyu.edu}{Allen} 
keeps an up-to-date version of MLRISC at NYU for private use.
This version includes everything but is under constant changes, so beware!
To access the CVS repository, set your \sml{CVSROOT} environment variable 
to
\begin{verbatim}
   :pserver:mlrisc@react-ilp.cs.nyu.edu:/home/leunga/mlrisc
\end{verbatim}
and checkout the repository using
\begin{verbatim}
   cvs co MLRISC++
\end{verbatim}
The password to use is \sml{mlrisc}.
\item
Generally speaking, you can get the latest version of MLRISC by asking
\href{mailto:george@research.bell-labs.com}{Lal}.
\end{enumerate}
MLRISC is \newdef{free, open source} software, and is released under the
\href{http://cm.bell-labs.com/cm/cs/what/smlnj/license.html}{SML/NJ license}.
