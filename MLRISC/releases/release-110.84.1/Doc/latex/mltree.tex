\section{The MLTREE Language}

\newdef{MLTree} is the 
register transfer language used in the MLRISC system.
It serves two important purposes:
\image{MLTree}{pictures/png/mlrisc-ir.png}{align=right}
\begin{enumerate}
\item As an intermediate representation for a compiler front-end 
  to talk to the MLRISC system,
\item As specifications for instruction semantics
\end{enumerate}
The latter is needed for optimizations which require precise knowledge of such;
for example, algebraic simplification and constant folding.

MLTree is a low-level \newdef{typed} language: 
all operations are typed by its width or precision.  
Operations on floating point, integer, and condition code 
are also segregated, to prevent accidental misuse. 
MLTree is also \emph{tree-oriented} so that it is possible to write efficient
MLTree transformation routines that uses SML pattern matching.

Here are a few examples of MLTree statements.
\begin{SML}
   MV(32,t,
      ADDT(32,
        MULT(32,REG(32,b),REG(32,b)),
        MULT(32,
          MULT(32,LI(4),REG(32,a)),REG(32,c))))
\end{SML}
computes \sml{t := b*b + 4*a*c}, all in 32-bit precision and overflow
trap enabled; while
\begin{SML}
   MV(32,t,
      ADD(32,
        CVTI2I(32,SIGN_EXTEND,8,
          LOAD(8,
            ADD(32,REG(32,a),REG(32,i))))))
\end{SML}
loads the byte in address \sml{a+i} and sign extend it to a 32-bit
value. 

The statement
\begin{SML}
   IF([],CMP(64,GE,REG(64,a),LI 0),
         MV(64, t, REG(64, a)),
         MV(64, t, NEG(64, REG(64, a)))
     )
\end{SML}
in more traditional form means:
\begin{verbatim}
   if a >= 0 then 
      t := a
   else
      t := -a
\end{verbatim} 
This example can be also expressed in a few different ways: 
\begin{enumerate}
   \item With the conditional move construct described in 
Section~\ref{sec:cond-move}:
     \begin{SML}
    MV(64, t, 
       COND(CMP(64, GE, REG(64, a)), 
            REG(64, a), 
            NEG(64, REG(64, a))))
     \end{SML}
  \item With explicit branching using the conditional branch
construct \verb|BCC|:
    \begin{SML}
     MV(64, t, REG(64, a));
     BCC([], CMP(64, GE, REG(64, a)), L1);
     MV(64, t, NEG(64, REG(64, a)));
     DEFINE L1;
    \end{SML}
\end{enumerate}
\subsection{The Definitions}

MLTree is defined in the signature \mlrischref{mltree/mltree.sig}{\sml{MLTREE}}
and the functor \mlrischref{mltree/mltree.sml}{\sml{MLTreeF}}

The functor \sml{MLTreeF} is parameterized in terms of
the label expression type, the client supplied region datatype,
the instruction stream type, and the client defined MLTree extensions.
\begin{SML}
  functor MLTreeF
    (structure LabelExp : \href{labelexp.html}{LABELEXP}
     structure Region : \href{regions.html}{REGION}
     structure Stream : \href{streams.html}{INSTRUCTION_STREAM}
     structure Extension : \mlrischref{mltree/mltree-extension.sig}{MLTREE_EXTENSION}
    ) : MLTREE
\end{SML}

\subsubsection{Basic Types}

  The basic types in MLTree are statements (\newtype{stm})
integer expressions (\newtype{rexp}), 
floating point expression (\newtype{fexp}), 
and conditional expressions (\newtype{ccexp}). 
Statements are evaluated for their effects,
while expressions are evaluated for their value. (Some expressions
could also have trapping effects.  The semantics of traps are unspecified.)
These types are parameterized by an extension
type, which we can use to extend the set of MLTree 
operators.  How this is used is described in Section~\ref{sec:mltree-extension}.

References to registers are represented internally as integers, and are denoted
as the type \sml{reg}. In addition, we use the types \sml{src} and \sml{dst}
as abbreviations for source and destination registers.
\begin{SML}
   type reg = int
   type src = reg
   type dst = reg
\end{SML}

All operators on MLTree are \emph{typed}
by the number of bits that 
they work on.  For example, 32-bit addition between \sml{a} and \sml{b}
is written as \sml{ADD(32,a,b)}, while 64-bit addition between the same
is written as \sml{ADD(64,a,b)}.  Floating point operations are
denoted in the same manner.  For example, IEEE single-precision floating
point add is written as \sml{FADD(32,a,b)}, while the same in
double-precision is written as \sml{FADD(64,a,b)} 

Note that these types are low level.  Higher level distinctions such
as signed and unsigned integer value, are not distinguished by the type.  
Instead, operators are usually partitioned into signed and unsigned versions, 
and it is legal (and often useful!) to mix signed and unsigned operators in
an expression.

Currently, we don't provide a direct way to specify non-IEEE floating point 
together with
IEEE floating point arithmetic.  If this distinction is needed then
it can be encoded using the extension mechanism described
in Section~\ref{sec:mltree-extension}.

We use the types \sml{ty} and \sml{fty} to stand for the number of
bits in integer and floating point operations.  
\begin{SML}
  type ty  = int
  type fty = int
\end{SML}

\subsubsection{The Basis}
The signature \mlrischref{mltree/mltree-basis.sig}{MLTREE\_BASIS}
defines the basic helper types used in the MLTREE signature.  
\begin{SML}
signature MLTREE_BASIS =
sig
 
  datatype cond = LT | LTU | LE | LEU | EQ | NE | GE | GEU | GT | GTU 

  datatype fcond = 
     ? | !<=> | == | ?= | !<> | !?>= | < | ?< | !>= | !?> |
     <= | ?<= | !> | !?<= | > | ?> | !<= | !?< | >= | ?>= |
     !< | !?= | <> | != | !? | <=> | ?<>

  datatype ext = SIGN_EXTEND | ZERO_EXTEND

  datatype rounding_mode = TO_NEAREST | TO_NEGINF | TO_POSINF | TO_ZERO

  type ty = int
  type fty = int

end
\end{SML}

The most important of these are the 
types \newtype{cond} and \newtype{fcond}, which represent the set of integer
and floating point comparisions.  These types can be combined with
the comparison constructors \verb|CMP| and \verb|FCMP| to form
integer and floating point comparisions.
\begin{Table}{|c|c|}{align=left} \hline
   Operator & Comparison \\ \hline
    \sml{LT}     & Signed less than \\
    \sml{LTU}    & Unsigned less than \\
    \sml{LE}     & Signed less than or equal \\
    \sml{LEU}    & Unsigned less than or equal \\
    \sml{EQ}     & Equal \\
    \sml{NE}     & Not equal \\
    \sml{GE}     & Signed greater than or equal \\
    \sml{GEU}    & Unsigned greater than or equal \\
    \sml{GT}     & Signed greater than \\
    \sml{GTU}    & Unsigned greater than \\
\hline
\end{Table}

Floating point comparisons can be ``decoded'' as follows.
In IEEE floating point, there are four different basic comparisons 
tests that we can performed given two numbers $a$ and $y$:
\begin{description}
   \item[$a < b$] Is $a$ less than $b$?
   \item[$a = b$] Is $a$ equal to $b$?
   \item[$a > b$] Is $a$ greater than to $b$?
   \item[$a ? b$] Are $a$ and $b$ unordered (incomparable)?
\end{description}
Comparisons can be joined together.  For example, 
given two double-precision floating point expressions $a$ and $b$,
the expression \verb|FCMP(64,<=>,a,b)| 
asks whether $a$ is less than, equal to or greater than $b$, i.e.~whether
$a$ and $b$ are comparable.  
The special symbol \verb|!| negates
the meaning the of comparison.    For example, \verb|FCMP(64,!>=,a,b)| 
means testing whether $a$ is less than or incomparable with $b$. 

\subsection{Integer Expressions}

A reference to the $i$th 
integer register with an $n$-bit value is written 
as \sml{REG(}$n$,$i$\sml{)}.  The operators \sml{LI}, \sml{LI32},
and \sml{LABEL}, \sml{CONST} are used to represent constant expressions 
of various forms.  The sizes of these constants are inferred from context.
\begin{SML}  
  REG   : ty * reg -> rexp
  LI    : int -> rexp
  LI32  : Word32.word -> rexp
  LABEL : LabelExp.labexp -> rexp
  CONST : Constant.const -> rexp
\end{SML}

The following figure lists all the basic integer operators and their
intuitive meanings.  All operators except \sml{NOTB, NEG, NEGT} are binary 
and have the type
\begin{SML}
  ty * rexp * rexp -> rexp
\end{SML}
The operators \sml{NOTB, NEG, NEGT} have the type
\begin{SML}
  ty * rexp -> rexp
\end{SML}

\begin{tabular}{|l|l|} \hline
   \sml{ADD} & Twos complement addition \\
  \sml{NEG}      & negation \\
  \sml{SUB}      & Twos complement subtraction \\
  \sml{MULS}     & Signed multiplication \\
  \sml{DIVS}     & Signed division, round to zero (nontrapping) \\
  \sml{QUOTS}    & Signed division, round to negative infinity (nontrapping) \\
  \sml{REMS}     & Signed remainder (???) \\
  \sml{MULU}     & Unsigned multiplication \\
  \sml{DIVU}     & Unsigned division \\
  \sml{REMU}     & Unsigned remainder \\
  \sml{NEGT}      & signed negation, trap on overflow \\
  \sml{ADDT}     & Signed addition, trap on overflow \\
  \sml{SUBT}     & Signed subtraction, trap on overflow \\
  \sml{MULT}     & Signed multiplication, trap on overflow \\
  \sml{DIVT}     & Signed division, round to zero,
   trap on overflow or division by zero \\
  \sml{QUOTT}    & Signed division, round to negative infinity, trap on overflow or division by zero \\
  \sml{REMT}     & Signed remainder, trap on division by zero \\
  \sml{ANDB}     & bitwise and \\
  \sml{ORB}      & bitwise or \\
  \sml{XORB}     & bitwise exclusive or \\
  \sml{NOTB}     & ones complement \\
  \sml{SRA}      & arithmetic right shift \\
  \sml{SRL}      & logical right shift \\
  \sml{SLL}      & logical left shift \\
\hline\end{tabular}

\subsubsection{Sign and Zero Extension}
Sign extension and zero extension are written using the operator
\sml{CVTI2I}. \sml{CVTI2I(}$m$,\sml{SIGN_EXTEND},$n$,$e$\sml{)} 
sign extends the $n$-bit value $e$ to an $m$-bit value, i.e. the
$n-1$th bit is of $e$ is treated as the sign bit.  Similarly,
\sml{CVTI2I(}$m$,\sml{ZERO_EXTEND},$n$,$e$\sml{)} 
zero extends an $n$-bit value to an $m$-bit
value.  If $m \le n$, then 
\sml{CVTI2I(}$m$,\sml{SIGN_EXTEND},$n$,$e$\sml{)} = 
\sml{CVTI2I}($m$,\sml{ZERO_EXTEND},$n$,$e$\sml{)}.

\begin{SML}
    datatype ext = SIGN_EXTEND | ZERO_EXTEND
    CVTI2I : ty * ext * ty * rexp -> rexp 
\end{SML}

\subsubsection{Conditional Move} \label{sec:cond-move}
Most new superscalar architectures incorporate conditional move 
instructions in their ISAs.  
Modern VLIW architectures also directly support full predication.  
Since branching (especially with data dependent branches) can
introduce extra latencies in highly pipelined architectures,
condtional moves should be used in place of short branch sequences. 
MLTree provide a conditional move instruction \sml{COND},
to make it possible to directly express conditional moves without using
branches. 
\begin{SML}
   COND : ty * ccexp * rexp * rexp -> rexp
\end{SML}

Semantically, \sml{COND(}\emph{ty},\emph{cc},$a$,$b$\sml{)} means to evaluate
\emph{cc}, and if \emph{cc} evaluates to true then the value of the entire expression is
$a$; otherwise the value is $b$.  Note that $a$ and $b$ are allowed to be
\emph{eagerly}
evaluated.  In fact, we are allowed to evaluate to \emph{both}
branches, one branch, or neither~\footnote{When possible.}. 

Various idioms of the \sml{COND} form are useful for expressing common
constructs in many programming languages.  For example, MLTree does not
provide a primitive construct for converting an integer value \sml{x} to a
boolean value (0 or 1).  But using \sml{COND}, this is expressible as
\sml{COND(32,CMP(32,NE,x,LI 0),LI 1,LI 0)}.  SML/NJ represents
the boolean values true and false as machine integers 3 and 1 respectively.
To convert a boolean condition $e$ into an ML boolean value, we can use
\begin{SML}
   COND(32,e,LI 3,LI 1)
\end{SML}

Common C idioms can be easily mapped into the \sml{COND} form. For example,
\begin{itemize}
  \item \verb|if (e1) x = y| translates into
  \sml{MV(32,x,COND(32,e1,REG(32,y),REG(32,x)))}
  \item
   \begin{verbatim}
     x = e1; 
     if (e2) x = y
   \end{verbatim}
    translates into 
  \sml{MV(32,x,COND(32,e2,REG(32,y),e1))}
  \item \verb|x = e1 == e2| translates into
  \sml{MV(32,x,COND(32,CMP(32,EQ,e1,e2),LI 1,LI 0)}
  \item \verb|x = ! e| translates into
   \sml{MV(32,x,COND(32,CMP(32,NE,e,LI 0),LI 1,LI 0)}
  \item \verb|x = e ? y : z| translates into
   \sml{MV(32,x,COND(32,e,REG(32,y),REG(32,z)))}, and
  \item \verb|x = y < z ? y : z| translates into
   \begin{alltt}
     MV(32,x,
         COND(32,
            CMP(32,LT,REG(32,y),REG(32,z)),
               REG(32,y),REG(32,z)))
   \end{alltt} 
\end{itemize}

In general, the \sml{COND} form should be used in place of MLTree's branching
constructs whenever possible, since the former is usually highly 
optimized in various MLRISC backends. 

\subsubsection{Integer Loads}

Integer loads are written using the constructor \verb|LOAD|.
\begin{SML}
   LOAD  : ty * rexp * Region.region -> rexp
\end{SML}
The client is required to specify a \href{regions.html}{region} that
serves as aliasing information for the load.  

\subsubsection{Miscellaneous Integer Operators}

An expression of the \sml{LET}($s$,$e$) evaluates the statement $s$ for
its effect, and then return the value of expression $e$.
\begin{SML}
  LET  : stm * rexp -> rexp
\end{SML}
Since the order of evaluation is MLTree operators are 
\emph{unspecified}
the use of this operator should be severely restricted to only 
\emph{side-effect}-free forms.

\subsection{Floating Point Expressions}

 Floating registers are referenced using the term \sml{FREG}.  The
$i$th floating point register with type $n$ is written 
as \sml{FREG(}$n$,$i$\sml{)}.
\begin{SML}
   FREG   : fty * src -> fexp
\end{SML}

Built-in floating point operations include addition (\sml{FADD}), 
subtraction (\sml{FSUB}), multiplication (\sml{FMUL}), division 
(\sml{FDIV}), absolute value (\sml{FABS}), negation (\sml{FNEG})
and square root (\sml{FSQRT}).
\begin{SML}
   FADD  : fty * fexp * fexp -> fexp
   FSUB  : fty * fexp * fexp  -> fexp
   FMUL  : fty * fexp * fexp -> fexp
   FDIV  : fty * fexp * fexp -> fexp
   FABS  : fty * fexp -> fexp
   FNEG  : fty * fexp -> fexp
   FSQRT : fty * fexp -> fexp
\end{SML}

A special operator is provided for manipulating signs.
To combine the sign of $a$ with the magnitude of $b$, we can
write \sml{FCOPYSIGN(}$a$,$b$\sml{)}\footnote{What should 
happen if $a$ or $b$ is nan?}.
\begin{SML}
   FCOPYSIGN : fty * fexp * fexp -> fexp
\end{SML}

To convert an $n$-bit signed integer $e$ into an $m$-bit floating point value,
we can write \sml{CVTI2F(}$m$,$n$,$e$\sml{)}\footnote{What happen to unsigned integers?}.
\begin{SML}
   CVTI2F : fty * ty * rexp -> fexp
\end{SML}

Similarly, to convert an $n$-bit floating point value $e$ to an $m$-bit
floating point value, we can write \sml{CVTF2F(}$m$,$n$,$e$\sml{)}\footnote{
What is the rounding semantics?}.
\begin{SML}
   CVTF2F : fty * fty * -> fexp
\end{SML}

\begin{SML}
  datatype rounding_mode = TO_NEAREST | TO_NEGINF | TO_POSINF | TO_ZERO
  CVTF2I : ty * rounding_mode * fty * fexp -> rexp
\end{SML}

\begin{SML}
   FLOAD : fty * rexp * Region.region -> fexp
\end{SML}

\subsection{Condition Expressions}
Unlike languages like C, MLTree makes the distinction between condition 
expressions and integer expressions.  This distinction is necessary for
two purposes:
\begin{itemize}
  \item It clarifies the proper meaning intended in a program, and
  \item It makes to possible for a MLRISC backend to map condition
expressions efficiently onto various machine architectures with different
condition code models.  For example, architectures like the Intel x86, 
Sparc V8, and PowerPC contains dedicated condition code registers, which
are read from and written to by branching and comparison instructions.
On the other hand, architectures such as the Texas Instrument C6, PA RISC,
Sparc V9, and Alpha does not include dedicated condition code registers.
Conditional code registers in these architectures
can be simulated by integer registers.
\end{itemize}


A conditional code register bit can be referenced using the constructors
\sml{CC} and \sml{FCC}.  Note that the \emph{condition} must be specified
together with the condition code register.
\begin{SML}
   CC   : Basis.cond * src -> ccexp 
   FCC  : Basis.fcond * src -> ccexp    
\end{SML}
For example, to test the \verb|Z| bit of the \verb|%psr| register on the
Sparc architecture, we can used \sml{CC(EQ,SparcCells.psr)}.  

The comparison operators \sml{CMP} and \sml{FCMP} performs integer and
floating point tests.  Both of these are \emph{typed} by the precision
in which the test must be performed under.
\begin{SML}
   CMP  : ty * Basis.cond * rexp * rexp -> ccexp  
   FCMP : fty * Basis.fcond * fexp * fexp -> ccexp
\end{SML}

Condition code expressions may be combined with the following
logical connectives, which have the obvious meanings.
\begin{SML}
   TRUE  : ccexp 
   FALSE : ccexp 
   NOT   : ccexp -> ccexp 
   AND   : ccexp * ccexp -> ccexp 
   OR    : ccexp * ccexp -> ccexp 
   XOR   : ccexp * ccexp -> ccexp 
\end{SML}

\subsection{Statements}

Statement forms in MLTree includes assignments, parallel copies,
jumps and condition branches, calls and returns, stores, sequencing,
and annotation.

\subsubsection{Assignments}

Assignments are segregated among the integer, floating point and
conditional code types.  In addition, all assignments are \emph{typed}
by the precision of destination register.

\begin{SML}
   MV   : ty * dst * rexp -> stm
   FMV  : fty * dst * fexp -> stm
   CCMV : dst * ccexp -> stm
\end{SML}  

\subsubsection{Parallel Copies}

Special forms are provided for parallel copies for integer and
floating point registers.  It is important to emphasize that
the semantics is that all assignments are performed in parallel.

\begin{SML}
   COPY  : ty * dst list * src list -> stm
   FCOPY : fty * dst list * src list -> stm
\end{SML}

\subsubsection{Jumps and Conditional Branches}  

Jumps and conditional branches in MLTree take two additional set of
annotations.  The first represents the \newdef{control flow} and is denoted
by the type \sml{controlflow}.  The second represent 
\newdef{control-dependence} and \newdef{anti-control-dependence} 
and is denoted by the type \sml{ctrl}.

\begin{SML}
   type controlflow = Label.label list
   type ctrl = reg list
\end{SML}
Control flow annotation is simply a list of labels, which represents
the set of possible targets of the associated jump.  Control dependence
annotations attached to a branch or jump instruction represents the
new definition of \newdef{pseudo control dependence predicates}.  These
predicates have no associated dynamic semantics; rather they are used
to constraint the set of potential code motion in an optimizer
(more on this later).

The primitive jumps and conditional branch forms are represented
by the constructors \sml{JMP}, \sml{BCC}.
\begin{SML}
   JMP : ctrl * rexp * controlflow  -> stm
   BCC : ctrl * ccexp * Label.label -> stm
\end{SML}

In addition to \sml{JMP} and \sml{BCC}, 
there is a \emph{structured} if/then/else statement.
\begin{SML}
   IF  : ctrl * ccexp * stm * stm -> stm
\end{SML}

Semantically, \sml{IF}($c,x,y,z$) is identical to
\begin{SML}
   BCC(\(c\), \(x\), L1)
   \(z\)
   JMP([], L2)
   DEFINE L1
   \(y\)
   DEFINE L2
\end{SML}
where \verb|L1| and \verb|L2| are new labels, as expected.

Here's an example of how control dependence predicates are used.
Consider the following MLTree statement:
\begin{SML}
   IF([p], CMP(32, NE, REG(32, a), LI 0),
        MV(32, b, PRED(LOAD(32, m, ...)), p),
        MV(32, b, LOAD(32, n, ...)))
\end{SML}
In the first alternative of the \verb|IF|, the \verb|LOAD|
expression is constrainted by the control dependence 
predicate \verb|p| defined in the \verb|IF|,
using the predicate constructor \verb|PRED|.  These states that
the load is \emph{control dependent} on the test of the branch,
and thus it may not be legally hoisted above the branch without
potentially violating the semantics of the program. 
For example,
semantics violation may happen  if the value of \verb|m| and \verb|a|
is corrolated, and whenever \verb|a| = 0, the address in \verb|m| is
not a legal address. 

Note that on architectures with speculative loads, 
the control dependence information can be used to 
guide the transformation of control dependent loads into speculative loads.

Now in constrast, the \verb|LOAD| in the second alternative is not
control dependent on the control dependent predicate \verb|p|, and
thus it is safe and legal to hoist the load above the test, as in
\begin{SML}
   MV(32, b, LOAD(32, n, ...));
   IF([p], CMP(32, NE, REG(32, a), LI 0),
        MV(32, b, PRED(LOAD(32, m, ...)), p),
        SEQ []
     )
\end{SML}
Of course, such transformation is only performed if the optimizer
phases think that it can benefit performance.  Thus the control dependence
information does \emph{not} directly specify any transformations, but it
is rather used to indicate when aggressive code motions are legal and safe.

\subsubsection{Calls and Returns}

Calls and returns in MLTree are specified using the constructors
\verb|CALL| and \verb|RET|, which have the following types.
\begin{SML}
   CALL : rexp * controlflow * mlrisc * mlrisc * 
          ctrl * ctrl * Region.region -> stm
   RET  : ctrl * controlflow -> stm
\end{SML}

The \verb|CALL| form is particularly complex, and require some explanation.
Basically the seven parameters are, in order:
\begin{description}
   \item[address] of the called routine.
   \item[control flow] annotation for this call.  This information 
specifies the potential targets of this call instruction.  Currently
this information is ignored but will be useful for interprocedural   
optimizations in the future.
   \item[definition and use]  These lists specify the list of
potential definition and uses during the execution of the call.
Definitions and uses are represented as the type \newtype{mlrisc} list.
The contructors for this type is:
\begin{SML}
  CCR : ccexp -> mlrisc
  GPR : rexp -> mlrisc
  FPR : fexp -> mlrisc
\end{SML}
   \item[definition of control and anti-control dependence] 
These two lists specifies definitions of control and anti-control dependence.
   \item[region] annotation for the call, which summarizes
the set of potential memory references during execution of the call.
\end{description}

The matching return statement constructor \verb|RET| has two
arguments.  These are:
\begin{description}
  \item[anti-control dependence]  This parameter represents
the set of anti-control dependence predicates defined by the return
statement.
  \item[control flow]  This parameter specifies the set of matching
procedure entry points of this return.  For example, suppose we have
a procedure with entry points \verb|f| and \verb|f'|.  
Then the MLTree statements 
\begin{verbatim}
  f:   ...
       JMP L1
  f':  ...
  L1:  ...
       RET ([], [f, f'])
\end{verbatim}
\noindent can be used to specify that the return is either from
the entries \verb|f| or \verb|f'|.  
\end{description}

\subsubsection{Stores}
Stores to integer and floating points are specified using the
constructors \verb|STORE| and \verb|FSTORE|.   
\begin{SML}
   STORE  : ty * rexp * rexp * Region.region -> stm
   FSTORE : fty * rexp * fexp * Region.region -> stm
\end{SML}

The general form is
\begin{SML}
   STORE(\(width\), \(address\), \(data\), \(region\))
\end{SML}

Stores for condition codes are not provided.
\subsubsection{Miscelleneous Statements}

Other useful statement forms of MLTree are for sequencing (\verb|SEQ|),
defining a local label (\verb|DEFINE|).
\begin{SML}
   SEQ    : stm list -> stm
   DEFINE : Label.label -> stm
\end{SML}
The constructor \sml{DEFINE L} has the same meaning as 
executing the method \sml{defineLabel L} in the 
\href{stream.html}{stream interface}.

\subsection{Annotations}
\href{annotations.html}{Annotations} are used as the generic mechanism for
exchanging information between different phases of the MLRISC system, and
between a compiler front end and the MLRISC back end.
The following constructors can be used to annotate a MLTree term with
an annotation:
\begin{SML}
   MARK : rexp * Annotations.annotation -> rexp
   FMARK : fexp * Annotations.annotation -> fexp
   CCMARK : ccexp * Annotations.annotation -> ccexp 
   ANNOTATION : stm * Annotations.annotation -> stm
\end{SML}
