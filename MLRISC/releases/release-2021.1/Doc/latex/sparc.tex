\section{The Sparc Back End}

The Sparc back end can function in two different modes:
\begin{description}
  \item[Sparc V8]  This is V8 instruction set is used.  In this mode the processor
behaves like a 32-bit processor.   In this mode we assume we
have 16 floating point registers numbered \verb|%f0, %f2, %f4, ..., %f30|.
These are all in IEEE double precision.
  \item[Sparc V9]  This generates code assuming the V9 instruction set is used.
In this mode the processor functions at 64-bit.  In this mode the 
floating point processors can number from \verb|%f0, %f2, %f4, ..., %f62|.
These are all in IEEE double precision.

  New V9 instructions include the 64-bit extended version of multiplications,
divisions, shifts, and load and store.
\begin{verbatim}
    MULX SMULX DIVX SLLX SRLX SRAX LDX STX
\end{verbatim}

  Also, V9 includes conditional moves and more general form of branches.
\begin{description}
  \item[MOVcc]  conditional moves on condition code 
  \item[FMOVcc] conditional moves on condition code 
  \item[MOVR]   conditional moves on integer condition 
  \item[BR]     branch on integer register with prediction 
  \item[BP]     branch on integer condition with prediction 
\end{description}
\end{description}

\subsection{General Setup for V8}

 The SPARC architecture has 32 general purpose registers 
 (\verb|%g0| is always 0)
 and 32 single precision floating point registers.

 Some Ugliness: double precision floating point registers are
 register pairs.  There are no double precision moves, negation and absolute
 values.  These require two single precision operations.  I've created
 composite instructions \verb|FMOVd|, 
  \verb|FNEGd| and 
  \verb|FABSd| to stand for these.
 
 All integer arithmetic instructions can optionally set the condition
 code register.  We use this to simplify certain comparisons with zero
 in the instruction selection process.

 Integer multiplication, division and conversion from integer to floating
 go thru the pseudo instruction interface, since older sparcs do not
 implement these instructions in hardware.

 In addition, the trap instruction for detecting overflow is a parameter.
 This allows different trap vectors to be used.

\subsection{General Setup for V9}

\subsection{Specializing the Sparc Back End}
