\chapter{Equality}

The semantics of equality testing is not yet final (at least in the
current implementation).  This chapter just describes the behavior of
the current implementation.  In one author's opinion, the testing for
equality of values whose type is not known at compile-time is a grave
mistake, as it severely constrains the implementation.

The equality function \verb"op = : 'a * 'a -> bool" is available at
all types \verb"'a".  However, function-values cannot successfully be
compared for equality; an attempt to test the equality of two
functions will raise an exception as described below.

Two values are tested for equality as follows, depending on the kind
of value:
\begin{description}
\item[Primitive types] like integers, reals, and strings have
equality functions with the conventional behavior.

\item[Function types] cannot be compared.  Testing two functions for equality
will raise the \verb"Equal" exception.

\item[Reference types:] On references, equality means identity; a
reference is equal to itself and to no other references, regardless
of similar contents.

\item[Record types] may be compared even if they have functions as
compenents.  Records containing functions may be found unequal (if
non-functional fields are unequal, and are examined first), or
the \verb"Equal" exception may be raised; testing is in alphabetical
order of field names.

\item[Datatypes] may be compared for equality even if built from
function types.  Two elements of a datatype are equal if they employ
the same constructor applied to equal values; testing the equality of
such values might lead to an exception being raised.

\item[Opaque types] from functor parameters and abstractions are
tested for equality as if they were not opaque.
\end{description}

The compiler may give a warning message on testing equality of a type
which must always raise an exception.

Some implementations of ML prohibit (statically) the compilation of
equality over types which are built from function-values.  To this
end, ``equality type variables'' are used to stand for the set of
types that admit equality, just as ordinary type variables stand for
the set of all types.  Equality type variables are written with two
initial apostrophes.  In implementations with the more permissive
equality mechanism, equality type variables may be used; but they might not
be checked more restrictively than ordinary type variables.
