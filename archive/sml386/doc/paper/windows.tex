\section{The 386/Windows platform}

In this section we discuss those aspects of the 80386 and Windows,
that are important in the context of SML\_NJ. Windows causes troubles
in implementing the system on a PC. This is because Windows still is
based on the old 8086 memory model where the memory is divided into
64k segments. As explained in Section \ref{sec:80386}, this fits badly
with the assumptions which SML\_NJ makes about storage usage; these
assumptions are described in Section \ref{sec:assumptions}. In Section
\ref{sec:windows} we describe how one can fit the two together.

\subsection{Assumptions that SML\_NJ relies on} \label{sec:assumptions}

The compiler operates with two basic datatypes --- integers and
pointers --- both 32-bit\footnote{actually only 31 because the least
significant bit is used as a tag bit to distinguish between integers
and pointers \cite{bib:runtimesystem}.}. Thus it is desirable that the
architecture supports 32-bit data and addresses. The way memory is
allocated and the way the garbage collection is done
\cite{bib:gc,bib:runtimesystem} assumes that the ML code and data are
located in a single continuous memory block. Since compiled ML code
accesses variables and functions in the runtime system through 32-bit
pointers and {\em vise versa}, the runtime system should be in the
same logical memory space as the ML code/data.

The compiled ML code occasionally makes calls to the operating system,
of which some are critical and other are of a more peripheral nature.
That is, some of the system calls are used by the compiler itself
whereas others just are for use in user applications. The critical
system calls must be supported by the operating system, or at least it
must be possible to simulate a corresponding action.

Languages like ML make heavy use of memory, so the operating system
should provide some kind of virtual memory.

In the next two sections we will see how the 80386 and Windows meet
these requirements. Only a brief overview together with the choices
made will be given here. For a more elaborate discussion of the
problems with choosing a memory model and how to build the system, the
reader is referred to \cite{bib:ysgsml}.

\subsection{The 80386} \label{sec:80386}

The 80386 processor has 32-bit registers for manipulation of code and
data. It is possible to address up to $2^{32}$ bytes (i.e.  4 Giga
bytes), and the built-in paging mechanism makes virtual memory
possible.

The memory organization is of interest to us. The 80386 operates with two
kinds of segments:

\begin{descit}{USE16} these are 16-bit segments. An address consists 
of a 16-bit segment address and a 16-bit offset. Code running in this
type of segments by default uses 16-bit addresses and 16-bit data.
Segment registers contain indices into a descriptor table that
contains information about the type and location of the segment. With
these segments the code and data has to be divided into 64K pieces,
causing the well-known problems with managing large programs.
\end{descit}
\begin{descit}{USE32} these are 32-bit segments. An address consists of a
 16-bit segment selector and a 32-bit offset. Code running in this
type of segments by default uses 32-bit data and 32-bit addresses.
This makes it possible to have a flat 32-bit memory model where the
memory consists of a single contiguous block.
\end{descit}

The same binary machine code will cause different actions in the two
types of segments. When using USE16 segments, the 386 operates much
like the predecessor 80286, only now 32-bit registers for manipulation
of data are available. When using USE32 segments the processor acts
like a real 32-bit processor.

The 80386 operates with separate code and data. It is not possible to
write into a code segment nor to execute code from a data segment. But
by letting a code segment register and a data segment register point
to the same physical memory, self-modifying code becomes possible.
This is known as {\em segment aliasing\/}.

The instruction set is comprehensive enough to implement the abstract
machine that the front end generates code for. There are some problems
in that the 386 has inherited many of the peculiarities from its
predecessors. Some registers are dedicated to special purposes in some
instructions and different addressing modes are available in different
instructions. For details, see \cite{bib:ysgsml}.

In short, the 80386 has what is needed to support the compiler.
Unfortunately, Windows does not make all its capabilities available to
the programmer.

\subsection{Windows} \label{sec:windows}

Although Windows in enhanced mode exploits some of the processors
32-bit facilities, it continues to adhere the segmented memory model.
It is not possible to implement a Windows application using an
exclusively flat 32-bit memory model. Windows itself relies on 16-bit
segments and a Windows application must contain at least one USE16
segment to interact with Windows.

Considerations about implementing the system using the USE16 memory
model only, are given in \cite{bib:ysgsml}. Here I'll just state that
it is not possible to get a satisfactory system within that model.
Instead I'll explain how to run 32-bit programs under Windows.

The dynamic-link library WINMEM32.DLL which comes with the Software
Development Kit (SDK) provides a set of functions that allow an
application to make use of the 32-bit capabilities of the 80386
processor. It contains a function {\em Global32Alloc} that allocates a
USE32 data segment (up to 16Mbytes) for which a code alias can be made
with the function {\em Global32CodeAlias}, thus enabling segment
aliasing.  This makes it possible to generate and run 32-bit ML code
within Windows.

As mentioned above (parts of) the runtime system should lie in the
same address space as the ML code. We shall somehow move the runtime
system into the ML-heap allocated with the {\em Global32Alloc}
function. This can be done by collecting the relevant variables and
functions of the runtime system in an assembler module, and then
compiling and linking this module into a segment that can be copied
into the ML-heap when initializing the system (the Microsoft C
compiler cannot generate 32-bit code so we have to use the assembler).
It is possible to access an USE32 segment from C code through a 64K
"window" obtained by the {\em Global16PointerAlloc\/} function in the
WINMEM32 library. This function allocates an USE16 alias to a portion
(up to 64K) of the USE32 segment, but it is inconvenient and slow to
use this technique when operating on larger portion of the ML heap.
With the assembler one has completely control over and access to all
the different kinds of segments.

There are three types of functions in the runtime system:
\begin{itemize}
\item 
functions that are called directly from the compiled ML-code must be
placed in the ML-heap and therefore must be implemented in assembler.
\item 
functions that interact closely with the ML-heap; these could be
implemented in C by using USE16 aliases to access the ML heap, but
have been implemented in assembler for greater efficiency.
\item
functions that hardly interact with the ML-heap; these can without any
 loss of efficiency be implemented in C.
\end{itemize}

The heart of the garbage collector is implemented in assembler
together with some few utility functions (as are the functions that
are called directly from ML). The rest of the runtime system is
implemented in C.  Access to the ML-heap is obtained through 64K
"windows" allocated with {\em Global16PointerAlloc\/} function.







