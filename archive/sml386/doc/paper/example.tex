\section{Example} \label{sec:example}

In this sections we show how a simple function is compiled with the
80386 code generator. The assembly code shown below is generated with
the assemble command in the batch compiler (see the file {\em
BATCHINSTALL\/}).

\begin{verbatim}

fun f x = if x=0 then 1 else x*f(x-1)

\end{verbatim}

is compiled into:

\begin{footnotesize}
\begin{verbatim}
; [esp+00] is the datalimit ``register'' and holds the highest available addr.
; [esp+12] holds the address of the rutine to initiate g.c.
: [eps+20] holds the address of the rutine to handle overflow
; [esp+40] is the standard argument ``register''
; [esp+44] is the standard continuation ``register''
; [esp+48] is the standard closure ``register''
  ...

L3:
cmp   dword ptr [esp+0],edi     ; check the available memory
jns @f
call    dword ptr [esp+12]      ; and call garbage collection if necessary
@@:
cmp   dword ptr [esp+40],1      ; if x=0
jne   L14
mov   edx,dword ptr [ebp+0]     ; then continue with 1
mov   dword ptr [esp+44],ebp
mov   dword ptr [esp+40],3
jmp   edx
L14:                            ; else x*f(x-1)
mov   eax,49
stos  eax                       ; setup a closure with the argument x and
lea   eax,L4                    ; the continuation for the multiplication L4
stos  eax
mov   eax,dword ptr [esp+40]
stos  eax
mov   eax,ebp
stos  eax
lea   ebp,dword ptr [edi+-12]
sub   dword ptr [esp+40],2      ; x=x-1
jno @f                          ; check for overflow
call    dword ptr [esp+20]
@@:
jmp   L3                        ; make the call f(x-1)

...

L4:
cmp   dword ptr [esp+0],edi     ; check the available memory
jns @f
call    dword ptr [esp+12]      ; and call G.C. if neseccary
@@:
mov   eax,dword ptr [esp+44]    ; make the multiplications
mov   eax,dword ptr [eax+4]
mov   dword ptr [esp+48],eax
mov   ebx,dword ptr [esp+48]
sar   ebx,1
sub   dword ptr [esp+40],1
mov   ecx,dword ptr [esp+40]
imul  ecx,ebx
mov   dword ptr [esp+40],ecx
jno @f
call    dword ptr [esp+20]
@@:
add   dword ptr [esp+40],1
mov   eax,dword ptr [esp+44]
mov   eax,dword ptr [eax+8]
mov   dword ptr [esp+44],eax
mov   eax,dword ptr [esp+44]
mov   edx,dword ptr [eax+0]
jmp   edx                      ; and continue
...
\end{verbatim}
\end{footnotesize}

Notice how the stack is used to simulate registers, and how the $EAX$
and $ECX$ are used as tempory registers. In SML\_NJ an integer $i$ is
represented as $i*2+1$, so the integer one (1) is represented as the
integer three (3).





