\section{Introduction} \label{sec:intro}

Standard ML of New Jersey \cite{bib:MLComp} is an implementation of
the programming language Standard ML \cite{bib:smldef}. So far,
Standard ML of New Jersey (SML\_NJ) has solely been available on
machines that run under the UNIX operation system. The purpose of
this paper is to report a port of SML\_NJ which runs on the Intel 80386
processor under Microsoft Windows 3.0, one of the most common
combinations for PC's.


Our port of SML\_NJ consists of two parts: 
\begin{enumerate} 
\item Standard ML modules which the SML\_NJ code generator calls in order to
produce 80386 machine code.
\item A new runtime system written in C and 80386 assembly language.
\end{enumerate}

The first part is largely independent of Windows, i.e. it could be
used for generating 80386 code that would run under other operating
systems. The second part was necessitated by the fact that the runtime
system has to run under Windows, which is very different from UNIX.

The reson we wrote parts of the runtime system in assembly language is
that SML\_NJ generates 32-bit code but no 32-bit C compiler is
currently available under Windows.

When a 32-bit version of Windows becomes available, a new runtime
system, which more closly resembles the original runtime system, can
be written.

Few prerequisites are required to read this paper: familiarity with ML
is neccesary and some knowledge about the Intel 80386 assembler is
useful.

The rest of this paper is organized as follows: in Section 2 we review
those aspects of the 80386/Windows platform that are of particular
interest in this context. In Section 3 we report the new runtime
system and in Section 4 we describe the new code generator. Finally in
Section 5 we show an example.

Many of the technical details that are not covered in this paper can
be found as comments in the source code. References are made to the
relevant files, which are located in the {\em sml\/} distribution directory.






