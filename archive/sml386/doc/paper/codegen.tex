\section{The code generator}

The SML\_NJ is nicely divided into a machine-independent front end and
a machine-dependent back end. A single signature (called {\em
CMACHINE\/}) defines the interface between the two. The back end
consists of a few structures that --- together with a few structures
in the front end --- are described below. At the end of this section
some special aspects of the 80386 that has influence on the code
generator are examined more closely.  The intention with this section
is to give a survey of the code generator and the description is
therefore kept at a very high level.

\subsection{From CPS-expressions to machine code} \label{sec:CPStoMach}

The front end transforms SML source code into CPS-expressions
\cite{bib:ContClos,bib:ysgsml,bib:orbit,bib:rabbit}. The
transformation of CPS-expressions into machine code is handled by the
machine-independent functor {\em CPSgen\/}. {\em CPSgen\/} is
parameterized on elementary code generating functions specified by the
signature {\em CMACHINE\/}. That is, {\em CMACHINE\/} defines names
for datatypes, variables, and functions that the front end will use
when transforming CPS-expressions into machine code (see
Figure~\ref{fig:CPSgen}).
\input{cpsgen.fig} The {\em CPSgen\/} functor is located in the file
{\em generic.sml\/}, and {\em CMACHINE\/} is found in {\em
cmachine.sig\/}.

\subsection{Backpatch and jumpsize-optimization}

The back end has to handle relative addresses. That is, we have to
backpatch relative jumps and other instructions that use relative
addressing. To help in this, a machine-independent functor, {\em
Backpatch\/}, is included in the compiler. {\em Backpatch\/} is
parameterized on a machine-dependent structure {\em Jump\/} which
contains the information needed to backpatch on a particular machine.
{\em Backpatch} needs to know the size in bytes of the
instructions that use relative addressing and how to emit code for
these. The signatures {\em BACKPATCH\/} and {\em JUMP\/} are given
in Figure~\ref{fig:BackPatch}.  \input{backpatc.fig} 

The machine code generator can make use of the primitives named in
{\em BACKPATCH\/} ({\em emitstring, newlabel, etc\/}) when
implementing the functions in {\em CMACHINE\/}. For example the {\em
emitstring()\/} is used every time a string (code or data) is put into
the code (see Figure \ref{fig:I386xCod}). The full {\em Backpatch\/}
functor, which is part of the general SML\_NJ system, can be found in
the file {\em backpatch.sml\/}. The {\em Jump\/} structure for the
80386 is found in the file {\em 386jumps.sml\/}.

\subsection{The 80386 code generator}

As mentioned in section \ref{sec:CPStoMach} we have to write a
structure that matches the {\em CMACHINE\/} signature. A functor {\em
CMach386} is used for this. The front end includes a batch compiler
that can generate assembler code, so we need to generate both assembly
code and machine code. The `argument' given to {\em CMach386}
determines which one is generated.

The signature {\em CODER386} defines a subset of the 80386
instruction set and addressing modes, that is used in the code
generation. Two functors which match this signature
\begin{descit}{MCode386} generates machine code \end{descit}
\begin{descit}{ACode386} generates assembly code \end{descit} 
are outlined in Figure \ref{fig:I386xCod}. \input{i386xcod.fig}

We see how the functor {\em MCode386\/} is parameterized on the
machine-dependent {\em Jump\/} structure mentioned above, and how this
structure is passed on to {\em Backpatch\/}. This is how we get access
to the primitives named in the {\em BACKPATCH\/} signature. When
generating assembly code we do not need to backpatch and therefore the
{\em Jump\/} structure is unnecessary.  Instead symbolic labels are
used. Notice that the primitives in {\em Backpatch\/} are used in the
{\em MCode386} module only.

These two functors are used to make a 80386 code generator to be
passed on to {\em CPSgen\/}. Applying the functor {\em CMach386} to
one of them results in a structure whose functions when called,
generates code (machine or assembly) for the 80386.

The signature {\em CODER386\/} is found in the file {\em
386coder.sig\/}. The {\em Mcode386\/} and the {\em ACode386\/}
functors are found in the files {\em 386mcode.sml\/} and {\em
386acode.sml\/}.

\subsection{Putting it together} \label{sec:glue}

SML\_NJ includes an interactive system and a batchcompiler. The front
end functor {\em IntShare\/} defines the interactive system, and {\em
Batch\/} defines the batchcompiler. Figure~\ref{fig:glue}
\input{glue.fig} shows how to build the interactive system and the
batchcompiler for the 80386.  {\em Batch} takes 2 arguments; {\em M}
that generates machine code and {\em A} that generates assembly code.
{\em Intshare} takes three argument where one of them ({\em Machm\/})
is the structure that generates the machine code. {\em Comp386} is the
batchcompiler module and {\em Int386} is the interactive module.

\subsection{Special conditions in 80386}

The compiler was originally designed for the Vax and M680x0 family.
Later on, code generators have been made for other architectures like
MIPS and SPARC. The 80386 lacks some of their facilities and this has
caused some problems.

One problem is the number of general registers. The {\em CMACHINE\/}
signature specifies a set of variables. These are supposed to be in
registers. But the 80386 has only seven general registers, so we have
to simulate some extra registers in the memory. Where should these be
placed? Under Windows it is possible to use absolute addresses because
we manually allocate a segment to hold the ML code and data, and
therefore have control over what goes where. Thus we can allocate say
the $4n$ lowest addresses in the ML-heap to hold the $n$ "memory
registers". But in general (i.e. in other operating systems) we cannot
use absolute addresses.

The runtime system (written in C) defines a set of variables that
compiled ML code uses. These (or theirs addresses) are placed in a
C-structure whose address is passed on to the ML code when
initializing the system. In the same way we could allocate space in
the runtime system (e.g.  {\tt\mbox{int MemRegs[n]}}) and pass {\em
MemRegs\/} to the ML code. But this would require some changes in the
front end, and the goal was to implement the code generator without
any changes at all in the front end.

As a result of Continuation Passing Style, ML does not use the stack.
We therefore have a static stack when running ML code, so the "memory
registers" can be put on the stack in the same way as one would
allocate space for local variables on the stack in a C-compiler.  This
is what has been done here. But care must be taken when referring to
these variables. Some registers in the 80386 are dedicated to special
purposes in some instructions. Consequently, the stack is used by a few
functions in the code generator to save the value of these registers,
when necessary. But this is not a problem because we always know
exactly how much has been pushed onto the stack.

Using the stack to hold the memory registers has another advantage
over absolute addresses in that we save 2 bytes in the code size on
every reference to memory variables. The code size is a serious
problem when running under Windows because we are restricted to 16
Mbyte.

Because the 80386 cannot perform memory to memory operations, a lot of
moving to temporary registers takes place, which again contributes to
the code size. This is a serious problem with the 80386 when running
under Windows.

Another problem with the Intel 80386 microprocessor is that it does
not directly support PC-relative addressing. SML\_NJ provides
facilities to handle this problem. The front end keeps track of the
addressing requirements of each function and calls the {\em
beginStdFn\/} in {\em CMACHINE\/} if it uses PC-relative addressing.
One of the parameters to {\em beginStdFn\/} is the closure of the
function so {\em beginStdFn} can load the base address and later use
it for relative addressing (see the file {\em cmachine.sig} for
details). But in version 0.66 of the compiler, the {\em needPC\/}
function (located in {\em CPSgen\/}) that determine whether a function
uses PC-relative addressing is constantly {\em true\/}:
\begin{verbatim}
    fun needPC cexp = true
\end{verbatim}
Thus, if the above technique is used, every function gets the extra
overhead with calculating a function's base address. Hoping that many
functions don't use PC-relative addressing I have chosen another
technique. We need to generate code for instructions like {\em load
effective address}:
\begin{verbatim}
    LEA reg1, (PC,offs)    ; reg1 := PC+offs
\end{verbatim}
This can be done by the following piece of code:
\begin{verbatim}
    1000: call  0           ; 1005 pushed (relative call)
    1005: pop   eax         ; eax = 1005
    1006: sub   eax, offs+5 ; +5 because the size of the call 
                            ; instruction is 5 bytes
\end{verbatim}
This way we can simulate PC-relative addressing. The size of these
three instructions is 12 bytes, so if this kind of addressing is
performed often it is better to use {\em beginStdFn\/} as described
above. The frequency of PC-relative addressing ought to be measured to
determine which technique is better.

This concludes the description of the 80386 code generator.