\section{The runtime system} \label{sec:runtime}

In this section the organization of the new runtime system is
described in terms of the existing SML\_NJ runtime systems.

The SML\_NJ runtime system is written in C and assembler. It consists of:
\begin{itemize}
\item A carbage collector
\item Small library functions written i assembler
\item Small library functions written i C
\item Facilities to handle Export and Import
\item Facilities to handle signals (interrupts)
\end{itemize}

The runtime systems for the different UNIX versions are very alike.
The differences are mostly expressed using \#IFDEF OPSYS ...
\#ENDIF declarations in the C-code. When writing a runtime system for
Windows we cannot simply reuse the original C-code. As mentioned in
the previous section, there are problems with mixing USE16 C-code and
USE32 ML-code. However, I have tried to maintain the original
structure of the runtime system, and only made modifications where
needed because of the mixed memory model and the restrictions made by
Windows. (Before reading the C code of the new runtime system, the
reader is advised first to look at the original runtime system written
for UNIX, because the many non-standard C details in the Windows
version reduce the clarity.)

\subsection{Segments} \label{sec:segments}

One of the first things the system does, when started, is to setup the
USE32 segments that constitute the ML-heap. This is done by the code
in the file {\em segments.c\/}. A USE32 segment (called {\em
Use32Data\/}) is allocated and a code alias (called {\em Use32Code\/})
is established.  These two constitute the ML-heap. The part of the
runtime system that must lie on the ML-heap are placed in an assembler
module (in the file {\em prim.asm}) which is compiled and linked into
a segment (called {\em \_RUNCODE\/}). The contents of this segment is
copied into the ML-heap by the code that set up the USE32 segments.

\subsection{The inteface between ML and the runtime system}

After the system is initialized, control is passed to the compiled ML
code. A C structure which contains the addresses of relevant variables
and functions in the runtime system is passed on to the ML code. To
call a C library function the ML code looks up its address {\em
faddr\/} in a table and calls the {\em call\_c\/} function with {\em
faddr\/} as an argument.

Our interface must handle the context switch from USE16 to USE32
associated with the transfer of control from the runtime system
(USE16) to the compiled ML code (USE32). When the ML code is entered,
registers that the C compiler uses are saved on the stack. A C
structure ({\em MLState\/}) holds the ML state. Before jumping to ML,
registers are loaded from this structure.  When returning to the
runtime system things happens in reverse order.

When executing code in the runtime system the values in the segment
registers are taken care of by the C compiler/linker and the Windows
loader. Before jumping to the compiled ML-code we must load the
segment registers with the correct values. DS is used as implicit
segment register in most machine instructions. ES is used in string
operations, and the SS segment register is used when ESP and EBP serve
as index registers in memory references. DS, ES, and SS are loaded
with the {\em Use32Data\/} value returned by {\em Global32Alloc\/},
before jumping to the ML-code.  Care must be taken when switching
stack in this way.  When the USE16 stack is used, the upper 16 bits of
the stack pointer (ESP) are not used, and can have random values. When
switching to the USE32 stack we must ensure that the ESP register
contains a legal stack address. After a move to the SS register the
80386 interrupt is disabled in one instruction, so we can handle this
by the following code sequence:
\begin{verbatim}
    ...
    mov ax, Use32Data
    mov ss, ax                     ; no interrupts between
    mov esp, Use32StackPointer     ; these two instructions
    ...
\end{verbatim}

The context switch is handled by the function {\em restoreregs\/} in
the file {\em interfce.asm\/} and by the functions {\em enterUse32\/}
and {\em saveregs\/} in the file {\em prim.asm\/}, which also contains
the variables and functions which must be placed in the ML-heap.

\subsection{C library functions}

The UNIX runtime system contains a number of C library functions that
handle the interaction with the operating system. Some of these are
UNIX specific and have no equivalent in Windows. In the Windows
runtime system, these will generate a system exception if called:
\begin{verbatim}
    void foo(ML_val_t arg)
    { raise_syserror("foo is not implemented under Windows!"); }
\end{verbatim}

Certain functions that are not crucial to the compiler and that are
difficult to implement under Windows are not implemented yet, and will
also generate a system exception, if called.

The functions crucial to the compiler are all implemented. Some of
these functions are UNIX specific and do not have an equivalent
meaning in Windows, e.g. the masking and unmasking of signals. When
called, those functions will perform some neutral action but will not
cause a system exception. This will in some cases mean wrong return
values to the ML system but these are not crucial. For example,
Windows cannot perform all of the timer functions that are available
in UNIX, so timing is not accurate in the Windows system. The C
library functions are located in the file {\em callc.c\/}.

Details about the interaction between the runtime system and the ML
code and data are found in \cite{bib:ysgsml}, and in the comments in
the files {\em prim.asm\/}, {\em gc.asm\/}, {\em interfce.asm\/}, and
{\em util.asm\/}.

\subsection{Assembly library functions}

There are a number of assembler library functions that are called
directly from compiled ML code. These are:
\begin{itemize}
\item {\em array(n,x)\/} allocates an array of length $n$, and initializes
 its elements to $x$.
\item {\em callc(f,a)\/} calls the C function whose address is $f$ with
 the argument $a$.
\item {\em create\_b(n)\/} allocates an uninitialized byte-array of length
 $n$.
\item {\em create\_s(n)\/} allocates a string of length $n$.
\item {\em floor(x)\/} returns the "floor" of the real number $x$.
\item {\em logb(x)\/} returns the exponent of the real number $x$.
\item {\em scalb(x)\/} inserts a new exponent into the real number $x$.
\end{itemize}

Floating point operations are executed by using the 80387 math
co-processor. We have added utility functions to handle the mixed
memory model. Se comments in the file {\em util.asm\/}.


\subsection{Garbage collection}

The original garbage collector operates on the entire ML heap. Because
it is difficult and inefficient to access USE32 data from the C code,
We have implemented the heart of the garbage collector in assembler. It
is actually just a "hand compilation" of the original garbage
collector written in C, where 32-bit addresses and registers are used.
The heart of the garbage collector is located in the file {\em
gc.asm\/}, whereas other g.c. related functions are found in the file
{\em callgc.c\/}.

\subsection{Signal handling}

In the UNIX version signal handlers have been implemented for a number
of hardware signals. This is not possible in the Windows version. The
whole signal machinery has been neutralized in the Windows version.
The lack of signal handlers has some annoying consequences. For
example it is not possible to stop an infinite loop with Ctrl-C.

As shown in \cite{bib:gc} signals can, in a smooth way, be used to
initiate garbage collection. All the existing implementations for UNIX
use signals this way, either by allocating ahead until a pagefault
occur, or by an explicit test for the available memory followed by an
"interrupt on overflow". This can't be done in the Windows version.
Instead we must make an explicit test and jump to a routine that can
initiate the garbage collection when necessary.
\begin{verbatim}
                     ...
                     cmp    datalimit, allocptr     |
                     jno    no_overflow             |  8 bytes
                     jmp    initiate_gc             |
        no_overflow: ...
\end{verbatim}

instead of:
\begin{verbatim}
                     ...
                     cmp    datalimit, allocptr     |
                     into                           | 4 bytes
                     ...
\end{verbatim}

This is unpleasant because it takes up space (we shall perform this
check at the beginning of every function).

\subsection{Export and Import}

In the UNIX version it is possible to export the state to an
executable file (a.out format). This is used to make stand alone
programs, and in particular it is used when bootstrapping the system
to make "ready to run" versions of the interactive system and the
batch compiler. Because of the problems that Windows has in handling
USE32 segment this is not possible in Windows.

The ability to export the state to an executable file, has influence
on the execution speed and the memory requirements. If exported to an
executable file, the compiler code is located together with the runtime
system below the base address of the ML heap, and will therefore not be
collected when a major collection is performed \cite{bib:gc}. This
would in our system mean higher execution speed because most of the
time is spent doing major collections. Because the heap always is at
least three times the size of the living data
\cite{bib:gc,bib:runtimesystem}, the presence of the whole compiler on
the heap contributes considerably to the memory requirements.

In principle, we could simulate the export by writing the contents of
the ML-heap together with the contents of the runtime system's data
segment to a file, and then use a special version of the runtime
system to load and execute this file. This has not been done yet;
instead the runtime system can load *.mo files as described in the
{\em howto-boot\/} file in the SML\_NJ documentation directory.



