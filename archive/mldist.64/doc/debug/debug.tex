\documentstyle [11pt]{article}
\setlength{\textheight}{8.8in}
\setlength{\textwidth}{6.5in}
\setlength{\oddsidemargin}{0in}
\setlength{\topmargin}{-.1in}
\setlength{\headheight}{0in}
\setlength{\headsep}{0in}

\title{Debugging in Standard ML of New Jersey}
\author{Andrew P. Tolmach
and Adam T. Dingle \\ Princeton University}
\begin{document}
\maketitle
\section{Introduction}
This is a {\em preliminary} reference manual for the Standard ML of New Jersey 
Debugger, designed and implemented by Andrew P. Tolmach with an Emacs interface
by Adam T. Dingle.  

The system is new; many aspects of it, including key bindings and the details
of commands, are likely to change.  Your comments and suggestions are welcomed.
Bug reports are also welcome, but no guarantees are offered as to when 
they will be dealt with.

\section{The Debugger}
The debugger is a specially-designed extension to the SML-NJ compiler.
In its basic concepts and capabilities, the debugger resembles typical 
source-level debuggers such as dbx.  You can set breakpoints, single-step 
execution, and display identifier values, while examining your source code 
in a separate window (managed by GNU Emacs).  The debugger is implemented
using a {\em source code instrumentation} technique.  When directed to 
process programs
for execution under debugger control, the compiler automatically inserts
instrumentation code at all interesting {\em events}, including function
applications, identifier bindings, the tops of function bodies, and so on.
This instrumentation code allows the debugger to collect information and 
optionally gain control at these points in the program.  Single-stepping 
units and breakpoints are defined in terms of events; in this sense, they
take the role of line numbers in debuggers for conventional languages.

The debugger's most unusual feature is that it supports 
{\em reverse execution}; that is, it gives you the illusion of being able
to execute your program backward as well as forward.  This feature is 
implemented by taking periodic checkpoints of program state and logs of
all I/O activities.  To jump back to a previous point in the program's 
execution, the debugger restores the last known checkpoint prior to the
target point, and then re-executes forwards.  (Checkpoints consist of
stored continuations, as captured by the \verb'callcc' primitive, plus
a description of the mutable \verb'ref' and \verb'array' data in the program.)
To keep track of where it is in the program's execution, the debugger 
increments a counter each time an event is encountered. 
We refer to the value of this counter as the current {\em time}, 
and many commands take or return times.

In addition to being a valuable 
user-level feature, reverse execution is used internally by the debugger
to support location-based breakpoints and identifier value lookup.
More details on the implementation philosophy may be found in \cite{us}.

\section{Setting Up}
The SML-NJ debugger is intended to be run in GNU Emacs (or Epoch), 
allowing screen- and mouse-oriented ``visual'' debugging.
Events are displayed within the source text
itself; an ML program can be single-stepped, breakpoints can be set,
identifiers' values can be displayed and the call traceback can be
examined by typing key sequences in the source text window.
The implementation extends Lars Bo Nielsen's Emacs mode for
editing SML programs, which is now part of the GNU Emacs distribution.

To install, build a version of the compiler that includes the debugger if
one does not already exist.  See the instructions on building the
compiler; include the -debug command line option to the makeml
program.  The executable image will be called smld by default.  You will be
able to use this image to compile code in debug mode or in the regular
compilation mode; you may wish to hold onto the version of the compiler
that does not support debugging since it is a few hundred kilobytes
smaller.  When compiling code outside
debug mode, compilation and execution times should not differ
noticeably between the two compiler images unless you are very tight on memory.

Secondly, place the files sml-mode.el, sml-init.el and sml-debug.el
(found in the lib/emacs directory of the SML distribution) in
a directory where Emacs finds its Lisp code; you can find out what
directories Emacs looks in by typing \verb'C-h v load-path' in Emacs.
You can alternatively place these three files in a directory of your own
and tell Emacs to include that directory in its search path for Lisp
code; to do so, enter the following line in your .emacs file (where
\verb'/u/me/myEmacsDir' is the directory where you would like Emacs to
look):

\begin{verbatim}
(setq load-path (cons "/u/me/myEmacsDir" load-path))
\end{verbatim}

Finally, add the following line to your .emacs file:

\begin{verbatim}
(load-library "sml-init")
\end{verbatim}

sml-init.el includes code that sets up both sml-mode (the sml editing
mode written by Lars Bo Neilsen) and sml-debug (debugging extensions
to sml-mode).

You should be running GNU Emacs with a version number of at least
18.55.  The system will probably work on earlier versions of Emacs as
well, but has only been tested on 18.55.  If you are running Epoch (a
version of Emacs for the X Window System developed by Alan Carroll and
Simon Kaplan at the University of Illinois at Urbana-Champaign), the
debug mode will take advantage of Epoch's highlighting capabilities.
You can obtain Epoch via anonymous ftp to cs.uiuc.edu.  The debugger 
interface has been tested using Epoch 3.2-beta.

\section{Getting Started}
From within Emacs, type the command \verb'M-x sml' to start up a
window with SML running in it.  In that window, type
\verb'usedbg "<filename>"' to compile and execute the contents of a
file in passive debug mode.  Typically, the file will contain function
definitions.  (When you \verb'usedbg' a file, all files \verb'use'd by
that file will be compiled with debugging as well.  The semantics of
\verb'use' are tricky, even when the debugger is not involved; it is 
strongly recommended that any file containing one or more \verb'use's 
contain {\em only} \verb'use's.)

Type \verb'run "<declaration-or-expression>"' to compile and evaluate a
given declaration or expression in active debug mode, which will allow you 
to step through program execution in a visual way.  
Typically, the argument to \verb'run' will be an expression that applies
one of the functions that was loaded using the \verb'usedbg' command. 
(The argument should be given just as you would normally type
it to the interactive system, except that double quotes (\verb'"') must 
be preceded by a backslash (\verb'\'). No trailing semicolon is needed.)

In response to your \verb'run' command, an Emacs window named
\verb'*sml-debug-command*' will be created containing the declaration or
expression to be evaluated. The first {\em event} will be displayed 
in brackets in that window; it will look something like

\begin{verbatim}
[START:4]
\end{verbatim}
where START indicates the type of the event and 4 is the current
time.  (For various technical reasons time does not begin at 1.)

Events are places in the ML program where you can stop execution to
look at identifier values.  You can use the Emacs key sequence \verb'M-s' to
{\em step} (execute for a short time) until the next event is reached.
(\verb'M-s', pronounced ``meta-s'', means that you should hold down
the \verb'META' key (often labeled \verb'ALT') and press the \verb's'
key.  If your keyboard does not have a \verb'META' key,
type \verb'ESC' followed by \verb's'.  For more information, see an
Emacs reference manual, or type \verb'C-h i' for information while in Emacs.)
Try it.  Try stepping a number of times; see what different types of
events you reach.

As noted above, the SML-NJ debugger also permits you to step {\em}
backward in time.  The key sequence \verb'M-b' steps backward one unit in
time.  Try stepping backward to the beginning of your program.

You can also jump forward or backward to arbitrary times in the 
program's execution: type \verb'M-<time> M-t' (that is,
hold down the \verb'META' key while you type the time number,
then type \verb't' while still holding down \verb'META'.  If your
keyboard does not have a META key, type \verb'C-u <time> ESC t'.)
Your jumps will always be truncated so that they lie between the
beginning and ending times of program execution.

You may occasionally notice that a step command moves for 2 time
units, or that the \verb'M-t' command brings you to a time that is one
unit away from the time you requested.  This happens because certain
unimportant internal events are assigned time numbers but are ignored 
by the debugger.  This inconsistency in time numbering may go away in
a future version.

\section{Selecting Events}
You can use the commands \verb'M-n' and \verb'M-p'
to browse through events in your ML
program.  \verb'M-n' moves the {\em event cursor} to the next event in the
source code; \verb'M-p' moves to the previous event.  These commands do not
cause any part of your program to be executed, or change the current
time; they are simply a way of looking at the events that exist.  The
position of the event cursor is denoted by an event surrounded by
square brackets; when you move the event cursor away from the current
event, the current event remains visible, but is surrounded by a pair
of angle brackets (\verb'<' \verb'>') to distinguish it from the event
cursor.  Typing \verb'M-c' will move the event cursor back to the
current event.

Sometimes you may wish to move the event cursor to a point that is far
away in your program text, or in a different source file than the one
in which the event cursor currently rests.  To do so, move the Emacs
cursor to the area of source code to which you wish to move the
event cursor, then
type the command \verb'M-e'.  The event cursor will jump to the event nearest
to the Emacs cursor; you can then use the \verb'M-n' and
\verb'M-p' commands to
browse the events in that area.

If you are running Emacs under X, you can select events even more
easily.  Point to a character in your source code that is near the
event that you would like to select, then hold down META and press the
middle pointer button.

\section{Breakpoints}
To set a breakpoint at an event, move the event cursor to it and type
the command \verb'M-k'.  The string \verb'bk:' will appear inside the event,
indicating that a breakpoint is set at it.  The event will continue to
be displayed even if you move the event cursor away from it.  Type \verb'M-k'
again to remove the breakpoint at the event.

Use the \verb'C-M-f' command to continue execution until the next breakpoint
(or the end of your program) is reached.  The \verb'C-M-b' command moves 
backward in time until a breakpoint (or the beginning of your program) is 
reached.

You can also set breakpoints at particular times: type
\verb'M-<time> M-k'
(that is, hold down the \verb'META' key while you type
the time number, then type \verb'k' while still holding down \verb'META').
Repeating the same command will delete the breakpoint
you just created.  Type \verb'C-M-k' to get a list of time breakpoints.

The \verb'M-- M-k' command deletes all existing breakpoints.
(\verb'M--' is ``meta-hyphen''.)

You can associate an ML function with a breakpoint; the function is
called whenever the breakpoint is reached.  The function might print
out the value of one or more identifiers.  To associate a function with
a breakpoint at a given event, move the event cursor to the event and
type \verb'breakFunc <function>;' in the SML process window.  The
function should have type \verb'unit -> unit'.

For example, you might type

\begin{verbatim}
- breakFunc (fn () => showVal "a");
\end{verbatim}

To reset the function associated with a breakpoint at an event, move
the event cursor to the event and type \verb'M-k M-k' (removing and
reinstalling the breakpoint at the event).

To associate a function with a breakpoint at a given time, type

\begin{verbatim}
- breakTimeFunc <time> <function>;
\end{verbatim}
To reset the function associated with a time breakpoint, type
\begin{verbatim}
- nofunc <time>;
\end{verbatim}

Soon, the debugger will allow you evaluate arbitrary ML expressions in the
context of the current time;  then, you will be able to write 
break functions that choose
to continue execution (using forward()) if a given ML expression is
satisfied.  In this way, you will be able to make breakpoints conditional.

\section{Examining identifiers and the call traceback}
Type \verb'M-l' to examine the value of any identifier that is currently
in scope.  Enter the identifier's name in the Emacs minibuffer, and its
value will be printed out in the SML process window.  The event cursor will
move to the event at which the identifier's value was bound.  This is 
particularly handy if the identifier represents a function.  You may notice
that if you
move the event cursor away from the event and then try to return to it, 
you might not be able to find the event again;
this is because not all events are normally
displayed (see the introduction to section ~ref{event types} for details).

If you are running Emacs under X, you can examine an identifier by
pointing to its name, holding down META and pressing the first mouse button.

The {\em backtrace cursor} is used to examine the chain of function
calls that led to the point where execution is currently stopped.
\verb'M-u' moves the backtrace cursor, which initially rests on the current
event, up one level in the function call chain.  The
backtrace cursor contains the string \verb':bt' inside it, and also
displays the time at which the corresponding function call occurred.
The command \verb'M-d' moves the backtrace cursor down one level in
the function call chain.  As a convenience, typing \verb'M-t' while a
backtrace event is selected will jump to the time of the backtrace event.

The command \verb'C-M-l' can be used to examine the value of a
identifier at the scope and time of the current backtrace event.  Under
X, you can alternatively point to the name of the identifier and press
the first mouse button while holding down CTRL and META.

\section{Interrupting Execution}
You can halt your program at any time during forward execution 
(e.g., if it enters an infinite loop)
by typing the \verb'C-c C-c' command in the window containing the sml process.
The program will always halt on an event boundary, and the location and time
should be shown in source code window.  You can then use the traceback 
commands to find out where you are in the execution.

\section{Finishing Up}
Two commands are available to exit from an active debugging session. The
\verb'C-c C-d' command aborts execution of the program run you
started with the \verb'run' command; no changes are made to the top-level
environment. The \verb'C-M-c' command completes 
execution of the program run; if the argument you gave to \verb'run' 
included declarations, they are added to the top-level environment.
Both commands remove all event labels from source code windows and delete
the \verb'*sml-debug-command*' buffer.  The window containing the SML
process will remain, so that you can load new code or specify another
command to run.  

While source code windows have events displayed in them, they are
placed into Label mode, a read-only Emacs minor mode.  This prevents
you from editing them and unintentionally saving event labels into
your source files.  Before you edit a section of source code that is
being debugged, you should use the \verb'C-c C-d' command to abort
the debugging session and remove the labels.  If for some reason a
window remains in Label mode after you have exited the debugging
session (this should not happen) you can use the command \verb'C-x
C-q' (exit-label-mode) to remove the labels from the window.

\section{Command and Function Summary}
The following commands are available within the window containing the
sml process as well as within windows containing source code.
\verb'<left>' and \verb'<middle>' refer to the left and middle pointer
buttons when running under the X window system.

\begin{center}
\begin{tabular}{|lll|}    \hline
Command & Default Key Binding & Description \\ \hline
\verb'sml-step' & \verb'M-s'             & single-step \\
\verb'sml-step-backward' & \verb'M-b'           & single-step backwards \\
\verb'sml-goto-time' & \verb'M-<time> M-t'    & go to specified time \\
\verb'sml-select-next' & \verb'M-n'             & select next event \\
\verb'sml-select-previous' & \verb'M-p'             & select previous event \\
\verb'sml-select-current' & \verb'M-c'  & move event cursor to current event \\
\verb'sml-select-near' & \verb'M-e' &
	select event near cursor (source windows \\
    & & only) \\
\verb'sml-mouse-select-near' & \verb'M-<middle>' &
	select event near pointer (source windows \\
    & & only) \\
\verb'sml-break' & \verb'M-k' & toggle breakpoint at selected event \\
\verb'sml-break' & \verb'M-<time> M-k' & toggle breakpoint at given time \\
\verb'sml-show-breaktimes' & \verb'C-M-k'           & list breakpoint times \\
\verb'sml-break' & \verb'M-- M-k'         & delete all breakpoints \\
\verb'sml-proceed-forward' & \verb'C-M-f' 
	& execute forward to next breakpoint \\
\verb'sml-proceed-backward' & \verb'C-M-b' &
	execute backward to previous breakpoint \\
\verb'sml-up-call-chain' & \verb'M-u' &
	move backtrace cursor up the call chain \\
\verb'sml-down-call-chain' & \verb'M-d' &
	move backtrace cursor down the call chain \\
\verb'sml-goto-time' & \verb'M-t' & go to time of selected backtrace event \\
\verb'sml-variable-value' & \verb'M-l or M-<left>' &
	show variable value and binding event \\
\verb'sml-bt-variable-value' & \verb'C-M-l or C-M-<left>' &
	show variable value and binding event at \\ 
                         & & scope and time of backtrace event \\
\verb'sml-abort' & \verb'C-c C-d' &
	abort command execution and \\
    & & exit debugging session \\
\verb'sml-complete' & \verb'C-M-c' & 
	complete command execution and \\
    & & exit debugging session \\
\hline
\end{tabular}
\end{center}

The following useful ML functions are available in the sml process window:

\begin{verbatim}
usedbg: string -> unit
\end{verbatim}

Compile and execute a file in passive debug mode, so that compiled
code will be instrumented for debugging.  All files used by
the file will also be executed in passive debug mode.

\begin{verbatim}
usedbg_stream: instream -> unit
\end{verbatim}

Compile and execute the text of an input stream in passive debug mode.

\begin{verbatim}
uselive: string -> unit
\end{verbatim}

Compile and execute a file in active debug mode.  For a typical file
containing definitions of functions, structures, and functors, this
is chiefly useful as a way to stepping through functor applications.

\begin{verbatim}
run: string -> unit
\end{verbatim}

Compile and execute a command in active debug mode.

\begin{verbatim}
bfunc: (unit -> unit) -> unit
\end{verbatim}

Sets the break function for the breakpoint at the selected event.  (To
reset the break function, select the event and type \verb'M-k' twice.) 

\begin{verbatim}
tfunc: int -> (unit -> unit) -> unit
\end{verbatim}

Sets the break function for the breakpoint at a given time.

\begin{verbatim}
nofunc: int -> unit
\end{verbatim}

Resets the break function for the breakpoint at a given time.

\begin{verbatim}
C-c C-c
\end{verbatim}

Halt forward execution of the current program.


\section{Event Types}
This section describes the event types may be displayed 
inside the event cursor; an example is given for the common event types.

Note that, although all these types of events are produced internally, they
are not all visible to \verb'M-n', \verb'M-p', and \verb'M-e'.  When a set of
several events is very closely related (generally when the execution of 
one necessarily implies the execution of the others), only the last event
(in execution order) is normally visible.  (The others may appear as 
binding sites for identifiers when \verb'M-l' is used.)  However, enough 
events will always be visible to allow you to set a breakpoint to 
examine any identifier value of interest. For example, in the code

\begin{verbatim}
let val a = 5 [VAL]
    val b = 6 [VAL]
in f [APP] (a+b)
end
\end{verbatim}

only the APP event will be visible, and you can examine 
the value of \verb'a' or \verb'b' by breaking there; you cannot break
at either VAL event.
 
A further note: displaying event markers in the text at the correct spots
is complicated by the presence of derived forms; we try to do the best
we can in these cases, but there may still be some confusions.

The most common events are as follows:
\verb'APP': Application event.  Occurs just before a function is applied to
its argument; appears between the function and its argument.  When an
infixed operator is applied, the event appears immediately following
the operator.

\begin{verbatim}
let val a = myfun [APP] 7 in ...

infix myOp
let val b = 2 myOp [APP] 8 in ...
       
\end{verbatim}

\vspace{0.25in}
\verb'VAL': Value binding event.  Occurs just after a value has been bound.

\begin{verbatim}
let val a = f(7) [VAL] in ...
\end{verbatim}

\verb'VALREC': Recursive value binding event.  Analogous to VAL.

\begin{verbatim}
let val rec f = fn x => if x = 0 then 1 else x * (f(x-1)) [VALREC] in ...
\end{verbatim}

\verb'FN': Function entry event.  Occurs just after a function has been
entered.  If a function's argument consists of several different
patterns, there will be a FN event for each.

\begin{verbatim}
fun f (SOME x) = [FN] g x
  | f NONE = [FN] 0
\end{verbatim}

\verb'CASE': Case entry event; analogous to FN.

\begin{verbatim}
case d of SOME x => [CASE] g x
        | NONE => [CASE] 0
\end{verbatim}


\verb'RAISE': Exception raising event.  Occurs just before an exception is
raised.  You may notice RAISE events at various unexpected points in
your code; RAISE events are generated implicitly whenever a pattern
match may fail.  In the current implementation, RAISE events are
sometimes generated even when they don't need to be there.

\begin{verbatim}
case v of SOME x => g x
        | NONE => raise [RAISE] MyException
\end{verbatim}

\verb'HANDLE': Exception handling event.  Occurs when an exception is
handled, before the exception handling code is actually executed.

\begin{verbatim}
fun myHd x = SOME (hd x) handle Hd [HANDLE] => NONE
\end{verbatim}

\verb'START': Start of program execution.

\verb'END': End of program execution.

The following types of events are normally not visible:

\verb'LET': LET body event.  Occurs after a set of LET-bound declarations
but before the expression body has been evaluated; in other words at the 
point where the \verb'IN' keyword appears in the program text.

\verb'LOCAL': Occurs before a set of local declarations are evaluated.

\verb'LOCAL IN': Occurs after a set a local declarations are evaluated,
but before the set of declarations which contain the local
declarations in their score are evaluated.  In other words, occurs at
the point where the \verb'IN' keyword appears in the program text.

\verb'LOCAL END': Occurs at the end of a local block.

The remaining types of events are encountered less often, since code
that contains these events is usually not executed in active debug mode.

\verb'STRUCTURE': Structure binding event.

\verb'ABSTRACTION': Analagous to STRUCTURE.

\verb'FUNCTOR': Functor binding event.

\verb'FUNCTOR ENTRY': Occurs upon entry to a functor.

\verb'FUNCTOR APP': Occurs just before a functor is applied to its
argument.

\verb'STRUCTURE END': Occurs at the end of the declarations in a
structure.

\verb'STRUCTURE VAR': Occurs just before a structure variable is used.

\verb'OPEN': Occurs just before a structure is opened.

\section{Environmental Issues}
``Debuggable'' code (i.e., code compiled via  \verb'usedbg', \verb'run', etc.)
uses a special version of the pervasive environment.  In particular, it 
redefines:
\begin{itemize}
\item the REF and ARRAY structures to support checkpointing of the mutable
store;
\item the IO structure to support logging of IO events;
\item various other structures that use REF, ARRAY, or IO; and
\item the LIST structure, for reasons explained below.
\end{itemize}
These redefinitions should be transparent to you, unless, of course, you
rebind these structures yourself. (Hint: Don't).  The debugger also makes
and uses top-level definitions of two structures named DEBUGN and DEBUGZ,
so your code shouldn't redefine these at top level.

The debugger will almost certainly not work correctly on code that performs its
own \verb'callcc's.

It is possible to mix non-debuggable with debuggable code under the
following (rather heavy) restrictions:
\begin{enumerate}
  \item The non-debuggable code must not update any arrays or refs created by
debuggable code, and must not create arrays or refs which are then passed
to debuggable code.  (N.B. There are many ways to pass an object, including:
as an argument to a function, as the value of a ref or array, or as the
argument to an exception constructor.)
  \item The non-debuggable code should not apply debuggable functions. 
\end{enumerate}
Violating restriction 2 will probably not cause the debugger to crash, but
it may make it produce misleading information.  A special pre-instrumented 
version of the LIST structure is provided because it would otherwise violate
this restriction; execution within function in this structure can be traced
in the normal manner, except that the source code is not available for display.

\section{Performance}
Because it expands the effective size of the source code by instrumentation,
the debugger can put heavy demands on system resources.  Compilation times,
in particular, will suffer a many-fold increase, particularly in systems
where memory is tight.  One partial solution is to use interpreted execution 
mode where possible.

The I/O and reference memory logs can also occupy a great deal of memory. In
particular, remember that every byte of input read (from a terminal or
a file, in the present implementation) is stored in the I/O log and 
becomes a permanent part of the program's live data.

Execution times for debuggable code should normally be 2-4 times slower than
ordinary code.  Heavy use of I/O and mutable store (particularly the creation
of references and arrays) may lead to poorer performance.

\section{The Future}
In addition to fixing some of the problems mentioned in this document
(as well as the problems we don't know about yet) we plan the following
major extensions to the debugger's functionality:
\begin{itemize}
  \item We will shortly be able to evaluate arbitrary ML expressions
in the context of the current time.  This will make it much easier to examine
complicated data structures, and will permit conditional breakpoints. 
  \item It will be possible to change the values of ref and array variables, 
by evaluating assignment expressions.
  \item There will be proper support for integrating debuggable code with 
non-debuggable code in a single execution.
  \item We are investigating integration with the separate compilation 
mechanism.
  \item We'll support more efficient logging of input from files, and more
efficient logging of the mutable store using a new multi-generational garbage
collector.
\end{itemize}

\begin{thebibliography}{9}
\bibitem{us} A.P.Tolmach, and A.W.Appel, ``Debugging Standard ML Without
Reverse Engineering,'' {\em Proc. 1990 ACM Conference on Lisp and Functional
Programming}, June, 1990.
\end{thebibliography}

\end{document}


