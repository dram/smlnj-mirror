\documentstyle [11pt]{article}
\setlength{\textheight}{8.8in}
\setlength{\textwidth}{6.5in}
\setlength{\oddsidemargin}{0in}
\setlength{\topmargin}{-.1in}
\setlength{\headheight}{0in}
\setlength{\headsep}{0in}

\title{Visual Debugging for Standard ML of New Jersey}
\author{Adam T. Dingle \\ Princeton University}
\begin{document}
\maketitle
\section{Introduction}
I have implemented an Emacs mode in which SML-NJ programs can be
debugged in a visual way.  Events are displayed within the source text
itself; an ML program can be single-stepped, breakpoints can be set,
variables' values can be displayed and the call traceback can be
examined by typing key sequences in the source text window.

My implementation extends Lars Bo Nielsen's Emacs mode for
editing SML programs, which is now part of the GNU Emacs distribution.

The system is new, and this documentation is preliminary; your
comments and suggestions are welcomed.

\section{Setting Up}
To set up, you must tell Emacs where the sml-mode Lisp code resides.
Add the following lines to your .emacs file (located in your home
directory).  (If you don't have an .emacs file, create one with these
lines in it.)

\begin{verbatim}
;; sml-mode initializations

; /u/atd/lib/emacs is where sml-mode.el (and sml-debug.el) live.
(setq load-path (cons "/u/atd/lib/emacs" load-path))

; Emacs should automatically enter sml-mode for any buffer whose
; filename ends with ".sml".
(setq auto-mode-alist (cons '("\\.sml$" . sml-mode) auto-mode-alist)) 

; sml-mode.el should be automatically loaded when the commands
; "sml-mode" or "sml-shell" are invoked.
(autoload 'sml-mode "sml-mode" "Major mode for editing SML." t)
(autoload 'sml-shell "sml-mode" "Inferior shell invoking SML." t)

; Use "sml" as an abbreviation for "sml-shell", which is too long.
(defun sml () (interactive) (sml-shell))

;; sml-debug initializations

; Until the version of sml that supports visual debugging is
; merged with the sml you usually run, use this line to tell
; Emacs where the debugging version is located.
(setq sml-prog-name "/u/atd/t60/src/smld")	; version with debugging

(setq sml-mode-hook 'sml-debug-mode)
(setq sml-shell-hook 'sml-debug-shell)
(autoload 'sml-debug-mode "sml-debug")
(autoload 'sml-debug-shell "sml-debug")

\end{verbatim}

\section{Getting Started}
From within Emacs, type the command \verb'M-x sml' to start up a
window with SML running in it.  In that window, type
\verb'usedbg "<filename>"' to compile and execute the contents of a
file in passive debug mode.  Typically, the file will contain function
definitions.  Type \verb'run "<command>"' to compile and execute a
given command in active debug mode, which will allow you to step
through the execution of the command in a visual way.  Typically, the
command will call one of the functions that was loaded using the
\verb'usedbg' command.

In response to your \verb'run' command, an Emacs window named
\verb'*sml-debug-command*' will be created containing the command to
be run.  The first {\em event} will be displayed in brackets in that
window; it will look something like

\begin{verbatim}
[START:4]
\end{verbatim}
where START indicates the type of the event and 4 is the current
time.  (For various technical reasons time does not begin at 1.)

Events are places in the ML program where you can stop execution to
look at variable values.  You can use the Emacs key sequence \verb'M-s' to
{\em step} (execute for a short time) until the next event is reached.
(\verb'M-s', pronounced ``meta-s'', means that you should hold down
the \verb'META' key (often labeled \verb'ALT') and press the \verb's'
key.  If your keyboard does not have a \verb'META' key,
type \verb'ESC' followed by \verb's'.  For more information, see an
Emacs reference manual, or type \verb'C-h i' for information while in Emacs.)
Try it.  Try stepping a number of times; see what different types of
events you reach.

Remarkably, the SML-NJ debugger also permits you to step {\em}
backward in time.  The key sequence \verb'C-M-s' steps backward one unit in
time.  Try stepping backward to the beginning of your program.

You can also jump to arbitrary times in the program's execution: type
\verb'M-<time> M-t' (that is,
hold down the \verb'META' key while you type the time number,
then type \verb't' while still holding down \verb'META'.  If your
keyboard does not have a META key, type \verb'C-u <time> ESC t'.)

You may occasionally notice that a step command moves for 2 time
units, or that the \verb'M-t' command brings you to a time that is one
unit away from the time you requested.  This happens because certain
unimportant internal events (i.e. I/O events) are assigned time
numbers but are ignored by the debugger.  This inconsistency in time
numbering is a bug and will be fixed soon.

\section{Selecting Events}

You can use the commands \verb'M-n' and \verb'M-p'
to browse through events in your ML
program.  \verb'M-n' moves the {\em event cursor} to the next event in the
source code; \verb'M-p' moves to the previous event.  These commands do not
cause any part of your program to be executed, or change the current
time; they are simply a way of looking at the events that exist.  The
position of the event cursor is denoted by an event surrounded by
square brackets; when you move the event cursor away from the current
event, the current event remains visible, but is surrounded by a pair
of angle brackets (\verb'<' \verb'>') to distinguish it from the event
cursor.  Typing \verb'M-c' will move the event cursor back to the
current event.

Sometimes you may wish to move the event cursor to a point that is far
away in your program text, or in a different source file than the one
in which the event cursor currently rests.  To do so, move the Emacs
cursor to the area of source code to which you wish to move the
event cursor, then
type the command \verb'M-e'.  The event cursor will jump to the event nearest
to the Emacs cursor; you can then use the \verb'M-n' and
\verb'M-p' commands to
browse the events in that area.

If you are running Emacs under X, you can select events even more
easily.  Point to a character in your source code that is near the
event that you would like to select, then hold down META and press the
middle pointer button.

\section{Breakpoints}

To set a breakpoint at an event, move the event cursor to it and type
the command \verb'M-k'.  The string \verb'bk:' will appear inside the event,
indicating that a breakpoint is set at it.  The event will continue to
be displayed even if you move the event cursor away from it.  Type \verb'M-k'
again to remove the breakpoint at the event.

Use the \verb'C-M-f' command to continue execution until the next breakpoint
is reached.  The \verb'C-M-b' command moves backward in time until a
breakpoint is reached.

You can also set breakpoints at particular times: type
\verb'M-<time> M-k'
(that is, hold down the \verb'META' key while you type
the time number, then type \verb'k' while still holding down \verb'META').
As the \verb'M-t' command can be used to jump to a given time, time
breakpoints are really only useful in conjunction with event
breakpoints.  Repeating the same command will delete the breakpoint
you just created.  Type \verb'C-M-k' to get a list of time breakpoints.

The \verb'M-- M-k' command deletes all existing breakpoints.
(\verb'M--' is ``meta-hyphen''.)

You can associate an ML function with a breakpoint; the function is
called whenever the breakpoint is reached.  The function might print
out the value of one or more variables.  To associate a function with
a breakpoint at a given event, move the event cursor to the event and
type \verb'breakFunc <function>;' in the SML process window.  The
function should have type \verb'unit -> unit'.

For example, you might type

\begin{verbatim}
- breakFunc (fn () => showVal "a");
\end{verbatim}

To reset the function associated with a breakpoint at an event, move
the event cursor to the event and type \verb'M-k M-k' (removing and
reinstalling the breakpoint at the event).

To associate a function with a breakpoint at a given time, type

\begin{verbatim}
- breakTimeFunc <time> <function>;
\end{verbatim}
To reset the function associated with a time breakpoint, type
\begin{verbatim}
- nofunc <time>;
\end{verbatim}

Soon, you will be able to evaluate arbitrary ML expressions in the
debugger; then, you will be able to write break functions that choose
to continue execution (using forward()) if a given ML expression is
satisfied.  In this way, you will be able to make breakpoints conditional.

\section{Examining variables and the call traceback}

Type \verb'M-l' to examine the value of any variable that is currently
in scope.  Enter the variable's name in the Emacs minibuffer, and its
value will be printed out in the SML process window.  The event at which the
variable's value was bound will become the selected event.  You may notice
that if you
move the event cursor away from the event and then return to it, you might
not be able to find the event again;
this is because not all events are normally
displayed (see the introduction to section ~ref{event types} for details).

If you are running Emacs under X, you can examine a variable by
pointing to its name, holding down META and pressing the first mouse button.

The {\em backtrace cursor} is used to examine the chain of function
calls that led to the point where execution is currently stopped.
\verb'M-u' moves the backtrace cursor, which initially rests on the current
event, up one level in the function call chain.  The
backtrace cursor contains the string \verb':bt' inside it, and also
displays the time at which the corresponding function call occurred.
The command \verb'M-d' moves the backtrace cursor down one level in
the function call chain.  As a convenience, typing \verb'M-t' while a
backtrace event is selected will jump to the time of the backtrace event.

The command \verb'C-M-l' can be used to examine the value of a
variable at the scope and time of the current backtrace event.  Under
X, you can alternatively point to the name of the variable and press
the first mouse button while holding down CTRL and META.

\section{Finishing Up}

The \verb'C-c C-d' command is used to exit an active debugging session.  It
aborts execution of the command
being run, removes all event labels from source code windows and deletes
the \verb'*sml-debug-command*' buffer.  The window containing the SML
process will remain, so that you can load new code or specify another
command to run.

While source code windows have events displayed in them, they are
placed into Label mode, a read-only Emacs minor mode.  This prevents
you from editing them and unintentionally saving event labels into
your source files.  Before you edit a section of source code that is
being debugged, you should use the \verb'C-c C-d' command to complete
the debugging session and remove the labels.  If for some reason a
window remains in Label mode after you have exited the debugging
session (this should not happen) you can use the command \verb'C-x
C-q' (exit-label-mode) to remove the labels from the window.

\section{Command and Function Summary}
The following commands are available within the window containing the
sml process as well as within windows containing source code.
\verb'<left>' and \verb'<middle>' refer to the left and middle pointer
buttons when running under the X window system.

\begin{center}
\begin{tabular}{|ll|}    \hline
Command & Description \\ \hline
\verb'M-s'             & single-step \\
\verb'C-M-s'           & single-step backwards \\
\verb'M-<time> M-t'    & go to specified time \\
\verb'M-n'             & select next event \\
\verb'M-p'             & select previous event \\
\verb'M-c'             & move event cursor to current event \\
\verb'M-e'             & select event near cursor (source windows only) \\
\verb'M-<middle>'      & select event near pointer (source windows only) \\
\verb'M-k'             & toggle breakpoint at selected event \\
\verb'M-<time> M-k'    & toggle breakpoint at given time \\
\verb'C-M-k'           & list breakpoint times \\
\verb'M-- M-k'         & delete all breakpoints \\
\verb'C-M-f'           & execute forward to next breakpoint \\
\verb'C-M-b'           & execute backward to previous breakpoint \\
\verb'M-u'             & move backtrace cursor up the call chain \\
\verb'M-d'             & move backtrace cursor down the call chain \\
\verb'M-t'             & go to time of selected backtrace event \\
\verb'M-l or M-<left>'    & show variable value and binding event \\
\verb'C-M-l or C-M-<left>' & show variable value and binding event at \\ 
                           & scope and time of backtrace event
\verb'C-c C-d'         & exit debugging session, aborting command execution \\
\hline
\end{tabular}
\end{center}

The following ML functions are available and useful.

\begin{verbatim}
bfunc: (unit -> unit) -> unit
\end{verbatim}

Sets the break function for the breakpoint at the selected event.  (To
reset the break function, select the event and type \verb'M-k' twice.) 

\begin{verbatim}
tfunc: int -> (unit -> unit) -> unit
\end{verbatim}

Sets the break function for the breakpoint at a given time.

\begin{verbatim}
nofunc: int -> unit
\end{verbatim}

Resets the break function for the breakpoint at a given time.

\section{Event Types}
The following event types may be displayed inside the event cursor; an
example is given for each event type.

Note that not every possible event exists in the program; often,
several closely related events are combined into one event when no
additional information would be obtainable by having several different
events.  For example, in the code

\begin{verbatim}
let val a = 5
    val b = 6
    val c = 8
[VAL] in ...
\end{verbatim}

only one VAL event is placed, not three; by stopping at the single VAL
event, you will be able to examine any of the three declared values.

\vspace{0.25in}
\verb'VAL': Value binding event.  Occurs just after a value has been bound.

\begin{verbatim}
let val a = f(7) [VAL] in ...
\end{verbatim}

\verb'VALREC': Recursive value binding event.  Analagous to VAL.

\begin{verbatim}
let val rec f = fn x => if x = 0 then 1 else x * (f(x-1)) [VALREC] in ...
\end{verbatim}

\verb'FN': Function entry event.  Occurs just after a function has been
entered.  If a function's argument consists of several different
patterns, there will be a FN event for each.

\begin{verbatim}
fun f (SOME x) = [FN] g x
  | f NONE = [FN] 0
\end{verbatim}

\verb'CASE': Case entry event; analogous to FN.

\begin{verbatim}
case d of SOME x => [CASE] g x
        | NONE => [CASE] 0
\end{verbatim}

\verb'APP': Application event.  Occurs just before a function is applied to
its argument; appears between the function and its argument.  When an
infixed operator is applied, the event appears immediately following
the operator.

\begin{verbatim}
let val a = myfun [APP] 7 in ...

infix myOp
let val b = 2 myOp [APP] 8 in ...
       
\end{verbatim}

\verb'RAISE': Exception raising event.  Occurs just before an exception is
raised.  You may notice RAISE events at various unexpected points in
your code; RAISE events are generated implicitly whenever a pattern
match may fail.  In the current implementation, RAISE events are
sometimes generated even when they don't need to be there.

\begin{verbatim}
case v of SOME x => g x
        | NONE => raise [RAISE] MyException
\end{verbatim}

\verb'HANDLE': Exception handling event.  Occurs when an exception is
handled, before the exception handling code is actually executed.

\begin{verbatim}
fun myHd x = SOME (hd x) handle Hd [HANDLE] => NONE
\end{verbatim}

\verb'STRUCTURE', \verb'ABSTRACTION', \verb'FUNCTOR',
\verb'FUNCTOR ENTRY',
\verb'FUNCTOR APP', \verb'STRUCTURE END', \verb'STRUCTURE VAR',
\verb'OPEN', \verb'LOCAL', \verb'LOCAL IN', \verb'LOCAL END' events:
Straightforward, and less often encountered than the preceding events.
To be documented later.

\end{document}
