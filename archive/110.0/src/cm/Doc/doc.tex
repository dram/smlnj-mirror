\documentstyle{article}

\author{Matthias Blume\\
Department of Computer Science\\
Princeton University}

\title{{\bf CM}\\
A Compilation Manager for SML/NJ\\
{\em User Manual}}

\begin{document}

\maketitle

\section{Overview}

An overview.

\section{Getting started}

\subsection{A rudimentary entity description}

For the casual user it is very easy to take advantage of {\em CM}.  Given a
collection of {\sc Sml} source files one just needs to specify their names
in a so-called {\em entity description file}.  For example, an entity
comprised of files {\tt a.sml}, {\tt a.sig}, and {\tt main.sml} would be
specified as

\begin{verbatim}
Group is
      a.sml
      a.sig
      main.sml
\end{verbatim}

{\em CM}\/'s default behavior is to look for the entity description in a
file named {\tt sources.cm}.  While it is possible to override the default
it seems to be convenient to take advantage of its existence.  Filenames
within {\tt sources.cm} (or any other entity description file) are either
absolute filenames or filenames {\em relative to the directory the
description file appears in}.

\subsection{Admissible input language}

Normally --- without {\em CM} --- the programmer would have to remember
which order these files have to be processed in when loading them into an
interactive {\em SML/NJ} session.  The order clearly depends upon the
contents of these files, in particular upon the interrelationship between
{\em imported} and {\em exported} identifiers.  With {\em CM} this is no
longer necessary, since it automatically determines the correct ordering
among source files.

To make it possible to find a feasible ordering within a reasonable amount
of time and to make sure that the resulting meaning of the program is the
one the programmer had in mind there must be some restrictions placed on
the admissible input language.

\begin{enumerate}
\item
In each entity\footnote{An entity is the part of the program described by
one particular entity description file.} and for each top-level identifier
there can be at most one file in the entity containing definition's for
this identifier.  However, there can be multiple definitions of the same
name within the same file, because only the last one will count as a
top-level definition.
\item
No source file can contain an {\bf open} declaration at top-level.  This is
an important new restriction, which wasn't present before.  However, it
must be noted that its absence in CM's predecessors is to be considered a
bug.
\item
The program must not depend on the ordering of top-level side-effects,
i.e. those which occur while the source files are being executed.  Note,
that this restriction is not automatically enforced by {\em CM}, while the
other ones are.
\item
{\em CM} only tracks {\bf signature}-, {\bf structure}-, {\bf functor}-,
and {\bf funsig}-declarations.  This is not really a restriction, but one
has to be aware of the fact that {\em CM} will miss dependencies created
due to references to top-level {\bf val}-bindings etc.
\end{enumerate}

\subsection{Re-compilation and {\em make}}

Once the entity description file has been set up and the sources are in
place you must start an interactive {\em SML/NJ} session with {\em CM}
loaded.  Usually this can be done by typing

\begin{verbatim}
sml-cm
\end{verbatim}

at the operating system's prompt.  This starts an ordinary interactive {\em
SML/NJ} session with three additional predefined structures --- {\bf
structure CM}, {\bf structure CMB}, and {\bf structure CMR}.  Unless you
are modifying {\em SML/NJ}\/'s compiler sources you can focus on {\bf
structure CM}.

If your entity description file is named {\tt sources.cm}, then you
can simply type

\begin{verbatim}
CM.make ();
\end{verbatim}

This will analyze your program and its interdependencies, determine an
ordering among the various compile- and load-steps, and carry them out if
necessary.  Provided there haven't been any errors or exceptions {\em CM}
will introduce new top-level bindings afterwards.

A similar command is

\begin{verbatim}
CM.recompile ();
\end{verbatim}

It performs the same analysis- and re-compilation-steps but avoids
executing any of the compiled code.  The interactive top-level environment
remains unaffected by {\tt CM.recompile ();}.

\subsection{Transient and persistent by-products of the compilation}

The most important by-products of compilation under {\em CM} are {\em
object files}.  These are usually referred to as {\em bin-files} and
contain machine code for the native architecture along with the
accompanying information about changes to the static environment.

In subsequent runs of {\tt CM.make ();} --- even after shutting down and
restarting {\tt sml-cm} --- the compilation steps for source files which
have a valid bin-file will be skipped.  Moreover, if you don't leave {\em
CM}, then it won't even be necessary to re-load bin-files from the ambient
file system.  {\em CM} uses an in-core cache for compiled units to avoid
unnecessary input/output traffic.

Bin-Files are stored in a sub-directory of the source file's directory.
The name of this sub-directory is derived from the native machine
architecture.  For example, on an Alpha workstation {\em CM} will store
your bin-files in a directory called {\tt .alpha32}.

Other by-products of {\em CM}\/'s work are the so-called {\em decl-files}.
If nothing is known about a source file, then {\em CM} needs to parse it
and extract all information relevant to dependency analysis.  The so
obtained information is cached both in-core and in the file system to avoid
repeatedly parsing the same file.  Decl-files are stored in a subdirectory
called {\tt .depend}.

\section{Concepts}

{\em CM} is built around the concept of an {\em entity}.  Entities consist
of collections of source files and are roughly divided into {\em groups}
and {\em libraries}.  An associated {\em imports-from}---{\em exports-to}
relation among entities arranges them into a {\em directed acyclic graph
(DAG)}.  This relation is given {\em explicitely} by {\em entity
descriptions}.

{\em CM}\/'s task is to determine the per-source-file import--export
relation. In doing so it takes advantage of the given per-entity relation.

\subsection{Properties}

But let's start by looking at the properties, which can be used to
distinguish between different kinds of entities:

\begin{description}

\item[transparent] Definitions exported by a {\em transparent} entity are
automatically re-exported by entities importing them, unless an explicit
{\em export filter} prevents this.  Under certain conditions symbols
defined by transparent entities are ultimately visible at the interactive
top-level.

\item[opaque] No definitions of an {\em opaque} entity will become visible
at the interactive top-level.  Entities importing names from opque entities
will not automatically re-export them. However, by including some or all of
those names into the export filter of the importing entity one can
explicitely force the re-export to take place. \\
Entities are either {\em transparent} or opaque.

\item[export filter] An explicitely given export filter limits the set of
names exported by an entity.  If the export filter mentions a name which
has been imported from another entity, then this name will be re-exported
regardless of whether the other entity was {\em transparent} or {\em
opaque} (see discussion below).

\item[load-on-demand] Modules from {\em load-on-demand} entities are only
incorporated into the final software system, if other parts of this system
refer to them.

\item[root] Some of the other properties do not mix well with an entity
being used as the root of the dependency hierarchy.  In particular, opaque
and load-on-demand entities\footnote{We will see that these two are really
the same.} can't be used purposefully in this context, because
load-on-demand will prevent loading anything at all while opaqueness will
prevent making anything at all visible.\footnote{As a matter of fact, in
{\em CM} you can use any entity as root entity: Under these special
circumstances opaque or load-on-demand entities will be treated as being
transparent and {\em load-always}.}

\item[stability] Development on a {\em stable} entity has come to a
(preliminary) conclusion --- the dependencies among sources of a stable
entity are fixed, and the binaries do not change anymore.  This can be used
to optimize {\em CM}\/'s work.

\item[no sources] Provided the binary file and other supporting information
is in place and up-to-date {\em CM} will not need to consult the contents
of certain source files.  In fact, such source files do not even need to
exist.  A special common case where this is guaranteed to be true is a {\em
stable} entity.  Source files can be deleted to save disk space.

\end{description}

\subsection{Discussion}

\subsubsection*{Symbol visibility}

The description of {\em transparent} entities above purposefully omitted
the exact set of conditions, which must be met in order for a symbol's
definition to become visible at the interactive top-level.  When we just
look at those rules alone, then they might seem to be chosen somewhat
ad-hoc.  Therefore, let's discuss the intention which led us to chosing
them before we are actually going to write them down.

The purpose of every entity is to provide definitions of certain symbols to
its {\em clients}.  A client is any direct ancestor of the given entity in
the per-entity import--export relation.  The difference between different
kinds of entities are largely determined by the way imported symbols are
treated within clients.  Definitions of {\em transparent} entities will
normally be re-exported by the client, while definitions of {\em opaque}
entities generally will not.  {\em Export filters} can be used to modify
this behavior at a symbol-by-symbol level.

In order for a symbol to become visible at the top-level it must be
re-exported along {\em some} path from the point if its definition all the
way up to the root of the hierarchy.  Therefore, the general rule is: there
must be at least one path from the definition of the symbol to the root of
the hiearchy within the DAG defined by the per-entity import--export
relation, which satisfies:
\begin{itemize}
\item every entity on this path must export this symbol;
\item clients of opaque entities on the path\footnote{this means that both
the opaque entity and the client are on the path} export the symbol {\em
explicitely}, i.e. by including it into their export filters; and
\item no entity along this path exports a different definition of the same
symbol
\end{itemize}

\subsubsection*{Interconnections between properties}

As hinted earlier it turns out that some of the properties must be treated
as a unit rather than separately.

{\em Opaque} entities should be {\em load-on-demand}, because otherwise we
will waste precious system resources by loading modules, which can never be
referenced.  It could be argued that garbage collection can take care of
this problem, but the internal caching of {\em CM} will pose as an obstacle
to the garbage collector.  Also, in addition to taking up memory resources
one must consider the time spent compiling and loading such modules.

Conversely, {\em load-on-demand} entities should be {\em opaque}.  If they
were {\em transparent}, then we would be confronted with the question of
whether only ``demanded'' parts or everything will become visible at the
top-level.  In the former case the set of new top-level defintions would
depend on the particular set of demands created by the software system,
which in turn change with changes in the implementation.  In the latter
case we can't avoid loading {\em everything}, because we can't predict
which defintions will be used interactively.

The {\em root} property is similarly linked to both {\em opaqueness} and
{\em load-on-demand}.  This has already been explained in the previous
section.

On the other hand, {\em export filters}, {\em stability}, and the ability
to overcome the {\em absence of sources} are relatively independend from
the above and from each other.

In conclusion we use the term {\em group} to denote {\em
transparent} entities, the members of which are {\em always
loaded}, while {\em opaque}, {\em load-on-demand} entities
are called {\em libraries}.  Both groups and libraries can
appear as the root of the dependency hierarchy, but
libraries will be treated as if they were groups in this case.

Source files from both groups and libraries can be missing,
in particular if the entity is stable.  Groups {\em can} and
libraries {\em must} have an explicit export filter.

\section{Entity and their description files}

An entity consists of a set of sources and a set of {\em
sub-entities}.  The entity will be a client in the sense of
the previous section's discussion of precisely those
sub-entities.  Names exported by them can be referred to within
sources of the entity.  It is necessary to require that

\begin{itemize}
\item the sets of exported names of the sub-entities are disjoint
\item the sets of top-level names defined by each source are disjoint
\end{itemize}

Note that there {\em can} be multiple definitions of the same symbol, but
they are confined to the case where a definition imported from a sub-entity
is overridden within the client.

Most sources maintained by {\em CM} will be written in {\sc Sml}, but
sometimes we want to use program generators and other tools to compute the
actual {\sc Sml} source from a source written in a different language.
{\em CM}\/'s built-in dependency analyzer only understands {\sc Sml}, but
it also provides hooks to run the appropriate tool should the {\sc Sml}
source not be up-to-date.

Currently there is support for the tools {\em ML-Yacc} --- a parser
generator, {\em ML-Lex} --- a generator for lexical analyzers, and {\em
ML-Burg} --- a code-generator-generator.

Within entity description files one can associate source files with tools,
but there are also suitable defaults which allow {\em CM} to infer the tool
from the name of the source file.  This means that explicitly specified
tool names will only occasionally be necessary.

Sub-entities are specified by simply including the name of their respective
description file into the set of sources.  A special built-in tool takes
care of treating those files separately.\footnote{There are really {\em
three} of those special tools, two of which have been included to provide
some kind of backward compatibility to {\em SC}, which is {\em CM}\/'s
precursor.}

\end{document}
