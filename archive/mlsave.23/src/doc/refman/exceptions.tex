\chapter{Exceptions}
\label{exception}
The rough and ready rule for understanding how exceptions are handled
is as follows.  If an exception is raised by a \verb"raise"
expression
\begin{quote}
\verb"raise" $E$ \verb"(" {\it exp} \verb")"
\end{quote}
which lies in the textual scope of a declaration of the exception
constructor $E$, then it may be handled by a handling rule
\begin{quote}
\verb"handle" $E$ \verb"(" pat \verb") =>" {\it exp'}
\end{quote}
but only if this handler is in the textual scope of the same
declaration.

Any exception, regardless of scope, is handled by a wildcard or
variable pattern, as
\begin{quote}
\verb"handle _ =>" {\it exp'}
\end{quote}
This rule is perfectly adequate for exceptions declared at top level;
some examples in Section~\ref{pathexn} below illustrate what may occur in other
cases.

\section{An example}
To illustrate the generality of exception handling,  suppose we have
declared some exceptions as follows:
\begin{verbatim}
exception    Oddlist of int    and   Oddstring of string
\end{verbatim}
and that a certain expression  exp:int  may raise either of these
exceptions and also runs the risk of dividing by zero.  The handler
in the following \verb"handle" expression would deal with these
expressions:
\begin{verbatim}
exp handle Oddlist []  => 0
         | Oddlist [x] => 2*x
         | Oddlist (x::y::_) => x div y
         | Oddstring "" => 0
         | Oddstring s => size(s)-1
         | Div => 10000
\end{verbatim}
Note that the whole expression is well-typed because in each handling
rule the type of the match-pattern is \verb"exn", and because the
result type of the match is \verb"int", the same as the type of exp.

Note also that the last handling rule will handle \verb"Div"
exceptions raised by exp, but will not handle the \verb"Div"
exception that may be raised by \verb"x div y" in the third handling
rule.  Finally, note that a universal handling rule
\begin{verbatim}
         | _ => 50000
\end{verbatim}
at the end would deal with all other exceptions raised by  exp.

\section{Exception constructors}
For an exception constructor E, the expression
\begin{quote}
$E$ \verb"(" {\it exp} \verb")"
\end{quote}
evaluates the expression {\it exp}, producing value $v$,
and then applies the constructor $E$ to it, yielding the value
$E(v)$, whose type is \verb"exn".

The \verb"raise" keyword may be applied to any expression of type
\verb"exn", and has the effect of ``raising'' that exception value.
The innermost (dynamically) enclosing expression $e = e_1
\verb" handle " m_1$ is found; all further evaluation of the
expression $e_1$ (and its subphrases) is aborted; and the match $m_1$
is applied to the exception value, yielding  the result of the
expression $e$.  

If the match in a handler fails, then the exception value is
\label{reraise}
re-raised, and another enclosing handler is found.

Exception constructors may be nullary (have no associated value), in
which case the {\it exp} and {\it pat} in the previous discussion are
omitted.

Exceptions may be constructed independently of raising them:
\begin{verbatim}
  exception A of int
  val e = A 6
  val x = raise e
\end{verbatim}
Handlers may be abstracted from the \verb"handle" keyword:
\begin{verbatim}
   val h = fn E 0 => "zero"
            | E _ => "nonzero"
            | v => raise v

   f(x)  handle  e => h e
\end{verbatim}
Note that it is advisable in this case to have a default clause in
the function \verb"h", since the default for a \verb"handle" match
(re-raising the exception) is different from the default for a \verb"fn" or
\verb"case" match (raising the \verb"Match" exception).

The ordinary wildcard pattern 
\verb"_" will handle any exception when it is used in a pattern, as
will any pattern consisting solely of a variable.  These should be
used with some care, bearing in mind that they will even handle
interrupts.

Nullary exception names, when misspelled, appear to the compiler to be
variables; these will then match any exception.  For this reason we
recommend the convention that exception names (and other
constructors) be written beginning with an uppercase character, and
variables be written beginning with a lowercase character.  The
compiler may remind the programmer of this convention when it is
violated.

\section{Some pathological examples}
\label{pathexn}
We now consider some possible misuses of exception handling, which
may arise from the fact that exception declarations have scope, and
that each evaluation of a generative exception binding creates a
distinct exception.  Consider a simple example:
\begin{verbatim}
exception  E of bool
fun f(x) = 
     let exception E of int
      in if x > 100 then raise E(x) else x+1
     end
val z = f(200) handle E(true) => 500 | E(false) => 1000
\end{verbatim}
The program is  well-typed, but useless.  The exception bound to the
outer \verb"E" is distinct from that bound to the inner \verb"E";
thus the exception raised by \verb"f(200)", with excepted value 200,
could only be handled by a handler within the scope of the inner
exception declaration---it will not be handled by the handler in the
program, which expects a boolean value.  So this exception will be
reported at top level.  This would apply even if the outer exception
declaration were also of type int; the two exceptions bound to
\verb"E" would be distinct.

On the other hand, if the last line of the program is changed to
\begin{verbatim}
f(200) handle _ => 500
\end{verbatim}
then the exception will be caught, and the value 500 returned.  A
universal handling rule (i.e. \verb"_" or a variable-identifier)
catches any exception---even one exported from the scope of the
declaration of the associated exception name---but cannot examine the
excepted value carried by the exception constructor, since the type
of this value cannot be statically determined.

Even a single textual exception binding---if for example it is
declared within a recursively defined function---may bind distinct
exceptions to the same identifier.  Consider another useless program:
\begin{verbatim}
fun f(x) =
     let exception E
      in if p(x) then a(x)
                 else if q(x) then f(b(x)) handle E => c(x)
                              else raise E
     end
val z = f v
\end{verbatim}
Now if p(v) is false but q(v) is true, the recursive call will
evaluate f(b(v)).  Then if both p(b(v) and q(b(v)) are false, this
evaluation will raise \verb"E".  But this exception will not be
handled, since the exception raised is that which is bound to
\verb"E" by the inner---not outer---evaluation of the exception
declaration.

These pathological examples should not leave the impression that
exceptions are hard to use or to understand.  The rough and ready
rule of Section~\ref{exception} will almost always  give the correct
understanding.
