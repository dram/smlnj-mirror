\chapter{Derived forms}
\label{derived}
ML is equipped with a number of {\em derived forms}, which in no way
add to the power of the language, as each is expressible in terms of
the more primitive constructs.
\section{Expressions and patterns}
\begin{tabular}{@{}l l}
{\bf Derived Form}&{\bf Equivalent Form} \\ \hline
\multicolumn{2}{l}{\bf Types:} \\
${\rm ty}_1$ \verb"*" \rep{2} \verb"*" ${\rm ty}_n$ &
\verb"{" 1 : ${\rm ty}_1$ , \rep{2} , $n$ : ${\rm ty}_n$ \verb"}"  \\ \hline
\multicolumn{2}{l}{\bf Expressions:} \\ 
\verb"()" & \verb"{ }"   {\it (no space between ``\/\verb"()"'')} \\ \xskip
\verb"(" ${\rm exp}_1$ \verb"," \rep{2} \verb"," ${\rm exp}_n$ \verb")" &
\verb"{" 1 \verb"=" ${\rm exp}_1$ , \rep{2} ,
$n$ \verb"=" ${\rm exp}_n$  \verb"}" \\ \xskip
\verb"case" exp \verb"of" match & ( \verb"fn" match ) ( exp ) \\ \xskip
\verb"#" lab & \verb"fn {" lab \verb"= x , ...} => x" \\ \xskip
\verb"if" exp \verb"then" ${\rm exp}_1$ \verb"else" ${\rm exp}_2$ &
\verb"case" exp \verb"of true =>" ${\rm exp}_1$ \\
& \ \ \ \ \ \ \ \ \ \  \verb"| false =>" ${\rm exp}_2$ \\ \xskip
${\rm exp}_1$ \verb"orelse" ${\rm exp}_2$ &
\verb"if" ${\rm exp}_1$ \verb"then true else" ${\rm exp}_2$ \\ \xskip
${\rm exp}_1$ \verb"andalso" ${\rm exp}_2$ &
\verb"if" ${\rm exp}_1$ \verb"then" ${\rm exp}_2$ \verb"else false" \\ \xskip
( ${\rm exp}_1$ ; \rep{1} ; ${\rm exp}_n$ ; exp) &
\verb"case"  ${\rm exp}_1$ \verb"of _ =>" \underline{\ \ \ } \verb"=>" \\
& \ \ \  \verb"case"  ${\rm exp}_n$ \verb"of _ =>" exp \\ \xskip
\verb"let" dec \verb"in" ${\rm exp}_1$ ; \rep{1} ; ${\rm exp}_n$ \verb"end"
&
\verb"let" dec \verb"in" ( ${\rm exp}_1$ ; \rep{1} ; ${\rm exp}_n$) \verb"end"
\\ \xskip
\verb"while" ${\rm exp}_1$ \verb"do" ${\rm exp}_2$ &
\parbox[t]{2.5in}{\begin{raggedright}
\verb"let val rec f = fn () =>" \\
\ \ \verb"if"  ${\rm exp}_1$ \verb"then (" ${\rm exp}_2$ \verb"; f()) else ()"
\\
\verb" in f() end"
\end{raggedright}} \\ \xskip
\verb"[" ${\rm exp}_1$ , \rep{0} , ${\rm exp}_n$ \verb"]" &
${\rm exp}_1$ \verb"::" \rep{0} \verb"::" ${\rm exp}_n$ \verb":: nil" \\
\hline
\pagebreak[1]
{\bf Derived Form}&{\bf Equivalent Form} \\ \hline
\multicolumn{2}{l}{\bf Patterns:} \\ 
\verb"()" & \verb"{ }"   {\it (no space between ``\/\verb"()"'')} \\ \xskip
\verb"(" ${\rm pat}_1$ \verb"," \rep{2} \verb"," ${\rm pat}_n$ \verb")" &
\verb"{" 1 \verb"=" ${\rm pat}_1$ , \rep{2} ,
$n$ \verb"=" ${\rm pat}_n$  \verb"}" \\ \xskip

\verb"[" ${\rm pat}_1$ , \rep{0} , ${\rm pat}_n$ \verb"]" &
${\rm pat}_1$ \verb"::" \rep{0} \verb"::" ${\rm pat}_n$ \verb":: nil" \\ \xskip
\verb"{" \underline{\ \ \ } , id , \underline{\ \ \ } \verb"}" &
\verb"{" \underline{\ \ \ } , id \verb"=" id , \underline{\ \ \ } \verb"}" \\ \xskip
\verb"{" \underline{\ \ \ } , id \verb"as" pat, \underline{\ \ \ } \verb"}" &
\verb"{" \underline{\ \ \ } , id \verb"=" id \verb"as" pat, \underline{\ \ \ } \verb"}" \\ \xskip
\verb"{" \underline{\ \ \ } , id : ty , \underline{\ \ \ } \verb"}" &
\verb"{" \underline{\ \ \ } , id \verb"=" id : ty , \underline{\ \ \ } \verb"}" \\
\hline
\end{tabular}

Each derived form is identical semantically to its ``equivalent
form.''  The type-checking of each derived form is also defined by
that of its equivalent form.  The precedence among all primitive and
derived forms is shown in Appendix~\ref{grammar}.

The derived type ${\rm ty}_1$ \verb"*" \rep{2} \verb"*" ${\rm ty}_n$
is called an (n--)tuple type, and the values of this type are called
(n--)tuples.

The final derived pattern allows a label and its associated value to
be elided in a record pattern, when they are the same identifier.

\section{Bindings and declarations}

A syntax class {\bf fb} of function bindings is used as a convient
form of value binding for (possibly recursive) function declarations.
The equivalent form of each function binding is an ordinary value
binding.  These new function bindings must be declared by \verb"fun",
not by \verb"val"; however, functions may still be declared using
\verb"val" or \verb"val rec" along with \verb"fn" expressions.

\begin{tabular}{@{}l l}
\multicolumn{1}{c}{\bf Derived Form}&
\multicolumn{1}{c}{\bf Equivalent Form} \\ \hline
\multicolumn{2}{l}{\bf Function bindings {\rm fb}:} \\
& id = \verb"fn" $x_1$ \verb"=>" \rep{1} \verb"=> fn" $x_n$ \verb"=>" \\
 & \ \ \verb"case (" $x_1,$ \underline{\ \ \ } $, x_n$ \verb")" \\
\ id ${\rm apat}_{11}$ \rep{1} ${\rm apat}_{1n}$ cst = ${\rm exp}_1$ &
\ \ \verb"of" ( ${\rm apat}_{11}$ , \rep{1} , ${\rm apat}_{1n}$  \verb"=>" ${\rm exp}_1$ cst \\
\verb"|" \underline{\ \ \ } & \ \ \verb"|" \underline{\ \ \ } \\
\verb"|" id ${\rm apat}_{m1}$ \rep{1} ${\rm apat}_{mn}$ cst = ${\rm exp}_m$ &
\ \ \ \verb"|" ( ${\rm apat}_{m1}$ , \rep{1} , ${\rm apat}_{mn}$  \verb"=>" ${\rm exp}_m$ cst \\

 & \\
${\rm fb}_1$ and \rep{1} and ${\rm fb}_n$ &
${\rm vb}_1$ and \rep{1} and ${\rm vb}_n$ \\
&{\it (where ${\rm vb}_i$ is the equivalent of\/ ${\rm fb}_i$) } \\
\hline
\multicolumn{2}{l}{\bf Declarations:} \\
\verb"fun" fb & \verb"val rec" vb \\
&{\it (where \/{\rm vb} is the equivalent of\/ {\rm fb}) } \\
& \\
exp & \verb"val it =" exp \\
\hline
\end{tabular}
In the table above, ``cst'' stands for an optional type constraint---a colon
followed by a type expression.
The last derived declaration (using ``it'') is only allowed at
top-level, for treating top-level expressions as degenerate
declarations; ``it'' is just a normal value variable.
