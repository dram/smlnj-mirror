\documentstyle [twoside]{report}
%\includeonly{library,compat}
\title{Standard ML Reference Manual (PRELIMINARY)}
\author{}
\date{\today}


\begin{document}
\newcommand{\xskip}{\vspace{1ex}}
\newcommand{\res}[1]{{\tt #1}}
\newcommand{\rep}[1]{\underline{\  {\footnotesize #1}  \ }}
\newcommand{\lhs}[1]{\pagebreak[1] \item[#1 \  \( \rightarrow \) ] }
\def\description{\list{}{\labelwidth 1.3in \labelsep 0.3in
                         \leftmargin 1.6in
 \let\makelabel\descriptionlabel}}
\maketitle
\begin{abstract}
This manual is a major revision by Andrew W. Appel and David B. MacQueen
of the {\em Standard ML} reference manual (ECS-LFCS-86-2).  That
document is divided into three parts: the Core language description
by Robin Milner, the standard I/O library by Robert W. Harper,
and the Module system by David B. MacQueen.  This new manual reflects
a slightly evolved language, and describes the library functions
(initial environment) built into the Standard ML of New Jersey system,
developed at AT\&T Bell Laboratories and Princeton University.

At present the manual is in a very rough form, and should be
considered a preliminary draft.  This version is distributed
primarily as documentation of the mid-1989 distribution of the
{\it Standard ML of New Jersey} compiler.
\end{abstract}
\chapter{Introduction}

This is a preliminary draft of the 1988 Standard ML manual.  It is a
revision of the Milner/Harper/MacQueen Edinburgh technical report,
taking account of changes to the language definition in the last year
or two.

The document at present has quite a few gaps.  It should be treated
as a guide to the 1987 pre-release, rather than as a complete
reference manual.

\chapter{Lexical analysis}
\section{Reserved words}
The following are reserved words.   They may not be used as
identifiers.  In this document the alphabetic reserved words are always
shown in boldface.
\begin{quote}
\raggedright
\tt
abstraction abstype and andalso
as case datatype else end exception do fn fun functor handle if in 
infix infixr let local nonfix of op open overload 
raise rec sharing sig signature
struct structure then type val while with withtype orelse

\verb"{  }  [  ]  ,  ;  (  )  ->  *  |  :  ...  =  =>  #  _"
\end{quote}
\section{Special constants}
An integer constant is any non-empty sequence of digits, possibly preceded
by a negation symbol (\verb|~|).

A real constant is an integer constant, possibly followed by a point (.)
and one or more digits, possibly followed by an exponent symbol(E) and
an integer constant; at least one of the optional parts must occur,
hence no integer constant is a real constant.  Examples: \verb|0.7| ,
\verb|~3.32E5| , \verb|3E~7| .  Non-examples: \verb|23| , \verb|.3| ,
\verb|4.E5| , \verb|1E2.0| .

A string constant is a sequence, between quotes (\verb|"|), of zero or more
printable characters, spaces, or escape sequences.  Each escape sequence
is introduced by the escape character \verb|\|, and stands for a character
sequence.  The allowed escape sequences are as follows (all other
uses of \verb|\| being incorrect):
\begin{tabular}{l p{3.9in}}
\verb|\n| & A single character interpreted by the system as end-of-line.\\
\verb|\t| & Tab. \\
\verb|\^c| & The control character c, for any appropriate c.\\
\verb|\ddd| &  The single character with ASCII code ddd (3 decimal digits).\\
\verb|\"| & The double-quote character (\verb'"'). \\
\verb|\\| &  The backslash character (\verb"\").\\
\verb|\f___f\| & This sequence is ignored, where f\_\_\_f stands for a
sequence of one or more formatting characters (a subset of the
non-printable characters including at least space, tab, newline,
formfeed).  This allows one to write long strings on more than one
line, by writing \verb"\" at the end of one line and at the start of the
next.
\end{tabular}

\section{Identifiers}

An identifier is either {\em alphanumeric}: any sequence of letters,
digits, primes (\verb"'"), and underbars (\verb"_") starting with a letter or a
prime, or {\em symbolic}: any sequence of the following symbols
\begin{quote}
\verb"! % & $ + - / : < = > ? @ \ ~ \^ | # * `"
\end{quote}
In either case, however, reserved words are excluded.  This means
that for example \verb"_" and \verb"|" are not identifiers, but
\verb"also_ran" and \verb"|=|" are identifiers.

Identifiers are used to stand for 9 different classes of objects,
which occupy 6 different name spaces, as follows:
\begin{enumerate}
\item value variables ({\it var}), value constructors ({\it con}), \\
exception constructors ({\it exncon})
\item type variables ({\it tyvar})
\item type constructors ({\it tycon})
\item record labels ({\it lab})
\item structures ({\it str}), functors ({\it fct})
\item signatures ({\it sgn})
\end{enumerate}
Thus, an identifier could not in the same scope stand for both a
value variable and a constructor, but an identifier can
be bound simultaneously to a type constructor and a signature.

To remove some ambiguity, it is recommended that constructors start
with an uppercase letter, and variables start with a lowercase
letter; but this is a convention, not an enforced rule  (it is
confounded, for example, by symbolic identifiers).

A type variable ({\it tyvar}) may be any alphanumeric identifier starting
with a prime.  The other eight classes ({\it var, con, tycon, ...})
are represented by identifiers not starting with a prime.  The class
lab is also extended to include the numeric labels 1, 2, 3, ... .

Type variables are therefore disjoint from the other classes.
Otherwise, the class of an occurrence of an identifier is determined
from context.

Spaces or parentheses are sometimes needed 
to separate symbolic identifiers and reserved words.  Two examples are

\begin{tabular}{c c c c c}
\verb"a:= !b" &or& \verb"a:=(!b)" &but not& \verb"a:=!b"\\
\verb"~ :int->int" &or& \verb"(~):int->int" &but not& \verb"~:int->int"
\end{tabular}

These punctuation characters cannot be constituents of identifiers
and therefore never need spaces around them:
\begin{quotation}
\verb| " ( ) , . ; [ ] { } |
\end{quotation}

\section{Comments}
A comment is a character sequence (outside of a string)
within comment brackets (* *) in which comment brackets are properly
nested.

\section{The bare syntax}
The Standard ML bare language is obtained by stripping the full
language of any {\em derived} forms (those that may be defined in
terms of other constructs in the language), and of any constructs
related to the module system.  The bare language will be explained
in Chapters \ref{eval} and \ref{types},
and successive chapters describe augmentations
of it that yield the full language.

Figure~\ref{bare} shows the syntax of the bare language.  The notation
\begin{quotation}
phrase x \rep{k} x phrase
\end{quotation}
indicates the repetition of the {\em phrase} at least $k$  times,
separated by the punctuation character $x$.

\label{bare}
\twocolumn[Figure \protect\ref{bare}: Syntax of the Bare Language \\ \\]
{\samepage
\begin{list}{}{\setlength{\listparindent}{0pt}
               \setlength{\labelwidth}{5em}
%              \setlength{\parsep}{0pt}
               \setlength{\labelsep}{0.5em}}
\lhs{lab}       INT
\\      id
\lhs{const}     INT
\\      REAL
\\      STRING
\lhs{exp}       id
\\      const
\\      \{ lab%
\verb"="%
exp , \rep{0} , lab%
\verb"="%
exp \}
\\      \res{let} dec \res{in} exp \res{end}
\\      exp exp
\\      exp : ty
\\      \res{fn} match
\\      \res{raise} exp
\\      exp \res{handle} match
\lhs{match}     pat \verb"=>" exp $\mid$ \rep{1} $\mid$ pat \verb"=>" exp
\lhs{apat}      id
\\      const
\\      \verb"_"
\\      ( pat )
\\      \{ lab%
\verb"="%
pat , \rep{0} , lab%
\verb"="%
pat \}
\\      \{ lab%
\verb"="%
pat , \rep{0} , lab%
\verb"="%
pat , ... \}
\lhs{pat}       apat
\\      id apat
\\      pat : ty
\\      id constraint \res{as} pat
\lhs{ty}        \res{tyvar}
\\      \{ lab : ty , \rep{0} , lab : ty \}
\\      ( ty )
\\      ( ty , \rep{2} , ty ) id
\\      ty id
\\      id
\\      ty \verb"->" ty
\lhs{vb}        pat \verb"=" exp
\\ vb \res{and} \rep{1} \res{and} vb
\lhs{constraint}        {\it empty}
\\          : ty
\lhs{rvb}       id constraint \verb"=" \res{fn} match
\\              rvb \res{and} \rep{1} \res{and} rvb
\lhs{tb}        tyvars id \verb"=" ty
\\              tb \res{and} \rep{1} \res{and} tb
\lhs{tyvars}    {\it empty}
\\      \res{tyvar}
\\      ( \res{tyvar} , \rep{2} , \res{tyvar} )
\lhs{db}        tyvars id \verb"=" constr $\mid$ \rep{1} $\mid$ constr
\\ db \res{and} \rep{1} \res{and} db
\lhs{constr}    id
\\      id \res{of} ty
\lhs{eb}        id
\\      id \res{of} ty
\\      id \verb"=" id
\\       eb  \res{and} \rep{1} \res{and} eb
\lhs{dec}       \res{val} vb
\\      \res{val} \res{rec} rvb
\\      \res{type} tb
\\      \res{datatype} db
\\      \res{exception} eb
\\      \res{local} dec \res{in} dec \res{end}
\\      dec ;
\\      dec dec

\end{list}
}
\onecolumn

\chapter{Evaluation}
\label{eval}
\section{Environments and Values}
Evaluation of phrases takes place in the presence of an {\em environment} and a
{\em store}.  An {\em value environment} E maps identifiers to values, value constructors, and
exception constructors.  A {\em store} S maps references to values; references
are themselves values.  (Type environments TE, are
ignored here since they are relevant only to
type-checking and compilation, not to evaluation.  Other environments
having to do with the module system are also ignored here.)

A value $v$ is either a constant (a nullary constructor), a
construction (a constructor applied to a value), a record, a
reference, or a function value.  A record value is a set of
label-value pairs, written \{ label = value , \rep{0} , label = value \},
in which the labels are distinct; note that the order of components
is immaterial.  Labels may be identifiers or integer numerals; both
kinds of labels may appear in the same record value.  A function
value is a partial function which, given a value, may return a
value or may raise an exception; it may also change the store as a side-effect.

An exception $e$, associated to an exception name  {\em exn} in a value
environment, is a special kind of constructor.  Exception
constructors may be nullary or value-carrying, just as may ordinary
constructors.  Nullary exception constructors, and constructed
exceptions (exception constructors applied to values), are ordinary
values of type {\em exn}.

Evaluation of a phrase returns a {\em result}---a value $v$,
a value-environment $E$, or a store $S$ as follows:

\begin{tabular}{l l}
{\bf \ \ Phrase}&{\bf \  Value} \\
Expression & v and S, or raise(v) and S\\
Value binding & E and S, or raise(v) and S\\
Type binding & {\it no effect on E or S} \\
Datatype binding & E \\
Exception binding & E \\
Declaration & E and S, or raise(v) and S
\end{tabular}

The remainder of this chapter describes the semantics of various
phrases in terms of values $v$ and environments $E$.

The semantics of stores ($S$) are discussed in
Chapter~\ref{reference};
for the remainder of this chapter, the
effect of various phrases on stores will be ignored.
The semantics of exceptions is discussed in
Chapter~\ref{exception}.


For every phrase except a \verb"handle" expression, whenever its
evaluation demands the evaluation of an immediate subphrase which
returns a raised exception {\em raise(v)} as a result, no further
evaluation of subphrases occurs and {\em raise(v)} is also the result
of the phrase.  This rule should be remembered while reading the
evaluation rules below.

In presenting the rules, explicit type
constraints (:ty) have been ignored since they have no effect on
evaluation.

\section{Environment manipulation}
We may write $\{ i_1 \mapsto v_1 , ... , i_n \mapsto v_n \}$ for a
value environment E (the identifiers $i_k$ being distinct).  Then
$E(i_k)$ denotes $v_k$, $\{\}$ is the empty value environment, and
$E+E'$ means the environment in which the associations of $E'$
supersede those of $E$.

\section{Matching patterns}
Patterns serve in ML both as formal parameters of functions, and
as indices in case statements.  When a function or a case statement
is applied to a particular value, one of its patterns may match the
value.

A nullary constructor (like \verb"nil") used as a pattern
will match only the corresponding nullary constructor value.  A
value-carrying constructor $c$, applied to a pattern $p_1$, makes a pattern
$c(p_1)$;
this constructed pattern will match a value $c(v_1)$
built with the same value-carrying constructor, and only if the
pattern $p_1$ matches the value $v_1$.

A record pattern 
\verb"{" ${\it lab}_1$ \verb"=" ${\it pat}_1$ , \underline{\ \ \ } , ${\it lab}_n$ \verb"=" ${\it pat}_n$ \verb"}"
matches a record value whose labels are the same, and only if each
pattern matches the corresponding value.
If the ellipsis (\verb"...") is used at the end of a record pattern,
then it will match a record value with {\it at least} the labels
specified in the pattern.

An identifier (i.e. a variable)
used as a pattern will match any value; when this happens, the
variable will also be bound to that value throughout the variable's
scope.  When a variable is a component of a record or constructor
pattern, then it is bound to a particular component of the record
value or constructed value.

An underscore will match any value.

Sometimes it is desired to bind a variable to a value only
if the value matches a particular pattern.  The pattern 
{\it id}\verb" as "{\it pat} binds the value to the variable {\it
id}, but only if the {\it pat} matches.

Type constraints may be applied to patterns.  {\it pat\verb":"ty}
matches the same values that {\it pat} does, but the compiler will
guarantee that the pattern will only be applied to values  of type
{\it ty}.

\subsection*{A more formal description}
The matching of a pattern to a value $v$ either {\em fails} or yields
a value environment.  Failure is distinct from raising an exception,
but an exception will be raised when all patterns fail in applying a
match to a value (see Sections~\ref{raisematch}, \ref{raisebind}, and \ref{reraise}).  In the following rules, if any
component pattern fails to match then the whole pattern fails to
match.

The following is the effect of matching a pattern  to a value $v$ in the
environment $E$, for
each of the kinds of pattern (with failure if any condition is not
satisfied):
\begin{description}
\item[\protect\verb"\_"\hfill]  {\it (underscore)} the empty value environment $\{\}$ is
returned
\item[id\hfill] if $E(id)$ is not a constructor, then the value environment
$\{id \mapsto v\}$ is returned
\item[id\hfill] if $E(id)$ is a nullary constructor, then if $E(id)=v$,
then $\{\}$ is returned, else failure
\item[id pat\hfill]  if $E(id)$ is a value-carrying constructor $c$, and 
$v = c v'$, then pat is matched to $v'$, else failure.
\item[id \protect\verb"as" pat\hfill]  pat is matched to $v$ returning $E$;
then $\{id \mapsto v\}+E$ is returned.
\item[\protect\verb"\{" ${\it lab}_1$ \protect\verb"=" ${\it pat}_1$ , \underline{\ \ \ } , ${\it lab}_n$ \protect\verb"=" ${\it pat}_n$ \protect\verb"\}" \hfill]  
if $v = \{ {\it lab}_1 = v_1 , ... , {\it lab}_n = v_n \}$ then 
${\it pat}_i$ is matched to $v_i$ returning $E_i$, for each $i$; then 
$E_1 + ... + E_n$ is returned.

\item[\protect\verb"\{" ${\it lab}_1$ \protect\verb"=" ${\it pat}_1$ , \underline{\ \ \ } , ${\it lab}_n$ \protect\verb"=" ${\it pat}_n$ \protect\verb", ... \}" \hfill]  
if $v = \{ {\it lab}'_1 = v_1 , ... , {\it lab}'_m = v_m \}$ then if
the ${\it lab}_i$ are a subset of the ${\it lab}'_j$ then
${\it pat}_i$ is matched to $v_j$ returning $E_i$, for each $i,j$ such that
${\it lab}_i = {\it lab}'_j$; then 
$E_1 + ... + E_n$ is returned.
\end{description}
No pattern may contain the same variable twice.
No record pattern, record expression, or record type may use the same
label twice.
\section{Applying a match}
Assume environment $E$.  Applying a match
$ {\it pat}_1 \protect\verb"=>" {\it exp}_1 \mid ... 
\mid {\it pat}_n \protect\verb"=>" {\it exp}_n $ to value $v$ returns a value
or raises an exception as follows.  Each ${\it pat}_i$ is matched to
$v$ in turn, from left to right, until one succeeds returning $E_i$;
then ${\it exp}_i$ is evaluated in $E+E_i$.  If none succeeds, then
an exception is raised, depending on the syntactic context in which
the match occurs (see Sections~\ref{raisematch}, \ref{raisebind}, and \ref{reraise}).
\label{matchwarn}
If a match contains a redundant pattern (where any value that could
satisfy it will be matched by a previous pattern in the match), the
compiler will issue a warning message.  If the match (except those
that form exception-handlers) is not exhaustive (some value matches
none of the patterns) then the compiler will issue a warning message.

Thus, for each $E$, a match denotes a function value.

\section{Evaluation of expressions}
Evaluating an expression in the environment $E$ returns a value (or raises an
exception) as follows, in each of the cases for exp:
\begin{description}
\item[id\hfill]  the value $E(id)$ is returned; id may be a variable-id or a
constructor-id
\item[const\hfill]  the value denoted by the constant (an integer, real, or
string literal) is returned.
\item[${\bf exp}_1 {\bf exp}_2$\hfill] ${\bf exp}_1$ is evaluated, 
returning function $f$; then ${\it exp}_2$
is evaluated, returning value $v$; then $f(v)$ is returned.
\item[\protect\verb"\{" ${\it lab}_1$ \protect\verb"=" ${\it exp}_1$ , \underline{\ \ \ } , ${\it lab}_n$ \protect\verb"=" ${\it exp}_n$ \protect\verb"\}" \hfill]  
the ${\it exp}_i$ are evaluated in sequence, from left to right,
returning $v_i$ respectively; then the record 
$\{ {\it lab}_1 = v_1 , ... , {\it lab}_n = v_n \}$ is returned.
\item[\protect\verb"raise" exp\hfill]  exp is evaluated, returning $v$; then 
the exception-value $v$ is raised.

\item[exp \verb"handle" match\hfill]  exp is evaluated; if exp returns a
value $v$, then $v$ is returned.  If exp raises an exception $e$ then
the match is applied to $e$.  If the match fails, then $e$ is raised
(as the value of the \verb"handle" expression).  If the match
succeeds, then the resulting value is returned.

\item[\verb"let" dec \verb"in" exp \verb"end"\hfill]  dec is evaluated,
returning $E'$; then exp is evaluated in the environment $E+E'$.

\item[\verb"fn" match\hfill]  $f$ is returned, where $f$ is the function of
$v$ gained by applying match to $v$ in the environment $E$, and such
that if the match fails, an exception \verb"Match" will be raised.
\label{raisematch}
But matches that may fail are to be detected by the
compiler and flagged with a warning; see Section~\ref{matchwarn}.
\end{description}

\section{Evaluation of value bindings}
Evaluating a value binding in environment $E$ returns a value
environment $E'$  or raises an exception as follows:
\begin{description}
\item[pat \verb"=" exp\hfill]  exp is evaluated in $E$, returning value
$v$; then pat is matched to $v$; if this returns $E'$ then $E'$ is
returned.  If the pattern fails to match then the exception 
\verb"Bind" will be raised.
\label{raisebind}
But bindings that may fail are to be detected by the
compiler and flagged with a warning; see Section~\ref{bindwarn}.

\item[${\bf vb}_1$ \verb"and" \underline{\ \ \ } \verb"and" ${\bf vb}_n$\hfill]
The value bindings ${\bf vb}_1$ through ${\bf vb}_n$ are evaluated in
$E$ from left to right, returning $E_1 , ... E_n$; then $E_1 + ... +
E_n$ is returned.

\item[\verb"rec" vb\hfill]  {\bf vb} is evaluated in $E'$, returning $E''$,
where $E' = E + E''$.  Because the values bound by \verb"rec" {\bf
vb} must be function values, $E'$ is well defined by ``tying knots''
(Landin).
\end{description}
No binding may bind the same identifier twice.

\label{bindwarn}
For each value binding ``pat = exp'' the compiler will issue a
warning message if {\it either} pat is not exhaustive {\it or} pat
contains no variable.  This will (on both counts) detect a mistaken
declaration like \verb"val nil = f(x)" in which the user expects to
declare a new variable nil (whereas the language dictates that
\verb"nil" here is a constant pattern, so no variable gets declared).
However, these warnings need not be given at top level in the
interactive system.
\section{Evaluation of type and datatype bindings}
The value environment $E$ does not affect the evaluation of type
bindings or datatype bindings ($TE$ affects their type-checking and
compilation).  Evaluation of a type binding just returns the empty
value environment $\{\}$; the purpose of type bindings is merely to
provide an abbreviation for a compound type.

Evaluation of a
datatype binding {\bf db} returns a value environment $E'$ (it cannot
raise an exception) as follows:

\begin{description}
\item[tyvars id = ${\bf constr}_1 \mid \underline{\ \ \ } \mid {\bf constr}_n$\hfill]
New constructors ${\bf con}_1, ... , {\bf con}_n$ are created.
The value environment $E' = \{ id_1 \mapsto v_1 , ... , id_n \mapsto
v_n \} $ is returned, where $v_i$ is either the constant value
${\bf con}_i$ (if ${\bf constr}_i$ is just ${\bf id}_i$), or else the
function which maps $x$ to ${\bf con}_i(x)$ (if ${\bf constr}_i$ is
${\bf id}_i$ \verb"of" {\bf ty}).

\item[${\bf db}_1$ \verb"and" \underline{\ \ \ } \verb"and" ${\bf db}_n$\hfill]
The bindings ${\bf db}_1 , ... , {\bf db}_n$
are evaluated from left to right, returning $E_1 , ... , E_n$; then
$E_1 + ... + E_n$ is returned.
\end{description}
In the left hand side ``tyvars id'' of a type of datatype binding,
no type-variable may appear twice in ``tyvars.''
The right hand side may contain only the type variables
mentioned on the left.  Within the scope of the declaration of
``id,'' any occurrence of the type constructor ``id'' must be
accompanied by as many type arguments as indicated by the (possibly empty)
tyvar sequence ``tyvars'' in the declaration.

No binding may bind the same identifier twice.
\section{Evaluation of exception bindings}
The evaluation of an exception binding in an environment $E$ returns an
environment $E'$ as follows:
\begin{description}
\item[id\hfill]   A new exception constructor {\bf con} is generated, and
the environment $\{{\bf id} \mapsto {\bf con} \}$ is returned.

\item[id \verb"of" ty \hfill]  A new exception constructor {\bf con} is generated,
and the environment $\{{\bf id} \mapsto v\}$ is returned, where $v$ is the
function of $x$ that returns ${\bf con}(x)$.

\item[${\bf id}_1$ \verb"=" ${\bf id}_2$ \hfill]  The environment
$\{ {\bf id}_1 \mapsto E({\bf id}_2) \}$ is returned; that is, 
${\bf id}_1$ is bound to the same exception constructor as ${\bf id}_2$.

\item[${\bf eb}_1$ \verb"and" \underline{\ \ \ } \verb"and" ${\bf eb}_n$\hfill]
The bindings ${\bf eb}_1 , ... , {\bf eb}_n$
are evaluated from left to right, returning $E_1 , ... , E_n$; then
$E_1 + ... + E_n$ is returned.
\end{description}
No binding may bind the same identifier twice.
\section{Evaluation of declarations}

Evaluating a declaration in a value environment $E$ returns a value
environment $E'$ or raises an exception as follows (as usual in this chapter,
the effect on type environments is ignored):

\begin{description}
\item[\verb"val" vb\hfill] The value binding {\rm vb} is evaluated,
returning $E'$; then $E'$ is returned.

\item[\verb"type" tb\hfill] The empty environment $\{\}$ is returned.

\item[\verb"datatype" db\hfill]  {\rm db} is evaluated, returning
$E'$, which is returned.

\item[\verb"exception" eb\hfill]  {\rm eb} is evaluated, returning
$E'$, which is returned.

\item[\verb"local" ${\bf dec}_1$ \verb"in" ${\bf dec}_2$ \verb"end"\hfill]
${\rm dec}_1$ is evaluated in $E$, returning $E_1$; then 
${\rm dec}_2$ is evaluated in $E+E_1$, returning $E_2$; then $E+E_2$
is returned.

\item[${\bf dec}_1$ ${\bf dec}_2$\hfill]
${\rm dec}_1$ is evaluated in $E$, returning $E_1$; then 
${\rm dec}_2$ is evaluated in $E+E_1$, returning $E_2$; then $E_1+E_2$
is returned.

\item[${\bf dec}$ \verb";"\hfill] has the same effect as {\bf dec}.
\end{description}
\section{Evaluation of a program}
The evaluation of a program ${\bf dec}_1$ \verb";" \underline{\ \ \ }
\verb";" ${\bf dec}_n$ takes place in the initial presence of the
standard top-level environment $E_0$ containing all the standard
bindings (see Appendix~\ref{library}).  For $i>0$ the top-level environment
$E_i$, present after the evaluation of ${\bf dec}_i$ in the program,
is defined recursively as follows:  ${\bf dec}_i$ is evaluated in
$E_{i-1}$ returning environment ${E'}_i$, and then $E_i =
E_{i-1}+{E'}_i$.

\chapter{The type system}
\label{types}
Type checking is much the same as in the previous system.  This
chapter has not yet been written.

\chapter{Directives}
Directives are included in ML as (syntactically) a subclass of declarations.
They possess scope, as do all declarations.  The directives
\begin{quote}
\verb"infix" {\it d} ${\bf id}_1$ \underline{\ \ \ } ${\bf id}_n$

\verb"infixr" {\it d} ${\bf id}_1$ \underline{\ \ \ } ${\bf id}_n$
\end{quote}
introduce infix status for the identifiers  ${\bf id}_1$  through ${\bf id}_n$.
The digit $d$ (optional, with a default of 0) determines the
precedence, and an infixed identifier associates to the left if
introduced by \verb"infix", and to the right if introduced by
\verb"infixr".  Different infixed operators of equal precedence
associate to the left.  As indicated in Appendix~\ref{grammar}, the precedence
of infixed application is just weaker than that of application.

The directive
\begin{quote}
\verb"nonfix" ${\bf id}_1$ \underline{\ \ \ } ${\bf id}_n$
\end{quote}
cancels infix status for the named identifiers.

While an identifier has infix status, each occurrence of it (as a
value variable or as a constructor) must be infixed or else preceded
by \verb"op".  Note that this includes occurrences of the identifier
within patterns, even binding occurrences of variables.

Several standard functions and constructors have infix status (see
Appendix~\ref{grammar}) with precedence; these are all left associative except
``\verb"::"''.

\chapter{Standard bindings}
The initial top-level environment is comprised of a set of standard
bindings; the ``core'' bindings are described in this chapter, and
all bindings are described in Appendix~\ref{library}

The language provides the record type constructor 
$ \verb"{" {\bf lab}_1 \verb":" {\bf ty}_1 , \underline{\ \ \ }
 , {\bf lab}_n \verb":" {\bf ty}_n \verb"}"$ for any set 
$\{ {\bf lab}_i \} $ of labels and corresponding set 
$\{ {\bf ty}_i \} $ of types.  The language also provides the infixed
function-type constructor \verb"->".  Otherwise, type constructors
are postfixed.  The following are standard:

\begin{description}
\item[Type {\em constants} (nullary constructors):]  unit, bool, exn, int,
real, string
\item[Unary type constructors:]  list, ref
\end{description}

The constructors unit, bool, and list are fully defined by the
following assumed declaration
\begin{verbatim}
infixr 5 ::
type unit = {}
datatype    bool = true | false
datatype  'a list = nil |  :: of {1 : 'a, 2 : 'a list}
\end{verbatim}

The word ``unit'' is chosen since the type contains just one value
``\verb"{}"'', the empty record.  This is why it is preferred to the
word ``void'' of Algol-68.

The type constants \verb"int", \verb"real", and \verb"string"
are equipped with special
constants as described in section 2.3.  The type constructor
\verb"ref" is for constructing reference types; see
Chapter~\ref{reference}.
The type constant \verb"exn" is the type of all exceptions, and
is a datatype containing an unbounded number of constructors
generated by \verb"exception" bindings (see Chapter~\ref{exception}).

All standard functions, constants, and exceptions are listed in
Appendix~\ref{library}.


\chapter{Derived forms}
\label{derived}
ML is equipped with a number of {\em derived forms}, which in no way
add to the power of the language, as each is expressible in terms of
the more primitive constructs.

The $n$-tuple type 
\verb"("$ty_1$\verb"*"$ty_2$\verb"*"~$\cdots$~\verb"*"$ty_n$\verb")"
for $n\geq 2$ is an abbreviation for the record type with numeric labels
\verb"{1:"$ty_1$\verb",2:"$ty_2$\verb","~$\cdots$~\verb","$n$\verb":"$ty_n$\verb"}".
Similarly the $n$-tuple expression 
\verb"("$exp_1$\verb","$exp_2$\verb","~$\cdots$~\verb","$exp_n$\verb")"
is an abbreviation for the record expression 
verb"{1="$exp_1$\verb",2="$exp_2$\verb","~$\cdots$~\verb","$n$\verb"="$exp_n$\verb"}",
and the $n$-tuple pattern
\verb"("$pat_1$\verb","$pat_2$\verb","~$\cdots$~\verb","$pat_n$\verb")"
is an abbrevation for the record pattern 
verb"{1="$pat_1$\verb",2="$pat_2$\verb","~$\cdots$~\verb","$n$\verb"="$pat_n$\verb"}".

The ``empty record'' \verb"{}" can also be written as \verb"()".  This
value is conventionally returned from functions that have side-effects
but don't return an interesting value.  The type of empty records
is named \verb"unit" (because the type only contains one value).

The expression \verb"#"{\it lab}  for any label (symbolic or numeric)
is a selector function that extracts the named field from a record.

The \verb"case"~{\it exp}~\verb"of"~{\it match} expression is completely
equivalent to a \verb"(fn"~{\it match}\verb")" expression applied to
the {\it exp}.

The \verb"if"-\verb"then"-\verb"else" expression has conventional semantics:
it evaluates a boolean condition, then evaluates either the \verb"then"
expression or the \verb"else" expression; this behavior can be defined
in terms of a \verb"case" with patterns \verb"true" and \verb"false".

The \verb"orelse" and \verb"andalso" boolean operators evaluate their
right-hand expressions only if the left-hand expressions are insufficient
to determine the answer (i.e. \verb"false" for \verb"orelse" and \verb"true"
for \verb"andalso").  Their syntactic precedence is weaker than all other
infix operators, and \verb"orelse" binds weaker than \verb"andalso".

Several expressions may be separated by semicolons, and the whole enclosed
in parentheses;  all the expressions will be evaluated and the result
of the whole will is the result of the last expression.  When such an
{\it expression sequence} is the entire body of a \verb"let" expression,
the parentheses may be omitted.

The expression \verb"while"~$exp_1$~\verb"do"~$exp_2$ repeatedly evaluates
$exp_1$ followed by $exp_2$ until $exp_1$ evaluates to \verb"false" 
(just as in Pascal); the \verb"while" expression can be expressed 
in terms of a recursive function.

The expression 
\verb"["$exp_1$\verb","$exp_2$\verb","$\cdots$\verb","$exp_n$\verb"]" is
an abbreviation for the list 
$exp_1$\verb"::"$exp_2$\verb"::"$\cdots$\verb"::"$exp_n$\verb"::nil",
and similarly the pattern
\verb"["$pat_1$\verb","$pat_2$\verb","$\cdots$\verb","$pat_n$\verb"]" is
an abbreviation for the list pattern
$pat_1$\verb"::"$pat_2$\verb"::"$\cdots$\verb"::"$pat_n$\verb"::nil".
In both patterns and expressions, \verb"[]" is an abbrevation for \verb"nil".

In a record pattern (but not in a record expression), an element
$id$\verb"="$id$, where the pattern is a variable with the same identifier
as the label, can be abbreviated as just $id$.  Similarly, 
$id$\verb"=~"$id$~verb"as"~{\it pat} can be abbreviated
as "$id$~verb"as"~{\it pat},
$id$\verb"=~"$id$~verb":"~{\it ty} can be abbreviated
"$id$~verb":"~{\it ty}.

Recursive clausal function definitions can be defined conviently with
the \verb"fun" keyword.  The declaration
\begin{verbatim}
fun f pat1a pat2a pat3a = exp1
  | f pat1b pat2b pat3b = exp2
  | f pat1c pat2c pat3c = exp2
\end{verbatim}
is an abbreviation for the curried function
\begin{verbatim}
val rec f = fn pat1 => fn pat2 => fn pat3 =>
             case (pat1,pat2,pat3)
              of (pat1a,pat2a,pat3a) => exp1
               | (pat1b,pat2b,pat3b) => exp1
               | (pat1c,pat2c,pat3c) => exp1
\end{verbatim}
The patterns must all be atomic patterns (including, of course, 
arbitrary patterns enclosed in parentheses) to avoid syntactic ambiguity.

A special form of \verb"fun" declaration is permitted for infixed
function identifiers, of which an example is shown here:
\begin{verbatim}
fun a + b = b - ~a
\end{verbatim}

The derived forms are summarized in the tables below.

\section{Expressions and patterns}
\begin{tabular}{@{}l l}
{\bf Derived Form}&{\bf Equivalent Form} \\ \hline
\multicolumn{2}{l}{\bf Types:} \\
${\rm ty}_1$ \verb"*" \rep{2} \verb"*" ${\rm ty}_n$ &
\verb"{" 1 : ${\rm ty}_1$ , \rep{2} , $n$ : ${\rm ty}_n$ \verb"}"  \\ \hline
\multicolumn{2}{l}{\bf Expressions:} \\ 
\verb"()" & \verb"{ }"\\ \xskip
\verb"(" ${\rm exp}_1$ \verb"," \rep{2} \verb"," ${\rm exp}_n$ \verb")" &
\verb"{" 1 \verb"=" ${\rm exp}_1$ , \rep{2} ,
$n$ \verb"=" ${\rm exp}_n$  \verb"}" \\ \xskip
\verb"case" exp \verb"of" match & ( \verb"fn" match ) ( exp ) \\ \xskip
\verb"#" lab & \verb"fn {" lab \verb"= x , ...} => x" \\ \xskip
\verb"if" exp \verb"then" ${\rm exp}_1$ \verb"else" ${\rm exp}_2$ &
\verb"case" exp \verb"of true =>" ${\rm exp}_1$ \\
& \ \ \ \ \ \ \ \ \ \  \verb"| false =>" ${\rm exp}_2$ \\ \xskip
${\rm exp}_1$ \verb"orelse" ${\rm exp}_2$ &
\verb"if" ${\rm exp}_1$ \verb"then true else" ${\rm exp}_2$ \\ \xskip
${\rm exp}_1$ \verb"andalso" ${\rm exp}_2$ &
\verb"if" ${\rm exp}_1$ \verb"then" ${\rm exp}_2$ \verb"else false" \\ \xskip
( ${\rm exp}_1$ ; \rep{1} ; ${\rm exp}_n$ ; exp) &
\verb"case"  ${\rm exp}_1$ \verb"of _ =>" \underline{\ \ \ } \verb"=>" \\
& \ \ \  \verb"case"  ${\rm exp}_n$ \verb"of _ =>" exp \\ \xskip
\verb"let" dec \verb"in" ${\rm exp}_1$ ; \rep{1} ; ${\rm exp}_n$ \verb"end"
&
\verb"let" dec \verb"in" ( ${\rm exp}_1$ ; \rep{1} ; ${\rm exp}_n$) \verb"end"
\\ \xskip
\verb"while" ${\rm exp}_1$ \verb"do" ${\rm exp}_2$ &
\parbox[t]{2.5in}{\begin{raggedright}
\verb"let val rec f = fn () =>" \\
\ \ \verb"if"  ${\rm exp}_1$ \verb"then (" ${\rm exp}_2$ \verb"; f()) else ()"
\\
\verb" in f() end"
\end{raggedright}} \\ \xskip
\verb"[" ${\rm exp}_1$ , \rep{0} , ${\rm exp}_n$ \verb"]" &
${\rm exp}_1$ \verb"::" \rep{0} \verb"::" ${\rm exp}_n$ \verb":: nil" \\
\hline
\pagebreak[1]
{\bf Derived Form}&{\bf Equivalent Form} \\ \hline
\multicolumn{2}{l}{\bf Patterns:} \\ 
\verb"()" & \verb"{ }"   {\it (no space between ``\/\verb"()"'')} \\ \xskip
\verb"(" ${\rm pat}_1$ \verb"," \rep{2} \verb"," ${\rm pat}_n$ \verb")" &
\verb"{" 1 \verb"=" ${\rm pat}_1$ , \rep{2} ,
$n$ \verb"=" ${\rm pat}_n$  \verb"}" \\ \xskip

\verb"[" ${\rm pat}_1$ , \rep{0} , ${\rm pat}_n$ \verb"]" &
${\rm pat}_1$ \verb"::" \rep{0} \verb"::" ${\rm pat}_n$ \verb":: nil" \\ \xskip
\verb"{" \underline{\ \ \ } , id , \underline{\ \ \ } \verb"}" &
\verb"{" \underline{\ \ \ } , id \verb"=" id , \underline{\ \ \ } \verb"}" \\ \xskip
\verb"{" \underline{\ \ \ } , id \verb"as" pat, \underline{\ \ \ } \verb"}" &
\verb"{" \underline{\ \ \ } , id \verb"=" id \verb"as" pat, \underline{\ \ \ } \verb"}" \\ \xskip
\verb"{" \underline{\ \ \ } , id : ty , \underline{\ \ \ } \verb"}" &
\verb"{" \underline{\ \ \ } , id \verb"=" id : ty , \underline{\ \ \ } \verb"}" \\
\hline
\end{tabular}

Each derived form is identical semantically to its ``equivalent
form.''  The type-checking of each derived form is also defined by
that of its equivalent form.  The precedence among all primitive and
derived forms is shown in Appendix~\ref{grammar}.

The derived type ${\rm ty}_1$ \verb"*" \rep{2} \verb"*" ${\rm ty}_n$
is called an (n--)tuple type, and the values of this type are called
(n--)tuples.

The final derived pattern allows a label and its associated value to
be elided in a record pattern, when they are the same identifier.

\section{Bindings and declarations}

A syntax class {\bf fb} of function bindings is used as a convient
form of value binding for (possibly recursive) function declarations.
The equivalent form of each function binding is an ordinary value
binding.  These new function bindings must be declared by \verb"fun",
not by \verb"val"; however, functions may still be declared using
\verb"val" or \verb"val rec" along with \verb"fn" expressions.

\begin{tabular}{@{}l l}
\multicolumn{1}{c}{\bf Derived Form}&
\multicolumn{1}{c}{\bf Equivalent Form} \\ \hline
\multicolumn{2}{l}{\bf Function bindings {\rm fb}:} \\
& id = \verb"fn" $x_1$ \verb"=>" \rep{1} \verb"=> fn" $x_n$ \verb"=>" \\
 & \ \ \verb"case (" $x_1,$ \underline{\ \ \ } $, x_n$ \verb")" \\
\ id ${\rm apat}_{11}$ \rep{1} ${\rm apat}_{1n}$ cst = ${\rm exp}_1$ &
\ \ \verb"of" ( ${\rm apat}_{11}$ , \rep{1} , ${\rm apat}_{1n}$  \verb"=>" ${\rm exp}_1$ cst \\
\verb"|" \underline{\ \ \ } & \ \ \verb"|" \underline{\ \ \ } \\
\verb"|" id ${\rm apat}_{m1}$ \rep{1} ${\rm apat}_{mn}$ cst = ${\rm exp}_m$ &
\ \ \ \verb"|" ( ${\rm apat}_{m1}$ , \rep{1} , ${\rm apat}_{mn}$  \verb"=>" ${\rm exp}_m$ cst \\

 & \\
${\rm fb}_1$ and \rep{1} and ${\rm fb}_n$ &
${\rm vb}_1$ and \rep{1} and ${\rm vb}_n$ \\
&{\it (where ${\rm vb}_i$ is the equivalent of\/ ${\rm fb}_i$) } \\
\hline
\multicolumn{2}{l}{\bf Declarations:} \\
\verb"fun" fb & \verb"val rec" vb \\
&{\it (where \/{\rm vb} is the equivalent of\/ {\rm fb}) } \\
& \\
exp & \verb"val it =" exp    {\it (only at top level)}\\
\hline
\end{tabular}
In the table above, ``cst'' stands for an optional type constraint---a colon
followed by a type expression.
The last derived declaration (using ``it'') is only allowed at
top-level, for treating top-level expressions as degenerate
declarations; ``it'' is just a normal value variable.

\chapter{Equality}

The equality function \verb"op = : ''a * ''a -> bool" is available at
all types \verb"''a" except function types, abstract types, and the
types constructed from them.  In fact, type variables that begin
with two primes are special: they stand only for types that admit
equality.

Two values are tested for equality as follows, depending on the kind
of value:
\begin{description}
\item[Primitive types] like integers, reals, and strings have
equality functions with the conventional behavior.

\item[Function types] cannot be compared (``do not admit equality'').

\item[Reference types:] On references, equality means identity; a
reference is equal to itself and to no other references, regardless
of similar contents.

\item[Record types] may be compared if all their components admit equality.

\item[Datatypes] may be compared for equality if all of their
constructed types admit equality.

\item[Opaque types] from functor parameters and abstractions do not
admit equality unless the \verb"eqtype" keyword is used (instead of
the \verb"type" keyword) in the
signature defining them.
\end{description}

The function \verb"op <> : ''a * ''a -> bool" is the inequality function;
it is applicable to any equality type.

The comparison functions \verb">", verb"<", \verb"<=", and \verb">=" do not
have this behavior; they are overloaded just for the types
\verb"int", \verb"real", and \verb"string".

\chapter{Exceptions}
\label{exception}
The rough and ready rule for understanding how exceptions are handled
is as follows.  If an exception is raised by a \verb"raise"
expression
\begin{quote}
\verb"raise" $E$ \verb"(" {\it exp} \verb")"
\end{quote}
which lies in the textual scope of a declaration of the exception
constructor $E$, then it may be handled by a handling rule
\begin{quote}
\verb"handle" $E$ \verb"(" {\it pat} \verb") =>" {\it exp'}
\end{quote}
but only if this handler is in the textual scope of the same
declaration.

Any exception, regardless of scope, is handled by a wildcard or
variable pattern, as
\begin{quote}
\verb"handle _ =>" {\it exp'}
\end{quote}
This rule is perfectly adequate for exceptions declared at top level;
some examples in Section~\ref{pathexn} below illustrate what may occur in other
cases.

\section{An example}
To illustrate the generality of exception handling,  suppose we have
declared some exceptions as follows:
\begin{verbatim}
exception    Oddlist of int    and   Oddstring of string
\end{verbatim}
and that a certain expression  exp:int  may raise either of these
exceptions and also runs the risk of dividing by zero.  The handler
in the following \verb"handle" expression would deal with these
expressions:
\begin{verbatim}
exp handle Oddlist []  => 0
         | Oddlist [x] => 2*x
         | Oddlist (x::y::_) => x div y
         | Oddstring "" => 0
         | Oddstring s => size(s)-1
         | Div => 10000
\end{verbatim}
Note that the whole expression is well-typed because in each handling
rule the type of the match-pattern is \verb"exn", and because the
result type of the match is \verb"int", the same as the type of exp.

Note also that the last handling rule will handle \verb"Div"
exceptions raised by exp, but will not handle the \verb"Div"
exception that may be raised by \verb"x div y" in the third handling
rule.  Finally, note that a universal handling rule
\begin{verbatim}
         | _ => 50000
\end{verbatim}
at the end would deal with all other exceptions raised by  exp.

\section{Exception constructors}
For an exception constructor E, the expression
\begin{quote}
$E$ \verb"(" {\it exp} \verb")"
\end{quote}
evaluates the expression {\it exp}, producing value $v$,
and then applies the constructor $E$ to it, yielding the value
$E(v)$, whose type is \verb"exn".

The \verb"raise" keyword may be applied to any expression of type
\verb"exn", and has the effect of ``raising'' that exception value.
The innermost (dynamically) enclosing expression 
$e = e_1~\verb"handle"~m_1$ is found; all further evaluation of the
expression $e_1$ (and its subphrases) is aborted; and the match $m_1$
is applied to the exception value, yielding  the result of the
expression $e$.  

If the match in a handler fails, then the exception value is
\label{reraise}
re-raised, and another enclosing handler is found.

Exception constructors may be nullary (have no associated value), in
which case the {\it exp} and {\it pat} in the previous discussion are
omitted.

Exceptions may be constructed independently of raising them:
\begin{verbatim}
  exception A of int
  val e = A 6
  val x = raise e
\end{verbatim}
Handlers may be abstracted from the \verb"handle" keyword:
\begin{verbatim}
   val h = fn E 0 => "zero"
            | E _ => "nonzero"
            | v => raise v

   f(x)  handle  e => h e
\end{verbatim}
Note that it is advisable in this case to have a default clause in
the function \verb"h", since the default for a \verb"handle" match
(re-raising the exception) is different from the default for a \verb"fn" or
\verb"case" match (raising the \verb"Match" exception).

The ordinary wildcard pattern 
\verb"_" will handle any exception when it is used in a pattern, as
will any pattern consisting solely of a variable.  These should be
used with some care, bearing in mind that they will even handle
interrupts.

Nullary exception names, when misspelled, appear to the compiler to be
variables; these will then match any exception.  For this reason we
recommend the convention that exception names (and other
constructors) be written beginning with an uppercase character, and
variables be written beginning with a lowercase character.  The
compiler may remind the programmer of this convention when it is
violated.

\section{Some pathological examples}
\label{pathexn}
We now consider some possible misuses of exception handling, which
may arise from the fact that exception declarations have scope, and
that each evaluation of a generative exception binding creates a
distinct exception.  Consider a simple example:
\begin{verbatim}
exception  E of bool
fun f(x) = 
     let exception E of int
      in if x > 100 then raise E(x) else x+1
     end
val z = f(200) handle E(true) => 500 | E(false) => 1000
\end{verbatim}
The program is  well-typed, but useless.  The exception bound to the
outer \verb"E" is distinct from that bound to the inner \verb"E";
thus the exception raised by \verb"f(200)", with excepted value 200,
could only be handled by a handler within the scope of the inner
exception declaration---it will not be handled by the handler in the
program, which expects a boolean value.  So this exception will be
reported at top level.  This would apply even if the outer exception
declaration were also of type int; the two exceptions bound to
\verb"E" would be distinct.

On the other hand, if the last line of the program is changed to
\begin{verbatim}
f(200) handle _ => 500
\end{verbatim}
then the exception will be caught, and the value 500 returned.  A
universal handling rule (i.e. \verb"_" or a variable-identifier)
catches any exception---even one exported from the scope of the
declaration of the associated exception name---but cannot examine the
excepted value carried by the exception constructor, since the type
of this value cannot be statically determined.

Even a single textual exception binding---if for example it is
declared within a recursively defined function---may bind distinct
exceptions to the same identifier.  Consider another useless program:
\begin{verbatim}
fun f(x) =
     let exception E
      in if p(x) then a(x)
                 else if q(x) then f(b(x)) handle E => c(x)
                              else raise E
     end
val z = f v
\end{verbatim}
Now if p(v) is false but q(v) is true, the recursive call will
evaluate f(b(v)).  Then if both p(b(v) and q(b(v)) are false, this
evaluation will raise \verb"E".  But this exception will not be
handled, since the exception raised is that which is bound to
\verb"E" by the inner---not outer---evaluation of the exception
declaration.

These pathological examples should not leave the impression that
exceptions are hard to use or to understand.  The rough and ready
rule of Section~\ref{exception} will almost always  give the correct
understanding.

\chapter{Reference values}
\label{reference}
References are cells whose contents may be changed after creation by
assignment.  The \verb"ref" ``datatype'' constructor, and its
corresponding value constructor, are almost as if defined by the declaration
\begin{verbatim}
datatype 'a ref = ref of 'a
\end{verbatim}
Thus, a reference whose initial contents are the string \verb|"abc"|
may be created by \verb|val r = ref "abc"|.  Subsequently, the
contents of \verb"r" may be altered by assignment: \verb|r := "def"|.
The contents of a reference may be examined by using the \verb"ref"
constructor in a pattern:
\begin{verbatim}
let val (ref s) = r
 in print s
end
\end{verbatim}
The function \verb"!" is defined to take the contents of a reference;
that is,
\begin{verbatim}
fun ! (ref x) = x
\end{verbatim}

References are not fully polymorphic; see Chapter~\ref{reftype}.

Formally, we say that phrases in ML are evaluated in the presence of
an {\em environment} $E$ and a {\em store} $S$.  The effect on $E$ of
evaluating declarations, expressions, etc.  is described in
Chapter~\ref{eval}.  Here we summarize the effect on $S$.

The store $S$ maps reference values to their contents.  Evaluation of
an expression in the store $S$ yields, depending on the form of the
expression,
\begin{description}
\item[\verb"ref" exp\hfill] exp is evaluated in $S$,
producing a value $v$ and a store $S'$; the 
reference value $r$ is returned with the store $S'+\{r \mapsto v\}$.

\item[${\rm exp}_1$~${\rm exp}_2$\hfill]  ${\rm exp}_1$ is evaluated in $S$
yielding the function $f_1$ and store $S'$;
${\rm exp}_2$ is evaluated in $S'$ yielding $v_2$ and $S''$;
finally the body of $f_1$ is evaluated with its variable bound to $v_2$,
in the store $S''$, yielding the result $v$ and the store $S'''$.

\item[\verb"op := "$({\rm exp}_1,{\rm exp}_2)$\hfill]  The expression 
$({\rm exp}_1,{\rm exp}_2)$ is evaluated in $S$, yielding the pair
$(r,v)$ and the store $S'$; then the unit value \{\} is returned with
the store $S'+\{r \mapsto v\}$.

\item[\protect\verb"\{" ${\rm lab}_1$ \protect\verb"=" ${\rm exp}_1$ , \underline{\ \ \ } , ${\rm lab}_n$ \protect\verb"=" ${\rm exp}_n$ \protect\verb"\}" \hfill]  
${\rm exp}_1$ is evaluated in $S$, yielding $v_1$ and the store
$S_1$; then each ${\rm exp}_i$ is evaluated in $S_{i-1}$, yielding $v_i$ and the store $S_i$; then the record
$\{ {\rm lab}_1 = v_1 , ... , {\rm lab}_n = v_n \}$ is returned with
the store $S_n$.  Note that the expressions are evaluated in the
sequence they are written, not in alphabetical order of the labels.

\item[\protect\verb"raise" exp\hfill]  exp is evaluated in $S$, returning $v$
and $S'$; then the exception-packet $(v,S')$ is raised.

\item[exp \verb"handle" match\hfill]  exp is evaluated; if exp returns a
value $v$ with state $S'$, then $v$ is returned with $S'$.  
If exp raises an exception-packet $(e,S'')$ then
the match is applied to $e$ in the state $S''$.
If the match fails, then $(e,S'')$ is raised
(as the value of the \verb"handle" expression).  If the match
succeeds, then the resulting value is returned.
\end{description}

Matching a pattern to a value has no effect on the store.  Evaluating
a value binding has an effect on the store just from the evaluation of
the constituent expressions.  Evaluation of type, datatype, or exception
bindings has no effect on the store.


\chapter{Reference and exception types}
\label{reftype}
The treatment of references and exceptions with ``open'' types
is based on the fact that the contents of a reference
cell cannot be constrained to be polymorphic, and so must be considered
to be monomorphic.  The following example illustrates the problem.
\begin{verbatim}
let val s = ref (fn x => x)
 in s := (fn x => x+1); (!s) true
end
\end{verbatim}
If s were given the polymorphic type 
$\forall \alpha . ( \alpha  \rightarrow \alpha ) {\bf ref}$, then this
expression would type-check, permitting an obvious type error.  To prevent
this, we insist that the type of an applied occurrence of the ref
constructor should always be given a ``ground'' type (one with no locally-bound
type variables).

However, functions whose application can create
reference variables can still have polymorphic types of a restricted
kind.  Consider the declaration
\begin{verbatim}
val F = fn x => let val r = ref x
                 in !r
                end
\end{verbatim}
Here the function F can be given polymorphic type
$\forall \alpha ^ 1 . (\alpha ^ 1 \rightarrow \alpha ^ 1)$
where $\alpha ^ 1$
is a special kind of type variable
called a {\em weak} type variable (the superscript ``1'' indicates that
there is one lambda abstraction suspending the creation of the
ref cell).  When F is applied to an argument, a reference value
of type $\alpha ^ 1$ is created, and hence this weak type
variable must be instantiated to a ground type.  This means that
an expression like \verb"(F nil)" would not be properly typed.  In contrast,
the type 
$\alpha ^ 1 {\bf ref}$
assigned to r is permissible because $\alpha ^ 1$ is implicitly
bound in an outer scope and within the scope of its binding is treated as
a constant type.

In ML, weak type variables will be written \verb"'1a", \verb"'2a",
etc., where the integer after the apostrophe denotes the level of
suspension.

Exception declarations raise similar problems, which are handled
by an analogous use of weak type variables.

\chapter{Modules}
ML provides a powerful module system, which can be used to partition
programs along clean interfaces.

\section{Structures}

In its simplest form, a module is
(syntactically) just a collection of declarations viewed as a unit,
or (semantically) the environment defined by those definitions.
This is one form  of a {\em structure--expression}:
\verb"struct"~dec~\verb"end".  For example, the following structure--expression represents an implementation of stacks:
\begin{verbatim}
struct
   datatype 'a stack = Empty | Push of 'a * 'a stack
   exception Pop and Top
   fun empty(Empty) = true | empty _ = false
   val push = Push
   fun pop(Push(v,s)) = s | pop(Empty) = raise Pop
   fun top(Push(v,s)) = v | top(Empty) = raise Top
end
\end{verbatim}
Structure--expressions and ordinary expressions are distinct classes;
structure--expressions may be bound using the \verb"structure"
keyword to structure--names, while ordinary expressions are
bound using \verb"val" to value--variables.
The form of a \verb"structure" binding is as follows:
\begin{quote}
\verb"structure" name \verb"=" structure--expression
\end{quote}
Thus, we might make a structure Stack using the structure--expression
shown above:
\begin{verbatim}
structure Stack = struct
                    datatype . . .
                    exception . . .
                     . . .
                  end
\end{verbatim}

The environment $E$ that binds the identifiers \verb"stack",
\verb"Pop", \verb"empty", etc. is now itself bound to the
structure--identifier \verb"Stack".  To refer to names in $E$, {\em
qualified identifiers} must be used.  A qualified identifier consists
of a structure--name, a dot, and the name of a structure component,
e.g. \verb"Stack.empty" (a value), \verb"Stack.stack" (a type),
\verb"Stack.Pop" (an exception), etc.

{\bf Structure closure:}  In order to isolate the interface between a
structure and its context, a \verb"struct" phrase is not allowed to
contain global references to types, values, or exceptions, except for
pervasive primitives of the language like \verb"int", \verb"nil",
etc.  It can, however, contain global references to other structures,
signatures, and functors, including qualified names referring to
compenents (values, types, etc.) of other structures.

There are three forms of structure-expression:
\begin{enumerate}
\item An environment enclosed in \verb"struct" ... \verb"end" (as
above),

\item An identifer that has been previously bound in a
\verb"structure" declaration, and

\item A functor application \verb"F("{\it str}\verb")", where
\verb"F" is the name of a functor and {\it str} is a structure
expression.
\end{enumerate}
Thus, the declaration \verb"structure~Pushdown~=~Stack" binds the
name \verb"Pushdown" to the same structure that \verb"Stack" is bound
to; here, \verb"Stack" is an example of the second kind of structure
expression.

\subsection{Accessing structure components}
The bindings making up a structure define named {\em components} of
the structure, as in a record.  To refer to such components we use
qualified names, which are formed by appending a period followed by a
component name to the name of a structure.  For instance,
\verb"Stack.empty" refers to the function \verb"empty" defined in the
structure \verb"Stack".  If the qualified name designates a
substructure of a structure, then it too has components; {\em e.g.}
\verb"A.B.x" denotes the component \verb"x" of the substructure
\verb"B" of a structure \verb"A".

Qualifiers can be attached only to names; they do not apply to other
forms of structure expressions.  Qualified names are treated as
single lexical units; the dot is not an infix operator.

Direct access to the bindings of a structure is provided by the
\verb"open" declaration, which is analagous to the ``with'' clause of
Pascal.  For example, in the scope (determined in the usual way) of
the declaration
\begin{quote}
\verb"open Stack"
\end{quote}
the names \verb"stack", \verb"empty", \verb"pop", etc. refer to the
corresponding components of the \verb"Stack" structure.  It is as
though the body of the structure definition had been inserted in the
program at that point, except that the bindings are not recreated,
but are instead simply ``borrowed'' from the opened structure.
\verb"open" declarations follow the usual rules for visibility, so
that if \verb"A" and \verb"B" are two structures containing a binding
for \verb"x" (of the same flavor, of course), then after opening both
\verb"A" and \verb"B" with the declaration
\begin{quote}
\verb"open A" \\
\verb"open B"
\end{quote}
the unqualified identifier \verb"x"  will be equivalent to
\verb"B.x".  The \verb"x" component of \verb"A" can still be referred
to as \verb"A.x", unless \verb"B" also contains a substructure named
\verb"A".

Qualified identifiers do not have infix status.  If \verb"+" is
declared infix in a structure \verb"A", the qualified identifier
\verb"A.+" is not an infix identifier.  However, when an identifier
is made visible by opening a structure, it retains its infix status,
if any.  AMBIGUOUS.

The declaration \verb"open A B C" is equivalent to \verb"open A; open
B; open C".

\subsection{Evaluating structure expressions}
The evaluation of a structure expression {\em str} depends on its
form, and assumes a current structure environment $SE$ that binds
structures and functors to names.  Informally, evaluation proceeds as
follows:
\begin{enumerate}
\item If {\em str} is an encapsulated declaration ({\it i.e.}
\verb"struct"...\verb"end"), then the body declarations are evaluated
relative to $SE$ and the {\em pervasive} value, exception, and type
environments of ML (that is, the environments binding the built-in
primitives of the language).  The resulting environment is packaged
as a structure and returned.  The evaluation of value bindings may
have an effect on the store (the mapping of references to contents);
the new store is returned as well, to be used in subsequent
expression evaluations.

\item If {\em str} is a simple name, then its binding in $SE$ is
returned.  If it is qualified name, then it is used as an access path
starting with $SE$ and the designated substructure is returned.

\item If {\em str} is a functor application
\verb"F("~$str'$~\verb")", where the functor \verb"F" is declared
by \verb"functor F ( A ) = "~$body$,
the parameter structure $str'$ is
evaluated in $SE$ yielding structure $s_1$; then the ``body'' of the
definition of \verb"M", which is a structure expression, is evaluated
in $SE+\{A \mapsto s_1 \}$.  In other words, functor applications are
evaluated in a conventional call-by-value fashion.
\end{enumerate}

\subsection{Evaluating structure declarations}
To evaluate  a simple structure declaration, one evaluates the
defining structure expression in the current environment $SE$ and
returns the binding of the name of the left hand side to the
resulting structure.  If evaluation of a structure expression raises
an (untrapped) exception, then the declaration has no effect.

\subsection{Structure equivalence}
For certain purposes, such as checking sharing constraints
(Section~\ref{sharing}) we must be able to determine whether two
(references to) structures are equal or ``the same.''  Here
structures are treated somewhat like datatypes; each evaluation of an
encapsulated declaration or functor application creates a distinct
new structure, and all references to this structure are considered
equal.  Thus after the following declarations:
\begin{verbatim}
structure S1 = struct ... end
structure S2 = S1
structure S3 = struct val x = 4 end
structure S4 = struct val x = 4 end
\end{verbatim}
the names \verb"S1" and \verb"S2" refer to the same structure and are
``equal,'' whereas \verb"S3" and \verb"S4" are different structures
and are not equal, even though the right-hand-sides are identical.

\section{Signatures}
It is often useful to explicitly constrain a structure binding to
limit the visibility of its fields.  This is done with a {\em
signature}, which is to a structure binding as a type constraint is to a
value binding.  For example, we might write a signature for the
\verb"Stack" module as
\begin{verbatim}
   sig type 'a stack
       exception Pop and Top
       val Empty : 'a stack
       val push : 'a * 'a stack -> 'a stack
       val empty : 'a stack -> bool
       val pop : 'a stack -> 'a stack
       val top : 'a stack -> 'a
   end
\end{verbatim}
The signature mentions the structure components that will be visible
outside the structure.

Signatures may be bound to identifiers by a signature declaration,
\begin{quote}
\verb"signature" {\it sig-Id} \verb"=" {\it sig-expr}
\end{quote}
where {\it sig-Id} is an identifier and {\it sig-expr} is a signature
expression---either a \verb"sig"...\verb"end" phrase or a previously
bound signature identifier.  Thus, the signature above could be bound
to the identifier \verb"STACK" by the declaration
\begin{verbatim}
signature STACK =
   sig type 'a stack
       exception Pop and Top
	. . .
   end
\end{verbatim}

A signature can be used to constrain a structure by including it in a
structure declaration:
\begin{quote}
\verb"structure" {\it str-id} \verb":" {\it sig-expr} \verb"=" {\it str}
\end{quote}
For example, we could write
\begin{verbatim}
structure Stack1 : STACK = Stack
\end{verbatim}
Now the constructor \verb"Push" is not a visible component of the
\verb"Stack1" structure, since it doesn't appear in the signature;
the qualified identifier \verb"Stack1.Push" is erroneous.
Furthermore, since \verb"stack" is mentioned in the signature only as
a \verb"type" constructor and not as a \verb"datatype" constructor,
the identifier \verb"Stack1.stack" is usable as a type but not a datatype.
Finally, since the constructor \verb"Empty" is mentioned as a
\verb"val" in the signature, but not as a constructor ({\it i.e.} as
part of a datatype specification), then \verb"Stack1.Empty" may be
applied as a function but not matched in a pattern.

There are many signatures that can match the structure \verb"Stack".
One of the ``broadest'' is
\begin{verbatim}
structure Stack2 : sig
		       datatype 'a stack = Empty | Push of 'a * 'a stack
		       exception Pop and Top
		       val empty : 'a stack -> bool
		       val push : 'a * 'a stack -> 'a stack
		       val pop : 'a stack -> 'a stack
		       val top : 'a stack -> 'a
		   end
		= Stack
\end{verbatim}
and the ``narrowest'' is
\begin{verbatim}
structure Stack3 : sig end = Stack
\end{verbatim}
Now, the structure \verb"Stack2" is equivalent to \verb"Stack"; it is
the ``same'' structure, and all the same fields are visible.  The
structure \verb"Stack3" has no components; there are no qualified
identifiers beginning with \verb"Stack3."  However, \verb"Stack3" is
the ``same'' for structure-equivalence purposes as \verb"Stack",
\verb"Stack1", and \verb"Stack2"; signature constraints do not change
the identity of a structure, just which fields are visible.


\appendix
\chapter{Syntax of the full language}
\label{grammar}
{\samepage
\renewcommand{\lhs}[1]{\pagebreak[1] \\ \vspace{2ex}
			 #1 \  \( \rightarrow \) \' }
\newcommand{\also}[0]{\nopagebreak[3] \\}
\begin{tabbing}
longword \( \rightarrow \) \= \{ lab = exp , \rep{0} , lab = exp \} \
\ \ \ \= \kill
\+
\lhs{ide}        ID \> {\em symbolic or alphabetic}
\also      \verb"*" \> {\em \verb"*" is legal as a value-identifer}
\also      =	   \> {\em \verb"=" may be used but not rebound }
\lhs{opid}       ide \>
\also      \res{op} ide \> {\em removes infix status }
\lhs{qualid}     ide
\also      ID . qualid
\lhs{ident}      opid
\also      qualid
\lhs{lab}      ID \>
\also      INT \> {\em numeric labels}
\lhs{const}      INT  \>
\also      REAL   \>
\also      STRING  \>
\also      ()     \>
\also      ident \> {\em nullary constructor}
\lhs{exp}        ident \> {\em variable}
\also      const
\also      \# lab \> {\em field selector }
\also      \{ lab = exp , \rep{0} , lab = exp \} \> {\em record}
\also      ( exp , \rep{2} , exp )  \> {\em tuple }
\also      ( exp ; \rep{1} ; exp )  \> {\em sequence }
\also      \verb"[" exp , \rep{0} , exp \verb"]" \> {\em list}
\also      \res{let} decs \res{in} expseq \res{end} \> {\em local
declaration}
\also      exp exp \> {\em application; left--associative}
\also      exp ide exp \>{\em infixed application}
\also      exp : ty \>{\em type constraint}
\also      exp \res{andalso} exp \>{\em conjunction}
\also      exp \res{orelse} exp \>{\em disjunction}
\also      \res{fn} match \>{\em function}
\also      \res{case} exp \res{of} match \>{\em case expression}
\also      \res{while} exp \res{do} exp \>{\em iteration}
\also      \res{if} exp \res{then} exp \res{else} exp \>{\em conditional}
\also      exp \res{handle} match \>{\em handle exception; right--associative}
\also      \res{raise} exp \>{\em raise exception}
\lhs{match}      pat \verb"=>" exp $\mid$ \rep{1} $\mid$ pat \verb"=>" exp \>
\lhs{apat}       ident  \>{\em  variable binding }
\also      const \>{\em constant pattern}
\also      \verb"_" \>{\em wildcard}
\also      ( pat )
\also      ( pat , \rep{2} , pat) \>{\em tuple}
\also      \{ patfield , \rep{0} , patfield \} \>{\em record}
\also      \{ patfield , \rep{0} , patfield , ... \} \>{\em flexible record}
\also      \verb"[" pat , \rep{0} , pat \verb"]" \>{\em list}
\lhs{pat}        apat \>{\em atomic}
\also      ident apat \>{\em construction; left--associative}
\also      pat ide pat \>{\em infixed construction}
\also      pat : ty \>{\em type constraint}
\also      opid constraint \res{as} pat \>{\em layered}
\lhs{patfield}   lab = pat \>{\em normal}
\also        ID \>{\em abbreviation}
\also        ID \res{as} pat \>{\em abbreviation}
\lhs{ty}         \res{tyvar} \>{\em type variable}
\also      \{ lab : ty , \rep{0} , lab : ty \} \>{\em record type}
\also      ( ty )
\also      ( ty , \rep{2} , ty ) qualid \>{\em type construction}
\also      ty qualid \>{\em unary type construction}
\also      qualid \>{\em nullary type construction}
\also      ty \verb"*" \rep{2} \verb"*" ty \>{\em typle type}
\also      ty \verb"->" ty   \>{\em function type; right--associative}
\lhs{vb}         pat = exp  \>{\em simple }
\also vb \res{and} \rep{1} \res{and} vb \>{\em multiple}
\lhs{constraint}       \>  {\em absent}
\also          : ty \>{\em type constraint}
\lhs{rvb}        opid constraint = \res{fn} match \>{\em simple recursive}
\also rvb \res{and} \rep{1} \res{and} rvb \>{\em mutually recursive}
\lhs{clause}     opid apat \rep{1} apat constraint = exp \>{\em prefix}
\also      pat ide pat constraint = exp \>{\em infix}
\lhs{fb}         clause $\mid$ \rep{1} $\mid$ clause \>{\em clausal function}
\also fb \res{and} \rep{1} \res{and} fb \>{\em mutually recursive}
\lhs{tb}         tyvars ID = ty  \>{\em simple }
\also tb \res{and} \rep{1} \res{and} tb \>{\em multiple}
\lhs{tyvars}    \> {\it absent} 
\also      \res{tyvar} \>{\em single}
\also      ( \res{tyvar} , \rep{2} , \res{tyvar} ) \>{\em multiple}
\lhs{db}         tyvars ID = constr $\mid$ \rep{1} $\mid$ constr \>{\em
simple}
\also  db \res{and} \rep{1} \res{and} db \>{\em mutually recursive}
\lhs{constr}     opid \>{\em nullary (constant)}
\also      opid \res{of} ty \>{\em unary}
\lhs{eb}         ide   \>{\em nullary (constant)}
\also      ide \res{of} ty \>{\em unary}
\also      ide = qualid \>{\em re--naming}
\also eb \res{and} \rep{1} \res{and} eb \>{\em multiple}
\lhs{ldec}       \res{val} vb \>{\em value declaration}
\also      \res{val} \res{rec} rvb \>{\em recursive value declaration}
\also      \res{fun} fb \>{\em function declaration}
\also      \res{type} tb \>{\em type declaration}
\also      \res{datatype} db \>{\em datatype declaration}
\also      \res{exception} eb  \>{\em exception declaration}
\also      \res{local} ldec \res{in} ldec \res{end} \>{\em local
declaration}
\also      \res{open} qualid \rep{1} qualid \>{\em structure visibility}
\also      fixity ide \rep{1} ide  \>{\em directive}
\also      ldec ldec  \>{\em declaration sequence}
\also      ldec \verb";" \>{\em optional semicolon}
\lhs{fixity}     \res{infix} INT \>{\em declare infix,
left--associative}
\also      \res{infix} \>{\em declare infix, precedence 0}
\also      \res{infixr} INT \>{\em infix, right associative}
\also      \res{infixr}  \>{\em infix, right assoc., prec. 0}
\also      \res{nonfix}  \>{\em cancel infix status}
\lhs{sgn}        ID
\also      \res{sig} specs \res{end}
\lhs{specs}      spec
\also      specs spec
\also      specs ;
\lhs{spec}       \res{structure} ID : sgn \res{and} \rep{1} \res{and} ID : sgn
\also      \res{datatype} db \res{and} \rep{1} \res{and} db
\also      \res{type} tyvars ID \res{and} \rep{1} \res{and} tyvars ID
\also      \res{val} ide : ty \res{and} \rep{1} \res{and} ide : ty
\also      \res{exception} exnspec \res{and} \rep{1} \res{and} exnspec
\also      \res{sharing} sharspec \res{and} \rep{1} \res{and} sharspec
\lhs{exnspec}    ide
\also       ide \res{of} ty
\lhs{sharspec}   qualid = \rep{2} = qualid
\end{tabbing}
This description is at present incomplete, as it is missing the
grammar rules for structures and functors.
}

\chapter{The standard library}
\label{library}

A standard set of values, types, exceptions, etc. are {\em
pervasive}---they are in the initial environment and available in
every structure.  This {\em standard library} is grouped into
structures; each structure deals with the operations on one or two
abstract types.  The signature of each of these modules, with an
informal explanation of the semantics, is given in this chapter.

\begin{verbatim}
signature GENERAL =
  sig
    infix 3 o
    infix before
    exception Bind
    exception Match
    exception Interrupt
    exception SystemCall of string
    val callcc : ('a cont -> 'a) -> 'a
    val throw : 'a cont -> 'a -> 'b
    val o : ('b -> 'c) * ('a -> 'b) -> ('a -> 'c)
    val before : ('a * 'b) -> 'a
    datatype 'a option = NONE | SOME of 'a
    type 'a cont
    type exn
    type unit
    infix 4 = <>
    val = : ''a * ''a -> bool
    val <> : ''a * ''a -> bool
  end

abstraction General : GENERAL
\end{verbatim}
The structure \verb"General" contains various miscellaneous and
general-purpose values, types, and exceptions.  The infix \verb"o"
is the function composition operator.  The infix \verb"before" evaluates
both of its arguments and returns the first one.  The exceptions
\verb"Bind" and \verb"Match" are automatically raised by patterns
that fail to match, as explained in chapter~\ref{eval}.  The exception
\verb"Interrupt" is raised when the INTERRUPT signal is received
(e.g. when the user types Control-C or its equivalent).
The functions \verb"callcc" and \verb"throw" and the type \verb"'a cont"
are used for explicit manipulation of continuations, as explained nowhere.
The standard type \verb"exn" 
is the type of all exceptions and \verb"unit" is the type of empty
records.  The \verb"=" and \verb"<>" operators are defined here {\it pro
forma}.  And finally, the \verb"option" datatype is one we have often found
convenient.

\section{List}
\begin{verbatim}
signature LIST =
  sig
    infixr 5 :: @
    datatype 'a list = :: of ('a * 'a list) | nil
    exception Hd
    exception Tl
    exception Nth
    exception NthTail
    val hd : 'a list -> 'a
    val tl : 'a list -> 'a list 
    val null : 'a list -> bool 
    val length : 'a list -> int 
    val @ : 'a list * 'a list -> 'a list
    val rev : 'a list -> 'a list 
    val map :  ('a -> 'b) -> 'a list -> 'b list
    val fold : (('a * 'b) -> 'b) -> 'a list -> 'b -> 'b
    val revfold : (('a * 'b) -> 'b) -> 'a list -> 'b -> 'b
    val app : ('a -> 'b) -> 'a list -> unit
    val revapp : ('a -> 'b) -> 'a list -> unit
    val nth : 'a list * int -> 'a
    val nthtail : 'a list * int -> 'a list
    val exists : ('a -> bool) -> 'a list -> bool
  end
\end{verbatim}
The semantics of this module are defined by the
following implementation.
\begin{verbatim}
abstraction List: LIST =
  struct
    infixr 5 :: @ 
    infix 6 + -
    datatype 'a list = :: of ('a * 'a list) | nil
    exception Hd
    fun hd (a::r) = a | hd nil = raise Hd
    exception Tl
    fun tl (a::r) = r | tl nil = raise Tl    
    fun null nil = true | null _ = false
    fun length nil = 0 | length (a::r) = 1 + length r
    fun op @ (nil,l) = l | op @ (a::r, l) = a :: (r@l)
    fun rev l = let fun f (nil, h) = h 
                      | f (a::r, h) = f(r, a::h)
                in  f(l,nil)
                end
    fun map f = let fun m nil = nil | m (a::r) = f a :: m r
                in  m
                end
    fun fold f = let fun f2 nil = (fn b => b)
                       | f2 (e::r) = (fn b => f(e,(f2 r b)))
                 in  f2
                 end
    fun revfold f l = fold f (rev l)
    fun app f l = (map f l; ())
    fun revapp f l = app f (rev l)
    exception Nth
    fun nth(e::r,0) = e 
      | nth(e::r,n) = nth(r,n-1)
      | nth _ = raise Nth
    exception NthTail
    fun nthtail(e,0) = e
      | nth(e::r,n) = nthtail(r,n-1)
      | nth _ = raise NthTail
    fun exists f =
          let fun g nil = false | g (h::t) = f h orelse g t
          in  g
          end
  end
\end{verbatim}
\section{Array}
\begin{verbatim}
signature ARRAY =
  sig
    infix 3 sub
    type 'a array
    exception Subscript
    val array : int * '1a -> '1a array
    val sub : 'a array * int -> 'a
    val update : 'a array * int * 'a -> unit
    val length : 'a array -> int
    val arrayoflist : 'a list -> 'a array
  end
\end{verbatim}
Arrays may be made whose elements are any type.  \verb"array(n,x)"
returns a new array of $n$ elements, indexed from $0$ to $n-1$,
initialized to $x$.  \verb"a sub i" returns the $i^{th}$ element of
the array $a$.  \verb"update(a,i,z)" sets the $i^{th}$ element of the
array $a$ to the value $z$.

Two arrays are equal if and only if they are the same array (created
with the same call to \verb"array"); except that all arrays of length
0 may be equal to each other, depending on the implementation.

The following implementation defines the semantics of arrays, though
in practice arrays are implemented much more efficiently.
\begin{verbatim}
abstraction Array : ARRAY =
  struct
   type 'a array = 'a ref list
   exception Subscript
   fun array(0,x) = nil | array(n,x) = ref x :: array(n-1,x)
   fun a sub i = !(nth(a,i)) handle Nth => raise Subscript
   fun update(a,i,z) = nth(a,i) := z handle Nth => raise Subscript
   fun length a = List.length a
   fun arrayoflist l = l
  end
\end{verbatim}
\section{Input/Output}
The input/output primitives are intended as a simple basis that may
be compatibly superseded by a more comprehensive I/O system that
provides for streams of arbitrary type or a richer repertoire of I/O
operations.  The IO structure contains all I/O primitives; this
structure will in all implementation
match (with thinning) the BASICIO signature provided
below, but may contain other primitives as well.

\begin{verbatim}
signature BASICIO = 
  sig
    type instream 
    type outstream
    exception Io of string
    val std_in : instream
    val std_out : outstream
    val open_in : string -> instream
    val open_out : string -> outstream
    val close_in : instream -> unit
    val close_out : outstream -> unit
    val output : outstream -> string -> unit
    val input : instream -> int -> string
    val lookahead : instream -> string
    val end_of_stream : instream -> bool
  end

\end{verbatim}
The type \verb"instream" is the type of input streams and the type
\verb"outstream" is the type of output streams.  The exception
\verb"Io" is used to represent all of the errors that may
arise in the course of performing I/O.  The value associated with
this exception is a string representing the type of failure.  In
general, any I/O operation may fail if, for any reason, the host
system is unable to perform the requested task.  The value associated
with the exception should describe the type of failure, insofar as
this is possible.

The standard prelude binds \verb"std_in" to an instream and
\verb"std_out" to an outstream.  For interactive ML processes, these
are expected to be associated with the user's terminal.  However, an
implementation that supports the connection of processes to streams
may associate one process's \verb"std_in" to another's
\verb"std_out".

The \verb"open_in" and \verb"open_out" primitives are used to
associate a disk file with a stream.  The expression
\verb"open_in(s)" creates a new instream whose producer is the file
named \verb"s" and returns that stream as a value.
Similarly, \verb"open_out(s)" creates a new \verb"outstream"
associated with the file \verb"s", and returns that stream.

The \verb"input" primitive is used to read characters from a stream.
Evaluation of \verb"input s n" causes the removal of \verb"n"
characters from the input stream \verb"s".  If fewer than \verb"n"
characters are currently available, then the ML system will block
until they become available from the producer associated with
\verb"s"\footnote{The exact definition of ``available'' is
implementation-dependent.  For instance, operating systems typically
buffer terminal input on a line-by-line basis so that no characters
are available until an entire line has been typed.}.
If the end of stream is reached while processing an \verb"input",
fewer than \verb"n" characters may be returned.  
Attempting \verb"input" from a closed stream raises
\verb"Io".

The function \verb"lookahead(s)" returns the next character on
\verb"instream s" without removing it from the stream.  Input streams
are terminated by the \verb"close_in" operation.  This primitive is
provided primarily for symmetry and to support the re-use of
unused streams on resource-limited systems.  The end of an input
stream is detected by \verb"end_of_stream", a derived form that is
defined as follows:
\begin{verbatim}
fun end_of_stream(s) = (lookahead(s)="")
\end{verbatim}

Characters are written to an \verb"outstream" with the \verb"output"
primitive.  The string argument consists of the characters to be
written to the given outstream.  The function \verb"close_out" is
used to terminate an output stream.  Any further attempts to output
to a closed stream cause \verb"Io" to be raised.

\begin{verbatim}
signature IO =
  sig
    type instream 
    type outstream
    exception Io of string
    val std_in : instream
    val std_out : outstream
    val open_in : string -> instream
    val open_out : string -> outstream
    val open_append : string -> outstream
    val open_string : string -> instream
    val close_in : instream -> unit
    val close_out : outstream -> unit
    val output : outstream -> string -> unit
    val input : instream -> int -> string
    val input_line : instream -> string
    val lookahead : instream -> string
    val end_of_stream : instream -> bool
    val can_input : instream -> int
    val flush_out : outstream -> unit
    val is_term_in : instream -> bool
    val is_term_out : outstream -> bool
    val set_term_in : instream * bool -> unit
    val set_term_out : outstream * bool -> unit
    val execute : string -> instream * outstream
    val exportML : string -> bool
    val exportFn : string * (string list * string list -> unit) -> unit
    val use : string -> unit
    val use_stream : instream -> unit
    val reduce : ('a -> 'b) -> ('a -> 'b)
    val mtime : instream -> int
(* the following are temporary components *)
    val reduce_r : ((unit -> unit) -> (unit -> unit)) ref
    val cleanup : unit -> unit
    val use_f : (string -> unit) ref
    val use_s : (instream -> unit) ref
  end

structure IO : IO
\end{verbatim}

In addition to the basic I/O primitives, provision is made for some
extensions that are likely to be provided by many implementations.
The functions listed above are provided by Standard ML of New Jersey.

The function \verb"execute" is used to create a pair of streams, one an
\verb"instream" and one an \verb"outstream", and associate them with
a process.  The string argument to \verb"execute" is the
operating-system command that starts the process.

The function \verb"flush_out" ensures that the consumer associated
with an \verb"out_stream" has received all of the characters
associated with that stream.  It is provided primarily to allow the
ML user to circumvent undesirable buffering characteristics that may
arise in connection with terminals and other processes.  All output
streams are flushed when they are closed, and in many implementations
an output stream is flushed whenever a newline character is written
if that stream is connected to a terminal.

The function \verb"can_input" returns the number of characters
which may be read from its instream argument without blocking.
For instance, a command processor may wish
to test whether or not a user has typed ahead in order to avoid an
unnecessary prompt.  The exact definition of ``currently available''
is implementation specific, perhaps depending on such things as the
processing mode of a terminal.

The \verb"input_line" primitive returns a string consisting of the
characters from an \verb"instream" up through, and including, the
next end of line character.  If the end of stream is reached without
reaching an end of line character, all remaining characters from the
stream ({\em without} an end of line character) are returned.

Files may be open for output while preserving their contents by using
the \verb"open_append" primitive.  Subsequent \verb"output" to the
outstream returned by this primitive is appended to the contents of
the specified file.

Basic support for the complexities of terminal I/O are provided.  The
pair of functions \verb"is_term_in" and \verb"is_term_out" test
whether or not a stream is associated with a terminal; and \verb"set_term_in"
and \verb"set_term_out" tell the ML system that a stream is (or is not)
a terminal.  These
functions are especially useful with \verb"std_in" and \verb"std_out"
because they are opened as part of the standard prelude.  A terminal
may be associated with a stream using the ordinary \verb"open_in" and
\verb"open_out" functions; the naming convention to do this is
implementation-dependent.  

Given a name of a file, \verb"use" compiles
and executes its contents as if they were typed into the top-level
prompt of the interactive system.  \verb"use" may be nested
recursively.  Similarly, \verb"use_stream"
compiles an already-opened instream.

\verb"exportML" creates an executable file whose name is
given by the argument.  When this file is executed, it is an ML
system in exactly the same state as the one that wrote the file.  For
example, the command
\verb|(exportML "foo"; print "Hello");| writes a file that, when
executed, prints \verb"Hello" and then returns to the top-level
prompt.  exportML returns true when the executable file is run,
and false when simply returning.

\verb"exportFn"  creates an executable file whose name is given
by the first argument.  When this file is executed, it is an ML
system that calls the function given as the second argument, then
exits.  The ML system created will not have a compiler or a
top-level, so it will be significantly more compact.
The command-line arguments and environment
are passed as the string list arguments to the
function that is called.   \verb"exportFn" terminates
execution of the ML system that called it.

\section{Bool}
\begin{verbatim}
signature BOOL =
  sig
    datatype bool = true | false
    val not: bool -> bool
    val print: bool -> bool
    val makestring: bool -> string
  end
\end{verbatim}
These are quite straightforward, and can be defined as follows:
\begin{verbatim}
abstraction Bool: BOOL =
  struct
    datatype bool = true | false
    fun not true = false | not false = true
    fun makestring true = "true" | makestring false = "false"
    fun print b = (output(std_out, makestring b); b)
  end
\end{verbatim}
\section{ByteArray}
\begin{verbatim}
signature BYTEARRAY =
  sig
    infix 3 sub
    eqtype bytearray
    exception Subscript
    exception Range
    val array : int * int -> bytearray
    val sub : bytearray * int -> int
    val update : bytearray * int * int -> unit
    val length : bytearray -> int
    val extract : bytearray * int * int -> string
    val fold : ((int * 'b) -> 'b) -> bytearray -> 'b -> 'b
    val revfold : ((int * 'b) -> 'b) -> bytearray -> 'b -> 'b
    val app : (int -> 'a) -> bytearray -> unit
    val revapp : (int -> 'b) -> bytearray -> unit
  end
\end{verbatim}
Byte arrays are just like arrays of integers, with the restriction
that the values of the component integers must be between 0 and 255.
The intent is that the implementation may store them more efficiently
than the equivalent array.

Note that, by default, the ByteArray structure is present but 
not opened in the 
initial environment.  The declaration \verb"open ByteArray" may be
used to open it.  The use of ByteArray is discouraged; future versions of the
compiler may not support it, or (for example) debuggers might
not support it.

The semantics can be defined by this implementation:
\begin{verbatim}
abstraction ByteArray : BYTEARRAY =
  struct
    infix 3 sub
    type bytearray = int array
    exception Subscript = Array.Subscript
    exception Range
    fun check x = if x<0 orelse x>255 then raise Range else ()
    fun array(i,x) = (check x; Array.array(i,x))
    val length = Array.length
    fun update(a,i,x) = (check x; Array.update(a,i,x))
    val op sub = Array.sub
    fun extract(b,i,0) = if i<0 orelse i>length(b)
                          then raise Subscript  else ""
      | extract(b,i,n) = chr(b sub i) ^ extract(b,i,n-1)
    val fold =  . . .
    val revfold =  . . .
    val app = ...
    val revapp = ...
  end
\end{verbatim}

\section{Integer}
\begin{verbatim}
signature INTEGER = 
  sig
    infix 7 * div mod
    infix 6 + -
    infix 4 > < >= <=
    exception Div
    exception Overflow
    type int
    val ~ : int -> int
    val * : int * int -> int
    val div : int * int -> int
    val mod : int * int -> int
    val + : int * int -> int
    val - : int * int -> int
    val >  : int * int -> bool
    val >= : int * int -> bool
    val <  : int * int -> bool
    val <= : int * int -> bool
    val min : int * int -> int
    val max : int * int -> int
    val abs : int -> int
    val print : int -> unit
    val makestring : int -> string
  end
\end{verbatim}
This should be mostly self-explanatory.
The function \verb"div" raises \verb"Div" on divide by zero,
otherwise \verb"Overflow" if the result doesn't fit;  similarly
\verb"mod" may raise \verb"Div" or \verb"Overflow".  Other operators
may raise \verb"Overflow" if the result doesn't fit into their
representation.
Standard ML of New Jersey uses finite precision signed 31-bit integers,
which can represent a range from $-2^{30}$ to $2^{30}-1$.
\section{Real}
\begin{verbatim}
signature REAL =
  sig
    infix 7 * /
    infix 6 + -
    infix 4 > < >= <=
    type real
    exception Floor and Sqrt and Exp and Ln
    exception Real of string
    val ~ : real -> real 
    val + : (real * real) -> real 
    val - : (real * real) -> real 
    val * : (real * real) -> real 
    val / : (real * real) -> real 
    val > : (real * real) -> bool
    val < : (real * real) -> bool
    val >= : (real * real) -> bool
    val <= : (real * real) -> bool
    val abs : real ->  real
    val real : int -> real
    val floor : real -> int
    val truncate : real -> int
    val ceiling : real -> int
    val sqrt : real -> real
    val sin : real -> real
    val cos : real -> real
    val arctan : real -> real
    val exp : real -> real
    val ln : real -> real
    val print : real -> unit
    val makestring : real -> string
  end
structure Real : REAL
\end{verbatim}

This should be mostly self-explanatory.  Except for the special
exceptions \verb"Floor", \verb"Sqrt", \verb"Exp", \verb"Ln", raised by
the functions of the corresponding names, all real-number functions
raise only the \verb"Real" exception with some system dependent
argument string.

\section{Ref}
\begin{verbatim}
signature REF = 
  sig
    infix 3 :=
    val ! : 'a ref -> 'a
    val := : 'a ref * 'a -> unit
    val inc : int ref -> unit
    val dec : int ref -> unit
  end
\end{verbatim}

Reference values are described in chapter~\ref{reference}.  The functions
\verb"inc" and \verb"dec" can be defined as
\begin{verbatim}
  fun inc i = i := !i+1
  fun dec i = i := !i-1
\end{verbatim}

\section{String}
\begin{verbatim}
signature STRING =
  sig
    infix 6 ^
    infix 4 > < >= <=
    type string
    exception Substring
    val length : string -> int
    val size : string -> int
    val substring : string * int * int -> string
    val explode : string -> string list
    val implode : string list -> string
    val <= : string * string -> bool
    val <  : string * string -> bool
    val >= : string * string -> bool
    val >  : string * string -> bool
    val ^  : string * string -> string
    exception Chr
    val chr : int -> string 
    exception Ord
    val ord : string -> int 
    val ordof : string * int -> int 
    val print : string -> string
  end
\end{verbatim}
Strings can be explained by the implementation below; of course, in
practice a more efficient implementation is used.
\begin{verbatim}
abstraction String : STRING =
  struct
    infix 6 ^
    infix 4 > < >= <=
    type string = int list
    exception Substring
    val length = List.length
    val size = length
    fun substring(s,0,0) = nil
      | substring(a::b,0,len) = a::substring(b,0,len-1)
      | substring(nil,_,_) = raise Substring
      | substring(a::b,i,len) = substring(b,i-1,len)
    fun explode nil = nil
      | explode (i::l) = [i] :: explode l
    fun implode nil = nil
      | implode (s::l) = s @ implode l
    fun (_::_) > nil = true
      | nil > (_::_) = false
      | (i::r) > (j::s) = Integer.>(i,j) orelse i=j andalso r>s
    fun a <= b = not (a>b)
    fun a < b = b > a
    fun a >= b = b <= a
    val op ^ = op @
    exception Chr
    fun chr i = if i<0 orelse i>255 then raise Chr else [i]
    exception Ord
    fun ord nil = raise Ord | ord (i::r) = i
    fun ordof(s,i) = nth s handle Nth => raise Ord
    fun print s = (output(std_out,s); s)
  end
\end{verbatim}
\section{Bits}
\begin{verbatim}
signature BITS =
  sig
    type int
    val orb : int * int -> int
    val andb : int * int -> int
    val xorb : int * int -> int
    val lshift : int * int -> int
    val rshift : int * int -> int
    val notb : int * int -> int
  end
structure Bits : BITS
\end{verbatim}
The structure Bits allows shifting and masking of integers (viewed as 
strings of binary digits).  This structure is present but
{\em not} opened in the standard environment; its use is discouraged.
The right shift (rshift) operator may shift 0's, 1's, or sign bits
into the left end of an integer at its discretion.
\section{System}
\begin{verbatim}
signature SYSTEM =
  sig
    structure Control : CONTROL
    structure Tags : TAGS
    structure Timer : TIMER
    structure Stats : STATS
    structure Unsafe : UNSAFE
    val exn_name : exn -> string
    val version : string
    val interactive : bool ref
    val cleanup : unit -> unit
    val system : string -> unit
    val cd : string -> unit
    val argv : unit -> string list
    val environ : unit -> string list
  end
structure System : SYSTEM
\end{verbatim}
Features of Standard ML of New Jersey
that should not be expected in any other implementation of ML
are grouped into the System structure, which is present but not opened
in the standard environment.

The substructures \verb"Control" of \verb"System" are not documented.

The function \verb"exn_name" returns the name of the exception constructor
that was used to build a given exception value.  The \verb"version" string
indicates which version of Standard ML of New Jersey is running.
The variable \verb"interactive" may be set to indicate whether the
compiler's input stream should be treated as interactive (i.e. issue
primary and secondary prompts, read a line at a time) or non-interactive
(i.e. no prompts, read a large block at a time).  The \verb"cleanup"
function closes all files.  The \verb"system" function runs an operating-system
(shell) command specified by its argument.  \verb"cd" changes the
current working directory.  \verb"argv" and \verb"environment" return
the command-line
argument-list and (Unix) environment with which the Standard ML 
process was created.

\chapter{Compatibility}
The language definition has changed a bit over the past year
(1986--87), particularly in the area of exception syntax.  Some
attempt has been made to allow old programs to continue running.
\section{Exceptions}
Three keywords: \verb"exceptionx", \verb"raisex", and \verb"handlex"
are provided.  They implement the old-style exception mechanism, and
are treated as derived forms.
\begin{tabular}{@{}l l}
\multicolumn{1}{c}{\bf Derived Form}&
\multicolumn{1}{c}{\bf Equivalent Form} \\ \hline
\verb"exceptionx" identifier & \verb"exception" Identifier \\
\verb"exceptionx" identifier \verb":" ty & \verb"exception" Identifier \verb"of" ty \\

\verb"raisex" identifier & \verb"raise" Identifier \\
\verb"raisex" identifier \verb"with" exp& \verb"raise" Identifier \verb"(" exp \verb")" \\

\verb"handlex" ident \verb"=>" exp & \verb"handle" Ident \verb"=>" exp \\
\verb"handlex" ident \verb"=>" exp & \verb"handle" Ident () \verb"=>" exp \\

\verb"handlex" ident \verb"with" pat \verb"=>" exp & \verb"handle" Ident pat \verb"=>" exp \\
\hline
\end{tabular}

The derivations for \verb"handlex" are a bit more intricate than
shown above.  The intent is that any program that works under the old
scheme will continue to work if all instances of \verb"exception",
\verb"handle", and \verb"raise" are changed to 
 \verb"exceptionx", \verb"handlex", and \verb"raisex" respectively.

Note that the derived forms change the exception identifier in
converting to the standard forms; they capitalize the first letter.
This is to simulate the old scheme, in which exception names lived in
a different space from value names.  This will cause problems in any
programs with a capitalized value-name that happens to conflict with
a (capitalized or uncapitalized) exception name.

\end{document}
