\documentstyle [twoside]{report}
%\includeonly{library,compat}
\title{Standard ML Reference Manual (PRELIMINARY)}
\author{}
\date{\today}


\begin{document}
\newcommand{\xskip}{\vspace{1ex}}
\newcommand{\res}[1]{{\tt #1}}
\newcommand{\rep}[1]{\underline{\  {\footnotesize #1}  \ }}
\newcommand{\lhs}[1]{\pagebreak[1] \item[#1 \  \( \rightarrow \) ] }
\def\description{\list{}{\labelwidth 1.3in \labelsep 0.3in
                         \leftmargin 1.6in
 \let\makelabel\descriptionlabel}}
\maketitle
\begin{abstract}
This manual is a major revision by Andrew W. Appel
of the {\em Standard ML} reference manual (ECS-LFCS-86-2).  That
document is divided into three parts: the Core language description
by Robin Milner, the standard I/O library by Robert W. Harper,
and the Module system by David B. MacQueen.

I have attempted to integrate the three parts, add more detail to the
description of the standard library, bring the documentation
up to date with the recent changes to the language ({\em e.g.}
exceptions), more clearly explain the semantics ({\em e.g.}
references), etc.

At present the manual is in a very rough form, and should be
considered a preliminary draft.  This version is distributed
primarily as documentation of the February 1988 distribution of the
{\it Standard ML of New Jersey} compiler.
\end{abstract}
%!TEX root = main.tex


\chapter{Introduction}\label{ch:intro}

Program modularity is not a radical idea. From early on, programmers have established from experience that breaking programs apart into smaller more managable pieces is a good idea. Steadily, this idea found its way into the design of programming languages. The ML module system is a powerful language for constructing and abstracting over modules in statically-typed languages. Decomposing a program into modules should not compromise the type-safety guarantees afforded by modern statically-typed languages. The module system itself must have a type system that ensures type-safe composition. 

A module system typically partitions code into interfaces (also called signatures) and implementations. Signatures describe the shape of the implementation by listing values and types exported from modules. Implementations rely on signatures to help typecheck occurrences of imported names. A central question in module system design is what should go into signatures. On the one hand, signatures must provide information to typecheck external module dependencies. On the other, signatures should not contain too much information so as to compel the programmer to recapitulate the implementation of the dependency in the corresponding signature. 

The main subject of this dissertation is the type system and semantics of an intuitive variant of the ML module system, the true higher-order module system (THO). As I will show, this module system is both type-safe and intuitive in that its semantics is based on $\beta$-reduction. 
The implementation of the higher-order module system has evolved considerably since the seminal MacQueen-Tofte paper~\cite{mt94} which was the original description of true higher-order semantics. The goal of the rest of this dissertation is to formalize, simplify, extend, and improve the static semantics of true higher-order module systems. 
%I focus on changes in module representations and improvements in elaboration algorithms. 
This dissertation studies what this true higher-order semantics means precisely and formally by giving a modern formal account. These results open the way to exploration of the design space of \emph{full transparency} (\emph{i.e.}, exactly what can and should propagate through higher-order functor applications) and separate compilation. 
	
\section{Higher-Order Module System}\label{sec:hofct}
	Higher-order modules extend the ML module system with higher-order functors, a natural extension. There are many different applications and three general approaches for higher-order modules~ \cite{tofte92,tofte:jfp94,mt94,biswas95,leroy95,Leroy:generativity,LillibridgeThesis,russothesis,dreyerthesis} which differ mainly on how they handle type sharing in the apply functor (fig.~\ref{fig:applyfctocaml}). 

\lstdefinestyle{ocamlcode}{language=[Objective]Caml}%, numbers=left, numberstyle=\tiny, stepnumber=1, numbersep=5pt}
\begin{figure} 
\begin{lstlisting}[style=ocamlcode]
module type T = sig type t end
module Id = functor(X:T) -> X

module type FS = functor (X:T) -> T

module Apply = functor(X:sig module F : FS module M:T end) -> X.F(X.M)

module M0 = struct type t = unit end	
module M1 = Apply(struct module F=Id module M=M0 end)

let x : M1.t = ();;
\end{lstlisting}
\caption{Apply functor example in OCaml}
\label{fig:applyfctocaml}
\end{figure}
		
In the apply functor example, the higher-order functor Apply applies functor F to a module M. Tofte~\cite{tofte92} introduced the first cut at the semantics for higher-order modules with \emph{non-propagating} functors. Under Tofte's semantics, the apply functor does not propagate the type M0.t through the functor application X.F(X.M), thus M1.t $\ne$ M0.t = unit. It assigns the signature in fig.~\ref{fig:tofteapplysig}. to the apply functor. Notice that the signature for the body does not say anything more about \lstinline{type t}. 
\begin{figure}
\begin{lstlisting}
functor Apply(X: sig functor F : (X: T) : T structure M : T end) : T
\end{lstlisting}
\caption{Signature assigned to Apply functor under MacQueen-Tofte semantics}
\label{fig:tofteapplysig}
\end{figure}

MacQueen-Tofte~\cite{mt94} argues that the type sharing M1.t = M0.t = unit should hold. The strong sums model of modules also predicts this behavior of full type propagation \cite{macqueen:popl86}. The semantics in MacQueen-Tofte, \emph{true higher-order functors} or \emph{fully transparent generative functors}, re-elaborates the functor body of apply given the contents of X.F and X.M at the point of the functor application on line 14. Although this re-elaboration has the desired effect of propagating types, MacQueen-Tofte assigns exactly the same functor signature as the Tofte semantics. Leroy~\cite{leroy95} offered an alternative approach, \emph{applicative functors}, that enriched the notion of type paths in functor signatures with functor applications such as F(M).t. Applicative functor semantics assigns the functor signature in fig.~\ref{fig:appapplysig}. 

\begin{figure}
\begin{lstlisting}
functor Apply(X : sig functor F : FS structure M : T end) :
          sig type t = X.F(X.M).t end
\end{lstlisting}
\caption{Signature assigned to Apply functor under applicative functor semantics}
\label{fig:appapplysig}
\end{figure}

However, applicative functors only solve the type propagation problem under certain circumstances and loses the generative semantics of functors, \ie, functor applications do not generate fresh tycons to enforce abstraction. To address this shortcoming, Moscow ML and Dreyer's module system~\cite{dhc03} combined applicative and non-propagating generative higher-order functors in the same language. Fully transparent generative functors both solve the type propagation problem under all circumstances and do not compromise on generative functor semantics. In the last decade, researchers have gained significant experience in engineering non-fully transparent generative functors and transparent applicative functors in compilers such as Moscow ML and OCaml. Although they differ internally, both SML/NJ and MLton compilers support some variant of true higher-order functors semantics. I take OCaml and SML/NJ as representatives of the former and latter groups respectively.
	 
Under both OCaml and SML/NJ compilers, the apply functor example typechecks. Furthermore, although applicative functors cannot directly handle applications to nonpaths, lambda lifting the offending nonpath generally solves that issue. However, if I modify the apply functor slightly by applying a formal functor F to \lstinline{struct type t = int end} as in fig.~\ref{fig:hoapplyfct}, then applicative functors fail to propagate enough type information. When the typechecker gets to line 3 in fig.~\ref{fig:hoapplyfct}, it does not have enough information about F to give a stronger signature for ApplyToInt's functor body. Neither is there a path to the argument \lstinline{struct type t = int end} used to construct an applicative functor path. Because the typechecker must give the functor body a signature immediately in order to give HO functor ApplyToInt a complete signature, it can only give the weakest signature, \lstinline{sig type t end}. A-normalization gets the program to typecheck but unnecessarily clutters up the code. Thus, in this sense, applicative functors cannot be said to be fully transparent. The SML/NJ compiler has no such restrictions.

\begin{figure}
\hrule
\begin{lstlisting}[style=ocamlcode]
module ApplyToInt = functor (F : functor (X:T) -> T) -> F(struct type t = int end)

module R = ApplyToInt(Id)
let x : R.t = 5;;
\end{lstlisting}

OCaml functor signature for ApplyToInt
\begin{lstlisting}
module ApplyToInt : functor (F : functor (X : T) -> T) -> sig type t end
\end{lstlisting}
OCaml type error:
\begin{verbatim}
This expression has type unit but is here 
used with type R.t = ApplyToInt(Id).t
\end{verbatim}
SML/NJ functor signature for ApplyToInt
\begin{lstlisting}
functor ApplyToInt(functor F : (X: T) : T end) : T

structure R = ApplyToInt(Id)
val x : R.t = 5
\end{lstlisting}
SML/NJ types x as R.t 
\hrule
\caption{OCaml and SML/NJ versions of ApplyToInt functor}
\label{fig:hoapplyfct}
\end{figure}

The example in figs.~\ref{fig:hoapplyfct} illustrates the fundamental problem with applicative functors. A higher-order ApplyToInt functor fails to properly propagate types under Leroy's applicative functor semantics. Consequently, the last line fails to typecheck. They work well as long as the relationship between functor parameter and body is simple, \ie, can be captured in the extended notion of a path with functor application or by A-normalizing the argument. However, not all possible functors fit into this mold. Having to A-normalize arguments of functor applications is an unnecessary shortcoming. Making an argument structure a local definition for the functor application does not help because then after the functor application, the argument structure name would be out of scope. True higher-order functors propagate types across all functor applications with no change to the source. The solution that MacQueen and Tofte \cite{mt94} advocate is a re-elaboration of the functor body given the actual argument module.

 \begin{figure}
\begin{center}
 \includegraphics[scale=0.5]{../design/figs/appandtruetypeprogvenn.pdf}
\end{center}
 \caption{Examples of applicative functors and true higher-order functors}
 \label{fig:appandtruetypeprogvenn}
 \end{figure}

% Show example. explanation of related work should be self-contained "mini-essays"
In instances such as the SymbolTable functor (fig.~\ref{fig:symtbl}), applicative functors admit too much sharing as noted by Dreyer \cite{dreyerthesis}. Applicative functors would permit the symbols from one SymbolTable ST1 to be used to index another ST2 despite the sealing of the functor body to signature SYMBOL\_TABLE. Both OCaml and SML/NJ elaborators will make the symbol type abstract, but in OCaml it is only abstract with respect to external clients and not other instances of SymbolTable. Because SymbolTable is applied to the same empty argument for both ST1 and ST2, the two instances also share the same symbol type according to applicative functor semantics. This behavior breaks an important abstraction. Generative functors are more appropriate for enforcing the exact kind of abstraction desired. In contrast, applicative functor semantics are appropriate for some purposes such as the Set functor in fig.~\ref{fig:setfct} where the type sharing of Item.item is desirable and it is acceptable to use items in Sets of the same type interchangeably. However, it is debatable whether the Set functor is a common case. The set of programs T for which true higher-order functors propagate types is exactly those one would want to propagate types. T partially overlaps with the set A the programs such that applicative functor semantics propagates types. In contrast, one should not propagate types in the set $A\\T$. Dreyer~\cite{dreyerthesis} noted correctly that applicative functors and non-fully transparent generative functors are incomparable. Neither applicative nor true higher-order functors can subsume the other. However, there remains the pragmatic question whether the programs that propagate types under applicative functors but not under true higher-order functors ought to have propagated the types in those cases. 

\begin{figure}
\hrule
~
\begin{center}
\begin{tabular}{c}
\begin{lstlisting}
signature SYMBOL_TABLE = 
sig 
  type symbol 
  val string2symbol : string -> symbol 
  val symbol2string : symbol -> string 
  ... 
end 

functor SymbolTable () = 
struct 
  type symbol = int 
  val table : HashTable.t = HashTable.create (initial size, NONE) 
		(* allocate internal hash table *)		
  fun string2symbol x = 
		(* lookup (or insert) x *) ... 
  fun symbol2string n = 
	(case HashTable.lookup (table, n) of 
		  SOME x => x 
		| NONE => raise (Fail "bad symbol")) 
    ... 
end :> SYMBOL_TABLE 

structure ST1 = SymbolTable() 
structure ST2 = SymbolTable() 
\end{lstlisting}
\end{tabular}
\end{center}
\hrule
\caption{SymbolTable functor example from Dreyer~\cite{dreyerthesis}}
\label{fig:symtbl}	
\end{figure}

\begin{figure}
\hrule
~
\begin{center}
\begin{tabular}{c}
\begin{lstlisting}
signature COMPARABLE = 
sig 
  type item 
  val compare : item * item -> order 
end 
functor Set (Item : COMPARABLE) = 
struct 
  type set = Item.item list 
  val emptyset : set = [] 
  fun insert (x : Item.item, S : set) : set = x::S 
  fun member (x : Item.item, S : set) : bool = ... Item.compare(x,y) ... 
  ... 
end 
\end{lstlisting}
\end{tabular}
\end{center}	
\hrule
\caption{Set functor example from Dreyer~\cite{dreyerthesis}}
\label{fig:setfct}
\end{figure}

The original criticisms of the MacQueen-Tofte semantics are that it lacks
support for true separate compilation and that the stamp-based
operational semantics makes it difficult to extend the module system
and to reason about it. Many recent treatments of ML module systems
abandon true higher-order functors completely due to these
issues. The claim is that the type-theoretic presentations of the
module system with applicative functors address these problems. This
dissertation will consider the question whether an operational
semantics account must necessarily be more complicated and if so,
why. In contrast to recent work, this dissertation will take true
higher-order module behavior as the starting point for developing a formal semantics while addressing these criticisms and concerns. My formalism follows behavior of the SML/NJ compiler which enriches the internal representation of functors and functor signatures to express the static actions of the functor, thus avoiding a re-elaboration of the functor body. This approach will also yield some practical benefits. The SML/NJ and MLton implementations have not kept up with the pace of the progress in module system design at least partially due to the fact that most of the research has been a radical departure from true higher-order module semantics. Reframing the state-of-the-art in terms of true higher-order module semantics will bring recent developments closer a practical extension in these production-quality compilers. 

\section{Full Transparency and True Separate Compilation}
		Since MacQueen and Tofte introduced true higher-order modules, many researchers~\cite{leroy94,russothesis,mixml} have studied how to downgrade higher-order functors to regain true separate compilation, \ie, Modula-2-style separate compilation. To workaround this limitation, SML/NJ \cite{am:pldi94,hlpr:tr94} uses a powerful compilation manager CM \cite{blume95:cm}, which uses a pickled static environments that include complete static descriptions going beyond syntactic signatures. However, using a pickled static environment does not solve the true separate compilation problem. 
		
		The separate compilation problem can be reframed as a completeness problem for the signature language, \ie, can the source-level signature language adequately describe all possible modules including functors. Currently, the SML/NJ compiler elaborates module syntax into internal semantic objects. These semantic objects are expressive enough to encode the functor body relationships that eluded the source signature language. The source and these semantic objects are then compiled to a predicative System F$_\omega$-like calculus \cite{shao98}. This suggests that an F$_\omega$-like calculus should be expressive enough to characterize all the static semantic actions of functors. In chapter~\ref{ch:entitycalc}, I describe the entity calculus, a small language that precisely complements the signature calculus and fully describes module types. The semantics is a foundation for future work on the question of separate compilation in higher-order module systems. 
		
		Intuitively, true higher-order modules cannot be fully expressed in the syntactic signature language because it is limited to definitional specs and type sharing. For example, there is no way to express a functor signature for the apply functor that accounts for all sharing due to full transparency. Therefore true higher-order modules cannot in general be separately compiled. In particular, HO functor applications cannot always be separately compiled from the functor definition. In the past, various researchers have approached this problem by incorporating applicative functors into the language to varying degrees \cite{leroy95,biswas95,russothesis,dhc03} sometimes limiting the generative functors in the process. In this dissertation, I will argue that applicative functors cannot replace true higher-order functors in the general case. Moreover, if fully transparent generativity is the goal, then applicative functors only serve to support true separate compilation in a limited number of cases. 
		
		% In order to have true separate compilation, the surface signature language must be able to express the full signature of all structures and functors. Even in a module language that only supports first-order functors, this requirement proves to be a problem because the signature language is be unable to express generative types in the body of a functor. Generative types in the body of a functor do not have externally expressible names prior to functor application. 
		
		
% \subsection{Signature calculus}
% 	Since the study of the true separate compilation problem points in the direction of the signature calculus, it will be fruitful to take this opportunity to reconsider the design of ML's signature language. After Harper-Lillibridge and Leroy, despite the continuing pace of the development of ML module systems, the signature language generally did not see much attention except for Ramsey \etal's paper\cite{ramsey05}. Ramsey \etal~\cite{ramsey05} describe a signature language that includes operations for post hoc manipulation such as adding, removing, rebinding components, and merging signatures. SML/NJ's semantics for \lstinline{include} is richer than the simple syntactic inclusion found in the Definition\cite{mthm97}. In particular, certain kinds of compatible signatures can be merged. In the SML/NJ 110.68 compiler, two signatures are {\bf compatible} when their overlapping specifications (\ie, specifications with the same name) have the same arity and follow the rules summarized in table~\ref{tbl:njsigmerge}. However, the current compatibility rules are inconsistent and incomplete. For example, merging an eqtype and a type specification results in an eqtype in one direction and a type in the other as shown in fig.~\ref{fig:compatmerge}. 

% \begin{figure}
% \begin{lstlisting}
% signature S0 = sig eqtype t end
% signature S1 = sig type t end
% signature S2 = sig include S0 include S2 end 

% S2 : sig eqtype t end

% signature S0 = sig type t end
% signature S1 = sig eqtype t end
% signature S2 = sig include S0 include S2 end

% S2 : sig type t end	
% \end{lstlisting}
% \caption{Unsound behavior of SML/NJ signature merging by \lstinline{include}}
% \label{fig:compatmerge}
% \end{figure}
	
% Despite its present incomplete state, SML/NJ can do the appropriate consistent merge for Garcia \etal's example (fig.~\ref{fig:garciasigmerge}). In the case of Garcia \etal's example, the typechecker needed to do is to note that the repeated components $t$ and $u$ due to inclusion are identical specifications. 
	
% 	The SML/NJ semantics goes further and merges consistent yet unequal specifications such as abstract types and datatypes. The merging semantics can be substantially improved by making the table more symmetrical and adjusting some of the precedences to something more sensible. Both Ramsey \cite{ramsey05} and Dreyer and Rossberg \cite{mixml} offer language support for a signature calculus that can safely compose signatures, effectively permitting a kind of multiple signature inheritance. Both accounts only model signature merging for a small language without support to features such as eqtype and generative datatypes. In particular, a fine-grain merging of eqtype can be nontrivial. For example, it is safe to merge \lstinline{eqtype t} and \lstinline{datatype t = K} where K is a data constructor. In contrast, merging \lstinline{eqtype t} and \lstinline{datatype t = K of int -> int} is unsafe. Consistent merge rules of this flavor can already be found elsewhere in the compiler, namely in signature matching. This dissertation will develop a formal semantics for a safe but flexible consistent signature merging that covers these features of ML. 

% 				\begin{table}
% 				\begin{tabular}{|l|l|l|l|l|}
% 				\hline
% 				     & type & eqtype & datatype & deftype\\
% 				\hline 
% 				type & \chk & eqtype & \ex & \ex\\
% 				\hline
% 				eqtype & type & \chk & \ex & \ex\\
% 				\hline
% 				datatype & \chk & datatype & \ex & \ex\\
% 				\hline
% 				deftype & \ex & \ex & \ex & \ex\\
% 				\hline
% 				datatype withtype & \chk & datatype withtype & \ex & \ex\\	
% 				\hline
% 				\end{tabular} 
% 				\caption{SML/NJ 110.68 Signature elaboration consistent signature merging: \chk~can be merged, \ex~cannot be merged, otherwise indicates specs mergable but indicated spec takes precedence}
% 				\label{tbl:njsigmerge}
% 				\end{table}
				
% \begin{figure}[ht]
% \hrulefill\\
% \begin{minipage}[b]{0.5\linewidth}
% \begin{lstlisting}[frame=none]
% signature S0 = 
%   sig 
%     type t
%     eqtype u
%   end
% \end{lstlisting}
% \end{minipage}
% \hspace{0.1em}
% \begin{minipage}[b]{0.5\linewidth}
% \begin{lstlisting}[frame=none]
% signature S1 = 
%   sig 
%     include S0
%     val x : int
%   end
% \end{lstlisting}
% \end{minipage}\\
% \begin{minipage}[b]{0.5\linewidth}
% \begin{lstlisting}[frame=none]
% signature S2 = 
%   sig 
%     include S0 
%     val y : unit 
%   end
% \end{lstlisting}
% \end{minipage}
% \hspace{0.1em}
% \begin{minipage}[b]{0.5\linewidth}
% \begin{lstlisting}[frame=none]
% signature S3 = 
%   sig 
%     include S1 
%     include S2 
%   end
% \end{lstlisting}
% \end{minipage}
% \hrulefill
% \caption{The naive macro expansion semantics of the Definition rejects S3. SML/NJ accepts it. This example was derived from Garcia \etal's GraphSig, IncidenceGraphSig, and VertexListGraphSig\cite{garcia05:extendedcomparing05}.}
% \label{fig:garciasigmerge}
% \end{figure}

% The signature merging semantics found in Ramsey \etal~is quite aggressive. In one example (fig.~\ref{fig:ramseymerge}), the merging semantics creates a new definitional type spec \lstinline{type u = t} in order to merge two signatures that disagree on an entangled value specification \lstinline{val x : t list} and \lstinline{val x : u list}. This kind of aggressiveness likely goes beyond the intention or expectation of the programmer. The programmer may have difficulty deciphering typechecking errors relating to S2.u and S2.x after this point because of this aggressive induced type sharing. It would be more sensible to have the typechecker complain that S0.x and S1.x are incompatible value specifications because as far as the typechecker and programmer are concerned, S0.t and S1.u are simply flexible type specifications. 

% \begin{figure}[ht]
% \hrulefill\\
% \begin{minipage}[b]{0.5\linewidth}
% \begin{lstlisting}[frame=none]
% signature S0 =
% sig
%   type t
%   type u 
%   val x : t list
% end
% \end{lstlisting}
% \end{minipage}
% \hspace{0.1cm}
% \begin{minipage}[b]{0.5\linewidth}
% \begin{lstlisting}[frame=none]
% signature S1 = 
% sig
%   type t
%   type u 
%   val x : u list
% end	
% \end{lstlisting}	
% \end{minipage}\\
% \begin{lstlisting}[frame=none]
% signature S2 =
% sig
%   type t 
%   type u = t
%   val x : t list
% end	
% \end{lstlisting}
% \hrulefill
% \caption{This is an example from Ramsey \etal \cite{ramsey05}. S2 is the merge (greatest lower bound) of S0 and S1 according to their semantics.}
% \label{fig:ramseymerge}
% \end{figure}

% Inspired by Ramsey \etal, my study of signature calculi will go beyond the semantics of consistent signature merging to consider the design implications of adding parameterized signatures\cite{jones96}, signature variables, and related features to the ML signature language.
% The ML signature language permits type definitions that may refer to general type expressions. Type expressions may involve both primitive type constructors such as $\rightarrow$ and programmer-defined type operators. It is the inclusion of type operators that gives the signature language much of its expressiveness. The semantics of type sharing constraints differs significantly between SML90 and SML97. Type sharing constraints could be imposed on two type constructors without restriction in SML90. In SML97, the designers partitioned the semantics of type sharing into type definitions which expressed sharing between an abstract type and an arbitrary type expression, and regular type sharing constraints which can only be imposed between two flexible (or primary) types whose names must be in scope. 

% 				A module system that permits both type definitions and type sharing constraints in signatures introduces significant new complexity. For example, whereas in Leroy's \cite{Leroy:generativity} TypModl language, which only permits SML90-style definitional type sharing constraints and no type definitions, type sharing constraints can be ``normalized'' by pushing them up the signature and eliminated by turning them into type definitions, type sharing constraints cannot be eliminated in a language that permits both type definitions and type sharing constraints. 

% In ML modules, structures can be arranged in a hierarchy. This feature enables flexible namespace management. In contrast, signatures cannot be arranged in such a hierarchy. Signatures must be defined at the top-level and can never be enclosed in any other signature or module. For complex hierarchies such the SML/NJ's Control module that contains layers of submodules, the corresponding signature CONTROL and the signatures of the submodules PRINT and ELAB are related only incidentally by occurrence in structure specifications in CONTROL. This shortcoming in the signature language unnecessarily pollutes the signature namespace and complicates browsing through and working with highly nested hierarchies. It would be desirable to permit (transparent) signature specifications within signatures. For added flexibility and perhaps increased expressiveness, it may be useful permit signature definitions within structures and functors. Furthermore, in order for modules to match these signatures enriched with signature specifications, modules must permit corresponding signature definitions. 

% \begin{figure}
% \begin{lstlisting}
% structure M = 
%   struct 
%     type t = int 
%     type u = bool * string 
%     val a : u * t
%   end 
% signature S = sign(M) removing u adding val b : t * t	
% \end{lstlisting}	
% \caption{Accessing and modifying (via Ramsey \etal~signature operations) an inferred signature}
% \label{fig:inferred}
% \end{figure}

% Since the semantics of ML already supports the extraction or inference of module signatures from the implementation, perhaps it makes sense to permit the programmer to operate directly on the inferred signatures of implementations (modules) or generate/synchronize module implementation based on the signature. Programmers often skip the step of writing proper signatures for modules because the necessary notation is cumbersome and potentially repetitive with respect to the module implementations of signatures. Some programming environments can help programmers automatically generate interfaces, but changes usually are not propagated bidirectionally. Synchronization of modules and signatures is ill-defined in the ML module system which supports a many-to-many relationship between modules and signatures. However, this is exactly where the existence of a principal signature or of a full signature might be useful. If programmers can leverage the structure of existing module structure when writing signatures, this might lower the barrier of entry for programmers with existing non-modularized source thus providing a path to ``gradual modularization''. For example, in fig.~\ref{fig:inferred} an inferred signature can be modified after the fact to serve as a template for future structures without having to explicitly write out any signature. This signature language will enable programmers to quickly integrate modular and non-modular code by facilitating rapid construction of variations on inferred signatures. Admittedly, this feature may run against the very spirit of splitting out explicit interfaces from implementations. 

% \section{Secondary Research Problems}
% \subsection{Static effects} including generative v. MixML mutation of types 
% \subsection{Module linking} functional and ad hoc linkage to show MixML is a radical departure from ML computational linking (the computation that happens before linking to produce the environments/closures that go along with the functions to be linked)

%% \section{Principal Goals of Research} Maybe here???
% \subsection{Polymorphism}
% \subsection{Generativity and abstraction}

				
% \section{Methodology}
% 	Informed by MixML, MacQueen-Tofte semantics, and FLINT semantics as the main sources of inspiration, my dissertation research will define a new formal semantics (including a dynamic semantics) and type system for true higher-order module system based on the current module system design in the SML/NJ compiler. The semantics will clarify and extend the implicit compiler semantics. Part of this study will include experimental prototypes evaluated according to the design criteria outlined above. This prototype module system will validate the practicality of the formal design. The prototype will include a module language elaborator including typechecker and basic compilation into a suitable typed intermediate language. 
	
% 	Through the course of formalizing the module system, the dissertation will establish type soundness of the module language for the dynamic semantics and type system in the style of Owens-Flatt but for the more powerful ML module system. Moreover, it will precisely define full transparency, separate compilation, and the relationship between the two. The hypothesis is that these two features are mutually exclusive because I conjecture that typechecking a signature language powerful enough to encode all possible relationships between functor parameter and body would be undecidable. If the hypothesis turns out to be false, then the dissertation should develop a signature calculus powerful enough to represent all possible static semantic actions of higher-order functor application. The final component of the dissertation will be the decidability and soundness of the signature calculus typechecking. 

\section{Organization}

Chapter~\ref{ch:background} gives an overview of the evolution of module systems and a discussion of related work including some of the latest variants of the ML module system. Chapter~\ref{ch:designspace} outlines and explores the design space of ML module systems. The focus is on the variants of higher-order functor semantics including applicative functors, type generativity, first- and second-class modules, separate compilation, and signature calculus design. It discusses the implications of true higher-order semantics with respect to the other major features of the module system and compares the design with other approaches to higher-order functors  and considers the implication of true higher-order semantics in light of the goal of separate compilation.  

Chapter~\ref{ch:typesystem} introduces the type system for the surface language. The type system is a first-order calculus supporting type constructors. The chapter covers the kind system and static semantics of the type systems. Chapter~\ref{ch:surfacelang} defines the core and module surface (\emph{i.e.}, syntactic) languages. The discussion covers important classifications of type constructors in the context of the module language. 

Chapter~\ref{ch:entitycalc} discusses the entity calculus, a small functional language that precisely defines the functor actions needed to fully express true higher-order semantics. The chapter gives a evaluation semantics for the entity calculus and establishes some key properties along with the semantics for the relativized type languages. The next part of the dissertation formally defines the semantics of the module system in two interweaving modes. Chapter~\ref{ch:homods} introduces the elaboration semantics that both constructs entity calculus expressions and typechecks programs using the result of the evaluation of those expressions. 

Chapter~\ref{ch:translation} defines a system that translates module syntax to an enriched variant of System F$_\omega$ and establishes the soundness of the elaboration semantics based on the soundness of System F$_\omega$'s type system. 

Chapter~\ref{ch:conclusion} outlines some directions for future work and concludes. Appendices~\ref{ch:smlnjsem} and~\ref{ch:impl} discuss semantics and implementation of higher-order modules in SML/NJ and contrasts it with that of the present work. Appendix~\ref{ch:proofs} provides the proofs of the formal properties in above chapters. Finally, Appendix~\ref{ch:notation-index} contains an index of the technical notations used in the body of the dissertation.



% Elsa Gunter, Faninguin (Mike Gordon's student) tried to develop a formal metatheory of Definition ML
% Peripheral goal of formalizing in 
% \section{Applications}
% \subsection{Type functionals vs. applicative functors}
% \subsection{Implied associative types}
% \subsection{Effects}

%%% Local Variables: 
%%% mode: latex
%%% TeX-master: "main"
%%% End: 

\include{lex}
\include{gram2}
\include{eval}
\include{type}
\include{direct}
\chapter{Standard bindings}
The initial top-level environment is comprised of a set of standard
bindings; the ``core'' bindings are described in this chapter, and
all bindings are described in Appendix~\ref{library}

The language provides the record type constructor 
$ \verb"{" {\bf lab}_1 \verb":" {\bf ty}_1 , \underline{\ \ \ }
 , {\bf lab}_n \verb":" {\bf ty}_n \verb"}"$ for any set 
$\{ {\bf lab}_i \} $ of labels and corresponding set 
$\{ {\bf ty}_i \} $ of types.  The language also provides the infixed
function-type constructor \verb"->".  Otherwise, type constructors
are postfixed.  The following are standard:

\begin{description}
\item[Type {\em constants} (nullary constructors):]  unit, bool, exn, int,
real, string
\item[Unary type constructors:]  list, ref
\end{description}

The constructors unit, bool, and list are fully defined by the
following assumed declaration
\begin{verbatim}
infixr 5 ::
type unit = {}
datatype    bool = true | false
datatype  'a list = nil |  :: of {1 : 'a, 2 : 'a list}
\end{verbatim}

The word ``unit'' is chosen since the type contains just one value
``\verb"{}"'', the empty record.  This is why it is preferred to the
word ``void'' of Algol-68.

The type constants \verb"int", \verb"real", and \verb"string"
are equipped with special
constants as described in section 2.3.  The type constructor
\verb"ref" is for constructing reference types; see
Chapter~\ref{reference}.
The type constant \verb"exn" is the type of all exceptions, and
is a datatype containing an unbounded number of constructors
generated by \verb"exception" bindings (see Chapter~\ref{exception}).

All standard functions, constants, and exceptions are listed in
Appendix~\ref{library}.


\chapter{Derived forms}
\label{derived}
ML is equipped with a number of {\em derived forms}, which in no way
add to the power of the language, as each is expressible in terms of
the more primitive constructs.
\section{Expressions and patterns}
\begin{tabular}{@{}l l}
{\bf Derived Form}&{\bf Equivalent Form} \\ \hline
\multicolumn{2}{l}{\bf Types:} \\
${\rm ty}_1$ \verb"*" \rep{2} \verb"*" ${\rm ty}_n$ &
\verb"{" 1 : ${\rm ty}_1$ , \rep{2} , $n$ : ${\rm ty}_n$ \verb"}"  \\ \hline
\multicolumn{2}{l}{\bf Expressions:} \\ 
\verb"()" & \verb"{ }"   {\it (no space between ``\/\verb"()"'')} \\ \xskip
\verb"(" ${\rm exp}_1$ \verb"," \rep{2} \verb"," ${\rm exp}_n$ \verb")" &
\verb"{" 1 \verb"=" ${\rm exp}_1$ , \rep{2} ,
$n$ \verb"=" ${\rm exp}_n$  \verb"}" \\ \xskip
\verb"case" exp \verb"of" match & ( \verb"fn" match ) ( exp ) \\ \xskip
\verb"#" lab & \verb"fn {" lab \verb"= x , ...} => x" \\ \xskip
\verb"if" exp \verb"then" ${\rm exp}_1$ \verb"else" ${\rm exp}_2$ &
\verb"case" exp \verb"of true =>" ${\rm exp}_1$ \\
& \ \ \ \ \ \ \ \ \ \  \verb"| false =>" ${\rm exp}_2$ \\ \xskip
${\rm exp}_1$ \verb"orelse" ${\rm exp}_2$ &
\verb"if" ${\rm exp}_1$ \verb"then true else" ${\rm exp}_2$ \\ \xskip
${\rm exp}_1$ \verb"andalso" ${\rm exp}_2$ &
\verb"if" ${\rm exp}_1$ \verb"then" ${\rm exp}_2$ \verb"else false" \\ \xskip
( ${\rm exp}_1$ ; \rep{1} ; ${\rm exp}_n$ ; exp) &
\verb"case"  ${\rm exp}_1$ \verb"of _ =>" \underline{\ \ \ } \verb"=>" \\
& \ \ \  \verb"case"  ${\rm exp}_n$ \verb"of _ =>" exp \\ \xskip
\verb"let" dec \verb"in" ${\rm exp}_1$ ; \rep{1} ; ${\rm exp}_n$ \verb"end"
&
\verb"let" dec \verb"in" ( ${\rm exp}_1$ ; \rep{1} ; ${\rm exp}_n$) \verb"end"
\\ \xskip
\verb"while" ${\rm exp}_1$ \verb"do" ${\rm exp}_2$ &
\parbox[t]{2.5in}{\begin{raggedright}
\verb"let val rec f = fn () =>" \\
\ \ \verb"if"  ${\rm exp}_1$ \verb"then (" ${\rm exp}_2$ \verb"; f()) else ()"
\\
\verb" in f() end"
\end{raggedright}} \\ \xskip
\verb"[" ${\rm exp}_1$ , \rep{0} , ${\rm exp}_n$ \verb"]" &
${\rm exp}_1$ \verb"::" \rep{0} \verb"::" ${\rm exp}_n$ \verb":: nil" \\
\hline
\pagebreak[1]
{\bf Derived Form}&{\bf Equivalent Form} \\ \hline
\multicolumn{2}{l}{\bf Patterns:} \\ 
\verb"()" & \verb"{ }"   {\it (no space between ``\/\verb"()"'')} \\ \xskip
\verb"(" ${\rm pat}_1$ \verb"," \rep{2} \verb"," ${\rm pat}_n$ \verb")" &
\verb"{" 1 \verb"=" ${\rm pat}_1$ , \rep{2} ,
$n$ \verb"=" ${\rm pat}_n$  \verb"}" \\ \xskip

\verb"[" ${\rm pat}_1$ , \rep{0} , ${\rm pat}_n$ \verb"]" &
${\rm pat}_1$ \verb"::" \rep{0} \verb"::" ${\rm pat}_n$ \verb":: nil" \\ \xskip
\verb"{" \underline{\ \ \ } , id , \underline{\ \ \ } \verb"}" &
\verb"{" \underline{\ \ \ } , id \verb"=" id , \underline{\ \ \ } \verb"}" \\ \xskip
\verb"{" \underline{\ \ \ } , id \verb"as" pat, \underline{\ \ \ } \verb"}" &
\verb"{" \underline{\ \ \ } , id \verb"=" id \verb"as" pat, \underline{\ \ \ } \verb"}" \\ \xskip
\verb"{" \underline{\ \ \ } , id : ty , \underline{\ \ \ } \verb"}" &
\verb"{" \underline{\ \ \ } , id \verb"=" id : ty , \underline{\ \ \ } \verb"}" \\
\hline
\end{tabular}

Each derived form is identical semantically to its ``equivalent
form.''  The type-checking of each derived form is also defined by
that of its equivalent form.  The precedence among all primitive and
derived forms is shown in Appendix~\ref{grammar}.

The derived type ${\rm ty}_1$ \verb"*" \rep{2} \verb"*" ${\rm ty}_n$
is called an (n--)tuple type, and the values of this type are called
(n--)tuples.

The final derived pattern allows a label and its associated value to
be elided in a record pattern, when they are the same identifier.

\section{Bindings and declarations}

A syntax class {\bf fb} of function bindings is used as a convient
form of value binding for (possibly recursive) function declarations.
The equivalent form of each function binding is an ordinary value
binding.  These new function bindings must be declared by \verb"fun",
not by \verb"val"; however, functions may still be declared using
\verb"val" or \verb"val rec" along with \verb"fn" expressions.

\begin{tabular}{@{}l l}
\multicolumn{1}{c}{\bf Derived Form}&
\multicolumn{1}{c}{\bf Equivalent Form} \\ \hline
\multicolumn{2}{l}{\bf Function bindings {\rm fb}:} \\
& id = \verb"fn" $x_1$ \verb"=>" \rep{1} \verb"=> fn" $x_n$ \verb"=>" \\
 & \ \ \verb"case (" $x_1,$ \underline{\ \ \ } $, x_n$ \verb")" \\
\ id ${\rm apat}_{11}$ \rep{1} ${\rm apat}_{1n}$ cst = ${\rm exp}_1$ &
\ \ \verb"of" ( ${\rm apat}_{11}$ , \rep{1} , ${\rm apat}_{1n}$  \verb"=>" ${\rm exp}_1$ cst \\
\verb"|" \underline{\ \ \ } & \ \ \verb"|" \underline{\ \ \ } \\
\verb"|" id ${\rm apat}_{m1}$ \rep{1} ${\rm apat}_{mn}$ cst = ${\rm exp}_m$ &
\ \ \ \verb"|" ( ${\rm apat}_{m1}$ , \rep{1} , ${\rm apat}_{mn}$  \verb"=>" ${\rm exp}_m$ cst \\

 & \\
${\rm fb}_1$ and \rep{1} and ${\rm fb}_n$ &
${\rm vb}_1$ and \rep{1} and ${\rm vb}_n$ \\
&{\it (where ${\rm vb}_i$ is the equivalent of\/ ${\rm fb}_i$) } \\
\hline
\multicolumn{2}{l}{\bf Declarations:} \\
\verb"fun" fb & \verb"val rec" vb \\
&{\it (where \/{\rm vb} is the equivalent of\/ {\rm fb}) } \\
& \\
exp & \verb"val it =" exp \\
\hline
\end{tabular}
In the table above, ``cst'' stands for an optional type constraint---a colon
followed by a type expression.
The last derived declaration (using ``it'') is only allowed at
top-level, for treating top-level expressions as degenerate
declarations; ``it'' is just a normal value variable.

\chapter{Equality}

The semantics of equality testing is not yet final (at least in the
current implementation).  This chapter just describes the behavior of
the current implementation.  In one author's opinion, the testing for
equality of values whose type is not known at compile-time is a grave
mistake, as it severely constrains the implementation.

The equality function \verb"op = : 'a * 'a -> bool" is available at
all types \verb"'a".  However, function-values cannot successfully be
compared for equality; an attempt to test the equality of two
functions will raise an exception as described below.

Two values are tested for equality as follows, depending on the kind
of value:
\begin{description}
\item[Primitive types] like integers, reals, and strings have
equality functions with the conventional behavior.

\item[Function types] cannot be compared.  Testing two functions for equality
will raise the \verb"Equal" exception.

\item[Reference types:] On references, equality means identity; a
reference is equal to itself and to no other references, regardless
of similar contents.

\item[Record types] may be compared even if they have functions as
compenents.  Records containing functions may be found unequal (if
non-functional fields are unequal, and are examined first), or
the \verb"Equal" exception may be raised; testing is in alphabetical
order of field names.

\item[Datatypes] may be compared for equality even if built from
function types.  Two elements of a datatype are equal if they employ
the same constructor applied to equal values; testing the equality of
such values might lead to an exception being raised.

\item[Opaque types] from functor parameters and abstractions are
tested for equality as if they were not opaque.
\end{description}

The compiler may give a warning message on testing equality of a type
which must always raise an exception.

Some implementations of ML prohibit (statically) the compilation of
equality over types which are built from function-values.  To this
end, ``equality type variables'' are used to stand for the set of
types that admit equality, just as ordinary type variables stand for
the set of all types.  Equality type variables are written with two
initial apostrophes.  In implementations with the more permissive
equality mechanism, equality type variables may be used; but they might not
be checked more restrictively than ordinary type variables.

\include{exceptions}
\include{reference}
\include{reftype}
\include{module}
\appendix
\include{gram}
\chapter{The standard library}
\label{library}

A standard set of values, types, exceptions, etc. are {\em
pervasive}---they are in the initial environment and available in
every structure.  This {\em standard library} is grouped into
structures; each structure deals with the operations on one or two
abstract types.  The signature of each of these modules, with an
informal explanation of the semantics, is given in this chapter.

\begin{verbatim}
signature GENERAL =
  sig
    infix 3 o
    infix before
    exception Bind
    exception Match
    exception Interrupt
    val o : ('b -> 'c) * ('a -> 'b) -> ('a -> 'c)
    val before : ('a * 'b) -> 'a
    val system : string -> unit
    val cd : string -> unit
    exception Equal
    datatype 'a option = NONE | SOME of 'a
    type exn
    type unit
  end

abstraction General : GENERAL
\end{verbatim}
The structure \verb"General" contains various miscellaneous and
general-purpose values, types, and exceptions.  The type \verb"exn"
is the type of all exceptions.

\section{List}
\begin{verbatim}
signature LIST =
  sig
    infixr 5 :: @
    datatype 'a list = :: of ('a * 'a list) | nil
    exception Hd
    exception Tl
    exception Nth
    val hd : 'a list -> 'a
    val tl : 'a list -> 'a list 
    val null : 'a list -> bool 
    val length : 'a list -> int 
    val @ : 'a list * 'a list -> 'a list
    val rev : 'a list -> 'a list 
    val map :  ('a -> 'b) -> 'a list -> 'b list
    val fold : (('a * 'b) -> 'b) -> 'a list -> 'b -> 'b
    val revfold : (('a * 'b) -> 'b) -> 'a list -> 'b -> 'b
    val app : ('a -> 'b) -> 'a list -> unit
    val revapp : ('a -> 'b) -> 'a list -> unit
    val nth : 'a list * int -> 'a 
    val exists : ('a -> bool) -> 'a list -> bool
  end
\end{verbatim}
The semantics of this module are defined by the
following implementation.
\begin{verbatim}
abstraction List: LIST =
  struct
    infixr 5 :: @ 
    infix 6 + -
    datatype 'a list = :: of ('a * 'a list) | nil
    exception Hd
    fun hd (a::r) = a | hd nil = raise Hd
    exception Tl
    fun tl (a::r) = r | tl nil = raise Tl    
    fun null nil = true | null _ = false
    fun length nil = 0 | length (a::r) = 1 + length r
    fun op @ (nil,l) = l | op @ (a::r, l) = a :: (r@l)
    fun rev l = let fun f (nil, h) = h 
                      | f (a::r, h) = f(r, a::h)
                in  f(l,nil)
                end
    fun map f = let fun m nil = nil | m (a::r) = f a :: m r
                in  m
                end
    fun fold f = let fun f2 nil = (fn b => b)
                       | f2 (e::r) = (fn b => f(e,(f2 r b)))
                 in  f2
                 end
    fun revfold f l = fold f (rev l)
    fun app f l = (map f l; ())
    fun revapp f l = app f (rev l)
    exception Nth
    fun nth(e::r,0) = e 
      | nth(e::r,n) = nth(r,n-1)
      | nth _ = raise Nth
    fun exists f =
          let fun g nil = false | g (h::t) = f h orelse g t
          in  g
          end
  end
\end{verbatim}
\section{Array}
\begin{verbatim}
signature ARRAY =
  sig
    infix 3 sub
    type 'a array
    exception Subscript
    val array : int * '1a -> '1a array
    val sub : 'a array * int -> 'a
    val update : 'a array * int * 'a -> unit
    val length : 'a array -> int
    val arrayoflist : 'a list -> 'a array
  end
\end{verbatim}
Arrays may be made whose elements are any type.  \verb"array(n,x)"
returns a new array of $n$ elements, indexed from $0$ to $n-1$,
initialized to $x$.  \verb"a sub i" returns the $i^{th}$ element of
the array $a$.  \verb"update(a,i,z)" sets the $i^{th}$ element of the
array $a$ to the value $z$.

Two arrays are equal if and only if they are the same array (created
with the same call to \verb"array"); except that all arrays of length
0 may be equal to each other, depending on the implementation.

The following implementation defines the semantics of arrays, though
in practice arrays are implemented much more efficiently.
\begin{verbatim}
abstraction Array : ARRAY =
  struct
   type 'a array = 'a ref list
   exception Subscript
   fun array(0,x) = nil | array(n,x) = ref x :: array(n-1,x)
   fun a sub i = !(nth(a,i)) handle Nth => raise Subscript
   fun update(a,i,z) = nth(a,i) := z handle Nth => raise Subscript
   fun length a = List.length a
   fun arrayoflist l = l
  end
\end{verbatim}
\section{Input/Output}
The input/output primitives are intended as a simple basis that may
be compatibly superseded by a more comprehensive I/O system that
provides for streams of arbitrary type or a richer repertoire of I/O
operations.  The IO structure contains all I/O primitives; this
structure will in all implementation
match (with thinning) the BASICIO signature provided
below, but may contain other primitives as well.

\begin{verbatim}
signature BASICIO = 
  sig
    type instream 
    type outstream
    exception Io_failure of string
    val std_in : instream
    val std_out : outstream
    val open_in : string -> instream
    val open_out : string -> outstream
    val close_in : instream -> unit
    val close_out : outstream -> unit
    val output : outstream -> string -> unit
    val input : instream -> int -> string
    val lookahead : instream -> string
    val end_of_stream : instream -> bool
  end

\end{verbatim}
The type \verb"instream" is the type of input streams and the type
\verb"outstream" is the type of output streams.  The exception
\verb"Io_failure" is used to represent all of the errors that may
arise in the course of performing I/O.  The value associated with
this exception is a string representing the type of failure.  In
general, any I/O operation may fail if, for any reason, the host
system is unable to perform the requested task.  The value associated
with the exception should describe the type of failure, insofar as
this is possible.

The standard prelude binds \verb"std_in" to an instream and
\verb"std_out" to an outstream.  For interactive ML processes, these
are expected to be associated with the user's terminal.  However, an
implementation that supports the connection of processes to streams
may associate one process's \verb"std_in" to another's
\verb"std_out".

The \verb"open_in" and \verb"open_out" primitives are used to
associate a disk file with a stream.  The expression
\verb"open_in(s)" creates a new instream whose producer is the file
named \verb"s" and returns that stream as a value.
Similarly, \verb"open_out(s)" creates a new \verb"outstream"
associated with the file \verb"s", and returns that stream.

The \verb"input" primitive is used to read characters from a stream.
Evaluation of \verb"input s n" causes the removal of \verb"n"
characters from the input stream \verb"s".  If fewer than \verb"n"
characters are currently available, then the ML system will block
until they become available from the producer associated with
\verb"s"\footnote{The exact definition of ``available'' is
implementation-dependent.  For instance, operating systems typically
buffer terminal input on a line-by-line basis so that no characters
are available until an entire line has been typed.}.
If the end of stream is reached while processing an \verb"input",
fewer than \verb"n" characters may be returned.  
Attempting \verb"input" from a closed stream raises
\verb"Io_failure".

The function \verb"lookahead(s)" returns the next character on
\verb"instream s" without removing it from the stream.  Input streams
are terminated by the \verb"close_in" operation.  This primitive is
provided primarily for symmetry and to support the re-use of
unused streams on resource-limited systems.  The end of an input
stream is detected by \verb"end_of_stream", a derived form that is
defined as follows:
\begin{verbatim}
fun end_of_stream(s) = (lookahead(s)="")
\end{verbatim}

Characters are written to an \verb"outstream" with the \verb"output"
primitive.  The string argument consists of the characters to be
written to the given outstream.  The function \verb"close_out" is
used to terminate an output stream.  Any further attempts to output
to a closed stream cause \verb"Io_failure" to be raised.

\begin{verbatim}
(* Extended IO *)
    structure BasicIO : BASICIO
    val execute : string -> BasicIO.instream * BasicIO.outstream
    val flush_out : BasicIO.outstream -> unit
    val can_input : BasicIO.instream * int -> bool
    val input_line : BasicIO.instream -> string
    val open_append : string -> BasicIO.outstream
    val is_term_in : BasicIO.instream -> bool
    val is_term_out : BasicIO.outstream -> bool
\end{verbatim}

In addition to the basic I/O primitives, provision is made for some
extensions that are likely to be provided by many implementations.
The functions listed above are provided by Standard ML of New Jersey.

The function \verb"execute" is used to create a pair of streams, one an
\verb"instream" and one an \verb"outstream", and associate them with
a process.  The string argument to \verb"execute" is the
operating-system command that starts the process.

The function \verb"flush_out" ensures that the consumer associated
with an \verb"out_stream" has received all of the characters
associated with that stream.  It is provided primarily to allow the
ML user to circumvent undesirable buffering characteristics that may
arise in connection with terminals and other processes.  All output
streams are flushed when they are closed, and in many implementations
an output stream is flushed whenever a newline character is written
if that stream is connected to a terminal.

The function \verb"can_input" returns the number of characters
which may be read from its instream argument without blocking.
For instance, a command processor may wish
to test whether or not a user has typed ahead in order to avoid an
unnecessary prompt.  The exact definition of ``currently available''
is implementation specific, perhaps depending on such things as the
processing mode of a terminal.

The \verb"input_line" primitive returns a string consisting of the
characters from an \verb"instream" up through, and including, the
next end of line character.  If the end of stream is reached without
reaching an end of line character, all remaining characters from the
stream ({\em without} an end of line character) are returned.

Files may be open for output while preserving their contents by using
the \verb"open_append" primitive.  Subsequent \verb"output" to the
outstream returned by this primitive is appended to the contents of
the specified file.

Basic support for the complexities of terminal I/O are provided.  The
pair of functions \verb"is_term_in" and \verb"is_term_out" test
whether or not a stream is associated with a terminal.  These
functions are especially useful with \verb"std_in" and \verb"std_out"
because they are opened as part of the standard prelude.  A terminal
may be associated with a stream using the ordinary \verb"open_in" and
\verb"open_out" functions; the naming convention to do this is
implementation-dependent.  Terminal I/O is, in general, more complex
than ordinary file I/O.   In most cases the \verb"ExtendedIO" module
provided by an implementation will have additional operations to
support mode control.  Since the details of such control operations
are highly host-dependent, the functions that may be provided are
left unspecified.
\section{Bool}
\begin{verbatim}
signature BOOL =
  sig
    datatype bool = true | false
    val not: bool -> bool
    val print: bool -> bool
    val makestring: bool -> string
  end
\end{verbatim}
These are quite straightforward, and can be defined as follows:
\begin{verbatim}
abstraction Bool: BOOL =
  struct
    datatype bool = true | false
    fun not true = false | not false = true
    fun makestring true = "true" | makestring false = "false"
    fun print b = (output(std_out, makestring b); b)
  end
\end{verbatim}
\section{ByteArray}
\begin{verbatim}
signature BYTEARRAY =
  sig
    infix 3 sub
    type bytearray
    exception Subscript
    exception Range
    val array : int * int -> bytearray
    val sub : bytearray * int -> int
    val update : bytearray * int * int -> unit
    val length : bytearray -> int
    val extract : bytearray * int * int -> string
    val fold : ((int * 'b) -> 'b) -> bytearray -> 'b -> 'b
    val revfold : ((int * 'b) -> 'b) -> bytearray -> 'b -> 'b
    val app : (int -> 'a) -> bytearray -> unit
    val revapp : (int -> 'b) -> bytearray -> unit
  end
\end{verbatim}
Byte arrays are just like arrays of integers, with the restriction
that the values of the component integers must be between 0 and 255.
The intent is that the implementation may store them more efficiently
than the equivalent array.

Note that, by default, the ByteArray structure is present but 
not opened in the 
initial environment.  The declaration \verb"open ByteArray" may be
used to open it.

Their semantics can be defined by this implementation:
\begin{verbatim}
abstraction ByteArray : BYTEARRAY =
  struct
    infix 3 sub
    type bytearray = int array
    exception Subscript = Array.Subscript
    exception Range
    fun check x = if x<0 orelse x>255 then raise Range else ()
    fun array(i,x) = (check x; Array.array(i,x))
    val length = Array.length
    fun update(a,i,x) = (check x; Array.update(a,i,x))
    val op sub = Array.sub
    fun extract(b,i,0) = if i<0 orelse i>length(b)
                          then raise Subscript  else ""
      | extract(b,i,n) = chr(b sub i) ^ extract(b,i,n-1)
    val fold =  . . .
    val revfold =  . . .
    val app = ...
    val revapp = ...
  end
\end{verbatim}

\section{Integer}
\begin{verbatim}
signature INTEGER = 
  sig
    infix 7 * div mod
    infix 6 + -
    infix 4 > < >= <=
    exception Div
    exception Mod
    exception Overflow
    type int
    val ~ : int -> int
    val * : int * int -> int
    val div : int * int -> int
    val mod : int * int -> int
    val + : int * int -> int
    val - : int * int -> int
    val >  : int * int -> bool
    val >= : int * int -> bool
    val <  : int * int -> bool
    val <= : int * int -> bool
    val min : int * int -> int
    val max : int * int -> int
    val abs : int -> int
    val print : int -> int
    val makestring : int -> string
  end
\end{verbatim}
This should be mostly self-explanatory.
The function \verb"div" raises \verb"Div" on divide by zero,
otherwise \verb"Overflow" if the result doesn't fit;  similarly
\verb"mod" may raise \verb"Mod" or \verb"Overflow".  Other operators
may raise \verb"Overflow" if the result doesn't fit into their
representation.  Implementations
may use infinite-precision integers, in which case \verb"Overflow"
won't be raised.  Finite precision implementations should allow
integers of at least 30 bits.  The author recommends that
infinite-precision be added as a separate package.
\section{Real}
\begin{verbatim}
signature REAL =
  sig
    infix 7 * /
    infix 6 + -
    infix 4 > < >= <=
    type real
    exception Floor and Sqrt and Exp and Ln
    exception Float of string
    val ~ : real -> real 
    val + : (real * real) -> real 
    val - : (real * real) -> real 
    val * : (real * real) -> real 
    val / : (real * real) -> real 
    val > : (real * real) -> bool
    val < : (real * real) -> bool
    val >= : (real * real) -> bool
    val <= : (real * real) -> bool
    val abs : real ->  real
    val real : int -> real
    val floor : real -> int
    val sqrt : real -> real
    val sin : real -> real
    val cos : real -> real
    val arctan : real -> real
    val exp : real -> real
    val ln : real -> real
    val print : real -> real
    val makestring : real -> string
  end
\end{verbatim}
\section{Ref}
\begin{verbatim}
signature REF = 
  sig
    infix 3 :=
    val ! : 'a ref -> 'a
    val := : 'a ref * 'a -> unit
    val inc : int ref -> unit
    val dec : int ref -> unit
  end
\end{verbatim}

\begin{verbatim}
abstraction Ref : REF = 
  struct
    fun ! (ref x) = x
    val := (* not explainable in terms of other primitives *)
    fun inc (x as ref i) = x := i+1
    fun dec (x as ref i) = x := i-1
  end
\end{verbatim}
\section{String}
\begin{verbatim}
signature STRING =
  sig
    infix 6 ^
    infix 4 > < >= <=
    type string
    exception Substring
    val length : string -> int
    val size : string -> int
    val substring : string * int * int -> string
    val explode : string -> string list
    val implode : string list -> string
    val <= : string * string -> bool
    val <  : string * string -> bool
    val >= : string * string -> bool
    val >  : string * string -> bool
    val ^  : string * string -> string
    exception Chr
    val chr : int -> string 
    exception Ord
    val ord : string -> int 
    val ordof : string * int -> int 
    val print : string -> string
  end
\end{verbatim}
Strings can be explained by the implementation below; though in
practice a more efficient implementation will often be used.
\begin{verbatim}
abstraction String : STRING =
  struct
    infix 6 ^
    infix 4 > < >= <=
    type string = int list
    exception Substring
    val length = List.length
    val size = length
    fun substring(s,0,0) = nil
      | substring(a::b,0,len) = a::substring(b,0,len-1)
      | substring(nil,_,_) = raise Substring
      | substring(a::b,i,len) = substring(b,i-1,len)
    fun explode nil = nil
      | explode (i::l) = [i] :: explode l
    fun implode nil = nil
      | implode (s::l) = s @ implode l
    fun (_::_) > nil = true
      | nil > (_::_) = false
      | (i::r) > (j::s) = Integer.>(i,j) orelse i=j andalso r>s
    fun a <= b = not (a>b)
    fun a < b = b > a
    fun a >= b = b <= a
    val op ^ = op @
    exception Chr
    fun chr i = if i<0 orelse i>255 then raise Chr else [i]
    exception Ord
    fun ord nil = raise Ord | ord (i::r) = i
    fun ordof(s,i) = nth s handle Nth => raise Ord
    fun print s = (output(std_out,s); s)
  end
\end{verbatim}
\section{Pervasives}
\begin{verbatim}
signature PERVASIVES =
  sig
    structure Ref : REF
    structure List : LIST
    structure Array : ARRAY
    structure ByteArray : BYTEARRAY
    structure IO : IO
    structure Bool : BOOL
    structure String : STRING
    structure Integer : INTEGER
    structure Real : REAL
    structure General : GENERAL
  end
\end{verbatim}
A module \verb"Pervasives", matching this signature, is defined.  The
initial environment contains this structure, which has been
recursively ``\verb"open"ed,'' so that (for example) \verb"nil" may
be referred to as \verb"nil", \verb"List.nil", or
\verb"Pervasives.List.nil".

\section{Extensions}
\begin{verbatim}
signature Extensions =
  sig
        val use : string -> unit
        val exportML : string -> bool
        val exportFn : string * (string list * string list -> unit) -> unit
        val exn_name : exn -> string
  end
\end{verbatim}
The extensions to the standard may vary from implementation to
implementation.  These are a suggested few.
\begin{description}
\item[\verb"use"]  Given a name of a file, \verb"use" compiles
and executes its contents as if they were typed into the top-level
prompt of the interactive system.  \verb"use" may be nested
recursively.

\item[\verb"exportML"]  Creates an executable file whose name is
given by the argument.  When this file is executed, it is an ML
system in exactly the same state as the one that wrote the file.  For
example, the command
\verb|(exportML "foo"; print "Hello");| writes a file that, when
executed, prints \verb"Hello" and then returns to the top-level
prompt.  exportML returns true when the executable file is run,
and false when simply returning.

\item[\verb"exportFn"]  Creates an executable file whose name is given
by the first argument.  When this file is executed, it is an ML
system that calls the function given as the second argument, then
exits.  The ML system created will not have a compiler or a
top-level, so it will be significantly more compact.
The command-line arguments and environment
are passed as the string list arguments to the
function that is called.   \verb"exportFn" may terminate
execution of the ML system that called it.

\item[\verb"exn\_name"]   Given an exception value as an argument,
returns the name of the exception constructor used to create it.
\end{description}

\include{compat}
\end{document}
