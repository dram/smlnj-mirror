\chapter{Compatibility}
The language definition has changed a bit over the past year
(1986--87), particularly in the area of exception syntax.  Some
attempt has been made to allow old programs to continue running.
\section{Exceptions}
Three keywords: \verb"exceptionx", \verb"raisex", and \verb"handlex"
are provided.  They implement the old-style exception mechanism, and
are treated as derived forms.
\begin{tabular}{@{}l l}
\multicolumn{1}{c}{\bf Derived Form}&
\multicolumn{1}{c}{\bf Equivalent Form} \\ \hline
\verb"exceptionx" identifier & \verb"exception" Identifier \\
\verb"exceptionx" identifier \verb":" ty & \verb"exception" Identifier \verb"of" ty \\

\verb"raisex" identifier & \verb"raise" Identifier \\
\verb"raisex" identifier \verb"with" exp& \verb"raise" Identifier \verb"(" exp \verb")" \\

\verb"handlex" ident \verb"=>" exp & \verb"handle" Ident \verb"=>" exp \\
\verb"handlex" ident \verb"=>" exp & \verb"handle" Ident () \verb"=>" exp \\

\verb"handlex" ident \verb"with" pat \verb"=>" exp & \verb"handle" Ident pat \verb"=>" exp \\
\hline
\end{tabular}

The derivations for \verb"handlex" are a bit more intricate than
shown above.  The intent is that any program that works under the old
scheme will continue to work if all instances of \verb"exception",
\verb"handle", and \verb"raise" are changed to 
 \verb"exceptionx", \verb"handlex", and \verb"raisex" respectively.

Note that the derived forms change the exception identifier in
converting to the standard forms; they capitalize the first letter.
This is to simulate the old scheme, in which exception names lived in
a different space from value names.  This will cause problems in any
programs with a capitalized value-name that happens to conflict with
a (capitalized or uncapitalized) exception name.
